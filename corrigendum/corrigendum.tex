\documentclass[11pt,a4paper]{article}
\usepackage[utf8]{inputenc}							% this is needed for umlauts
\usepackage[T1]{fontenc}							% this is needed for correct output of umlauts
\usepackage[ngerman]{babel}							% set document language to new german
\usepackage{amsfonts}								% for infinity sign
\usepackage{adjustbox}								% for adjustbox environment

\usepackage{tikz}									% for drawings
\usetikzlibrary{arrows,decorations.pathmorphing,backgrounds,positioning,fit,petri} % load options
\tikzstyle{every picture}+=[font=\sffamily]			% use sans serif font for tikz pictures -> http://tex.stackexchange.com/questions/4887/pgf-tikz-and-sans-serif-fonts

\usepackage[singlelinecheck = off,labelsep = period]{caption} 							% -> caption is aligned with table
\captionsetup[subfigure]{labelformat=simple, labelsep = period, listofformat=subsimple} % (a) -> a.
\captionsetup[figure]{labelfont=it, labelsep = period}									% italic 

\usepackage[left=118pt,right=118pt, top=50pt, bottom = 40pt]{geometry}		% for custom margins

\date{}
\title{\textbf{Corrigendum}:\\Der Zusammenhang zwischen Spatial-Suppression, Mental-Speed und psychometrischer Intelligenz}

\begin{document}
	\maketitle
	\thispagestyle{empty}

	\noindent Bei der Analyse der Spatial-Suppression-Aufgabe wurden die Erkennungsschwellen jeder Person mit einem exponentiellen Modell der Form $y=a \times e^{bx}$ vorhergesagt. Der Parameter $a$ wurde in der vorliegenden Arbeit Asymptote genannt (siehe auch Melnick et al., 2013).
	Diese Bezeichnung ist falsch, weil die horizontale Asymptote einer Funktion der Form $y=a \times e^{bx}$ im minus-Unendlichen immer Null beträgt ($e^{-\infty}=0$). Die korrekte Bezeichnung für den Parameter $a$ lautet Skalierungsparameter, weil er den gesamten Kurvenverlauf, unter Einhaltung der Verhältnisse der Datenpunkte zueinander, skaliert (siehe Abbildung 1).
	
	\begin{figure}[h]
		\centering
		%	\captionsetup{font = small}
		\begin{adjustbox}{width=.91\textwidth}
			% Created by tikzDevice version 0.10.1 on 2017-04-01 11:24:37
% !TEX encoding = UTF-8 Unicode
\begin{tikzpicture}[x=1pt,y=1pt]
\definecolor{fillColor}{RGB}{255,255,255}
\path[use as bounding box,fill=fillColor,fill opacity=0.00] (0,0) rectangle (505.89,505.89);
\begin{scope}
\path[clip] ( 54.00, 48.00) rectangle (505.89,505.89);
\definecolor{drawColor}{RGB}{0,0,0}

\path[draw=drawColor,line width= 0.4pt,line join=round,line cap=round] (  0.00,230.76) --
	(  0.66,230.83) --
	(  3.31,231.10) --
	(  5.93,231.37) --
	(  8.50,231.64) --
	( 11.04,231.91) --
	( 13.55,232.19) --
	( 16.02,232.46) --
	( 18.46,232.73) --
	( 20.87,233.00) --
	( 23.24,233.27) --
	( 25.59,233.54) --
	( 27.90,233.81) --
	( 30.19,234.08) --
	( 32.45,234.36) --
	( 34.68,234.63) --
	( 36.88,234.90) --
	( 39.06,235.17) --
	( 41.21,235.44) --
	( 43.33,235.71) --
	( 45.43,235.98) --
	( 47.51,236.26) --
	( 49.56,236.53) --
	( 51.59,236.80) --
	( 53.60,237.07) --
	( 55.59,237.34) --
	( 57.55,237.61) --
	( 59.49,237.88) --
	( 61.42,238.15) --
	( 63.32,238.43) --
	( 65.20,238.70) --
	( 67.07,238.97) --
	( 68.91,239.24) --
	( 70.74,239.51) --
	( 72.55,239.78) --
	( 74.34,240.05) --
	( 76.11,240.33) --
	( 77.87,240.60) --
	( 79.60,240.87) --
	( 81.33,241.14) --
	( 83.03,241.41) --
	( 84.72,241.68) --
	( 86.40,241.95) --
	( 88.06,242.22) --
	( 89.70,242.50) --
	( 91.33,242.77) --
	( 92.95,243.04) --
	( 94.55,243.31) --
	( 96.14,243.58) --
	( 97.71,243.85) --
	( 99.27,244.12) --
	(100.81,244.40) --
	(102.35,244.67) --
	(103.87,244.94) --
	(105.38,245.21) --
	(106.87,245.48) --
	(108.36,245.75) --
	(109.83,246.02) --
	(111.29,246.29) --
	(112.73,246.57) --
	(114.17,246.84) --
	(115.60,247.11) --
	(117.01,247.38) --
	(118.41,247.65) --
	(119.81,247.92) --
	(121.19,248.19) --
	(122.56,248.47) --
	(123.92,248.74) --
	(125.27,249.01) --
	(126.61,249.28) --
	(127.94,249.55) --
	(129.27,249.82) --
	(130.58,250.09) --
	(131.88,250.36) --
	(133.17,250.64) --
	(134.46,250.91) --
	(135.73,251.18) --
	(137.00,251.45) --
	(138.26,251.72) --
	(139.51,251.99) --
	(140.75,252.26) --
	(141.98,252.54) --
	(143.20,252.81) --
	(144.42,253.08) --
	(145.62,253.35) --
	(146.82,253.62) --
	(148.02,253.89) --
	(149.20,254.16) --
	(150.38,254.43) --
	(151.54,254.71) --
	(152.71,254.98) --
	(153.86,255.25) --
	(155.01,255.52) --
	(156.15,255.79) --
	(157.28,256.06) --
	(158.40,256.33) --
	(159.52,256.61) --
	(160.63,256.88) --
	(161.74,257.15) --
	(162.83,257.42) --
	(163.93,257.69) --
	(165.01,257.96) --
	(166.09,258.23) --
	(167.16,258.50) --
	(168.23,258.78) --
	(169.29,259.05) --
	(170.34,259.32) --
	(171.39,259.59) --
	(172.43,259.86) --
	(173.46,260.13) --
	(174.49,260.40) --
	(175.52,260.68) --
	(176.54,260.95) --
	(177.55,261.22) --
	(178.55,261.49) --
	(179.56,261.76) --
	(180.55,262.03) --
	(181.54,262.30) --
	(182.53,262.57) --
	(183.51,262.85) --
	(184.48,263.12) --
	(185.45,263.39) --
	(186.41,263.66) --
	(187.37,263.93) --
	(188.33,264.20) --
	(189.28,264.47) --
	(190.22,264.75) --
	(191.16,265.02) --
	(192.09,265.29) --
	(193.02,265.56) --
	(193.95,265.83) --
	(194.87,266.10) --
	(195.78,266.37) --
	(196.69,266.64) --
	(197.60,266.92) --
	(198.50,267.19) --
	(199.40,267.46) --
	(200.29,267.73) --
	(201.18,268.00) --
	(202.06,268.27) --
	(202.94,268.54) --
	(203.82,268.82) --
	(204.69,269.09) --
	(205.56,269.36) --
	(206.42,269.63) --
	(207.28,269.90) --
	(208.14,270.17) --
	(208.99,270.44) --
	(209.84,270.71) --
	(210.68,270.99) --
	(211.52,271.26) --
	(212.35,271.53) --
	(213.18,271.80) --
	(214.01,272.07) --
	(214.84,272.34) --
	(215.66,272.61) --
	(216.47,272.88) --
	(217.29,273.16) --
	(218.10,273.43) --
	(218.90,273.70) --
	(219.70,273.97) --
	(220.50,274.24) --
	(221.30,274.51) --
	(222.09,274.78) --
	(222.88,275.06) --
	(223.66,275.33) --
	(224.45,275.60) --
	(225.22,275.87) --
	(226.00,276.14) --
	(226.77,276.41) --
	(227.54,276.68) --
	(228.30,276.95) --
	(229.07,277.23) --
	(229.82,277.50) --
	(230.58,277.77) --
	(231.33,278.04) --
	(232.08,278.31) --
	(232.83,278.58) --
	(233.57,278.85) --
	(234.31,279.13) --
	(235.05,279.40) --
	(235.78,279.67) --
	(236.51,279.94) --
	(237.24,280.21) --
	(237.97,280.48) --
	(238.69,280.75) --
	(239.41,281.02) --
	(240.13,281.30) --
	(240.84,281.57) --
	(241.55,281.84) --
	(242.26,282.11) --
	(242.97,282.38) --
	(243.67,282.65) --
	(244.37,282.92) --
	(245.07,283.20) --
	(245.76,283.47) --
	(246.45,283.74) --
	(247.14,284.01) --
	(247.83,284.28) --
	(248.51,284.55) --
	(249.20,284.82) --
	(249.88,285.09) --
	(250.55,285.37) --
	(251.23,285.64) --
	(251.90,285.91) --
	(252.57,286.18) --
	(253.24,286.45) --
	(253.90,286.72) --
	(254.56,286.99) --
	(255.22,287.27) --
	(255.88,287.54) --
	(256.53,287.81) --
	(257.19,288.08) --
	(257.84,288.35) --
	(258.48,288.62) --
	(259.13,288.89) --
	(259.77,289.16) --
	(260.41,289.44) --
	(261.05,289.71) --
	(261.69,289.98) --
	(262.32,290.25) --
	(262.95,290.52) --
	(263.58,290.79) --
	(264.21,291.06) --
	(264.84,291.34) --
	(265.46,291.61) --
	(266.08,291.88) --
	(266.70,292.15) --
	(267.32,292.42) --
	(267.93,292.69) --
	(268.55,292.96) --
	(269.16,293.23) --
	(269.77,293.51) --
	(270.37,293.78) --
	(270.98,294.05) --
	(271.58,294.32) --
	(272.18,294.59) --
	(272.78,294.86) --
	(273.38,295.13) --
	(273.97,295.41) --
	(274.56,295.68) --
	(275.16,295.95) --
	(275.74,296.22) --
	(276.33,296.49) --
	(276.92,296.76) --
	(277.50,297.03) --
	(278.08,297.30) --
	(278.66,297.58) --
	(279.24,297.85) --
	(279.81,298.12) --
	(280.39,298.39) --
	(280.96,298.66) --
	(281.53,298.93) --
	(282.10,299.20) --
	(282.67,299.48) --
	(283.23,299.75) --
	(283.80,300.02) --
	(284.36,300.29) --
	(284.92,300.56) --
	(285.48,300.83) --
	(286.03,301.10) --
	(286.59,301.37) --
	(287.14,301.65) --
	(287.69,301.92) --
	(288.24,302.19) --
	(288.79,302.46) --
	(289.34,302.73) --
	(289.88,303.00) --
	(290.42,303.27) --
	(290.97,303.55) --
	(291.51,303.82) --
	(292.04,304.09) --
	(292.58,304.36) --
	(293.12,304.63) --
	(293.65,304.90) --
	(294.18,305.17) --
	(294.71,305.44) --
	(295.24,305.72) --
	(295.77,305.99) --
	(296.30,306.26) --
	(296.82,306.53) --
	(297.34,306.80) --
	(297.86,307.07) --
	(298.38,307.34) --
	(298.90,307.62) --
	(299.42,307.89) --
	(299.94,308.16) --
	(300.45,308.43) --
	(300.96,308.70) --
	(301.47,308.97) --
	(301.98,309.24) --
	(302.49,309.51) --
	(303.00,309.79) --
	(303.50,310.06) --
	(304.01,310.33) --
	(304.51,310.60) --
	(305.01,310.87) --
	(305.51,311.14) --
	(306.01,311.41) --
	(306.51,311.69) --
	(307.00,311.96) --
	(307.50,312.23) --
	(307.99,312.50) --
	(308.48,312.77) --
	(308.97,313.04) --
	(309.46,313.31) --
	(309.95,313.58) --
	(310.44,313.86) --
	(310.92,314.13) --
	(311.41,314.40) --
	(311.89,314.67) --
	(312.37,314.94) --
	(312.85,315.21) --
	(313.33,315.48) --
	(313.81,315.76) --
	(314.28,316.03) --
	(314.76,316.30) --
	(315.23,316.57) --
	(315.70,316.84) --
	(316.18,317.11) --
	(316.65,317.38) --
	(317.11,317.65) --
	(317.58,317.93) --
	(318.05,318.20) --
	(318.51,318.47) --
	(318.98,318.74) --
	(319.44,319.01) --
	(319.90,319.28) --
	(320.36,319.55) --
	(320.82,319.83) --
	(321.28,320.10) --
	(321.74,320.37) --
	(322.19,320.64) --
	(322.65,320.91) --
	(323.10,321.18) --
	(323.56,321.45) --
	(324.01,321.72) --
	(324.46,322.00) --
	(324.91,322.27) --
	(325.35,322.54) --
	(325.80,322.81) --
	(326.25,323.08) --
	(326.69,323.35) --
	(327.14,323.62) --
	(327.58,323.90) --
	(328.02,324.17) --
	(328.46,324.44) --
	(328.90,324.71) --
	(329.34,324.98) --
	(329.78,325.25) --
	(330.21,325.52) --
	(330.65,325.79) --
	(331.08,326.07) --
	(331.51,326.34) --
	(331.95,326.61) --
	(332.38,326.88) --
	(332.81,327.15) --
	(333.24,327.42) --
	(333.67,327.69) --
	(334.09,327.97) --
	(334.52,328.24) --
	(334.94,328.51) --
	(335.37,328.78) --
	(335.79,329.05) --
	(336.21,329.32) --
	(336.63,329.59) --
	(337.05,329.86) --
	(337.47,330.14) --
	(337.89,330.41) --
	(338.31,330.68) --
	(338.73,330.95) --
	(339.14,331.22) --
	(339.55,331.49) --
	(339.97,331.76) --
	(340.38,332.04) --
	(340.79,332.31) --
	(341.20,332.58) --
	(341.61,332.85) --
	(342.02,333.12) --
	(342.43,333.39) --
	(342.84,333.66) --
	(343.24,333.93) --
	(343.65,334.21) --
	(344.05,334.48) --
	(344.46,334.75) --
	(344.86,335.02) --
	(345.26,335.29) --
	(345.66,335.56) --
	(346.06,335.83) --
	(346.46,336.11) --
	(346.86,336.38) --
	(347.25,336.65) --
	(347.65,336.92) --
	(348.05,337.19) --
	(348.44,337.46) --
	(348.83,337.73) --
	(349.23,338.00) --
	(349.62,338.28) --
	(350.01,338.55) --
	(350.40,338.82) --
	(350.79,339.09) --
	(351.18,339.36) --
	(351.57,339.63) --
	(351.95,339.90) --
	(352.34,340.18) --
	(352.73,340.45) --
	(353.11,340.72) --
	(353.49,340.99) --
	(353.88,341.26) --
	(354.26,341.53) --
	(354.64,341.80) --
	(355.02,342.07) --
	(355.40,342.35) --
	(355.78,342.62) --
	(356.16,342.89) --
	(356.54,343.16) --
	(356.91,343.43) --
	(357.29,343.70) --
	(357.66,343.97) --
	(358.04,344.25) --
	(358.41,344.52) --
	(358.78,344.79) --
	(359.16,345.06) --
	(359.53,345.33) --
	(359.90,345.60) --
	(360.27,345.87) --
	(360.64,346.14) --
	(361.00,346.42) --
	(361.37,346.69) --
	(361.74,346.96) --
	(362.10,347.23) --
	(362.47,347.50) --
	(362.83,347.77) --
	(363.20,348.04) --
	(363.56,348.32) --
	(363.92,348.59) --
	(364.28,348.86) --
	(364.65,349.13) --
	(365.01,349.40) --
	(365.37,349.67) --
	(365.72,349.94) --
	(366.08,350.21) --
	(366.44,350.49) --
	(366.80,350.76) --
	(367.15,351.03) --
	(367.51,351.30) --
	(367.86,351.57) --
	(368.22,351.84) --
	(368.57,352.11) --
	(368.92,352.39) --
	(369.27,352.66) --
	(369.62,352.93) --
	(369.97,353.20) --
	(370.32,353.47) --
	(370.67,353.74) --
	(371.02,354.01) --
	(371.37,354.28) --
	(371.72,354.56) --
	(372.06,354.83) --
	(372.41,355.10) --
	(372.75,355.37) --
	(373.10,355.64) --
	(373.44,355.91) --
	(373.79,356.18) --
	(374.13,356.46) --
	(374.47,356.73) --
	(374.81,357.00) --
	(375.15,357.27) --
	(375.49,357.54) --
	(375.83,357.81) --
	(376.17,358.08) --
	(376.51,358.35) --
	(376.85,358.63) --
	(377.18,358.90) --
	(377.52,359.17) --
	(377.85,359.44) --
	(378.19,359.71) --
	(378.52,359.98) --
	(378.86,360.25) --
	(379.19,360.53) --
	(379.52,360.80) --
	(379.86,361.07) --
	(380.19,361.34) --
	(380.52,361.61) --
	(380.85,361.88) --
	(381.18,362.15) --
	(381.51,362.42) --
	(381.83,362.70) --
	(382.16,362.97) --
	(382.49,363.24) --
	(382.82,363.51) --
	(383.14,363.78) --
	(383.47,364.05) --
	(383.79,364.32) --
	(384.12,364.60) --
	(384.44,364.87) --
	(384.76,365.14) --
	(385.08,365.41) --
	(385.41,365.68) --
	(385.73,365.95) --
	(386.05,366.22) --
	(386.37,366.49) --
	(386.69,366.77) --
	(387.01,367.04) --
	(387.33,367.31) --
	(387.64,367.58) --
	(387.96,367.85) --
	(388.28,368.12) --
	(388.59,368.39) --
	(388.91,368.67) --
	(389.23,368.94) --
	(389.54,369.21) --
	(389.85,369.48) --
	(390.17,369.75) --
	(390.48,370.02) --
	(390.79,370.29) --
	(391.11,370.56) --
	(391.42,370.84) --
	(391.73,371.11) --
	(392.04,371.38) --
	(392.35,371.65) --
	(392.66,371.92) --
	(392.97,372.19) --
	(393.27,372.46) --
	(393.58,372.74) --
	(393.89,373.01) --
	(394.20,373.28) --
	(394.50,373.55) --
	(394.81,373.82) --
	(395.11,374.09) --
	(395.42,374.36) --
	(395.72,374.63) --
	(396.03,374.91) --
	(396.33,375.18) --
	(396.63,375.45) --
	(396.93,375.72) --
	(397.23,375.99) --
	(397.54,376.26) --
	(397.84,376.53) --
	(398.14,376.81) --
	(398.44,377.08) --
	(398.74,377.35) --
	(399.03,377.62) --
	(399.33,377.89) --
	(399.63,378.16) --
	(399.93,378.43) --
	(400.22,378.70) --
	(400.52,378.98) --
	(400.82,379.25) --
	(401.11,379.52) --
	(401.41,379.79) --
	(401.70,380.06) --
	(401.99,380.33) --
	(402.29,380.60) --
	(402.58,380.87) --
	(402.87,381.15) --
	(403.16,381.42) --
	(403.46,381.69) --
	(403.75,381.96) --
	(404.04,382.23) --
	(404.33,382.50) --
	(404.62,382.77) --
	(404.91,383.05) --
	(405.19,383.32) --
	(405.48,383.59) --
	(405.77,383.86) --
	(406.06,384.13) --
	(406.34,384.40) --
	(406.63,384.67) --
	(406.92,384.94) --
	(407.20,385.22) --
	(407.49,385.49) --
	(407.77,385.76) --
	(408.06,386.03) --
	(408.34,386.30) --
	(408.62,386.57) --
	(408.91,386.84) --
	(409.19,387.12) --
	(409.47,387.39) --
	(409.75,387.66) --
	(410.03,387.93) --
	(410.31,388.20) --
	(410.59,388.47) --
	(410.87,388.74) --
	(411.15,389.01) --
	(411.43,389.29) --
	(411.71,389.56) --
	(411.99,389.83) --
	(412.27,390.10) --
	(412.54,390.37) --
	(412.82,390.64) --
	(413.10,390.91) --
	(413.37,391.19) --
	(413.65,391.46) --
	(413.92,391.73) --
	(414.20,392.00) --
	(414.47,392.27) --
	(414.75,392.54) --
	(415.02,392.81) --
	(415.29,393.08) --
	(415.56,393.36) --
	(415.84,393.63) --
	(416.11,393.90) --
	(416.38,394.17) --
	(416.65,394.44) --
	(416.92,394.71) --
	(417.19,394.98) --
	(417.46,395.26) --
	(417.73,395.53) --
	(418.00,395.80) --
	(418.27,396.07) --
	(418.54,396.34) --
	(418.81,396.61) --
	(419.07,396.88) --
	(419.34,397.15) --
	(419.61,397.43) --
	(419.87,397.70) --
	(420.14,397.97) --
	(420.40,398.24) --
	(420.67,398.51) --
	(420.93,398.78) --
	(421.20,399.05) --
	(421.46,399.33) --
	(421.73,399.60) --
	(421.99,399.87) --
	(422.25,400.14) --
	(422.51,400.41) --
	(422.78,400.68) --
	(423.04,400.95) --
	(423.30,401.22) --
	(423.56,401.50) --
	(423.82,401.77) --
	(424.08,402.04) --
	(424.34,402.31) --
	(424.60,402.58) --
	(424.86,402.85) --
	(425.12,403.12) --
	(425.38,403.40) --
	(425.63,403.67) --
	(425.89,403.94) --
	(426.15,404.21) --
	(426.41,404.48) --
	(426.66,404.75) --
	(426.92,405.02) --
	(427.17,405.29) --
	(427.43,405.57) --
	(427.68,405.84) --
	(427.94,406.11) --
	(428.19,406.38) --
	(428.45,406.65) --
	(428.70,406.92) --
	(428.95,407.19) --
	(429.21,407.47) --
	(429.46,407.74) --
	(429.71,408.01) --
	(429.96,408.28) --
	(430.21,408.55) --
	(430.47,408.82) --
	(430.72,409.09) --
	(430.97,409.36) --
	(431.22,409.64) --
	(431.47,409.91) --
	(431.72,410.18) --
	(431.97,410.45) --
	(432.21,410.72) --
	(432.46,410.99) --
	(432.71,411.26) --
	(432.96,411.54) --
	(433.21,411.81) --
	(433.45,412.08) --
	(433.70,412.35) --
	(433.95,412.62) --
	(434.19,412.89) --
	(434.44,413.16) --
	(434.68,413.43) --
	(434.93,413.71) --
	(435.17,413.98) --
	(435.42,414.25) --
	(435.66,414.52) --
	(435.91,414.79) --
	(436.15,415.06) --
	(436.39,415.33) --
	(436.63,415.61) --
	(436.88,415.88) --
	(437.12,416.15) --
	(437.36,416.42) --
	(437.60,416.69) --
	(437.84,416.96) --
	(438.08,417.23) --
	(438.33,417.50) --
	(438.57,417.78) --
	(438.81,418.05) --
	(439.04,418.32) --
	(439.28,418.59) --
	(439.52,418.86) --
	(439.76,419.13) --
	(440.00,419.40) --
	(440.24,419.68) --
	(440.48,419.95) --
	(440.71,420.22) --
	(440.95,420.49) --
	(441.19,420.76) --
	(441.42,421.03) --
	(441.66,421.30) --
	(441.90,421.57) --
	(442.13,421.85) --
	(442.37,422.12) --
	(442.60,422.39) --
	(442.84,422.66) --
	(443.07,422.93) --
	(443.30,423.20) --
	(443.54,423.47) --
	(443.77,423.75) --
	(444.00,424.02) --
	(444.24,424.29) --
	(444.47,424.56) --
	(444.70,424.83) --
	(444.93,425.10) --
	(445.17,425.37) --
	(445.40,425.64) --
	(445.63,425.92) --
	(445.86,426.19) --
	(446.09,426.46) --
	(446.32,426.73) --
	(446.55,427.00) --
	(446.78,427.27) --
	(447.01,427.54) --
	(447.24,427.82) --
	(447.47,428.09) --
	(447.69,428.36) --
	(447.92,428.63) --
	(448.15,428.90) --
	(448.38,429.17) --
	(448.60,429.44) --
	(448.83,429.71) --
	(449.06,429.99) --
	(449.28,430.26) --
	(449.51,430.53) --
	(449.74,430.80) --
	(449.96,431.07) --
	(450.19,431.34) --
	(450.41,431.61) --
	(450.64,431.89) --
	(450.86,432.16) --
	(451.09,432.43) --
	(451.31,432.70) --
	(451.53,432.97) --
	(451.76,433.24) --
	(451.98,433.51) --
	(452.20,433.78) --
	(452.43,434.06) --
	(452.65,434.33) --
	(452.87,434.60) --
	(453.09,434.87) --
	(453.31,435.14) --
	(453.53,435.41) --
	(453.76,435.68) --
	(453.98,435.96) --
	(454.20,436.23) --
	(454.42,436.50) --
	(454.64,436.77) --
	(454.86,437.04) --
	(455.08,437.31) --
	(455.29,437.58) --
	(455.51,437.85) --
	(455.73,438.13) --
	(455.95,438.40) --
	(456.17,438.67) --
	(456.39,438.94) --
	(456.60,439.21) --
	(456.82,439.48) --
	(457.04,439.75) --
	(457.25,440.03) --
	(457.47,440.30) --
	(457.69,440.57) --
	(457.90,440.84) --
	(458.12,441.11) --
	(458.33,441.38) --
	(458.55,441.65) --
	(458.76,441.92) --
	(458.98,442.20) --
	(459.19,442.47) --
	(459.41,442.74) --
	(459.62,443.01) --
	(459.83,443.28) --
	(460.05,443.55) --
	(460.26,443.82) --
	(460.47,444.10) --
	(460.69,444.37) --
	(460.90,444.64) --
	(461.11,444.91) --
	(461.32,445.18) --
	(461.54,445.45) --
	(461.75,445.72) --
	(461.96,445.99) --
	(462.17,446.27) --
	(462.38,446.54) --
	(462.59,446.81) --
	(462.80,447.08) --
	(463.01,447.35) --
	(463.22,447.62) --
	(463.43,447.89) --
	(463.64,448.17) --
	(463.85,448.44) --
	(464.06,448.71) --
	(464.26,448.98) --
	(464.47,449.25) --
	(464.68,449.52) --
	(464.89,449.79) --
	(465.10,450.06) --
	(465.30,450.34) --
	(465.51,450.61) --
	(465.72,450.88) --
	(465.92,451.15) --
	(466.13,451.42) --
	(466.34,451.69) --
	(466.54,451.96) --
	(466.75,452.24) --
	(466.95,452.51) --
	(467.16,452.78) --
	(467.36,453.05) --
	(467.57,453.32) --
	(467.77,453.59) --
	(467.98,453.86) --
	(468.18,454.13) --
	(468.39,454.41) --
	(468.59,454.68) --
	(468.79,454.95) --
	(469.00,455.22) --
	(469.20,455.49) --
	(469.40,455.76) --
	(469.60,456.03) --
	(469.81,456.31) --
	(470.01,456.58) --
	(470.21,456.85) --
	(470.41,457.12) --
	(470.61,457.39) --
	(470.81,457.66) --
	(471.01,457.93) --
	(471.22,458.20) --
	(471.42,458.48) --
	(471.62,458.75) --
	(471.82,459.02) --
	(472.02,459.29) --
	(472.22,459.56) --
	(472.41,459.83) --
	(472.61,460.10) --
	(472.81,460.38) --
	(473.01,460.65) --
	(473.21,460.92) --
	(473.41,461.19) --
	(473.61,461.46) --
	(473.80,461.73) --
	(474.00,462.00) --
	(474.20,462.27) --
	(474.40,462.55) --
	(474.59,462.82) --
	(474.79,463.09) --
	(474.99,463.36) --
	(475.18,463.63) --
	(475.38,463.90) --
	(475.58,464.17) --
	(475.77,464.45) --
	(475.97,464.72) --
	(476.16,464.99) --
	(476.36,465.26) --
	(476.55,465.53) --
	(476.75,465.80) --
	(476.94,466.07) --
	(477.13,466.34) --
	(477.33,466.62) --
	(477.52,466.89) --
	(477.72,467.16) --
	(477.91,467.43) --
	(478.10,467.70) --
	(478.30,467.97) --
	(478.49,468.24) --
	(478.68,468.52) --
	(478.87,468.79) --
	(479.07,469.06) --
	(479.26,469.33) --
	(479.45,469.60) --
	(479.64,469.87) --
	(479.83,470.14) --
	(480.02,470.41) --
	(480.21,470.69) --
	(480.41,470.96) --
	(480.60,471.23) --
	(480.79,471.50) --
	(480.98,471.77) --
	(481.17,472.04) --
	(481.36,472.31) --
	(481.55,472.59) --
	(481.74,472.86) --
	(481.92,473.13) --
	(482.11,473.40) --
	(482.30,473.67) --
	(482.49,473.94) --
	(482.68,474.21) --
	(482.87,474.48) --
	(483.06,474.76) --
	(483.24,475.03) --
	(483.43,475.30) --
	(483.62,475.57) --
	(483.81,475.84) --
	(483.99,476.11) --
	(484.18,476.38) --
	(484.37,476.66) --
	(484.55,476.93) --
	(484.74,477.20) --
	(484.93,477.47) --
	(485.11,477.74) --
	(485.30,478.01) --
	(485.48,478.28) --
	(485.67,478.55) --
	(485.85,478.83) --
	(486.04,479.10) --
	(486.22,479.37) --
	(486.41,479.64) --
	(486.59,479.91) --
	(486.78,480.18) --
	(486.96,480.45) --
	(487.14,480.73) --
	(487.33,481.00) --
	(487.51,481.27) --
	(487.69,481.54) --
	(487.88,481.81) --
	(488.06,482.08) --
	(488.24,482.35) --
	(488.43,482.62) --
	(488.61,482.90) --
	(488.79,483.17) --
	(488.97,483.44);
\end{scope}
\begin{scope}
\path[clip] (  0.00,  0.00) rectangle (505.89,505.89);
\definecolor{drawColor}{RGB}{0,0,0}

\node[text=drawColor,anchor=base,inner sep=0pt, outer sep=0pt, scale=  1.20] at (279.95,  2.40) {Mustergr{"o}sse ($^\circ$)};

\node[text=drawColor,rotate= 90.00,anchor=base,inner sep=0pt, outer sep=0pt, scale=  1.20] at ( 15.60,276.94) {82\,\%-Erkennungsschwelle f{"u}r horizontale Bewegung (ms)};
\end{scope}
\begin{scope}
\path[clip] (  0.00,  0.00) rectangle (505.89,505.89);
\definecolor{drawColor}{RGB}{0,0,0}

\path[draw=drawColor,line width= 0.4pt,line join=round,line cap=round] (177.55, 48.00) -- (429.46, 48.00);

\path[draw=drawColor,line width= 0.4pt,line join=round,line cap=round] (177.55, 48.00) -- (177.55, 42.00);

\path[draw=drawColor,line width= 0.4pt,line join=round,line cap=round] (303.50, 48.00) -- (303.50, 42.00);

\path[draw=drawColor,line width= 0.4pt,line join=round,line cap=round] (377.18, 48.00) -- (377.18, 42.00);

\path[draw=drawColor,line width= 0.4pt,line join=round,line cap=round] (429.46, 48.00) -- (429.46, 42.00);

\node[text=drawColor,anchor=base,inner sep=0pt, outer sep=0pt, scale=  1.20] at (177.55, 30.00) {1.8};

\node[text=drawColor,anchor=base,inner sep=0pt, outer sep=0pt, scale=  1.20] at (303.50, 30.00) {3.6};

\node[text=drawColor,anchor=base,inner sep=0pt, outer sep=0pt, scale=  1.20] at (377.18, 30.00) {5.4};

\node[text=drawColor,anchor=base,inner sep=0pt, outer sep=0pt, scale=  1.20] at (429.46, 30.00) {7.2};
\end{scope}
\begin{scope}
\path[clip] (  0.00,  0.00) rectangle (505.89,505.89);
\definecolor{drawColor}{RGB}{0,0,0}

\path[draw=drawColor,line width= 0.4pt,line join=round,line cap=round] ( 70.74, 48.00) --
	(489.15, 48.00);
\end{scope}
\begin{scope}
\path[clip] (  0.00,  0.00) rectangle (505.89,505.89);
\definecolor{drawColor}{RGB}{0,0,0}

\path[draw=drawColor,line width= 0.4pt,line join=round,line cap=round] ( 54.00, 64.96) -- ( 54.00,488.93);

\path[draw=drawColor,line width= 0.4pt,line join=round,line cap=round] ( 54.00, 64.96) -- ( 48.00, 64.96);

\path[draw=drawColor,line width= 0.4pt,line join=round,line cap=round] ( 54.00,123.74) -- ( 48.00,123.74);

\path[draw=drawColor,line width= 0.4pt,line join=round,line cap=round] ( 54.00,306.34) -- ( 48.00,306.34);

\path[draw=drawColor,line width= 0.4pt,line join=round,line cap=round] ( 54.00,413.15) -- ( 48.00,413.15);

\path[draw=drawColor,line width= 0.4pt,line join=round,line cap=round] ( 54.00,488.93) -- ( 48.00,488.93);

\node[text=drawColor,anchor=base east,inner sep=0pt, outer sep=0pt, scale=  1.20] at ( 45.60, 60.83) {0};

\node[text=drawColor,anchor=base east,inner sep=0pt, outer sep=0pt, scale=  1.20] at ( 45.60,119.61) {50};

\node[text=drawColor,anchor=base east,inner sep=0pt, outer sep=0pt, scale=  1.20] at ( 45.60,302.20) {100};

\node[text=drawColor,anchor=base east,inner sep=0pt, outer sep=0pt, scale=  1.20] at ( 45.60,409.02) {150};

\node[text=drawColor,anchor=base east,inner sep=0pt, outer sep=0pt, scale=  1.20] at ( 45.60,484.80) {200};
\end{scope}
\begin{scope}
\path[clip] (  0.00,  0.00) rectangle (505.89,505.89);
\definecolor{drawColor}{RGB}{255,255,255}
\definecolor{fillColor}{RGB}{255,255,255}

\path[draw=drawColor,line width= 0.4pt,line join=round,line cap=round,fill=fillColor] ( 50.61, 92.55) rectangle ( 57.39, 99.42);
\definecolor{drawColor}{RGB}{0,0,0}

\path[draw=drawColor,line width= 0.4pt,line join=round,line cap=round] ( 50.61, 89.12) -- ( 57.39, 95.99);

\path[draw=drawColor,line width= 0.4pt,line join=round,line cap=round] ( 50.61, 95.99) -- ( 57.39,102.85);
\end{scope}
\begin{scope}
\path[clip] ( 54.00, 48.00) rectangle (505.89,505.89);
\definecolor{drawColor}{RGB}{0,0,0}
\definecolor{fillColor}{RGB}{0,0,0}

\path[draw=drawColor,line width= 0.4pt,line join=round,line cap=round,fill=fillColor] (177.55,261.22) circle (  2.25);

\path[draw=drawColor,line width= 0.4pt,line join=round,line cap=round,fill=fillColor] (303.50,310.06) circle (  2.25);

\path[draw=drawColor,line width= 0.4pt,line join=round,line cap=round,fill=fillColor] (377.18,358.90) circle (  2.25);

\path[draw=drawColor,line width= 0.4pt,line join=round,line cap=round,fill=fillColor] (429.46,407.74) circle (  2.25);

\path[draw=drawColor,line width= 0.4pt,dash pattern=on 4pt off 4pt ,line join=round,line cap=round] (  0.00,142.13) --
	(  0.66,142.19) --
	(  3.31,142.46) --
	(  5.93,142.73) --
	(  8.50,143.01) --
	( 11.04,143.28) --
	( 13.55,143.55) --
	( 16.02,143.82) --
	( 18.46,144.09) --
	( 20.87,144.36) --
	( 23.24,144.63) --
	( 25.59,144.91) --
	( 27.90,145.18) --
	( 30.19,145.45) --
	( 32.45,145.72) --
	( 34.68,145.99) --
	( 36.88,146.26) --
	( 39.06,146.53) --
	( 41.21,146.80) --
	( 43.33,147.08) --
	( 45.43,147.35) --
	( 47.51,147.62) --
	( 49.56,147.89) --
	( 51.59,148.16) --
	( 53.60,148.43) --
	( 55.59,148.70) --
	( 57.55,148.98) --
	( 59.49,149.25) --
	( 61.42,149.52) --
	( 63.32,149.79) --
	( 65.20,150.06) --
	( 67.07,150.33) --
	( 68.91,150.60) --
	( 70.74,150.87) --
	( 72.55,151.15) --
	( 74.34,151.42) --
	( 76.11,151.69) --
	( 77.87,151.96) --
	( 79.60,152.23) --
	( 81.33,152.50) --
	( 83.03,152.77) --
	( 84.72,153.05) --
	( 86.40,153.32) --
	( 88.06,153.59) --
	( 89.70,153.86) --
	( 91.33,154.13) --
	( 92.95,154.40) --
	( 94.55,154.67) --
	( 96.14,154.94) --
	( 97.71,155.22) --
	( 99.27,155.49) --
	(100.81,155.76) --
	(102.35,156.03) --
	(103.87,156.30) --
	(105.38,156.57) --
	(106.87,156.84) --
	(108.36,157.12) --
	(109.83,157.39) --
	(111.29,157.66) --
	(112.73,157.93) --
	(114.17,158.20) --
	(115.60,158.47) --
	(117.01,158.74) --
	(118.41,159.01) --
	(119.81,159.29) --
	(121.19,159.56) --
	(122.56,159.83) --
	(123.92,160.10) --
	(125.27,160.37) --
	(126.61,160.64) --
	(127.94,160.91) --
	(129.27,161.19) --
	(130.58,161.46) --
	(131.88,161.73) --
	(133.17,162.00) --
	(134.46,162.27) --
	(135.73,162.54) --
	(137.00,162.81) --
	(138.26,163.08) --
	(139.51,163.36) --
	(140.75,163.63) --
	(141.98,163.90) --
	(143.20,164.17) --
	(144.42,164.44) --
	(145.62,164.71) --
	(146.82,164.98) --
	(148.02,165.26) --
	(149.20,165.53) --
	(150.38,165.80) --
	(151.54,166.07) --
	(152.71,166.34) --
	(153.86,166.61) --
	(155.01,166.88) --
	(156.15,167.15) --
	(157.28,167.43) --
	(158.40,167.70) --
	(159.52,167.97) --
	(160.63,168.24) --
	(161.74,168.51) --
	(162.83,168.78) --
	(163.93,169.05) --
	(165.01,169.33) --
	(166.09,169.60) --
	(167.16,169.87) --
	(168.23,170.14) --
	(169.29,170.41) --
	(170.34,170.68) --
	(171.39,170.95) --
	(172.43,171.22) --
	(173.46,171.50) --
	(174.49,171.77) --
	(175.52,172.04) --
	(176.54,172.31) --
	(177.55,172.58) --
	(178.55,172.85) --
	(179.56,173.12) --
	(180.55,173.40) --
	(181.54,173.67) --
	(182.53,173.94) --
	(183.51,174.21) --
	(184.48,174.48) --
	(185.45,174.75) --
	(186.41,175.02) --
	(187.37,175.29) --
	(188.33,175.57) --
	(189.28,175.84) --
	(190.22,176.11) --
	(191.16,176.38) --
	(192.09,176.65) --
	(193.02,176.92) --
	(193.95,177.19) --
	(194.87,177.47) --
	(195.78,177.74) --
	(196.69,178.01) --
	(197.60,178.28) --
	(198.50,178.55) --
	(199.40,178.82) --
	(200.29,179.09) --
	(201.18,179.36) --
	(202.06,179.64) --
	(202.94,179.91) --
	(203.82,180.18) --
	(204.69,180.45) --
	(205.56,180.72) --
	(206.42,180.99) --
	(207.28,181.26) --
	(208.14,181.54) --
	(208.99,181.81) --
	(209.84,182.08) --
	(210.68,182.35) --
	(211.52,182.62) --
	(212.35,182.89) --
	(213.18,183.16) --
	(214.01,183.43) --
	(214.84,183.71) --
	(215.66,183.98) --
	(216.47,184.25) --
	(217.29,184.52) --
	(218.10,184.79) --
	(218.90,185.06) --
	(219.70,185.33) --
	(220.50,185.61) --
	(221.30,185.88) --
	(222.09,186.15) --
	(222.88,186.42) --
	(223.66,186.69) --
	(224.45,186.96) --
	(225.22,187.23) --
	(226.00,187.50) --
	(226.77,187.78) --
	(227.54,188.05) --
	(228.30,188.32) --
	(229.07,188.59) --
	(229.82,188.86) --
	(230.58,189.13) --
	(231.33,189.40) --
	(232.08,189.68) --
	(232.83,189.95) --
	(233.57,190.22) --
	(234.31,190.49) --
	(235.05,190.76) --
	(235.78,191.03) --
	(236.51,191.30) --
	(237.24,191.57) --
	(237.97,191.85) --
	(238.69,192.12) --
	(239.41,192.39) --
	(240.13,192.66) --
	(240.84,192.93) --
	(241.55,193.20) --
	(242.26,193.47) --
	(242.97,193.75) --
	(243.67,194.02) --
	(244.37,194.29) --
	(245.07,194.56) --
	(245.76,194.83) --
	(246.45,195.10) --
	(247.14,195.37) --
	(247.83,195.64) --
	(248.51,195.92) --
	(249.20,196.19) --
	(249.88,196.46) --
	(250.55,196.73) --
	(251.23,197.00) --
	(251.90,197.27) --
	(252.57,197.54) --
	(253.24,197.82) --
	(253.90,198.09) --
	(254.56,198.36) --
	(255.22,198.63) --
	(255.88,198.90) --
	(256.53,199.17) --
	(257.19,199.44) --
	(257.84,199.71) --
	(258.48,199.99) --
	(259.13,200.26) --
	(259.77,200.53) --
	(260.41,200.80) --
	(261.05,201.07) --
	(261.69,201.34) --
	(262.32,201.61) --
	(262.95,201.89) --
	(263.58,202.16) --
	(264.21,202.43) --
	(264.84,202.70) --
	(265.46,202.97) --
	(266.08,203.24) --
	(266.70,203.51) --
	(267.32,203.78) --
	(267.93,204.06) --
	(268.55,204.33) --
	(269.16,204.60) --
	(269.77,204.87) --
	(270.37,205.14) --
	(270.98,205.41) --
	(271.58,205.68) --
	(272.18,205.96) --
	(272.78,206.23) --
	(273.38,206.50) --
	(273.97,206.77) --
	(274.56,207.04) --
	(275.16,207.31) --
	(275.74,207.58) --
	(276.33,207.85) --
	(276.92,208.13) --
	(277.50,208.40) --
	(278.08,208.67) --
	(278.66,208.94) --
	(279.24,209.21) --
	(279.81,209.48) --
	(280.39,209.75) --
	(280.96,210.03) --
	(281.53,210.30) --
	(282.10,210.57) --
	(282.67,210.84) --
	(283.23,211.11) --
	(283.80,211.38) --
	(284.36,211.65) --
	(284.92,211.92) --
	(285.48,212.20) --
	(286.03,212.47) --
	(286.59,212.74) --
	(287.14,213.01) --
	(287.69,213.28) --
	(288.24,213.55) --
	(288.79,213.82) --
	(289.34,214.10) --
	(289.88,214.37) --
	(290.42,214.64) --
	(290.97,214.91) --
	(291.51,215.18) --
	(292.04,215.45) --
	(292.58,215.72) --
	(293.12,215.99) --
	(293.65,216.27) --
	(294.18,216.54) --
	(294.71,216.81) --
	(295.24,217.08) --
	(295.77,217.35) --
	(296.30,217.62) --
	(296.82,217.89) --
	(297.34,218.16) --
	(297.86,218.44) --
	(298.38,218.71) --
	(298.90,218.98) --
	(299.42,219.25) --
	(299.94,219.52) --
	(300.45,219.79) --
	(300.96,220.06) --
	(301.47,220.34) --
	(301.98,220.61) --
	(302.49,220.88) --
	(303.00,221.15) --
	(303.50,221.42) --
	(304.01,221.69) --
	(304.51,221.96) --
	(305.01,222.23) --
	(305.51,222.51) --
	(306.01,222.78) --
	(306.51,223.05) --
	(307.00,223.32) --
	(307.50,223.59) --
	(307.99,223.86) --
	(308.48,224.13) --
	(308.97,224.41) --
	(309.46,224.68) --
	(309.95,224.95) --
	(310.44,225.22) --
	(310.92,225.49) --
	(311.41,225.76) --
	(311.89,226.03) --
	(312.37,226.30) --
	(312.85,226.58) --
	(313.33,226.85) --
	(313.81,227.12) --
	(314.28,227.39) --
	(314.76,227.66) --
	(315.23,227.93) --
	(315.70,228.20) --
	(316.18,228.48) --
	(316.65,228.75) --
	(317.11,229.02) --
	(317.58,229.29) --
	(318.05,229.56) --
	(318.51,229.83) --
	(318.98,230.10) --
	(319.44,230.37) --
	(319.90,230.65) --
	(320.36,230.92) --
	(320.82,231.19) --
	(321.28,231.46) --
	(321.74,231.73) --
	(322.19,232.00) --
	(322.65,232.27) --
	(323.10,232.55) --
	(323.56,232.82) --
	(324.01,233.09) --
	(324.46,233.36) --
	(324.91,233.63) --
	(325.35,233.90) --
	(325.80,234.17) --
	(326.25,234.44) --
	(326.69,234.72) --
	(327.14,234.99) --
	(327.58,235.26) --
	(328.02,235.53) --
	(328.46,235.80) --
	(328.90,236.07) --
	(329.34,236.34) --
	(329.78,236.62) --
	(330.21,236.89) --
	(330.65,237.16) --
	(331.08,237.43) --
	(331.51,237.70) --
	(331.95,237.97) --
	(332.38,238.24) --
	(332.81,238.51) --
	(333.24,238.79) --
	(333.67,239.06) --
	(334.09,239.33) --
	(334.52,239.60) --
	(334.94,239.87) --
	(335.37,240.14) --
	(335.79,240.41) --
	(336.21,240.69) --
	(336.63,240.96) --
	(337.05,241.23) --
	(337.47,241.50) --
	(337.89,241.77) --
	(338.31,242.04) --
	(338.73,242.31) --
	(339.14,242.58) --
	(339.55,242.86) --
	(339.97,243.13) --
	(340.38,243.40) --
	(340.79,243.67) --
	(341.20,243.94) --
	(341.61,244.21) --
	(342.02,244.48) --
	(342.43,244.76) --
	(342.84,245.03) --
	(343.24,245.30) --
	(343.65,245.57) --
	(344.05,245.84) --
	(344.46,246.11) --
	(344.86,246.38) --
	(345.26,246.65) --
	(345.66,246.93) --
	(346.06,247.20) --
	(346.46,247.47) --
	(346.86,247.74) --
	(347.25,248.01) --
	(347.65,248.28) --
	(348.05,248.55) --
	(348.44,248.83) --
	(348.83,249.10) --
	(349.23,249.37) --
	(349.62,249.64) --
	(350.01,249.91) --
	(350.40,250.18) --
	(350.79,250.45) --
	(351.18,250.72) --
	(351.57,251.00) --
	(351.95,251.27) --
	(352.34,251.54) --
	(352.73,251.81) --
	(353.11,252.08) --
	(353.49,252.35) --
	(353.88,252.62) --
	(354.26,252.90) --
	(354.64,253.17) --
	(355.02,253.44) --
	(355.40,253.71) --
	(355.78,253.98) --
	(356.16,254.25) --
	(356.54,254.52) --
	(356.91,254.79) --
	(357.29,255.07) --
	(357.66,255.34) --
	(358.04,255.61) --
	(358.41,255.88) --
	(358.78,256.15) --
	(359.16,256.42) --
	(359.53,256.69) --
	(359.90,256.97) --
	(360.27,257.24) --
	(360.64,257.51) --
	(361.00,257.78) --
	(361.37,258.05) --
	(361.74,258.32) --
	(362.10,258.59) --
	(362.47,258.86) --
	(362.83,259.14) --
	(363.20,259.41) --
	(363.56,259.68) --
	(363.92,259.95) --
	(364.28,260.22) --
	(364.65,260.49) --
	(365.01,260.76) --
	(365.37,261.04) --
	(365.72,261.31) --
	(366.08,261.58) --
	(366.44,261.85) --
	(366.80,262.12) --
	(367.15,262.39) --
	(367.51,262.66) --
	(367.86,262.93) --
	(368.22,263.21) --
	(368.57,263.48) --
	(368.92,263.75) --
	(369.27,264.02) --
	(369.62,264.29) --
	(369.97,264.56) --
	(370.32,264.83) --
	(370.67,265.11) --
	(371.02,265.38) --
	(371.37,265.65) --
	(371.72,265.92) --
	(372.06,266.19) --
	(372.41,266.46) --
	(372.75,266.73) --
	(373.10,267.00) --
	(373.44,267.28) --
	(373.79,267.55) --
	(374.13,267.82) --
	(374.47,268.09) --
	(374.81,268.36) --
	(375.15,268.63) --
	(375.49,268.90) --
	(375.83,269.18) --
	(376.17,269.45) --
	(376.51,269.72) --
	(376.85,269.99) --
	(377.18,270.26) --
	(377.52,270.53) --
	(377.85,270.80) --
	(378.19,271.07) --
	(378.52,271.35) --
	(378.86,271.62) --
	(379.19,271.89) --
	(379.52,272.16) --
	(379.86,272.43) --
	(380.19,272.70) --
	(380.52,272.97) --
	(380.85,273.25) --
	(381.18,273.52) --
	(381.51,273.79) --
	(381.83,274.06) --
	(382.16,274.33) --
	(382.49,274.60) --
	(382.82,274.87) --
	(383.14,275.14) --
	(383.47,275.42) --
	(383.79,275.69) --
	(384.12,275.96) --
	(384.44,276.23) --
	(384.76,276.50) --
	(385.08,276.77) --
	(385.41,277.04) --
	(385.73,277.32) --
	(386.05,277.59) --
	(386.37,277.86) --
	(386.69,278.13) --
	(387.01,278.40) --
	(387.33,278.67) --
	(387.64,278.94) --
	(387.96,279.21) --
	(388.28,279.49) --
	(388.59,279.76) --
	(388.91,280.03) --
	(389.23,280.30) --
	(389.54,280.57) --
	(389.85,280.84) --
	(390.17,281.11) --
	(390.48,281.39) --
	(390.79,281.66) --
	(391.11,281.93) --
	(391.42,282.20) --
	(391.73,282.47) --
	(392.04,282.74) --
	(392.35,283.01) --
	(392.66,283.28) --
	(392.97,283.56) --
	(393.27,283.83) --
	(393.58,284.10) --
	(393.89,284.37) --
	(394.20,284.64) --
	(394.50,284.91) --
	(394.81,285.18) --
	(395.11,285.46) --
	(395.42,285.73) --
	(395.72,286.00) --
	(396.03,286.27) --
	(396.33,286.54) --
	(396.63,286.81) --
	(396.93,287.08) --
	(397.23,287.35) --
	(397.54,287.63) --
	(397.84,287.90) --
	(398.14,288.17) --
	(398.44,288.44) --
	(398.74,288.71) --
	(399.03,288.98) --
	(399.33,289.25) --
	(399.63,289.53) --
	(399.93,289.80) --
	(400.22,290.07) --
	(400.52,290.34) --
	(400.82,290.61) --
	(401.11,290.88) --
	(401.41,291.15) --
	(401.70,291.42) --
	(401.99,291.70) --
	(402.29,291.97) --
	(402.58,292.24) --
	(402.87,292.51) --
	(403.16,292.78) --
	(403.46,293.05) --
	(403.75,293.32) --
	(404.04,293.60) --
	(404.33,293.87) --
	(404.62,294.14) --
	(404.91,294.41) --
	(405.19,294.68) --
	(405.48,294.95) --
	(405.77,295.22) --
	(406.06,295.49) --
	(406.34,295.77) --
	(406.63,296.04) --
	(406.92,296.31) --
	(407.20,296.58) --
	(407.49,296.85) --
	(407.77,297.12) --
	(408.06,297.39) --
	(408.34,297.67) --
	(408.62,297.94) --
	(408.91,298.21) --
	(409.19,298.48) --
	(409.47,298.75) --
	(409.75,299.02) --
	(410.03,299.29) --
	(410.31,299.56) --
	(410.59,299.84) --
	(410.87,300.11) --
	(411.15,300.38) --
	(411.43,300.65) --
	(411.71,300.92) --
	(411.99,301.19) --
	(412.27,301.46) --
	(412.54,301.74) --
	(412.82,302.01) --
	(413.10,302.28) --
	(413.37,302.55) --
	(413.65,302.82) --
	(413.92,303.09) --
	(414.20,303.36) --
	(414.47,303.63) --
	(414.75,303.91) --
	(415.02,304.18) --
	(415.29,304.45) --
	(415.56,304.72) --
	(415.84,304.99) --
	(416.11,305.26) --
	(416.38,305.53) --
	(416.65,305.81) --
	(416.92,306.08) --
	(417.19,306.35) --
	(417.46,306.62) --
	(417.73,306.89) --
	(418.00,307.16) --
	(418.27,307.43) --
	(418.54,307.70) --
	(418.81,307.98) --
	(419.07,308.25) --
	(419.34,308.52) --
	(419.61,308.79) --
	(419.87,309.06) --
	(420.14,309.33) --
	(420.40,309.60) --
	(420.67,309.88) --
	(420.93,310.15) --
	(421.20,310.42) --
	(421.46,310.69) --
	(421.73,310.96) --
	(421.99,311.23) --
	(422.25,311.50) --
	(422.51,311.77) --
	(422.78,312.05) --
	(423.04,312.32) --
	(423.30,312.59) --
	(423.56,312.86) --
	(423.82,313.13) --
	(424.08,313.40) --
	(424.34,313.67) --
	(424.60,313.95) --
	(424.86,314.22) --
	(425.12,314.49) --
	(425.38,314.76) --
	(425.63,315.03) --
	(425.89,315.30) --
	(426.15,315.57) --
	(426.41,315.84) --
	(426.66,316.12) --
	(426.92,316.39) --
	(427.17,316.66) --
	(427.43,316.93) --
	(427.68,317.20) --
	(427.94,317.47) --
	(428.19,317.74) --
	(428.45,318.02) --
	(428.70,318.29) --
	(428.95,318.56) --
	(429.21,318.83) --
	(429.46,319.10) --
	(429.71,319.37) --
	(429.96,319.64) --
	(430.21,319.91) --
	(430.47,320.19) --
	(430.72,320.46) --
	(430.97,320.73) --
	(431.22,321.00) --
	(431.47,321.27) --
	(431.72,321.54) --
	(431.97,321.81) --
	(432.21,322.08) --
	(432.46,322.36) --
	(432.71,322.63) --
	(432.96,322.90) --
	(433.21,323.17) --
	(433.45,323.44) --
	(433.70,323.71) --
	(433.95,323.98) --
	(434.19,324.26) --
	(434.44,324.53) --
	(434.68,324.80) --
	(434.93,325.07) --
	(435.17,325.34) --
	(435.42,325.61) --
	(435.66,325.88) --
	(435.91,326.15) --
	(436.15,326.43) --
	(436.39,326.70) --
	(436.63,326.97) --
	(436.88,327.24) --
	(437.12,327.51) --
	(437.36,327.78) --
	(437.60,328.05) --
	(437.84,328.33) --
	(438.08,328.60) --
	(438.33,328.87) --
	(438.57,329.14) --
	(438.81,329.41) --
	(439.04,329.68) --
	(439.28,329.95) --
	(439.52,330.22) --
	(439.76,330.50) --
	(440.00,330.77) --
	(440.24,331.04) --
	(440.48,331.31) --
	(440.71,331.58) --
	(440.95,331.85) --
	(441.19,332.12) --
	(441.42,332.40) --
	(441.66,332.67) --
	(441.90,332.94) --
	(442.13,333.21) --
	(442.37,333.48) --
	(442.60,333.75) --
	(442.84,334.02) --
	(443.07,334.29) --
	(443.30,334.57) --
	(443.54,334.84) --
	(443.77,335.11) --
	(444.00,335.38) --
	(444.24,335.65) --
	(444.47,335.92) --
	(444.70,336.19) --
	(444.93,336.47) --
	(445.17,336.74) --
	(445.40,337.01) --
	(445.63,337.28) --
	(445.86,337.55) --
	(446.09,337.82) --
	(446.32,338.09) --
	(446.55,338.36) --
	(446.78,338.64) --
	(447.01,338.91) --
	(447.24,339.18) --
	(447.47,339.45) --
	(447.69,339.72) --
	(447.92,339.99) --
	(448.15,340.26) --
	(448.38,340.54) --
	(448.60,340.81) --
	(448.83,341.08) --
	(449.06,341.35) --
	(449.28,341.62) --
	(449.51,341.89) --
	(449.74,342.16) --
	(449.96,342.43) --
	(450.19,342.71) --
	(450.41,342.98) --
	(450.64,343.25) --
	(450.86,343.52) --
	(451.09,343.79) --
	(451.31,344.06) --
	(451.53,344.33) --
	(451.76,344.61) --
	(451.98,344.88) --
	(452.20,345.15) --
	(452.43,345.42) --
	(452.65,345.69) --
	(452.87,345.96) --
	(453.09,346.23) --
	(453.31,346.50) --
	(453.53,346.78) --
	(453.76,347.05) --
	(453.98,347.32) --
	(454.20,347.59) --
	(454.42,347.86) --
	(454.64,348.13) --
	(454.86,348.40) --
	(455.08,348.68) --
	(455.29,348.95) --
	(455.51,349.22) --
	(455.73,349.49) --
	(455.95,349.76) --
	(456.17,350.03) --
	(456.39,350.30) --
	(456.60,350.57) --
	(456.82,350.85) --
	(457.04,351.12) --
	(457.25,351.39) --
	(457.47,351.66) --
	(457.69,351.93) --
	(457.90,352.20) --
	(458.12,352.47) --
	(458.33,352.75) --
	(458.55,353.02) --
	(458.76,353.29) --
	(458.98,353.56) --
	(459.19,353.83) --
	(459.41,354.10) --
	(459.62,354.37) --
	(459.83,354.64) --
	(460.05,354.92) --
	(460.26,355.19) --
	(460.47,355.46) --
	(460.69,355.73) --
	(460.90,356.00) --
	(461.11,356.27) --
	(461.32,356.54) --
	(461.54,356.82) --
	(461.75,357.09) --
	(461.96,357.36) --
	(462.17,357.63) --
	(462.38,357.90) --
	(462.59,358.17) --
	(462.80,358.44) --
	(463.01,358.71) --
	(463.22,358.99) --
	(463.43,359.26) --
	(463.64,359.53) --
	(463.85,359.80) --
	(464.06,360.07) --
	(464.26,360.34) --
	(464.47,360.61) --
	(464.68,360.89) --
	(464.89,361.16) --
	(465.10,361.43) --
	(465.30,361.70) --
	(465.51,361.97) --
	(465.72,362.24) --
	(465.92,362.51) --
	(466.13,362.78) --
	(466.34,363.06) --
	(466.54,363.33) --
	(466.75,363.60) --
	(466.95,363.87) --
	(467.16,364.14) --
	(467.36,364.41) --
	(467.57,364.68) --
	(467.77,364.96) --
	(467.98,365.23) --
	(468.18,365.50) --
	(468.39,365.77) --
	(468.59,366.04) --
	(468.79,366.31) --
	(469.00,366.58) --
	(469.20,366.85) --
	(469.40,367.13) --
	(469.60,367.40) --
	(469.81,367.67) --
	(470.01,367.94) --
	(470.21,368.21) --
	(470.41,368.48) --
	(470.61,368.75) --
	(470.81,369.03) --
	(471.01,369.30) --
	(471.22,369.57) --
	(471.42,369.84) --
	(471.62,370.11) --
	(471.82,370.38) --
	(472.02,370.65) --
	(472.22,370.92) --
	(472.41,371.20) --
	(472.61,371.47) --
	(472.81,371.74) --
	(473.01,372.01) --
	(473.21,372.28) --
	(473.41,372.55) --
	(473.61,372.82) --
	(473.80,373.10) --
	(474.00,373.37) --
	(474.20,373.64) --
	(474.40,373.91) --
	(474.59,374.18) --
	(474.79,374.45) --
	(474.99,374.72) --
	(475.18,374.99) --
	(475.38,375.27) --
	(475.58,375.54) --
	(475.77,375.81) --
	(475.97,376.08) --
	(476.16,376.35) --
	(476.36,376.62) --
	(476.55,376.89) --
	(476.75,377.17) --
	(476.94,377.44) --
	(477.13,377.71) --
	(477.33,377.98) --
	(477.52,378.25) --
	(477.72,378.52) --
	(477.91,378.79) --
	(478.10,379.06) --
	(478.30,379.34) --
	(478.49,379.61) --
	(478.68,379.88) --
	(478.87,380.15) --
	(479.07,380.42) --
	(479.26,380.69) --
	(479.45,380.96) --
	(479.64,381.24) --
	(479.83,381.51) --
	(480.02,381.78) --
	(480.21,382.05) --
	(480.41,382.32) --
	(480.60,382.59) --
	(480.79,382.86) --
	(480.98,383.13) --
	(481.17,383.41) --
	(481.36,383.68) --
	(481.55,383.95) --
	(481.74,384.22) --
	(481.92,384.49) --
	(482.11,384.76) --
	(482.30,385.03) --
	(482.49,385.31) --
	(482.68,385.58) --
	(482.87,385.85) --
	(483.06,386.12) --
	(483.24,386.39) --
	(483.43,386.66) --
	(483.62,386.93) --
	(483.81,387.20) --
	(483.99,387.48) --
	(484.18,387.75) --
	(484.37,388.02) --
	(484.55,388.29) --
	(484.74,388.56) --
	(484.93,388.83) --
	(485.11,389.10) --
	(485.30,389.38) --
	(485.48,389.65) --
	(485.67,389.92) --
	(485.85,390.19) --
	(486.04,390.46) --
	(486.22,390.73) --
	(486.41,391.00) --
	(486.59,391.27) --
	(486.78,391.55) --
	(486.96,391.82) --
	(487.14,392.09) --
	(487.33,392.36) --
	(487.51,392.63) --
	(487.69,392.90) --
	(487.88,393.17) --
	(488.06,393.45) --
	(488.24,393.72) --
	(488.43,393.99) --
	(488.61,394.26) --
	(488.79,394.53) --
	(488.97,394.80);

\path[fill=fillColor] (174.17,172.58) --
	(177.55,175.96) --
	(180.92,172.58) --
	(177.55,169.21) --
	cycle;

\path[fill=fillColor] (300.13,221.42) --
	(303.50,224.80) --
	(306.88,221.42) --
	(303.50,218.05) --
	cycle;

\path[fill=fillColor] (373.81,270.26) --
	(377.18,273.64) --
	(380.56,270.26) --
	(377.18,266.89) --
	cycle;

\path[fill=fillColor] (426.08,319.10) --
	(429.46,322.48) --
	(432.83,319.10) --
	(429.46,315.73) --
	cycle;
\end{scope}
\end{tikzpicture}

		\end{adjustbox}
		\caption[]{Auswirkung des Skalierungsparameters $a$ auf den exponentiellen Zusammenhang zwischen der Mustergrösse und der $82\,\%$-Er\-ken\-nungs\-schwel\-le für horizontale Bewegung in der Spatial-Suppression-Aufgabe. Die durchgezogene Linie (---) beschreibt die Funktion $y=70 \times e^{.103x}$. Die gestrichelte Linie (- - -) beschreibt die Funktion $y=50 \times e^{.103x}$. Die beiden abgebildeten Funktionen unterscheiden sich folglich lediglich durch den Skalierungsparameter $a$. Mit dieser Abbildung wird deutlich, dass der Parameter $a$ den gesamten Kurvenverlauf skaliert. Die x- und die y-Achse sind beide logarithmiert.}
		\label{fig:scale}
	\end{figure}

\end{document}