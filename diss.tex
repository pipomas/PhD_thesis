% !TeX spellcheck = de_DE
% ===================================================================================
% P R E A M B E L 
% ===================================================================================
% Basic document settings -----------------------------------------------------------
\documentclass[11pt, twoside, a4paper]{book}		% option draft for no images and overfull hbox markers
\linespread{1.3}\selectfont							% use one and a half line spaces
%\renewcommand{\familydefault}{\sfdefault}			% sets file in sans serif
%\setlength{\marginparwidth}{0pt}					% removes margin notes area O_o
%\setlength{\parskip}{1.5ex plus 0.5ex minus 0.2ex}	% https://en.wikibooks.org/wiki/LaTeX/Page_Layout#Widows_and_orphans
\AtBeginDocument{\setlength{\parindent}{2em}}		% parindent 2 em units for new paragraph
%\usepackage{indentfirst}							% use if first paragraph should be indented
\usepackage[pass]{geometry}							% ’pass’ disregards the package layout, so the original ’book’ layout is memorized here
\usepackage{layout}									% show layout on page 1
%\usepackage{showframe}								% show frame on pages
%\usepackage{url}									% loaded internally by hyperref
%\usepackage{booktabs}								% Nicely formatted tables
%\usepackage{topcapt}								% Be able to put captions above tables 
%\usepackage{chngpage}								% Be able to change margins on the fly for big tables
%\usepackage{longtable}								% For tables that go over sereral pages
\usepackage[utf8]{inputenc}							% this is needed for umlauts
\usepackage[T1]{fontenc}							% this is needed for correct output of umlauts in pdf
\usepackage[ngerman]{babel}							% set document language to new german
\usepackage[german=swiss]{csquotes}					% enquote{} makes quoting in swiss german easy
\usepackage{pdflscape}								% https://en.wikibooks.org/wiki/LaTeX/Page_Layout#cite_ref-4
\usepackage{longtable}								% a multi-page environment for tabular
\usepackage{listings}								% for R code snippets
\usepackage{graphicx}								% needed for graphics
\renewcommand{\textfraction}	{.001}				% minimum fraction text per side
\renewcommand{\topfraction}		{.999}				% fraction of float on top page
\renewcommand{\bottomfraction}	{.999}				% fraction of float on bottum page
\usepackage[hang]{footmisc}							% footnotes are indented
\usepackage{cancel}									% for \cancel{expression} 
\usepackage{pifont}									% for "itemize" symbols
\usepackage{color}									% \textcolor{color}{text} for coloring text

% Mathpackages ---------------------------------------------------------------------
\usepackage{mathtools}								% handy tools for mathematical typesetting
\usepackage{amsmath}								% miscellaneous enhancements for mathematical formulas
\usepackage{amsfonts}								% - for certain mathematical fonts
\usepackage{amssymb}								% - for certain mathematical symbols
\usepackage{fixltx2e} 								% needed when $ and $ are used in section titles
\usepackage{array}    								% für >{}, d.h. fügt den Inhalt vor jeder Zelle ein. oder >{$}c{$}< 
\usepackage[Euler]{upgreek}							% for non-italic greek letters

% Setup for fancyhdr package -------------------------------------------------------
\usepackage{fancyhdr}								% customize the page layout of  LaTeX documents
\pagestyle{fancy}									% select pagestyle
%\renewcommand{\sectionmark}[1]{\markright{\thesection}}	% renew the sectionmark command
%\renewcommand{\chaptermark}[1]{\markboth{#1}{#1}}			% <- works as single line
\renewcommand{\chaptermark}[1]{\markboth{\small\textsc{#1}}{}}
\renewcommand{\sectionmark}[1]{\markright{\small\textsc{#1}}{}}
\fancyhf{}  										% delete current header and footer
\fancyhead[LE,RO]{\thepage}							% ...
\fancyhead[LO]{\normalfont\nouppercase{\rightmark}}	% ... tiefer als leftmark
\fancyhead[RE]{\normalfont\nouppercase{\leftmark}} 	% ... höher als rightmark
\renewcommand{\headrulewidth}{.1pt}					% set head rule width
\renewcommand{\footrulewidth}{0pt}					% set foot rule width
\addtolength{\headheight}{0.5pt} 					% space between header and page

% pagestyle for plain pages (chapters, titles etc.)
\fancypagestyle{plain}{%							
   \fancyfoot[C]{\thepage}							% show page number centered
   \fancyhead{}										% get rid of headers on plain pages
   \renewcommand{\headrulewidth}{0pt}				% and the line
}
\usepackage{emptypage}								% removes headers from plain pages


% Check this style, found on http://tex.stackexchange.com/questions/132469/showcase-of-nice-looking-headers
%\pagestyle{fancy}
%\fancyhead{}
%
%\renewcommand{\chaptermark}[1]{\markboth{\small\textsc{#1}}{}}
%\renewcommand{\sectionmark}[1]{\markright{\small\textsc{#1}}{}}
%\fancyhead[LE]{\rightmark}
%\fancyhead[RO]{\leftmark}

% Backup code in case setup doesn't work anymore ------------------------------
%\usepackage{emptypage}  % removes headers of empty pages

%\usepackage{fancyhdr}
%\pagestyle{fancy}
%\renewcommand{\chaptermark}[1]{\markboth{#1}{}}
%\renewcommand{\sectionmark}[1]{%
%        \markright{\thesection\ #1}}
%\fancyhf{}  % delete current header and footer
%\fancyhead[LE,RO]{\thepage}
%\fancyhead[LO]{\normalfont\nouppercase{\rightmark}}
%\fancyhead[RE]{\normalfont\nouppercase{\leftmark}}
%\renewcommand{\headrulewidth}{0.5pt}
%\renewcommand{\footrulewidth}{0pt}
%\addtolength{\headheight}{0.5pt} % space for the rule
%\fancypagestyle{plain}{%
%   \fancyfoot[C]{\thepage}
%   \fancyhead{} % get rid of headers on plain pages
%   \renewcommand{\headrulewidth}{0pt} % and the line
%}

% setup for tables--------------------------------------------------------------
\usepackage{rotating}								% performs most sorts of rotation floating figures and tables
\usepackage{threeparttable}							% this package facilitates tables with titles (captions) and notes.
\usepackage{adjustbox}								% this package allows to adjust general (LA)TEX material
\usepackage{multirow}								% for \multirow command in tabular environment

% Make landscape mode rotate properly in a twosided book -----------------------
% see http://stackoverflow.com/questions/4982219/how-to-make-landscape-mode-rotate-properly-in-a-twoside-book/5320962
\makeatletter	
\global\let\orig@begin@landscape=\landscape%
\global\let\orig@end@landscape=\endlandscape%
\gdef\@true{1}
\gdef\@false{0}
\gdef\landscape{%
    \global\let\within@landscape=\@true%
    \orig@begin@landscape%
}%
\gdef\endlandscape{%
    \orig@end@landscape%
    \global\let\within@landscape=\@false%
}%
\@ifpackageloaded{pdflscape}{%
    \gdef\pdf@landscape@rotate{\PLS@Rotate}%
}{
    \gdef\pdf@landscape@rotate#1{}%
}
\let\latex@outputpage\@outputpage
\def\@outputpage{
    \ifx\within@landscape\@true%
        \if@twoside%
            \ifodd\c@page%
                \gdef\LS@rot{\setbox\@outputbox\vbox{%
                    \pdf@landscape@rotate{-90}%
                    \hbox{\rotatebox{90}{\hbox{\rotatebox{180}{\box\@outputbox}}}}}%
                }%
            \else%
                \gdef\LS@rot{\setbox\@outputbox\vbox{%
                    \pdf@landscape@rotate{+90}%
                    \hbox{\rotatebox{90}{\hbox{\rotatebox{0}{\box\@outputbox}}}}}%
                }%
            \fi%
        \else%
            \gdef\LS@rot{\setbox\@outputbox\vbox{%
                \pdf@landscape@rotate{+90}%
                \hbox{\rotatebox{90}{\hbox{\rotatebox{0}{\box\@outputbox}}}}}%
            }%
        \fi%
    \fi%
    \latex@outputpage%
}
\makeatother

% Setup for figures ------------------------------------------------------------
\usepackage{tikz}					% load package
\usetikzlibrary{arrows,decorations.pathmorphing,backgrounds,positioning,fit,petri} % load options
\tikzstyle{every picture}+=[font=\sffamily]			% use sans serif font for tikz pictures -> http://tex.stackexchange.com/questions/4887/pgf-tikz-and-sans-serif-fonts
\usepackage[lofdepth=1]{subfig}															% for subfigures. Loads caption package internally. 
\usepackage[singlelinecheck = off,labelsep = period]{caption} 							% -> caption is aligned with table
\captionsetup[subfigure]{labelformat=simple, labelsep = period, listofformat=subsimple} % (a) -> a.
\captionsetup[figure]{labelfont=it, labelsep = period}									% italic labelfont in figures
\captionsetup[table]{labelsep = none}													% no . after table label
\usepackage[leftcaption]{sidecap}	% for side caption: outercaption, innercaption, leftcaption, rightcaption
\usepackage[section]{placeins}		% http://tex.stackexchange.com/questions/35125/how-to-use-the-placement-options-t-h-with-figures
\usepackage{float}					% for better floating of floats

% Setup for figures and tables -------------------------------------------------
\usepackage{chngcntr}							% package for avoiding the counter reset per chapter when using figs and tbls
\counterwithout{figure}{chapter}				% -> does it for figures
\counterwithout{table}{chapter}					% -> does it for tables
\counterwithout{footnote}{chapter}				% -> does it for tables
%\numberwithin{equation}{section}				% equations start with 1.1 1.2, 2.1 2.2 etc.
%\numberwithin{figure}{section} 				% the same with figures
\usepackage{siunitx}							% for decimal aligned columns in tables
\setlength{\abovecaptionskip}{3pt plus 3pt minus 2pt}	% modifiy vertical space between figure and caption
\setcounter{lofdepth}{2}			% include minicaptions in LOF (list of figures)

% Removes "Chapter X" string in chapters ---------------------------------------
\makeatletter
\def\@makechapterhead#1{%
  \vspace*{50\p@}%
  {\parindent \z@ \raggedright \normalfont
    \interlinepenalty\@M
    \Huge\bfseries  \thechapter\quad #1\par\nobreak
    \vskip 40\p@
  }}
\makeatother

% Removes spacing in list of figures and tables -------------------------------
% see http://tex.stackexchange.com/questions/69321/spacing-between-chapters-in-list-of-tables
\makeatletter
\def\@chapter[#1]#2{\ifnum \c@secnumdepth >\m@ne
	\if@mainmatter
	\refstepcounter{chapter}%
	\typeout{\@chapapp\space\thechapter.}%
	\addcontentsline{toc}{chapter}%
	{\protect\numberline{\thechapter}#1}%
	\else
	\addcontentsline{toc}{chapter}{#1}%
	\fi
	\else
	\addcontentsline{toc}{chapter}{#1}%
	\fi
	\chaptermark{#1}%
	%                    \addtocontents{lof}{\protect\addvspace{10\p@}}% NEW
	%                    \addtocontents{lot}{\protect\addvspace{10\p@}}% NEW
	\if@twocolumn
	\@topnewpage[\@makechapterhead{#2}]%
	\else
	\@makechapterhead{#2}%
	\@afterheading
	\fi}
\makeatother

% Setup for bibliography with the 'apacite' package ----------------------------
\usepackage[tocbib, natbibapa, nosectionbib]{apacite}	% enable options
%\newcommand\citepos[1]{\citeauthor{#1}s\ (\citeyear{#1})} % see http://tex.stackexchange.com/questions/137511/how-to-put-a-s-after-citing-author
\bibliographystyle{apacite-mod}							% see apacite-mod in directory http://tex.stackexchange.com/questions/304217/reference-list-suppressing-dots-after-company-names-apacite
\usepackage{doi}										% turns doi's into hyperlinks
\renewcommand\doiprefix{\ignorespaces}					% removes white space in front of doi
\AtBeginDocument{\urlstyle{APACsame}}					% removes monospaced font in url's
\AtBeginDocument{\renewcommand{\BRetrievedFrom}{Verfügbar unter\ }} % replaces  "Zugriff auf" with "Verfügbar unter

% Allow URL breaks after all letters
\AtBeginDocument{\renewcommand{\UrlBreaks}{\do\/\do\a\do\b\do\c\do\d\do\e\do\f\do\g\do\h\do\i\do\j\do\k\do\l\do\m\do\n\do\o\do\p\do\q\do\r\do\s\do\t\do\u\do\v\do\w\do\x\do\y\do\z\do\A\do\B\do\C\do\D\do\E\do\F\do\G\do\H\do\I\do\J\do\K\do\L\do\M\do\N\do\O\do\P\do\Q\do\R\do\S\do\T\do\U\do\V\do\W\do\X\do\Y\do\Z}}


% Correction in bilbiography: Removal of whitespace between volume and issue (APA vs DGPs) ------------
%\makeatletter                                    % removes white space between vol and issue		---
%\AtBeginDocument{%                               % uncomment for apa format  						---
%  \renewcommand{\APACjournalVolNumPages}[4]{%    % with ngerman, apacite switches to DGP rules...	---
%    \Bem{#1}%             journal																	---
%    \ifx\@empty#2\@empty																			---
%    \else																							---
%      \unskip, \Bem{#2}%  volume																	---
%    \fi																							---
%    \ifx\@empty#3\@empty																			---
%    \else																							---
%      \unskip({#3})%      issue number																---
%    \fi																							---
%    \ifx\@empty#4\@empty																			---
%    \else																							---
%      \unskip, {#4}%      pages																	---
%    \fi																							---
%  }																								---
%}																									---
%\makeatother																						---

% Setup for hyperref -----------------------------------------------------------
\usepackage{hyperref}		% load package
\hypersetup{colorlinks,		% define options
			citecolor=blue,
			filecolor=black,
			linkcolor=black,
			urlcolor=blue,
			pdfauthor={Philipp Thomas},
%			pdfsubject={Dissertation},
			pdftitle={Dissertation},
			pdfkeywords={psychometric intelligence, spatial suppression, mental speed, hick task, R, LaTeX}
			}


%\usepackage{cleveref}		% see http://tex.stackexchange.com/questions/62611/how-to-make-ref-references-include-the-word-figure	

% Setup for glossaries (must be placed AFTER the hyperref setup) ---------------------------
\usepackage[xindy, toc, automake, acronym, shortcuts]{glossaries}		% load options (what is 'xindy' for?)
\makeglossary							% to generate the glossary

% Glossary entries --------------------------------------------------------
% Glossaries -------------------------------
\newglossaryentry{ssans}
{	name=Spat\-ial-Sup\-pres\-sion-An\-satz,
	description={is}}

\newglossaryentry{ssauf}
{	name=Spat\-ial-Sup\-pres\-sion-Auf\-gabe,
	description=}

\newglossaryentry{ss}
{	name=Spat\-ial-Sup\-pres\-sion,
	description=}

\newglossaryentry{si}
{	name=Sup\-pres\-sion-Index,
	description=}

\newglossaryentry{ms}
{	name=Ment\-al-Speed,
	description=}

\newglossaryentry{msa}
{name=Mental-Spe\-ed-An\-satz,
	description=is}

\newglossaryentry{msm}
{name=Mental-Speed-Mass,
	plural=Mental-Speed-Massen,
	description=is}

\newglossaryentry{flm}
{	name=Fixed-Links-Modell,
	description=}

\newglossaryentry{ha}
{	name=Hick-Auf\-gabe,
	description=}

\newglossaryentry{ita}
{	name=Inspection-Time-Auf\-gabe,
	description=}

\newglossaryentry{gfaktor}
{	name= \textit{g}-Fak\-tor,
	description=}

\newglossaryentry{zwert}
{	name= \textit{z}-Wert,
	description=}

%\newglossaryentry{tst}
%{	name=Three-Stratum-Theorie,
%	description=}
%
%\newglossaryentry{}
%{	name=,
%	description=}

% Acronyms -------------------------------

\newacronym{gf}{Gf}{fluide Intelligenz}
\newacronym{gc}{Gc}{kristalline Intelligenz}

\newacronym{chct}{CHC-Theorie}{Cattell-Horn-Carroll-Theorie}
\newacronym{tst}{TS-Theorie}{Three-Stratum-Theorie}

%\newacronym{eka}{EKA}{elementaren kognitiven Aufgabe}
\newacronym[
plural=EKAn,
longplural={},
]
{eka}{EKA}{elementaren kognitiven Aufgabe}


\newacronym[
	plural=Vpn,
	longplural={Versuchspersonen},
%	\glsshortpluralkey=Vpn,	
	]
	{vp}{Vp}{Versuchsperson}

\newacronym[
	plural=RZn,
	longplural={Reaktionszeiten}]
	{rz}{RZ}{Reaktionszeit}



\newacronym{bist}{BIS-Test}{Berliner Intelligenzstruktur-Test}
\newacronym{bism}{BIS}{Berliner Intelligenzstrukturmodell}

\newacronym{ai}{AI}{Allgemeine Intelligenz}

\newacronym{k}{K}{Verarbeitungskapazität}
\newacronym{b}{B}{Bearbeitungsgeschwindigkeit}
\newacronym{M}{M}{Merkfähigkeit}
\newacronym{e}{E}{Einfallsreichtum}

\newacronym{v}		{V}				{sprach\-ge\-bund\-enes Denken}
\newacronym{n}		{N}				{zahl\-enge\-bund\-enes Denken}
\newacronym{f}		{F}				{an\-scha\-uungs\-ge\-bun\-denes, fi\-gural-bildhaftes Denken}

\newacronym{bd}		{BD}			{Buchstaben-Durchstreichen}
\newacronym{kw}		{KW}			{Klassifizieren von Wörtern}
\newacronym{oe}		{OE}			{Old English}
\newacronym{RZ}		{RZ}			{Rechen-Zeichen}
\newacronym{tg}		{TG}			{Teil-Ganzes}
\newacronym{uw}		{UW}			{Unvollständige Wörter}
\newacronym{xg}		{XG}			{X-Grösser}

\newacronym{soa}	{SOA}			{Stimulus-Onset Asynchrony}

%\newacronym{kfa}	{KFA}			{Konfirmatorischer Faktorenanalyse}
\newacronym{cst}	{$\upchi^2$-Test}{Chi Quadrat-Test}
\newacronym{cfi}	{CFI}			{Comparative Fit Index}
\newacronym{rmsea}	{RMSEA}			{Root Mean Square Error of Approximation}
\newacronym{rmse}	{\textit{RMSE}}	{Root Mean Square Error}
\newacronym{srmr}	{SRMR}			{Standardized Root Mean Square Residual}

\newacronym{m}		{\textit{M}}	{Mittelwert}
\newacronym{sd}		{\textit{SD}}	{Standardabweichung}
\newacronym{sds}	{\textit{SDs}}	{Standardabweichungen}

\newacronym{epq-rk}	{EPQ-RK}		{Eysenck Personality Questionnaire}
\newacronym{dii}	{DII}			{Dickman Impulsivity Inventory}












% ==========================================================================================
% ==========================================================================================
% ==========================================================================================
%
% B E G I N   D O C U M E N T 
%
% ==========================================================================================
% ==========================================================================================
% ==========================================================================================

\begin{document}
\layout								% prints layout page at the beginning of the document
\frontmatter						% triggers roman numerals 


% =================================================================
% T I T L E   P A G E
% =================================================================

% Declare new goemetry for the title page only --------------------
\newgeometry{top=2in,bottom=2in,right=1.5in,left=1.5in}

\begin{titlepage}
	
	\vspace*{2cm}
	
	\huge\centering Spatial-Suppression, Mental-Speed und psychometrische Intelligenz 
	
	\noindent\makebox[\textwidth]{\rule{\textwidth}{0.4pt}}
	
	\vspace{1.2cm}
	
	{\centering
		\Large \textit{Inauguraldissertation}
		
		\vspace{1.2cm}
		
		der Philosophisch-humanwissenschaftlichen Fakultät der Universität Bern zur Erlangung der Doktorwürde
		
		\vspace{1cm}
		vorgelegt von\\
		Philipp Thomas
		\vspace{1cm}
		
		bei\\
		Prof. Dr. Stefan Troche\\
		
		\vspace*{\fill}
		\Large Bern, Oktober 2016
		
		
	}

\end{titlepage}
% Ends the declared geometry for the titlepage
\restoregeometry


% =================================================================
% S U M M A R Y
% =================================================================
\chapter*{Zusammenfassung \label{cha:Zusammenfassung}}
\addcontentsline{toc}{chapter}{Zusammenfassung}
\noindent
DGP Richtlinien weisen auf folgende Punkte hin:
\begin{itemize}
	\item \textit{Vollständigkeit}
	\item \textit{Genauigkeit}
	\item \textit{Objektivität}
	\item \textit{Kürze}
	\item \textit{Verständlichkeit}
	\item Trotz Kürze sollte über die zu prüfenden psychologischen Hypothesen, die Methode, die Ergebnisse
		  und die Interpretation informiert werden
\end{itemize}

\noindent Generelle Hinweise:
\begin{itemize}
	\item Fragestellung und die zu prüfenden Hypothesen sollten dargestellt werden
	\item Zentrale Merkmale der Teilnehmer sollen angegeben werden (Anzahl, Alter, Geschlecht)
	\item Die experimentelle Methode inklusive verwendeter Apparaturen und Formen der Datenerhebung
	\item Zentrale Befunde angeben
	\item Schlussfolgerung aus den Befunden inklusive deren Bedeutung für die psychologische Hypothese
\end{itemize}
\pagebreak

% =================================================================
% T A B L E   O F   C O N T E N T S
% =================================================================
\renewcommand{\contentsname}{Inhalte}			% define name of toc
\setcounter{tocdepth}{3}						% set toc depth
\tableofcontents								% insert toc
%\addcontentsline{toc}{chapter}{Inhalte}		% add toc to toc? :D

% Insert the lof ---------------------------------------------------
\renewcommand\listfigurename{Abbildungen}		% define name of lof
\listoffigures									% insert lof
\addcontentsline{toc}{chapter}{Abbildungen}		% add lof to toc



% Insert the lot ---------------------------------------------------
\renewcommand\listtablename{Tabellen}			% define name of lot
\listoftables 									% insert lot
\addcontentsline{toc}{chapter}{Tabellen}		% add lot to toc



% =================================================================
% P R E F A C E
% =================================================================
\chapter*{Vorwort \label{cha:Vorwort}}
\addcontentsline{toc}{chapter}{Vorwort}
Diese Arbeit ist das Produkt meiner dreijährigen Forschungstätigkeit. Danken möchte ich 
allen Menschen, die mich in der Zeit unterstützt haben und dazu beigetragen haben, dass diese
%Arbeit erfolgreich zu Ende gebracht wurde. Namentlich möchte ich mich bei Herrn \mbox{Prof. Dr. Thomas
%Rammsayer} bedanken, der mir zu jeder Zeit als Ansprechsperson zur Verfügung stand. \mbox{Prof. Dr. Stefan
%Troche} möchte ich für seine zahlreichen Tipps und Anregungen bezüglich der statistischen Analysen
%danken. \mbox{Prof. Dr. Duje Tadin} ebenfalls. Ein grosser Dank gilt Personen aus meinem privaten Umfeld, die mich in den letzten Jahren begleitet
%und unterstützt haben.
\citep{Upper1974} 
\citep{Pahud2016}

Der \LaTeX-Code für die Reproduktion dieses pdfs und der R-Code für die Reproduktion der berichteten Resultate sind unter \url{http://www.github.com/pipomas} verfügbar.


\vspace{6 mm}

\begin{quote}
Philipp Thomas\\
\today
\end{quote}



\mainmatter				% page numbers in arabic
\widowpenalty=300		% avoid single lines (values 0-1000)
\clubpenalty=300		% avoid single lines (values 0-1000)


% =================================================================
% I N T R O D U C T I O N
% =================================================================
\chapter{Einleitung \label{cha:Einleitung}}


\section{Der \textit{g}-Faktor}

Die Existenz eines Generalfaktors der Intelligenz, kurz \gls{gfaktor} genannt, stellt eine der einflussreichsten Ideen in der Psychologie dar. \citet{Spearman1904, Spearman1927} machte mit seinen Untersuchungen zu Beginn des zwanzigsten Jahrhunderts die Entdeckung, dass zwischen Tests zur Erfassung intellektueller Fähigkeiten positive Zusammenhänge bestanden. Wenn eine Person in einem Test zur Erfassung von mathematischen Denken gut abgeschnitten hatte, so schnitt sie vergleichsweise gut in einem Test zur Erfassung von räumlichen Denkvermögen ab. \citeauthor{Spearman1904} erklärte die positiven Korrelationen zwischen unterschiedlichen Intelligenztest mit dem \gls{gfaktor}, der die Leistung in sämtlichen Bereichen intellektueller Fähigkeit beeinflusst. Diese Idee eines Generalfaktors der Intelligenz wurde einige Jahre später von \citet{Thurstone1938} in Frage gestellt. Er sah den \gls{gfaktor} nicht als die Ursache unterschiedlicher kognitiver Fähigkeiten, sondern vielmehr als die Folge von sieben primären mentalen Fähigkeiten. Die sieben primäre mentale Fähigkeiten von \citeauthor{Thurstone1938} bildeten damit die erste echte multifaktorielle Theorie menschlicher Intelligenz und widersprach der Theorie von \citeauthor{Spearman1904}. Nach diesen beiden einflussreichen Theorien folgten hierarchisch aufgebaute Modelle wie zum Beispiel das von \citet{Vernon1950}, das von \citet{Cattell1971},  oder das von \citet{Jaeger1984}. Die Vielzahl an unterschiedlichen Auffassungen und Modellen psychometrischer Intelligenz veranlasste \citet{Carroll1993} dazu eine umfassende, empirisch begründete Taxonomie der menschlichen kognitiven Fähigkeiten zu entwickeln. Er reanalysierte dafür nahezu alle Datensätze, die für die Entwicklung früherer Modelle verwendet wurden und versuchte ein Modell zu finden, das den Daten am besten entsprach. Seine Ergebnisse formulierte \citeauthor{Carroll1993} in der \gls{tst}, welche auf höchster Hierarchieebene einen \gls{gfaktor} identifizierte. Die Analysen von \citeauthor{Carroll1993} lieferten Evidenz dafür, dass die Annahme von \citeauthor{Spearman1904} ihre Richtigkeit hatten und werden bis heute als sehr wichtig erachtet \citep[für Erweiterungen der \gls{tst} siehe][]{McGrew2005, McGrew2009}.
Voraussetzung für die dafür ist eine grosse Bandbreite an unterschiedlichen Tests mentaler Fähigkeiten. Der \gls{gfaktor} beinhaltet das Gemeinsame und vernachlässigt das Aufgabenspezifische \citep{Jensen1998a}.
Untersuchungen haben gezeigt, dass der \gls{gfaktor} dafür gut eignet. Er ist invariant gegenüber der faktorenanalytischen Methode \citep{Jensen1994} und der eingesetzten Testbatterie \citep{Johnson2004, Johnson2008}. 

Alternative Erklärungen für das Auftreten des \gls{gfaktor}s sind spärlich. Die Mutualism-theory-of-\textit{g} \citep{VanDerMaas2006} scheint nicht adäquat zu sein \citep{Gignac2016} und das Bond-Modell von \citet{Thomson1916} wird heute noch aufgenommen und als plausible Alternativerklärung gehandelt \citep{Bartholomew2013}.





\section{Biologische und kognitive Grundlagen von Intelligenz}

Die Frage nach der Ursache für Intelligenzunterschiede stellte sich schon \citet{Galton1883}. Er hatte die Annahme, dass intelligente Personen über die Fähigkeit verfügen eine grosse Menge an eingehender Sinnesinformation zu verarbeiten, während nicht so intelligente Personen weniger scharfe Sinne besitzen. \citeauthor{Galton1883} operationalisierte die Schärfe der Sinne beispielsweise mit dem Seh- und Hörvermögen, der Farbdiskriminationsleistung oder dem Schmerzempfinden. \citet{Spearman1904} nahm diese Idee der besseren sensorischen Diskriminationsfähigkeit von intelligenten Personen auf und machte selbst Untersuchungen dazu. Er fand positive Zusammenhänge zwischen Diskriminationsleistung und Intelligenz. Es gibt noch andere Ansätze, welche im Folgenden ausführlich beschrieben werden.


\subsection{Der Mental-Speed-Ansatz}

Untersuchungen der letzten Jahrzehnte  haben gezeigt, dass Verarbeitungsgeschwindigkeit (\gls{ms}) und psychometrische Intelligenz zusammenhängen \citep[für Übersichtsarbeiten siehe][]{Deary2000a, Jensen2006, Sheppard2008}. \gls{ms} wird dabei oft mit Hilfe einer sogenannten \gls{eka} erfasst. 

Eine \gls{eka} (\citealp[S. 290]{Anderson2001}; \citealp[S. 11]{Carroll1993}; \citealp[S. 207--209]{Jensen2006}) ist eine Aufgabe, die gesunde Personen mit genügend Zeit ohne grosse mentale Anstrengung fehlerfrei lösen können. Die Stimuli sind gross abgebildet und klar erkennbar, sodass sie von allen Personen mit normalem Sehvermögen gut wahrzunehmen sind. Die \gls{vp} wird aufgefordert, so schnell wie möglich eine Antwort abzugeben und dabei Fehler zu vermeiden. Weil das Lösen der Aufgabe nur sehr simple mentale Prozesse beansprucht, werden interindividuelle Strategien, die das Lösen der Aufgabe erleichtern, unterdrückt. Unterschiede in der Reaktionszeit zwischen \glspl{vp} können dann nur durch die Geschwindigkeit verursacht werden, mit welcher die \glspl{vp} die Aufgabe verarbeiten und auf Stimuli reagieren.
Beispiele für solche reaktionszeitbasierten \glspl{eka} sind
die Coincidence-Timing-Aufgabe \citep{Smith1987},
die \gls{ha} \citep{Hick1952},
das Memory-Scan-Paradigma \citep{Sternberg1966, Sternberg1969},
die Odd-Man-Out-Aufgabe \citep{Frearson1986}, 
die Posner-Aufgabe \citep{Posner1969}
oder das Visual-Scan-Paradigma \citep[][S. 66--71]{Neisser1967}.


\subsubsection*{Die \gls{ha}}

Eine der beliebtesten und ältesten \gls{eka} ist die \gls{ha} \citep{Hick1952}. Die \gls{ha} erfasst einfache Reaktionszeit und die Reaktionszeit für eine Mehrfachauswahl. Die Reaktionszeit in der \gls{ha} kann gemäss \citet[S. 105]{Jensen1987a} mit der linearen Funktion \textit{Reaktionszeit}~$=a+b\log_{2}n$ beschrieben werden, wobei $a$ durch den y-Achsen\-ab\-schnitt, $b$ durch die Steigung der Regres\-sions\-geraden und $\log_{2}\,n$ durch den Logarithmus zur Basis 2 der Anzahl Antwortalternativen ($n$) bestimmt ist. Das Produkt $\log_{2}\,n$ wurde von \citet{Hick1952} als \textit{Bit} bezeichnet und entspricht derjenigen Menge an Information, welche die Entscheidung zwischen zwei gleich wahrscheinlichen Antwortalternativen ermöglicht\footnote{Entsprechend dieser Definition gab das \textit{Bit} den Bedingungen der \gls{ha} ihre Namen: In der 0-bit-Bedingung steht eine Antwortalternative zur Verfügung ($\log_{2}\,1=0$), in der 1-bit-Bedingung stehen zwei Antwortalternativen zur Verfügung ($\log_{2}\,2=1$), in der 2-bit-Bedingung stehen vier Antwortalternativen zur Verfügung ($\log_{2}\,4=2$) und in der 2.58-bit-Bedingung stehen sechs Antwortalternative zur Verfügung ($\log_{2}\,6=2.58$).} \citep[siehe auch][S. 27]{Jensen2006}.

Für die Differentielle Psychologie wurde die \gls{ha} mit der Untersuchung von Roth \citep[1964; zitiert nach][S. 105]{Jensen1987a} interessant. Er berichtete über einen Zusammenhang von $r=-.39$ zwischen der aus den Reaktionszeiten abgeleiteten Steigung und Intelligenz. Intelligentere \glspl{vp} zeigten also mit zunehmender Anzahl Antwortalternativen einen weniger starke Verlangsamung ihrer Reaktionszeit als weniger intelligente \glspl{vp}. Weiter fand Roth keinen Zusammenhang zwischen dem y-Achsenabschnitt und Intelligenz. Diese Resultate legten die Vermutung nahe, dass Unterschiede in komplexer kognitiver Leistung erfasst mit Intelligenztests von der Geschwindigkeit abhängen, mit der Informationen verarbeitet werden \citep{Jensen1987a}.

Spätere Untersuchungen haben gezeigt, dass auch der y-Achsenabschnitt mit Intelligenz negativ zusammenhängt \citep{Jensen1982b, Jensen1987a, Neubauer1997a, Neubauer1997b}. Intelligentere \glspl{vp} zeigten also mit eine kleinere Reaktionszeit als weniger intelligente \glspl{vp}.



Der y-Achsenabschnitt ($a$) widerspiegelt dann Prozesse wie die Wahrnehmungsgeschwindigkeit und die benötigte Zeit, um eine motorische Antwort abzugeben.


Die Aufteilung der Reaktionszeit in y-Achsenabschnitt und Steigung war der erste Versuch, MT DT zu unterscheiden \citep{Jensen1979}.



Nochmal verdeutlichen, das slope das wichtige ist (auf hick beziehen)





In der Steigung ($b$) wird die Geschwindigkeit der kognitiven Prozesse abgebildet die benötigt wird, um eine Antwort auszuwählen. 
Die Reaktionszeiten der \gls{ha} sind negativ korreliert \citep{Sheppard2008} (u.a.).
Slope ist interessant, weil das aufschluss über die beteiligten Prozesse der intelligenz geben könnte.


und die daraus abgeleiteten Aufgabenparameter (y-Achsenabschnitt und Steigung) korrelieren negativ mit psychometrischer Intelligenz \citep{Helmbold2006a, Jensen1982a, Jensen1982b, Jensen1987a, Neubauer1991, Neubauer1992, Neubauer1997a, Neubauer1997b, Rammsayer2007, Rammsayer2010a}.

%Siehe auch \citet{Jensen1998b} suppressor variable


Der Mental-Speed-Ansatz \citep{Jensen1982a, Jensen1982b, Vernon1983} erklärt den Zusammenhang zwischen kürzeren Reaktionszeiten und besserer mentaler Operationen damit, dass eine schnelle und effiziente Verarbeitung von Informationen sowohl bei einer Reaktionszeitaufgabe wie auch bei einem Intelligenztest benötigt wird. Als Grundlage für diesen Zusammenhang kommen höhere neuronale Effizienz \citep{Bates1995, Hendrickson1980}, eine bessere Myelinisierung von Neuronen \citep{Miller1994}, eine höhere neuronale Plastizität \citep{Garlick2002}, oder spezifische Aktivierung \citep{Neubauer1995} in Frage.





Mental-Speed als Erklärung für andere Korrelate. Jensen (AGK) schneller -> bessere AGK, Melnick schnellere Verarbeitung -> bessere Leistung.
Ulman Lindenberger / Saulberg: AG über mental-speed zu erklären 



\subsection{Der Spatial-Suppression-Ansatz}

Das Phänomen der Spatial-Suppression wurde von \citet{Tadin2003} unabhängig von psychometrischer Intelligenz entdeckt. \citet{Tadin2003} machten die Entdeckung, dass sich die Wahrnehmungsleistung mit zunehmender Grösse eines Stimulus verschlechtert. 

Zentrum-Umfeld-Antagonismus \citep{Allman1985a}\\

\subsubsection*{Die Spatial-Suppression-Aufgabe}























%\vspace{5cm}
%
%\citep{Galton1869}
%
%Die Existenz eines Generalfaktors der Intelligenz, kurz \gls{gfaktor} gennant, stellt eine der einflussreichsten Ideen in der Psychologie dar.
%
%Intelligenz ist eines der ältesten Konstrukte in der Forschungsgeschichte der Psychologie und beeinflusst den Menschen in den unterschiedlichsten Lebensbereichen.
%Intelligentere Personen sind im Vergleich zu weniger intelligenten Personen zum Beispiel politisch liberaler eingestellt \citep{Kanazawa2010}, weniger religiös \citep{Zuckerman2013}, sie leben gesünder und länger \citep{Gottfredson2004}, sind beruflich erfolgreicher \citep{Salgado2003} oder verhalten sich weniger aggressiv \citep{Ackerman1997}.
%Was aber genau ist unter Intelligenz zu verstehen? Während sich die Wissenschaftsgemeinde über die Bedeutung der Intelligenz für das Erleben und Verhalten des Menschen im klaren ist, besteht bezüglich der Definition des Konstrukts weniger Einigkeit \citep{Neisser1996}. 
%Trotz dem heterogenen Verständnis darüber, was Intelligenz ausmacht, lässt sich innerhalb der verschiedenen Ansichten einen gemeinsamen Nenner finden: Kognitive Fähigkeiten wie abstraktes Denken, Problemlösen und Entscheidungsfindung sind wiederkehrende Bestandteile in vielen Definitionen und können als Kernmerkmale von Intelligenz aufgefasst werden \citep{Sternberg1986b}. 
%
%\begin{itemize}
%	\item Diskriminationsfähigkeit \citep{Spearman1904}
%	\item IVG\ldots -> Roth \citep[1964; zitiert nach][S. 121]{Jensen1979}
%\end{itemize}
%
%
%\subsection{Struktur der Intelligenz}
%
%Intelligenz ist kein direkt beobachtbares Merkmal. Sie ist vielmehr ein theoretisches Konstrukt, das mit Hilfe von psychometrischer Tests erfasst werden muss. Psychometrische Intelligenztests gehören zu den am weitesten entwickelten Instrumente zur Erfassung interindividueller Fähigkeitsunterschiede und kamen  vor allem im 20. Jahrhundert mit Entwicklung der Faktorenanalyse \citep{Spearman1904} auf. Diese neu entwickelte Methode ermöglichte es, die strukturellen Bestandteile menschlicher kognitiver Fähigkeiten zu identifizieren und daraus Theorien zu bilden.
%
%Die Zwei-Faktorentheorie von \citet{Spearman1904, Spearman1927} führte die Idee einer allgemeinen Intelligenz (\textit{g}) ein. \citeauthor{Spearman1904} machte die Beobachtung, dass Leistungen zwischen unterschiedlichen (zum Beispiel räumlichen, sprachlichen oder mathematischen) Tests substanziell miteinander zusammenhingen. Mit Hilfe der Faktorenanalyse konnte er zeigen, dass ein Faktor (der \gls{gfaktor}) die Zusammenhänge zwischen den verschiedenen Tests erklären konnte. \citeauthor{Spearman1904} schloss daraus auf eine allgemeine Intelligenz, welche für jegliche Art von kognitiver Leistung verantwortlich sein muss. Der zweite Faktor seiner Theorie besteht aus dem aufgabenspezifischen Teil einer Aufgabe, welche unabhängig vom \gls{gfaktor} ist und ausschliesslich für die Bearbeitung der jeweiligen Aufgabe erforderlich ist. Für \citeauthor{Spearman1904} setzt sich die Leistung in einer Gedächtnisaufgabe also aus der aufgabenspezifischen Fähigkeit zusammen, Informationen im Gedächtnis zu erhalten respektive wiederzugeben und aus der allgemeinen, aufgabenunspezifischen Fähigkeit (\textit{g}) kognitive Leistung zu erbringen.
%
%Die Primärfaktorentheorie von \citet{Thurstone1938} stellte die Zwei-Fak\-tor\-en\-theo\-rie von \citet{Spearman1904, Spearman1927} in Frage. Er wies darauf hin, dass  \citeauthor{Spearman1904} nur gezeigt hatte, dass kognitive Leistungen in verschiedenen Bereichen positiv miteinander korrelierten. Die Annahme, dass der \gls{gfaktor} für alle Leistungen in allen Bereichen verantwortlich sei, schien \citeauthor{Thurstone1938} nicht plausibel. 
%Er postulierte deshalb eine Theorie mit sieben Primärfaktoren und sah den \gls{gfaktor} von \citeauthor{Spearman1904} vielmehr als ein Artefakt seiner Primärfaktoren. 
%
%%\citet{Cattell1971} wiederum anerkannte die Idee von \citeauthor{Spearman1904}, teilte den \gls{gfaktor} auf einer tieferen Ebene jedoch in \gls{gf} und \gls{gc}. \gls{gf} spiegelt die Fähigkeit wieder, schlussfolgernd zu denken, Informationen schnell aufzunehmen und zu verarbeiten sowie Informationen miteinander miteinander in Verbindung zu setzten. \gls{gc} hingegen spiegelt Wissen und Fähigkeiten wider, welche durch Lernen erworben wurden und kulturspezifisch sind. \citeauthor{Cattell1971} sah \gls{gc} also im Vergleich zu \gls{gf} als viel dynamischer an.
%
%Nach diesen ersten beiden einflussreichen Strukturmodellen folgten noch viele mehr, die alle mehr oder weniger gemeinsam hatten \citep[siehe][]{Vernon1950, Cattell1971, Guilford1977, Jaeger1984}. Diese Vielfalt an Strukturmodellen veranlasste \citet{Carroll1993} dazu eine umfassende, empirisch begründete Taxonomie der menschlichen kognitiven Fähigkeiten zu entwickeln. Er reanalysierte dafür nahezu alle Datensätze, die für die Entwicklung früherer Strukturmodelle verwendet wurden und versuchte ein Modell zu finden, das den Daten am besten entsprach. Seine Ergebnisse formulierte \citeauthor{Carroll1993} in der \gls{tst}. Das hierarchisches Strukturmodell postulierte auf dritter Ebene einen allgemeinen Intelligenzfaktor (\textit{g}), der alle kognitiven Fähigkeiten beeinflusst. Auf zweiter Hierarchieebene identifizierte \citeauthor{Carroll1993} acht generelle Fähigkeiten (\textit{broad abilities}): Fluide Intelligenz, kristalline Intelligenz, allgemeines Gedächtnis und Lernen, visuelle Wahrnehmung, auditive Wahrnehmung, Wiedergabefähigkeit, kognitive Schnelligkeit und Verarbeitungsgeschwindigkeit. Diese acht generellen Fähigkeiten der zweiten Ebene wurden faktorenanalytisch aus 68 spezifischen Fähigkeitskonstrukten (\textit{narrow abilities}) der ersten Ebene abgeleitet. 
%
%%Die \gls{chct} stellt die letzte \citep{McGrew2005, McGrew2009}
%\subsection{Der \gls{gfaktor}}
%
%Die Untersuchungen von \citet{Carroll1993} und deren Erweiterungen im Rahmen des Cattell-Horn-Carroll-Modells \citep[siehe][]{McGrew2005, McGrew2009} haben gezeigt, dass 
%
%
%
%\section{Erklärungsansätze für Intelligenzunterschiede}
%
%\subsection{Mental-Speed-Ansatz}
%
%Untersuchungen der letzten Jahrzehnte  haben gezeigt, dass Verarbeitungsgeschwindigkeit und psychometrische Intelligenz zusammenhängen \citep[für Übersichtsarbeiten siehe][]{Deary2000a, Jensen2006, Sheppard2008}. Verarbeitungsgeschwindigkeit wird dabei mit einer \gls{eka} erfasst.  \citep[S. 11]{Carroll1993}.
%
%
%
%
%
%
%
%
%\vspace{2cm}
%
%
%warum mental speed am stärksten mit g?
%warum ha als mass für mental speed?
%
%
%\subsection{Spatial-Suppression}
%
%Das Phänomen der Spatial-Suppression wurde von \citet{Tadin2003} unabhängig von psychometrischer Intelligenz entdeckt. \citet{Tadin2003} machten die Entdeckung, dass sich die Wahrnehmungsleistung mit zunehmender Grösse eines Stimulus verschlechtert. 
%
%Zentrum-Umfeld-Antagonismus \citep{Allman1985a}\\
%


\section{Das Impurity-Problem und Fixed-Links Modelle}

\begin{itemize}
	\item Ausführlicher, weil das ein zentraler Bestandteil der Arbeit ist
	\item 
\end{itemize}

Messungen von kognitiven Prozessen sind in der Regel keine reinen Masse. In vielen experimentellen Aufgaben werden verschiedene zum Teil aufeinanderfolgende kognitive Prozesse benötigt, um zu einer Lösung zu gelangen. Einer Aufgabe muss Aufmerksamkeit zugewendet werden, Informationen müssen enkodiert und verarbeitet werden und zum Schluss muss eine motorische Antwort erfolgen. Die motorische Antwort endet meistens in der Erfassung von Antwortakkuratheit oder Reaktionszeit als abhängige Variable, die dann als Indikator für den kognitiven Prozess verwendet wird. Der eigentlich interessierende kognitive Prozess fliesst also nur zu einem gewissen Grad in die abhängige Variable mit ein und wird von anderen Prozessen der Informationsverarbeitung überlagert, was zu ungültigen Schlussfolgerungen führen kann. Dieses Phänomen wird als Impurity-Problem bezeichnet.

Um dem Impurity-Problem entgegenzutreten, hat \citet{Schweizer2006a, Schweizer2006b} eine Variante der herkömmlichen Faktorenanalyse vorgeschlagen, die sich Fixed-Links-Modelle nennt. Die Idee von Fixed-Links-Modellen besteht darin, die Varianz eines (vermeintlich) eindimensionalen Konstrukts in verschiedene Varianzquellen bzw. Prozesse aufzutrennen. Mit dieser Trennung der Varianz des Konstrukts wird die Überlagerung von verschiedenen Prozessen auf statistischer Ebene rückgängig gemacht, so dass nach dieser Trennung der Varianz die Zusammenhänge der Variablen mit Drittvariablen bereinigt sind. Voraussetzung für eine solche Trennung der Varianz mit Fixed-Links-Modellen ist die experimentelle Manipulation von Aufgaben. 

%Die Idee der experimentellen Manipulation besteht darin, alle Prozesse bis auf den eigentlich interessierenden Prozess über alle Bedingungen hinweg konstant zu halten. 

Bei der Analyse der Aufgabe werden dann in der Regel zwei latente Variablen angenommen. Eine latente Variable beinhaltet aufgabenrelevante Prozesse, deren Einflüsse sich über die vier Bedingungen hinweg nicht verändert haben. Dieser gleichbleibende Einfluss wird hergestellt, indem die unstandardisierten Faktorladungen aller manifesten Variablen auf den Wert 1 fixiert wurden. Eine zweite latente Variable beinhaltet aufgabenrelevante Prozesse, die durch die Bedingungen systematisch manipuliert wurden. Der unterschiedlich starke Einfluss der in der latenten Variable abgebildeten Prozesse auf die Bedingungen wird durch sich unterscheidende unstandardisierte Faktorladungen hergestellt.

Anschliessend kann mit der Fixed-Links-Technik die Varianz des Konstrukts in unabhängige Varianzquellen aufgetrennt werden. Dabei lässt sich die Varianzquelle der experimentellen Manipulation, die den eigentlich interessierenden kognitiven Prozess abbildet, von der Varianzquelle aller anderen Prozesse, die nicht durch die experimentelle Manipulation verursacht werden, trennen.

Um das Impurity-Problem und die Fixed-Links-Technik verständlicher zu machen, wird als nächstes ein Beispiel gegeben. Als Grundlage für dieses Beispiel dient die Arbeit von Troche und Rammsayer (2009)

\section{Fragestellungen \label{sec:Fragestellungen}}

Der \gls{ssans} zur Erklärung individueller Intelligenzunterschiede ist neu und unterscheidet sich von der Art der Aufgabenstellung her deutlich von reaktionszeitbasierten \glspl{msm}. Das übergeordnete Ziel dieser Arbeit besteht darin, zu überprüfen, ob sich die \gls{ssauf} als Prädiktor psychometrischer Intelligenz bewährt und inwiefern der \gls{ssans} zur Aufklärung individueller Intelligenzunterschiede neuartige Erklärungsmöglichkeiten liefert, welche nicht bereits der \gls{msa} bietet. Dieses Ziel wird in fünf Punkten abgearbeitet:

\begin{enumerate}
	\item Die Arbeit von \citet{Melnick2013} berichtet bis heute als einzige über den Zusammenhang zwischen der \gls{ssauf} und psychometrischer Intelligenz. Um die Aufgabe in Zusammenhang mit psychometrischer Intelligenz als Prädiktor zu festigen, bedarf es zuerst einer Bestätigung dieses Befundes. Dafür werden für die vorliegende Arbeit die experimentellen Bedingungen von \citet{Melnick2013} bestmöglich übernommen und die Aufgabe wird einer grossen, betreffend der Intelligenzausprägung heterogenen Stichprobe vorgelegt. Die aus der Aufgabe abgeleitete abhängige Variable, der \gls{si}, wird entsprechend dem Vorgehen in der Originalarbeit gebildet. Der \gls{si} wurde in der Arbeit von \citeauthor{Melnick2013} mit IQ-Punkten in Zusammenhang gesetzt. Der IQ wurde dabei für jede Person aus der Kurzform der Wechsler-Adult-Intelligence-Scale III \citep{Axelrod2002} und aus der Wechsler-Adult-In\-tell\-igence-Scale IV \citep{Wechsler2008} gebildet \citep[siehe Studie 1 und 2 bei][]{Melnick2013}. Wenn die Annahme gilt, dass der \gls{si} -- IQ Zusammenhang robust ist, sollte dieser auch unter Einsatz eines anderen Instruments zur Erfassung der psychometrischen Intelligenz auftreten. In der  vorliegenden Arbeit wird der Berliner Intelligenzstruktur-Test \citep{Jaeger1997} eingesetzt, welcher sich empirisch als Indikator für psychometrische Intelligenz bewährt hat \citep{Beauducel2002, Valerius2014}. Die Verwendung von nicht exakt demselben Intelligenzmass erscheint hinsichtlich einer beabsichtigten Bestätigung des Befundes von \citeauthor{Melnick2013} als Schwachpunkt dieser Arbeit. Führt man sich aber vor Augen, dass die \gls{ssauf} beansprucht, einen grundlegenden Aspekt der menschlichen Informationsverarbeitung zu erfassen, erscheint die Verwendung eines Intelligenzmasses, welches noch nie mit der \gls{ssauf} in Zusammenhang gebracht wurde, weniger als Schwachpunkt, sondern vielmehr als eine Notwendigkeit.

	\item Der \gls{si}, die in der Arbeit von \citet{Melnick2013} abhängige Variable, wurde für jede Person als Differenz zwischen zwei Schwellenschätzungen gebildet. Dabei wurde nicht berücksichtigt, dass Differenzmasse unter bestimmten, in empirischen Studien oft vorliegenden Bedingungen, problematisch sind: Weisen der Minuend  beziehungsweise der Subtrahend keine perfekte Reliabilität auf und hängen sie zusammen, reduziert sich die Reliabilität des Differenzmasses. Beträgt beispielsweise die Reliabilität vom Minuend $r_{xx} = .80$, die Reliabilität vom Subtrahend $r_{yy} = .80$ und die Korrelation von Minuend und Subtrahend $r_{xy} = .50$, reduziert sich die Reliabilität der Differenz auf $r_{DD} = .60$ \citep[][ S. 145]{Murphy2005}. Wird der \gls{si} als Differenzmass gebildet, kann folglich nicht ausgeschlossen werden, dass ein verhältnismässig wenig reliables Mass vorliegt. Um diesem Umstand Rechnung zu tragen, wird in der vorliegenden Arbeit eine abhängige Variable eingesetzt, welche nicht auf einer Differenz zwischen zwei Schwellenschätzungen beruht. \citeauthor{Melnick2013} haben sich in ihrer Arbeit bereits bemüht, ein alternatives Mass herzuleiten. Sie haben die Wahrnehmungsschwellen jeder Person mit einer exponentiellen Regression vorhergesagt, jedoch nicht beide daraus resultierenden Parameter, die Asymptote und die Steigung, mit psychometrischer Intelligenz in Verbindung gesetzt. Um die \gls{ssauf} mit ihren Bestandteilen besser zu verstehen, werden deshalb in dieser Arbeit die aus der exponentiellen Regression abgeleiteten Aufgabenparameter benutzt, um psychometrische Intelligenz vorherzusagen.

	\item Eine weitere Möglichkeit zur Quantifizierung der \gls{ssauf} besteht darin, die Aufgabenbedingungen auf latenter Ebene zu analysieren und damit die Zusammenhänge der Aufgabenbedingungen auf einen oder mehrere unbeobachtete Faktoren zurückzuführen. Im Gegensatz zur manifesten Auswertung (vgl. Punkt 1 und 2) berücksichtigt die Analyse auf latenter Ebene die Tatsache, dass sich ein beobachteter Messwert immer aus einem wahren Anteil der Merkmalsausprägung und einem zufällig zustande gekommenen Fehleranteil, der unabhängig von der wahren Merkmalsausprägung ist, zusammensetzt. Ein latenter Faktor beinhaltet nur die wahren Merkmalsausprägungen von Personen, womit sich, verglichen mit einer Analyse auf manifester Ebene, Zusammenhänge mit anderen Variablen valider bestimmen lassen. Die Bedeutung der \gls{ssauf} als Prädiktor von \textit{g}, der latenten Operationalisierung psychometrischer Intelligenz, sollte demnach auf latenter Ebene deutlicher erkennbar sein als auf manifester Ebene.

	\item Um bei der Beschreibung der \gls{ssauf} auf latenter Ebene eine vergleichbare Trennung von Prozessen zu erhalten wie unter Punkt 2 auf manifester Ebene, wird versucht die Aufgabenbedingungen mit einem \gls{flm} (\textcolor{red}{/autoref{Verweis auf Einleitung, in welcher FLM beschrieben werden folgt noch}}) zu beschreiben. 
	Dafür werden zwei latente Variablen angenommen: Die erste latente Variable bildet durch konstantgehaltene Faktorladungen aufgabenrelevante Prozesse ab, deren Einflüsse sich über die vier Bedingungen hinweg nicht ändern. 
	Die zweite latente Variable weist sich unterscheidende Faktorladungen auf, welche  bestimmten Annahmen folgend gewählt werden.
	Durch die sich unterscheidenden Faktorladungen werden in der zweiten latenten Variable aufgabenrelevante Prozesse gebunden, die durch die vier Bedingungen systematisch manipuliert wurden. Weil die Aufgabe noch nie mit einem \gls{flm} beschrieben wurde, werden unterschiedliche Ladungsverläufe gegeneinander getestet und das beste Modell für die weiteren Analysen ausgewählt.
	Diese Trennung von aufgabenrelevanten Prozessen auf latenter Ebene kann dann zum einen benutzt werden um die \gls{ssauf} mit ihren Bestandteilen besser zu verstehen und zum anderen lässt sich damit der Zusammenhang  der Aufgabe mit dem \gls{gfaktor} differenzierter betrachten, als mit herkömmlichen Faktorenanalysen.

	\item Nach dieser ausführlichen, aber auch isolierten Aufarbeitung des Zusammenhangs zwischen der \gls{ssauf} und psychometrischer Intelligenz folgt in einem letzten Schritt die Einbettung der \gls{ssauf} in ihr nomologisches Netzwerk. Dafür wird die \gls{ha} als ein etabliertes \gls{msm} hinzugezogen. Die \gls{ssauf} kann auf manifester wie auch auf latenter Ebene mit der \gls{ha} und psychometrischer Intelligenz in Verbindung gebracht werden und es kann der Frage nachgegangen werden, welche Prozesse sich hinter den unter Punkt 2 und Punkt 4 erarbeiten Parametern (Asymptote und Steigung respektive latente Variable mit konstanten Faktorladungen und latente Variable mit ansteigenden Faktorladungen) verbergen. Mit der Einbettung der \gls{ssauf} in dieses nomologische Netzwerk soll sichergestellt werden, dass die Aufgabe in Zusammenhang mit psychometrischer Intelligenz einen Aspekt der menschlichen Informationsverarbeitung abbildet, der neuartig ist und nicht bereits von bestehenden, etablierten Aufgaben erfasst beziehungsweise erklärt wird. Schlussendlich soll dadurch die Frage beantwortet werden, ob der \gls{ssans} zur Aufklärung individueller Intelligenzunterschiede neuartige Erklärungsmöglichkeiten bietet oder ob der \gls{msa} den Zusammenhang zwischen der \gls{ssauf} und psychometrischer Intelligenz vollständig zu erklären vermag.

\end{enumerate}



% =================================================================
% M E T H O D
% =================================================================
\chapter{Methode \label{cha:Methode}}

\section{Stichprobe \label{sec:Stichprobe}}

An den Testungen haben $206$~\glspl{vp} teilgenommen, wovon $29$~\glspl{vp}~($14\,\%$) aufgrund von technischen Problemen, nicht auswertbarer Subtests oder im Vergleich zu den restlichen \glspl{vp} stark abweichenden Werten ausgeschlossen wurden (siehe \autoref{cha:Anhang_A} für eine genaue Erläuterung der Vorgehensweise).

Analysiert wurden die Daten von $177$ \glspl{vp}. Die $116$ Frauen und $61$ Männer waren zwischen $18$ und $30$ Jahre alt und wiesen ein mittleres Alter $\pm$ \gls{sd} von $21.14\,\pm\,2.71$ Jahren auf. 
Um eine bezüglich der Intelligenzausprägung heterogene Stichprobe zu erhalten, nahmen \glspl{vp} aus verschiedenen Bildungsgruppen an der Untersuchung teil:
Neun~\glspl{vp} haben als höchsten Bildungsabschluss die obligatorische Schulzeit genannt,
$55$~\glspl{vp} eine Berufslehre,
$31$~\glspl{vp} eine Berufsmatura,
$23$~\glspl{vp} eine gymnasiale Maturität,
$45$~\glspl{vp} ein Bachelor-Studium,
drei~\glspl{vp} ein Master-Studium und 
$11$~\glspl{vp} eine andere Ausbildung.
$160$ der $177$ \glspl{vp} waren deutscher Muttersprache. Die anderen $17$~\glspl{vp} sprachen akzentfrei deutsch. Alle \glspl{vp} berichteten über eine normale Sehschärfe, eine normale Hörfähigkeit, waren Nichtraucher, konsumierten keine Medikamente und waren nicht chronisch krank. Um Einflüsse von Koffein auf die Wahrnehmungsleistung \citep[][]{Stough1995} der \glspl{vp} zu minimieren, wurden die \glspl{vp} gebeten, bis zwei Stunden vor der Teilnahme keine koffeinhaltigen Getränke zu konsumieren. Die \glspl{vp} hatten keine Erfahrung mit den eingesetzten Testverfahren. 
Für die Teilname an der Untersuchung erhielten Berner Studierende des Fachs Psychologie vier~Ver\-suchs\-per\-sonen-Stun\-den, die sie an ihr Studium anrechnen lassen konnten. Alle anderen \glspl{vp} wurden für die Teilnahme mit CHF~$50.-$ entlöhnt.




\section{Die Spatial-Suppression-Aufgabe \label{sec:}}

Als Grundlage für die Aufgabe diente der \href{http://www.bcs.rochester.edu/people/duje/SuppressionCode.zip}{Programmcode} von \citet{Melnick2013}.

\subsection{Apparatur und Material \label{sub:ssas}}
Präsentiert wurde die Aufgabe auf einem ASUS Vento A2 Computer, der mit einem 2.6 GHz Prozessor, 4 GB Arbeitsspeicher und 512 MB Videospeicher (Nvidia GeForce 9800 GT) ausgestattet war. Als Betriebssystem diente Windows 7. Der verwendete ASUS VG248QE Monitor wies bei einer Bildschirmbreite von $53.2$ cm und einer Bildschirmhöhe von $29.9$ cm eine Auflösung von $1920 \times 1080$ Pixel auf. Er wurde linearisiert und mit einer Bildwiederholungsrate von 144 Hz betrieben. Die Antworten der \glspl{vp} wurden mit einer PC-Tastatur erfasst. 

Die visuellen Reize wurden in MATLAB\textsuperscript{\textregistered} \citep{matlab} erzeugt. Die vertikal schwarz-grau gestreiften Muster (räumliche Frequenz von $1^{\circ}$ Sehwinkel pro Periode) wurden mit einem Kontrast von $99\,\%$ auf einem grauen Hintergrund präsentiert, welcher eine Leuchtdichte von $178\,\textnormal{cd}/ \textnormal{m}^2$ aufwies. Die Leuchtdichte des Raumes betrug in unmittelbarer Umgebung des Monitors $9\,\textnormal{cd}/ \textnormal{m}^2$. Die drei in \citet{Melnick2013} verwendeten Mustergrössen mit den Sehwinkeln  $1.8^{\circ}$, $3.6^{\circ}$ und $7.2^{\circ}$ wurden um die Mustergrösse von $5.4^{\circ}$ ergänzt, wodurch sich für diese Arbeit die Mustergrössen mit den Sehwinkeln $1.8^{\circ}$, $3.6^{\circ}$, $5.4^{\circ}$ und $7.2^{\circ}$ ergaben (siehe \autoref{fig:spatial_suppression_stimuli}). 
Die Sehwinkel der Muster wurden mit einer Kinnstütze, die $61$~cm vom Monitor entfernt war, sichergestellt. 
%Die räumliche Frequenz aller vertikal schwarz-grau gestreiften Mustergrössen betrug wie bei \citeauthor{Melnick2013} $1^{\circ}$ pro Periodenlänge.
Der verwendete Ton wies bei einer Frequenz von $2900$~Hz und einer Lautstärke von $70$~dB eine Länge von $50$~ms auf.

%\begin{figure}[htb]
%	\centering
%	\subfloat[Sehwinkel $= 1.8^{\circ}$][Sehwinkel $= 1.8^{\circ}$]{\includegraphics[width=0.48\textwidth]{../jpg/s1_a}}~~
%	\subfloat[Sehwinkel $= 3.6^{\circ}$][Sehwinkel $= 3.6^{\circ}$]{\includegraphics[width=0.48\textwidth]{../jpg/s2_a}}
%
%	\subfloat[Sehwinkel $= 5.4^{\circ}$][Sehwinkel $= 5.4^{\circ}$]{\includegraphics[width=0.48\textwidth]{../jpg/s3_a}}~~
%	\subfloat[Sehwinkel $= 7.2^{\circ}$][Sehwinkel $= 7.2^{\circ}$]{\includegraphics[width=0.48\textwidth]{../jpg/s4_a}}
%
%	\caption[Die Spatial-Suppression-Bedingungen]{Die vier Mustergrössen $(a - d)$ der \gls{ssauf}.}
%	\label{fig:spatial_suppression_stimuli}
%\end{figure}

\subsection{Versuchsablauf \label{subsec:Prozedur}}

Ein Durchgang sah folgendermassen aus: Nach einer Zeitspanne von $440$~ms erschien in der Mitte des Monitors für $560$~ms ein Kreis, der sich über die ersten $200$~ms von einer Grösse von $1.6^{\circ}$ auf eine Grösse von $0.26^{\circ}$ zusammenzog, für $360$~ms diese Grösse beibehielt und anschliessend ausgeblendet wurde. Dieses Vorgehen diente dazu, den Blick der \glspl{vp} in die Bildschirmmitte zu lenken. Nach einem  Intervall von $300$~ms erschien in der Mitte des Monitors ein sich nach links oder rechts bewegendes vertikal schwarz-grau gestreiftes Musters. Die Stelle, an welcher die \glspl{vp} das Muster auf dem Monitor sahen, war stationär. Hinter dieser stationären Stelle bewegte sich das Muster mit einer Geschwindigkeit von $4^\circ / \textnormal{s} $  nach links oder nach rechts. Nach der Darbietungszeit mussten die \glspl{vp} mit einem Tastendruck entscheiden, in welche Richtung sich das Muster bewegt hat. Die \glspl{vp} erhielten die Instruktion, bei einer wahrgenommenen Bewegung nach links mit ihrem linken Zeigefinger die linke Pfeiltaste  und bei einer wahrgenommen Bewegung nach Rechts mit ihrem rechten Zeigefinger die rechte Pfeiltaste zu drücken. 
Bei einer korrekten Antwort wurde ein Ton abgegeben und die Darbietungszeit des nächsten Musters verringert, bei einer falschen Antwort wurde kein Ton abgegeben und die Darbietungszeit des nächsten Musters erhöht. 
Die Darbietungszeit des Musters wurde entsprechend dem QUEST-Ver\-fah\-ren \citep{Watson1983} angepasst.
Das QUEST-Ver\-fah\-ren ist adaptiv und arbeitet mit logarithmierten Werten, das heisst alle Berechnungen des Verfahrens finden im logarithmierten Raum statt. Der Algorithmus schätzt dabei mit Hilfe von Grundprinzipien der Bayes-Statistik nach jeder Antwort eine $\log_{10}$-Erkennungsschwelle für einen im Voraus bestimmten Prozentsatz an korrekten Antworten (in der hier vorliegenden Aufgabe betrug der Prozentsatz $82\,\%$).
Die geschätzte $\log_{10}$-Erkennungsschwelle wird dann vom Algorithmus benutzt, um die Darbietungszeit des nächsten Stimulus zu bestimmen. 
Die \glspl{vp} wurden instruiert, sich bei der Antwortabgabe genügend Zeit zu lassen und möglichst fehlerfrei zu arbeiten. Nach Antwortabgabe startete der nächste Durchgang.

Als Erstes bearbeiteten die \glspl{vp} eine Übungsaufgabe. Dabei wurden die vier Mustergrössen allen \glspl{vp} je drei Mal in einer pseudorandomisierten Abfolge präsentiert. Die Darbietungszeit aller Mustergrössen betrug zu Beginn der Aufgabe $80$~ms und wurde adaptiv angepasst. Die Übungsaufgabe dauerte etwa eine Minute und wurde nicht ausgewertet. Die $12$~Durchgänge der Übungsaufgabe dienten dazu, dass sich die \glspl{vp} mit der Art der Stimuluspräsentation, der Antworteingabe und dem Ton vertraut machen konnten. 

Als Zweites folgte eine etwas längere Aufgabe. Die \glspl{vp} bearbeiteten drei Wiederholungen, die durch eine Pause von etwa $30$~Sekunden getrennt waren. Eine Wiederholung bestand aus zwei Schätzungen der $82\,\%$-$\log_{10}$-Er\-ken\-nungs\-schwel\-le pro Mustergrösse. Jede der vier Mustergrössen wurde innerhalb einer Schätzung sieben Mal präsentiert. Gesamthaft bearbeiteten die \glspl{vp} folglich $3 \times 2 \times 4 \times 7 = 168$ Durchgänge. Die Mustergrössen wurde allen \glspl{vp} in einer pseudorandomisierten Abfolge präsentiert. Die Darbietungszeit der Mustergrössen betrug zu Beginn der Aufgabe $30$~ms und wurde für jede Mustergrösse einzeln über den gesamten Verlauf der $42$~Durchgänge adaptiv angepasst. Die Aufgabe dauerte etwa $7$~Minuten und wurde nicht ausgewertet, weil sich bei einigen \glspl{vp} die Wahrnehmungsleistung während der ersten Durchgänge stark verbessern kann (D. Tadin, persönl. Mitteilung, 19.08.2014). Dieser Aufgabenblock diente dazu, diese Trainingseffekte zuzulassen und die Leistung der \glspl{vp} auf ihrem individuellem Niveau zu festigen. 

Als Drittes wurde den \glspl{vp} die eigentliche Aufgabe vorgelegt. Die \glspl{vp} bearbeiteten drei Wiederholungen, die durch eine Pause von etwa einer Minute getrennt waren. Eine Wiederholung bestand aus zwei Schätzungen der $82\,\%$-$\log_{10}$-Er\-ken\-nungs\-schwel\-le pro Mustergrösse. Jede der vier Mustergrössen wurde innerhalb einer Schätzung $22$~Mal präsentiert. Gesamthaft bearbeiteten die \glspl{vp} somit $3 \times 2 \times 4 \times 22 = 528$ Durchgänge. 
Daraus resultierten für jede \gls{vp} 24 Schätzungen der $82\,\%$-$\log_{10}$-Er\-ken\-nungs\-schwel\-le (sechs pro Mustergrösse).
Die Mustergrössen wurde allen \glspl{vp} in einer pseudorandomisierten Abfolge präsentiert. Die Darbietungszeit der Mustergrössen betrug bei Start der Aufgabe $30$~ms und wurde für jede Mustergrösse einzeln über den gesamten Verlauf der $132$~Durchgänge adaptiv angepasst. 
Die Aufgabe dauerte etwa $25$~Minuten. 

Für jede \gls{vp} wurden die sechs pro Mustergrösse erhaltenen $82\,\%$-$\log_{10}$-Er\-ken\-nungs\-schwel\-len in eine Rangreihenfolge gebracht, die tiefste und höchste Schätzung entfernt und die restlichen vier $82\,\%$-$\log_{10}$-Er\-ken\-nungs\-schwel\-len gemittelt. Damit resultierte für jede \gls{vp} pro Mustergrösse ($1.8^{\circ}$, $3.6^{\circ}$, $5.4^{\circ}$ und $7.2^{\circ}$) eine $82\,\%$-$\log_{10}$-Er\-ken\-nungs\-schwel\-le für horizontale Bewegung. 
Alle Berechnungen wurden mit diesen $82\,\%$-$\log_{10}$-Er\-ken\-nungs\-schwel\-len getätigt. 
Ausnahme bildete die exponentielle Regression (siehe \autoref{sec:2Fragestellung}), bei welcher die vier $82\,\%$-$\log_{10}$-Er\-ken\-nungs\-schwel\-len auf Anraten von D. Tadin (persönl. Mitteilung, 11.02.2016) als Exponenten zur Basis 10 verrechnet und in dieser invertierten Form analysiert wurden.
Um die Interpretation der logarithmierten Werte zu erleichtern, wurden die Werte für die Ergebnisdarstellung (in Tabellen und Abbildungen) invertiert.
Der \gls{si} wurde gemäss der Vorgehensweise von \citet{Melnick2013} als Differenz zwischen der $82\,\%$-$\log_{10}$-Er\-ken\-nungs\-schwel\-le für die Mustergrösse $7.2^{\circ}$ und der $82\,\%$-$\log_{10}$-Er\-ken\-nungs\-schwel\-le für die Mustergrösse $1.8^{\circ}$ gebildet. 


\section{Die Hick-Aufgabe \label{sec:Hick}}

Angelehnt an die Versuchsanordnung von \citet{Rammsayer2007} wurde als Mass für \gls{ms} eine \gls{ha} eingesetzt.

\subsection{Apparatur und Material \label{sub:}}
Präsentiert wurde die Aufgabe auf dem in \autoref{sub:ssas} beschriebenen Computer, mit dem einzigen Unterschied, dass die Auflösung des Monitors für die \gls{ha} $1280 \times 1024$ Pixel betrug. Die Antworten der \glspl{vp} wurden mit einer Cedrus RB-830 Tastatur erfasst. 

Die Stimuli wurden mit E-Prime\textsuperscript{\textregistered} \citep{eprime} generiert. Die weissen Stimuli wurden auf einem schwarzen Hintergrund präsentiert, welcher eine Leuchtdichte von $2\,\textnormal{cd}/ \textnormal{m}^2$ aufwies. Der horizontale und vertikale Sehwinkel der verwendeten Rechtecke betrug $1.8^{\circ}$ respektive $1.5^{\circ}$. Die Rechtecke wurden auf dem Monitor zentriert dargeboten. Die Stimulianordnung der verwendeten Bedingungen sah folgendermassen aus (siehe \autoref{fig:hick_stimuli}):  In der $0$-Bit-Bedingung wurde ein Rechteck präsentiert. In der $1$-Bit-Bedingung wurden horizontal nebeneinander zwei Rechtecke präsentiert. Die beiden Rechtecke erschlossen zusammen einen horizontalen und vertikalen Sehwinkel von $4.5^{\circ}$ respektive $1.5^{\circ}$. In der $2$-Bit-Bedingung wurden in U-Form vier Rechtecke präsentiert. Die vier Rechtecke erschlossen gemeinsam einen horizontalen und vertikalen Sehwinkel von $7.5^{\circ}$ respektive $4.3^{\circ}$. In der $2.58$-Bit-Bedingung wurden zu den in U-Form angeordneten vier Rechtecken der $2$-Bit-Bedingung in der oberen Reihe je links und rechts ein Rechteck hinzugefügt. Die sechs Rechtecke erschlossen zusammen einen horizontalen und vertikalen Sehwinkel von $12.9^{\circ}$ respektive $4.3^{\circ}$. Der Sehwinkel des imperativen Reizes, einem \enquote{+}, betrug $0.5^{\circ}$ und wurde  immer in der Mitte eines Rechtecks präsentiert. Die Sehwinkel der Stimuli wurden mit einer Kinnstütze, die $61$ cm vom Monitor entfernt war, sichergestellt. Der verwendete Ton wies bei einer Frequenz von $1000$~Hz und einer Lautstärke von $70$~dB eine Länge von $200$~ms auf.

%\begin{figure}[htbp]
%	\centering
%	\subfloat[0-Bit]		[0-Bit-Bedingung]	{
%		\resizebox{.9\textwidth}{!}{
%			\begin{tikzpicture}
%			[scale=1, font=\sffamily, inner sep=0pt, baseline,
%			manifest/.style		= {draw, rectangle, thick, white, inner sep=0pt, minimum width=19mm, minimum height=16mm},
%			invisible/.style	= {draw, rectangle, thick, black!80, inner sep=0pt, minimum width=19mm, minimum height=16mm},
%			visual/.style 		= {draw, rectangle, thick, white, fill=white!100, minimum width= 10.65mm, minimum height=.5mm}]
%			
%			\node [invisible]	at (0,0)							(3)	{};
%			\node [invisible]	[right = 11mm of 3]	  				(4)	{};
%			\node [invisible]	[above left  = 15mm and -3mm of 3]	(2) {};
%			\node [invisible]	[above right = 15mm and -3mm of 4]	(5) {};
%			\node [invisible]	[left  = 11mm of 2]	  				(1)	{};
%			\node [invisible]	[right = 11mm of 5]	  				(6)	{};
%			
%			\node [manifest] at (1.5,1.5)							(9)	{\Huge $+$};
%			
%			\node [visual]		at (-5,-1)	{}							;
%			\node [white]		at (-5,-.6) {\Large$1^{\circ}$}		;
%			
%			\begin{scope}[on background layer]
%				\node [fill=black!80, inner sep= 20pt, fit=(1) (2) (3) (4) (5) (6)] {};
%			\end{scope}
%			\end{tikzpicture}
%}} \newline
%
%\subfloat[1-Bit]		[1-Bit-Bedingung]	{
%	\resizebox{.9\textwidth}{!}{
%		\begin{tikzpicture}
%		[scale=1, font=\sffamily, inner sep=0pt,
%		manifest/.style		= {draw, rectangle, thick, white, inner sep=0pt, minimum width=19mm, minimum height=16mm},
%		invisible/.style	= {draw, rectangle, thick, black!80, inner sep=0pt, minimum width=19mm, minimum height=16mm},
%		visual/.style 		= {draw, rectangle, thick, white, fill=white!100, minimum width= 10.65mm, minimum height=.5mm}]
%		
%		\node [invisible]	at (0,0)							(3)	{};
%		\node [invisible]	[right = 11mm of 3]	  				(4)	{};
%		\node [invisible]	[above left  = 15mm and -3mm of 3]	(2) {};
%		\node [invisible]	[above right = 15mm and -3mm of 4]	(5) {};
%		\node [invisible]	[left  = 11mm of 2]	  				(1)	{};
%		\node [invisible]	[right = 11mm of 5]	  				(6)	{};
%		
%		\node [manifest] at (0,1.5)				(9)		{\Huge $+$}	;
%		\node [manifest] [right = 11mm of 9] 	(10)	{}			;	
%		
%		\node [visual]		at (-5,-1)	{}							;
%		\node [white]		at (-5,-.6) {\Large$1^{\circ}$}		;
%		
%		\begin{scope}[on background layer]
%		\node [fill=black!80, inner sep= 20pt, fit=(1) (2) (3) (4) (5) (6)] {};
%		\end{scope}
%		\end{tikzpicture}
%	}} \newline
%	
%	\subfloat[2-Bit]		[2-Bit-Bedingung]	{
%		\resizebox{.9\textwidth}{!}{
%			\begin{tikzpicture}
%			[scale=1, font=\sffamily, inner sep=0pt,
%			manifest/.style		= {draw, rectangle, thick, white, inner sep=0pt, minimum width=19mm, minimum height=16mm},
%			invisible/.style	= {draw, rectangle, thick, black!80, inner sep=0pt, minimum width=19mm, minimum height=16mm},
%			visual/.style 		= {draw, rectangle, thick, white, fill=white!100, minimum width= 10.65mm, minimum height=.5mm}]
%			
%			\node [invisible]	at (0,0)							(3)	{};
%			\node [invisible]	[right = 11mm of 3]	  				(4)	{};
%			\node [invisible]	[above left  = 15mm and -3mm of 3]	(2) {};
%			\node [invisible]	[above right = 15mm and -3mm of 4]	(5) {};
%			\node [invisible]	[left  = 11mm of 2]	  				(1)	{};
%			\node [invisible]	[right = 11mm of 5]	  				(6)	{};
%			
%			\node [manifest]	at (0,0)							(3)	{};
%			\node [manifest]	[right = 11mm of 3]	  				(4)	{};
%			\node [manifest]	[above left  = 15mm and -3mm of 3]	(2) {\Huge $+$};
%			\node [manifest]	[above right = 15mm and -3mm of 4]	(5) {};	
%			
%			\node [visual]		at (-5,-1)	{}							;
%			\node [white]		at (-5,-.6) {\Large$1^{\circ}$}		;
%			
%			\begin{scope}[on background layer]
%			\node [fill=black!80, inner sep= 20pt, fit=(1) (2) (3) (4) (5) (6)] {};
%			\end{scope}
%			\end{tikzpicture}
%		}} \newline
%		
%		\subfloat[$2.58$-Bit]		[$2.58$-Bit-Bedingung]	{
%			\resizebox{.9\textwidth}{!}{
%				\begin{tikzpicture}
%				[scale=1, font=\sffamily, inner sep=0pt,
%				manifest/.style		= {draw, rectangle, thick, white, inner sep=0pt, minimum width=19mm, minimum height=16mm},
%				invisible/.style	= {draw, rectangle, thick, black!80, inner sep=0pt, minimum width=19mm, minimum height=16mm},
%				visual/.style 		= {draw, rectangle, thick, white, fill=white!100, minimum width= 10.65mm, minimum height=.5mm}]
%				
%				\node [invisible]	at (0,0)							(3)	{};
%				\node [invisible]	[right = 11mm of 3]	  				(4)	{};
%				\node [invisible]	[above left  = 15mm and -3mm of 3]	(2) {};
%				\node [invisible]	[above right = 15mm and -3mm of 4]	(5) {};
%				\node [invisible]	[left  = 11mm of 2]	  				(1)	{};
%				\node [invisible]	[right = 11mm of 5]	  				(6)	{};
%				
%				\node [manifest]	at (0,0)							(3)	{};
%				\node [manifest]	[right = 11mm of 3]	  				(4)	{};
%				\node [manifest]	[above left  = 15mm and -3mm of 3]	(2) {};
%				\node [manifest]	[above right = 15mm and -3mm of 4]	(5) {};
%				\node [manifest]	[left  = 11mm of 2]	  				(1)	{};
%				\node [manifest]	[right = 11mm of 5]	  				(6)	{\Huge $+$};
%				
%				\node [visual]		at (-5,-1)	{}							;
%				\node [white]		at (-5,-.6) {\Large$1^{\circ}$}		;
%				
%				\begin{scope}[on background layer]
%				\node [fill=black!80, inner sep= 20pt, fit=(1) (2) (3) (4) (5) (6)] {};
%				\end{scope}
%				\end{tikzpicture}
%			}} \newline
%			
%			\caption[Die Hick-Bedingungen]{Die vier Bedingungen $(a - d)$ der \gls{ha}. }
%			\label{fig:hick_stimuli}
%\end{figure}




\subsection{Versuchsablauf \label{subsec:hick_Versuchsablauf}}


In der $0$-Bit-Bedingung bearbeiteten die \glspl{vp} $32$ Durchgänge. Jeder Durchgang startete nach $1100$ ms mit der Präsentation eines Rechtecks. Nach einer variablen Zeitdauer, \gls{soa} genannt, welche $1000$, $1333$, $1666$ oder $2000$ ms betrug, wurde der imperative Reiz, ein \enquote{+}, eingeblendet. Die \glspl{vp} wurden angewiesen, mit dem Zeigefinger ihrer dominanten Hand so rasch als möglich auf die vorgesehene Antworttaste zu drücken. Bei einer Antwortabgabe nach Einblenden des imperativen Reizes folgte ein Ton. Bei einer Antwortabgabe vor Einblenden des imperativen Reizes folgte kein Ton. In beiden Fällen führte eine Antwortabgabe zur Ausblendung der Stimuli und zum Start des nächsten Durchganges.

Die $1$-Bit-Bedingung unterschied sich von der $0$-Bit-Bedingung in der Anzahl dargebotener Rechtecke und der Tonabgabe. Der imperative Reiz trat im linken oder im rechten Rechteck auf. Die \glspl{vp} erhielten die Anweisung, beim Auftreten des imperativen Reizes im linken Rechteck mit ihrem linken Zeigefinger und beim Auftreten des imperativen Reizes im rechten Rechteck mit ihrem rechten Zeigefinger so rasch als möglich auf die dem jeweiligen Finger zugewiesene Antworttaste zu drücken. Bei einer korrekten Antwortabgabe nach Einblendung des imperativen Reizes folgte ein Ton. Bei einer Antwortabgabe vor Einblendung des imperativen Reizes oder bei einer falschen Antwortabgabe folgte kein Ton.

Die $2$-Bit-Bedingung unterschied sich von der $1$-Bit-Bedingung lediglich in der Anzahl präsentierter Rechtecke. Der imperative Reiz trat entweder im oberen linken, unteren linken, oberen rechten oder unteren rechten Rechteck auf. Die \glspl{vp} wurden angewiesen, beim Auftreten des imperativen Reizes im oberen linken Rechteck mit ihrem linken Mittelfinger, beim Auftreten des imperativen Reizes im unteren linken Rechteck mit ihrem linken Zeigefinger,  beim Auftreten des imperativen Reizes im oberen rechten Rechteck mit ihrem rechten Mittelfinger und beim Auftreten des imperativen Reizes im unteren rechten Rechteck mit ihrem rechten Zeigefinger so rasch als möglich auf die dem jeweiligen Finger zugewiesene Antworttaste zu drücken.

Die $2.58$-Bit-Bedingung unterschied sich von der $2$-Bit-Bedingung nur in der Anzahl präsentierter Rechtecke. Der imperative Reiz trat entweder im oberen äusseren linken, oberen inneren linken, unteren linken, oberen äusseren rechten, oberen inneren rechten oder unteren rechten Rechteck auf. Die \glspl{vp} wurden angewiesen, beim Auftreten des imperativen Reizes im oberen äusseren linken Rechteck mit ihrem linken Ringfinger, beim Auftreten des imperativen Reizes im oberen inneren linken Rechteck mit ihrem linken Mittelfinger, beim Auftreten des imperativen Reizes im unteren linken Rechteck mit ihrem linken Zeigefinger, beim Auftreten des imperativen Reizes im oberen äusseren Rechteck mit ihrem rechten Ringfinger, beim Auftreten des imperativen Reizes oberen inneren rechten Rechteck mit ihrem rechten Mittelfinger und beim Auftreten des imperativen Reizes im unteren rechten Rechteck mit ihrem rechten Zeigefinger so rasch als möglich auf die dem jeweiligen Finger zugewiesene Antworttaste zu drücken.

Die Bedingungen wurden von allen \glspl{vp} in aufsteigender Reihenfolge ($0$-, $1$-, $2$-, $2.58$-Bit-Bedingung) bearbeitet. Jeder Bedingung gingen acht  Übungsdurchgänge voraus, damit sich die \glspl{vp} mit der Art der Stimuluspräsentation, der Antworteingabe und dem Ton vertraut machen konnten. 
Der imperative Reiz trat in der $1$-, $2$- und $2.58$-Bit-Bedingung für alle \glspl{vp} in einer pseudorandomisierten Abfolge mit der identischen, ausbalancierten \gls{soa} am identischen, über die $32$~Durchgänge der Bedingungen ausbalancierten Ort auf. Insgesamt dauerte die Aufgabe etwa $15$~Minuten. 

Pro Bedingung wurde für jede \gls{vp} der Mittelwert und die Standardabweichung aller korrekten Antworten bestimmt, die zwischen $100$ und $2500$~ms lagen. Basierend auf diesen Berechnungen wurden für jede \gls{vp} in jeder Bedingung diejenigen Durchgänge entfernt, welche eine \gls{rz} $\geq$ \gls{m} $+\,3\,\times$ \gls{sd} aufwiesen. Nach dieser intraindividuellen Ausreisserkontrolle wurden die verbliebenen Durchgänge innerhalb einer Bedingung gemittelt und für jede \gls{vp} als Leistungsmass der Bedingung der \gls{ha} verwendet.


\section{Erfassung der psychometrischen Intelligenz \label{sec:Erfassung_der_psychometrischen_Intelligenz}}

\glsunset{bist} % see http://tex.stackexchange.com/questions/30167/suppress-the-glossary-expansion-at-first-occurance


Psychometrische Intelligenz wurde mit einer modifizierten Kurzversion des \acrlong{bist} \citep[\gls{bist};][]{Jaeger1997} erfasst. Die fähigkeitstheoretische Grundlage des Tests ist das integrativ konzipierte bimodale und hierarchische \gls{bism} von \citet[][siehe \autoref{fig:bis_model}]{Jaeger1984}. 

\begin{figure}[htb]
	\centering
	\includegraphics[width=0.8\textwidth]{png/BIS}
	\caption[Das Berliner Intelligenzstrukturmodell]{Das Berliner Intelligenzstrukturmodell von \citet{Jaeger1984}.}
	\label{fig:bis_model}
\end{figure} 

Als integratives Modell ist das \gls{bism} zu bezeichnen, weil \citet{Jaeger1984} bei der Konstruktion des Modells versucht hat, die Vielfalt intellektueller Leistungsformen möglichst umfassend zu repräsentieren.
Bimodal ist das \gls{bism}, weil das Modell zwei Modalitäten aufweist, unter welchen Leistungen und Fähigkeiten klassifiziert werden können. 
Das \gls{bism} trennt dabei zwischen sogenannten Operationen und Inhalten. Innerhalb der Modalität Operationen werden die vier Fähigkeitsbündel Verarbeitungskapazität, Bearbeitungsgeschwindigkeit, Merkfähigkeit und Einfallsreichtum unterschieden. 
\gls{k} steht für die Fähigkeit, komplexe Informationen von Aufgaben zu verarbeiten, die nicht auf Anhieb zu lösen sind, sondern die erst durch vielfältiges Beziehungsstiften, formallogisch exaktes Denken und sachgerechtes Beurteilen von Informationen zu lösen sind. 
\gls{b} beschreibt das Arbeitstempo, die Auffassungsleichtigkeit und die Konzentrationskraft beim Lösen von einfach strukturierten Aufgaben mit geringem Schwierigkeitsgrad. 
\gls{M} spiegelt die Fähigkeit wider, sich etwas aktiv einzuprägen, etwas kurzfristig wieder zu erkennen oder zu reproduzieren. 
\gls{e} beschreibt die Fähigkeit, flexible Ideen zu produzieren und über vielfältige Vorstellungen von Problemen zu verfügen. 
Innerhalb der Modalität Inhalte lässt sich nach \citet{Jaeger1984} sprachgebundenes Denken von zahlengebundenem Denken und anschauungsgebundenem, figural-bildhaftem Denken unterscheiden.
\Gls{v} beschreibt den Grad der Aneignung und der Verfügbarkeit des Beziehungssystems Sprache.
\Gls{n} steht für das Ausmass der Aneignung und der Verfügbarkeit des Beziehungssystems Zahlen.
\Gls{f} spiegelt die Fähigkeit wider, Aufgabenmaterial zu verarbeiten, welches bildhaftes beziehungsweise räumliches Vorstellen erfordert.

Auf höchster Hierarchiestufe des \gls{bism} steht als Integral aller sieben Fähigkeiten (\gls{k}, \gls{b}, \gls{M}, \gls{e}, \gls{v}, \gls{n} und \gls{f}) die \gls{ai}. Die \gls{ai} und die Fähigkeiten unterscheiden sich aber lediglich im Differenzierungsgrad. \gls{ai} bildet Intelligenzleistungen gemäss \citet{Jaeger1984} aus grosser Distanz ab, während die sieben Fähigkeiten auf der Ebene darunter Intelligenzleistungen aus geringerer Distanz mit feinerem Auflösungsgrad abbilden. Untersuchungen zum \gls{bism} konnten die postulierte Struktur des \gls{bist} replizieren  und Zusammenhänge mit anderen Intelligenzmodellen wie denjenigen von \citet{Cattell1971}  oder von \citet{Carroll1993} herstellen \citep{Bucik1996, Beauducel2002, Suess2002}.

Die von \citet{Jaeger1997} vorgeschlagene Kurzversion des \gls{bist} enthält $15$ Subtests. Die Operationen \gls{b}, \gls{M} und \gls{e} werden darin mit je einem Subtest pro Inhalt erfasst, wobei \gls{k} mit zwei Subtests pro Inhalt erfasst wird. Bei der Modellierung der Daten mittels Strukturgleichungsmodellen hätte dies bei der vorliegenden Arbeit zu einer Überrepräsentation von \gls{k} im \gls{gfaktor} geführt. Um dies zu vermeiden, wurden die Operationen \gls{b} und \gls{M} um je einen Subtest pro Inhalt angereichert. Grundlage für die Auswahl der Subtests bildeten die Erkenntnisse von \citet{Wicki2014}, wobei bei der Entscheidung über die Aufnahme der Subtests ökonomische (Bearbeitungszeit der Subtests) und teststatistische (Trennschärfe und Reliabilität der Subtests)  Gesichtspunkte berücksichtigt wurden. Die Kurzversion von \citet{Jaeger1997} wurde mit folgenden Subtests ergänzt: Klassifizieren von Wörtern, Old English, Rechen-Zeichen, Wege-Erinnern, Worte Merken und Zweistellige Zahlen. 
\citet{Wicki2014} berichtet für diese modifizierte Kurzversion für die Operationen \gls{k}, \gls{b} und \gls{M} interne Konsistenzen von Cronbachs $\alpha=.61-.73$ und Konstruktreliabilitäten, gemessen mit \citeauthor{McDonald1999}'s \citeyearpar{McDonald1999} Omegakoeffizienten, von $\Omega = .58-.64$.
Auf Subtests der Operation \gls{e} wurde gänzlich verzichtet, weil zum einen unklar ist, wie Einfallsreichtum und Intelligenz zusammenhängen \citep{Kim2005} und zum anderen weil \citet{Jaeger1997} unbefriedigende Objektivitätswerte berichten. 
Alle eingesetzten Subtests, deren Beschreibung sowie Zuordnung zu den jeweiligen Operationen und Inhalte sind \autoref{tab:bis_subtest_description} zu entnehmen.

\begin{sidewaystable}
	\captionsetup{labelsep = none}
	\caption[Die verwendeten Subtests des \gls{bist}s]{\newline  \textit{Beschreibung und Reihenfolge der eingesetzten Subtests des \gls{bist}s} \vspace{.2cm}}
	\label{tab:bis_subtest_description}
	\begin{adjustbox}{width=\textwidth,totalheight=.9\textheight,keepaspectratio}
		\begin{threeparttable}
			\begin{tabular}{l l c c c c p{.0001cm} c c c p{20cm}}
				\hline
					&		&		& \multicolumn{3}{c}{Operation}	&	&	\multicolumn{3}{c}{Inhalt}	&		\\
				\cline{4-6}
				\cline{8-10}
				\multicolumn{1}{c}{Nr.}	&	\multicolumn{1}{c}{Name}	&	\multicolumn{1}{c}{Abkürzung}	&		K	&	B	&	M		&	&		V	&	N	&	F		&	\multicolumn{1}{c}{Beschreibung}	\\
				
				\hline
				1				&	Unvollständige Wörter*	&	UW			&&	\checkmark	&&&\checkmark&&& In vorgegebenen Wörtern fehlen einige Buchstaben, welche zu ergänzen sind (z.B. F\_scher)	\\
				2				&	Orientierungs-Gedächtnis	&	OG		&&&	\checkmark	&&&&\checkmark& Auf einem Stadtplanausschnitt markierte Gebäude müssen eingeprägt und unmittelbar danach wiedergegeben werden\\
				3				&	Zahlenreihen			&	ZN			&	\checkmark	&&&&&\checkmark&& Nach bestimmten Regeln aufgebaute Zahlenreihen sind um ein weiteres Glied zu ergänzen (z.B. 2 5 8 11 14 17 ?)\\
				4				&	Analogien				&	AN			&	\checkmark	&&&&&&\checkmark& Analogien mit Form $A:B=C:\,?$ müssen ergänzt werden, wobei die Analogien aus geometrischen Formen bestehen\\
				5				&	X-Grösser				&	XG			&&	\checkmark	&&&&\checkmark&& Zahlen, die um $3$ grösser sind als die unmittelbar vorangegangene Zahl müssen so schnell wie möglich durchgestrichen werden (z.B. 18 20 24 \cancel{27} 13 18 \cancel{21} \ldots)\\
				6				&	Wortanalogien			&	WA			&	\checkmark	&&&&\checkmark&&& Wortanalogien der Form \enquote{Huhn zu Küken} wie \enquote{Kuh zu ?} müssen vervollständigt werden\\
				7				&	Zahlenpaare				&	ZP			&&&	\checkmark	&&&\checkmark&& Zahlenpaare der Form 71 -- 918 sind einzuprägen. Das jeweils zweite Glied ist anschliessend unter vier Distraktoren zu identifizieren\\
				8				&	Tatsache-Meinung		&	TM			&	\checkmark	&&&&\checkmark&&& Sätze müssen daraufhin geprüft werden, ob sie eher eine Tatsache oder eher eine Meinung wiedergeben\\
				9				&	Buchstaben-Durchstreichen&	BD			&&	\checkmark	&&&&&\checkmark& Alle \enquote{x} müssen in Zeilen von Buchstaben durchgestrichen werden (z.B. sys\cancel{x}kdihj\cancel{x}\ldots)\\
				10				&	Schätzen				&	SC			&	\checkmark	&&&&&\checkmark&& Rechenaufgaben der Form $118492-3684-2106-4768=\,?$ müssen durch einfache rechnerische Überlegungen geschätzt bzw. gelöst werden\\
				11				&	Sinnvoller Text			&	ST			&&&	\checkmark	&&\checkmark&&& Verbale Detailangaben in einem Text sind einzuprägen und unmittelbar danach zu reproduzieren\\
				12				&	Charkow					&	CH			&	\checkmark	&&&&&&\checkmark& Eine Folge von Strichzeichnungen, die nach einer bestimmten Regel aufgebaut ist, ist um die beiden folgenden Glieder zu ergänzen\\
				13				&	Teil-Ganzes				&	TG			&&	\checkmark	&&&\checkmark&&& In Wortlisten sind zwei aufeinander folgende Wörter, die in der Beziehung Ganzes/zugehöriger Teil zueinander stehen zu markieren (z.B. Baum, \cancel{Blatt}, Stein, Haus, \cancel{Dach}, \ldots)\\
				14				&	Rechen--Zeichen			&	RZ			&&	\checkmark	&&&&\checkmark&& In  einfachen vorgegebenen Gleichungen stehen anstelle von Plus- oder Minuszeichen leere Kästchen. Die richtigen Rechenzeichen sind einzutragen\\
				15				&	Worte merken			&	WM			&&&	\checkmark	&&\checkmark&&& Eine Liste von Wörtern ist einzuprägen und unmittelbar danach in beliebiger Reihenfolge zu reproduzieren\\ 
				16				&	Klassifizieren von Wörtern&	KW			&&	\checkmark	&&&\checkmark&&& In Spalten von Wörtern sind alle Worte, die Pflanzen bezeichnen, durchzustreichen\\
				17				&	Zweistellige Zahlen		&	ZZ			&&&	\checkmark	&&&\checkmark&& Eine Reihe zweistelliger Zahlen ist einzuprägen und unmittelbar danach in beliebiger Reihenfolge zu reproduzieren\\
				18				&	Old English				&	OE			&&	\checkmark	&&&&&\checkmark& In Buchstabenreihen sind alle in einem vorgegebenen Schrifttyp gedruckten Buchstaben durchzustreichen\\
				19				&	Wege--Erinnern			&	WE			&&&	\checkmark	&&&&\checkmark& Ein in einem Stadtplanausschnitt eingezeichneter Weg ist einzuprägen und unmittelbar danach zu reproduzieren\\
				
				\hline
			\end{tabular}
			
			\begin{tablenotes}[flushleft]
				\footnotesize				% font size
				\setlength\labelsep{0pt}	% no indent on second line
				\item \textit{Anmerkungen.} K~=~Verarbeitungskapazität; B~=~Bearbeitungsgeschwindigkeit; M~=~Merkfähigkeit; V~=~verbal; N~=~numerisch; F~=~figural-bildhaft.\\
				{$^*$}Der Subtest UW wurde als Aufwärmaufgabe verwendet und floss nicht in die Auswertung mit ein.
			\end{tablenotes}
		\end{threeparttable}
	\end{adjustbox}
\end{sidewaystable}

Die $19$ Subtests wurden den \glspl{vp} nach der in \autoref{tab:bis_subtest_description} aufgeführten Reihenfolge vorgelegt und gemäss dem Manual des \gls{bist} instruiert. 
Die Bearbeitung der Subtests dauerte insgesamt $50$ Minuten.
Die Aufwärmaufgabe \gls{uw} wurde nicht ausgewertet. Die Rohwerte der restlichen $18$~Subtests wurden \textit{z}-standardisiert. 
Für die Beantwortung der Fragestellungen 1 und 2 wurden alle $18$~\textit{z}-stand\-ard\-isier\-ten Subtests gemittelt. Dadurch resultierte für jede \glspl{vp} ein \textit{z}-standardisiertes Mittel ihrer Leistung. 
Um für die Beantwortung der Fragestellungen 3, 4 und 5 einen \gls{gfaktor} zu bilden, wurden die $18$~\textit{z}-standardisierten Subtests innerhalb ihrer zugehörigen Operation gemittelt. Damit flossen in jede Operation (\gls{k}, \gls{b} und \gls{M}) zwei Subtests aus dem Bereich \gls{v}, zwei Subtests aus dem Bereich \gls{n} und zwei Subtests aus dem Bereich \gls{f} (insgesamt sechs Subtests) ein. Der \gls{gfaktor} wurde anschliessend aus den drei gemittelten \textit{z}-Werten der Operationen \gls{k}, \gls{b} und \gls{M} abgeleitet.


\section{Weitere Instrumente}

Im Rahmen der Untersuchung wurden den \glspl{vp} Fragebögen und weitere Com\-put\-er-Auf\-gaben zur Bearbeitung vorgelegt. Sie sind für die Fragestellungen dieser Arbeit nicht relevant und werden deshalb im folgenden Abschnitt nur kurz beschrieben.


\subsection{Fragebögen}

\subsubsection*{Persönliche Angaben}
Die Erfassung persönlicher Angaben fand in zwei Teilen statt. In einem ersten Teil machten die \glspl{vp} schriftlich Angaben zu ihrer Muttersprache, Seh- und Hörfähigkeit, ihren chronischen Krankheiten und ihrem Medikamenten- sowie Nikotinkonsum. In einem zweiten Teil machten sie computergestützt Angaben zu ihrem Alter, Geschlecht, Bildungsniveau, Koffeinkonsum,  Videospielhäufigkeit, Musikinstrumenterfahrung und Vertrautheit mit dem Zehnfingersystem beim Computerschreiben.


\subsubsection*{Kurzform der deutschen Übersetzung des revidierten \gls{epq-rk}}
Die \glspl{vp} haben  computergestützt die Kurzform der deutschen Übersetzung des \gls{epq-rk} von \citet{Ruch1999} bearbeitet. Der Fragebogen enthält insgesamt $50$~Fragen, darunter $14$~Items zur Erfassung von Psychotizismus, $12$~Items zur Erfassung von Extraversion, $12$~Items zur Erfassung von Neurotizismus und $12$~Items zur Erfassung der individuellen Neigung, sozial erwünschte Antworten abzugben.

\subsubsection*{Deutsche Übersetzung des \gls{dii}}
Die deutsche Übersetzung des \gls{dii} stammt von \citet{Kuhmann1996} und beinhaltet insgesamt $23$~Items, darunter $11$~Items zur Erfassung der funktionalen Impulsivität und  $12$~Items zur Erfassung der dysfunktionalen Impulsivität. Der Fragebogen wurde von den \glspl{vp} computergestützt bearbeitet.

\subsection{Zeitverarbeitungsaufgaben}


\subsubsection*{Zeitdauerdiskrimination im Millisekundenbereich mit gefüllten und leeren Intervallen}

Die \glspl{vp} bekamen über Lautsprecher hintereinander eine Standardtondauer und eine variable Vergleichstondauer dargeboten. Danach mussten die \glspl{vp} jeweils mit einem Tastendruck entscheiden, ob die erste oder die zweite Tondauer länger war. Bei einer korrekten Antwort verringerte sich die Differenz zwischen der Standard- und der Vergleichstondauer und bei einer falschen Antwort erhöhte sich diese Differenz. Die Aufgabe wurde einmal mit gefüllten Zeitintervallen (das heisst mit jeweils zwei kontinuierlichen Tönen) und einmal mit leeren Zeitintervallen (das heisst die Töne waren durch einen Klick am Anfang und einen Klick am Schluss des Intervalls gekennzeichnet) durchgeführt. Diese Aufgaben dauerte insgesamt etwa $15$ Minuten. Der Aufgabenaufbau war vergleichbar mit demjenigen von \citet{Stauffer2011}. 


\subsubsection*{Zeitgeneralisation im Millisekundenbereich}

Die Aufgabe der \glspl{vp} war es, in einer Lernphase die über Lautsprecher fünf Mal präsentierte Standardtonlänge einzuprägen. Danach folgte die eigentliche Aufgabe: Es wurden in zufälliger Reihenfolge die Standardtonlänge und sechs Vergleichstonlängen präsentiert. Die \glspl{vp} mussten nach jeder Tonlänge mit einem Tastendruck entscheiden, ob die präsentierte Tonlänge von gleicher Länge war wie die Standardtonlänge oder nicht. Diese Aufgabe dauerte insgesamt etwa $5$ Minuten \citep[siehe][]{Stauffer2011}.

\subsubsection*{Rhythmuswahrnehmung}

Die \glspl{vp} hatten die Aufgabe, sechs über Lautsprecher in unregelmässigen Abständen präsentierte Töne von jeweils $3$~ms Dauer auf rhythmische Darbietung hin zu beurteilen. 
Gaben die \glspl{vp} an, den Rhythmus als regelmässig wahrgenommen zu haben, wurde die Abweichung des Interstimulusintervalls beim nächsten Durchgang erhöht. Gaben die \glspl{vp} an, den Rhythmus als unregelmässig wahrgenommen zu haben, wurde die Abweichung des Interstimulusintervalls beim nächsten Durchgang verringert.
Die Aufgabe dauerte insgesamt etwa $5$ Minuten \citep[siehe][]{Stauffer2011}.

\subsection{Inspection-Time-Aufgabe}

Die auf einem Computermonitor präsentierten Stimuli der \gls{ita} \citep{Vickers1972} bestanden aus zwei ungleich langen vertikalen Linien, die an ihren oberen Enden mit einer horizontalen Linie verbunden waren. Bei jedem Durchgang wurde die kürzere vertikale Linie zufällig links oder rechts präsentiert und nach der Darbietungszeit mit einer Pi-förmigen Abbildung, die gleich lange vertikale Linien aufwies, maskiert. Die Aufgabe der \glspl{vp} bestand darin anzugeben, ob die linke oder die rechte vertikale Linie länger war. Eine korrekte Antwort verringerte und eine falsche Antwort erhöhte die Darbietungszeit des nächsten Stimulus. Die Aufgabe dauerte insgesamt etwa $5$ Minuten.


\section{Untersuchungsablauf \label{sec:Versuchsablauf}}

Die Untersuchung wurde vor Datenerhebungsbeginn von der Ethikkomission der philosophisch-humanwissenschaftlichen Fakultät der Universität Bern gutgeheissen. Die \glspl{vp} nahmen an zwei Sitzungen teil, welche $2$ bis $14$~Tage voneinander getrennt waren. Zwei \glspl{vp} hatten krankheitsbedingt ein längeres Intervall zwischen den beiden Sitzungen ($18$ und $30$ Tage).

\subsection{Sitzung 1}

Die \glspl{vp} wurden in Gruppen von zwei bis sechs Personen in einem $18\,\textnormal{m}^2$ grossen Raum an Einzeltische gesetzt. Die Tische waren so weit voneinander entfernt, dass die \glspl{vp} nicht durch den Nachbarn gestört werden oder abschreiben konnten. 
Ohne die Fragestellungen der Arbeit zu offenbaren, klärte der Versuchsleiter\footnote{In dieser Arbeit wird der Einfachheit halber nur die männliche Form verwendet. Die weibliche Form ist selbstverständlich immer mit eingeschlossen.} die \glspl{vp} über den Zweck der Untersuchung auf, informierte sie über den Ablauf der bevorstehenden Sitzung und nahm die Einverständniserklärungen der \glspl{vp} entgegen. Danach wurden der Reihenfolge nach folgende Daten erhoben und Instrumente eingesetzt:

\begin{enumerate}
	\item Persönliche Angaben Teil 1
	\item \acrshort{bist}
	\item Persönliche Angaben Teil 2
	\item \gls{epq-rk}
	\item \gls{dii}
\end{enumerate}

\noindent Diese erste Sitzung dauerte insgesamt etwa 90 Minuten.

\subsection{Sitzung 2}
Die zweite Sitzung fand als Einzeltestung in einer $5\,\textnormal{m}^2$ grossen, schallgedämpften Kabine statt. 
Der Versuchsleiter informierte die \glspl{vp} über den Ablauf der bevorstehenden Sitzung und legte ihnen am Computer der Reihenfolge nach folgende Aufgaben vor:

\begin{enumerate}
	\item	\gls{ssauf}
	\item	Die fünf Aufgaben
			\begin{dinglist}{43}
				\item \gls{ha}
				\item Zeitdauerdiskrimination im Millisekundenbereich mit gefüllten Intervallen
				\item Zeitdauerdiskrimination im Millisekundenbereich mit leeren Intervallen
				\item Zeitgeneralisation im Millisekundenbereich
				\item Rhythmuswahrnehmung
			\end{dinglist}
			wurden über alle \glspl{vp} hinweg vollständig permutiert, was in $5\,! = 120$ unterschiedlichen Reihenfolgen resultierte. Nach $120$~\glspl{vp} wurden die Reihenfolgen wiederholt, das heisst  \gls{vp} $121$ bearbeitete die Aufgaben in der gleichen Reihenfolge wie \gls{vp} 1, \gls{vp} $122$ bearbeitete die Aufgaben in der gleichen Reihenfolge wie \gls{vp} 2 und so weiter.
	\item	\gls{ita}
\end{enumerate}

Nach der letzten Aufgabe wurden die \glspl{vp} vollständig über das Ziel der Untersuchung aufgeklärt und entlöhnt. Diese zweite Sitzung dauerte inklusive einer fünfminütigen Pause nach 50 Minuten insgesamt etwa 120 Minuten.



%\clearpage
\section{Statistische Analyse \label{sec:StatistischeAnalyse}}

Alle Berechnungen wurden in R \citep{R} durchgeführt, dessen Basisfunktionen mit folgenden Paketen ergänzt wurde:
{coin} \citep{coin},
%colorspace \citep{colorspace},
dplyr \citep{dplyr},
effsize \citep{effsize},
ez \citep{ez},
ggplot2 \citep{ggplot2},
lavaan \citep{lavaan},
lm.beta \citep{lm.beta},
lmSupport \citep{lmSupport},
MASS \citep{MASS},
Metrics \citep{Metrics},		% für RMSE. Muss aufgeführt werden.
multcomp \citep{multcomp},
nlme \citep{nlme},
nlstools \citep{nlstools},
%nortest \citep{nortest},		% wird nicht mehr benötigt, weil parametrisch gerechnet wird. Shapiro-Wilk-Test in Tabelle sollte genügen.
pacman \citep{pacman},
pbapply \citep{pbapply},
plotrix \citep{plotrix},
ppcor \citep{ppcor},
psych \citep{psych},
readxl \citep{readxl},
reshape2 \citep{reshape2},
rprime \citep{rprime},
R.matlab \citep{R.matlab} und
semPlot \citep{semPlot}.
Als Editor diente RStudio \citep{RStudio}.

Die Fragestellungen 3, 4 und 5 werden mittels konfirmatorischer Faktorenanalysen beantwortet. Die Güte einer konfirmatorischen Faktorenanalyse kann anhand einer Vielzahl von unterschiedlichen Kennwerten beurteilt werden, weshalb hier die für diese Arbeit wichtigen Kennwerte kurz vorgestellt werden.










\subsubsection*{\gls{cst}}

Der \gls{cst} ist ein Modelltest, der angibt, wie stark sich die theoretische, vom Modell implizierte Var\-ianz-Ko\-var\-ianz\-ma\-trix von der empirischen Var\-ianz-Ko\-var\-ianz\-ma\-trix unterscheidet \citep{Kline2011}. Die dafür berechnete Teststatistik folgt in grossen Stichproben und unter der Voraussetzung der multivariaten Normalverteilung einer zentralen Chi Quad\-rat-Ver\-teil\-ung und wird deshalb auch als $\upchi^2_{m}$ bezeichnet. Die Freiheitsgrade für den $\upchi^2$-Test ergeben sich aus den Freiheitsgraden des zu testenden Modells ($df_{m}$). Wenn $\upchi^2_{m}=0$ ist, stimmt die empirische Var\-ianz-Ko\-var\-ianz\-ma\-trix mit der vom Modell implizierten Varianz-Kovarianzmatrix ohne Abweichung überein und die empirischen Daten passen perfekt zum theoretischen Modell. Bildet das Modell die Daten nicht gut ab, wird $\upchi^2_{m}>0$. Liegt $\upchi^2_{m}$ über dem kritischen $\upchi^2_{df}$, sind die Abweichungen zwischen der empirischen und der theoretischen Varianz-Kovarianzmatrix grösser als durch den Stichprobenfehler erwartet, und die Nullhypothese wird verworfen. Wenn ein korrekt spezifiziertes Modell mit mehreren Zufallsstichproben geprüft wird, liegt der Erwartungswert von $\upchi^2_{m}$ bei $df_{m}$ und $\upchi^2_{m}$ würde bei einem $\upalpha$-Fehler von $5\,\%$ bei 19 von 20 Stichproben im nicht-signifikanten Bereich liegen.

Bei der Bewertung der berichteten konfirmatorischen Faktorenanalysen wurde das Ergebnis des Modelltests (im Vergleich zu den weiter unten beschriebenen Kennwerten) am stärksten gewichtet. Diese Art der Modellbeurteilung entspricht der Vorstellung von Jöreskog \citep[1985, zitiert nach][S. 1620]{McIntosh2012}, der sich dafür aussprach alle andere Kennwerte weniger zu gewichten \citep[siehe auch][]{Hayduk2007}.

\subsubsection*{\gls{cfi}}
Der \gls{cfi} lässt sich der Klasse der inkrementellen Fit Indizes zuordnen und wurde von \citet{Bentler1990} entworfen. Die Formel lautet

$$ \textnormal{CFI} = 1 - \frac{\upchi^2_{m}-df_{m}}{\upchi^2_{b}-df_{b}} $$

\noindent Im Zähler wird $df_{m}$ von $\upchi^2_{m}$ subtrahiert. Im Nenner des Bruchs wird die gleiche Differenz mit den Werten des Baseline Modells ($df_{b}$ und $\upchi^2_{b}$) gebildet.
Das Baseline-Modell nimmt keinerlei Zusammenhänge zwischen den manifesten Variablen an und wird deshalb auch als \enquote{independence model} bezeichnet. Zieht man den beschriebenen Quotienten von Eins ab, ergibt sich ein Mass für die relative Verbesserung des angenommenen Modells gegenüber dem Baseline-Modell. Aus der Formel folgt, dass \gls{cfi} $= 1$ ergibt, wenn $\upchi^2_{m} \leq df_{m}$ ist. Das bedeutet aber auch, dass ein \gls{cfi} von Eins nicht mit einem perfekten Fit ($\upchi^2_{m} = 0$) gleichzusetzen ist. Ein \gls{cfi} von $.95$ ist laut \citet{Hu1999} als guter Fit zu bezeichnen.

\subsubsection*{\gls{rmsea}}
Die Anzahl Freiheitsgrade eines Modells geben an, auf wie vielen Dimensionen die empirischen Daten vom Modell abweichen können. Der RMSEA \citep{Steiger1990} ist ein Fit Index, der die durchschnittliche Abweichung des Modells pro mögliche Dimension der Abweichung angibt. Die Formel lautet

$$ \textnormal{RMSEA} = \sqrt{ \frac{\upchi^2_{m}-df_{m}}{df_{m}(N-1)} } $$

\noindent Wie beim \gls{cfi} ergibt sich der beste Wert, wenn $\upchi^2_{m} \leq df_{m}$ ist (dann ist \gls{rmsea} $= 0$). Das bedeutet jedoch wie beim \gls{cfi} auch, dass ein \gls{rmsea} von Null keinen perfekten Modell-Fit ($\upchi^2_{m} = 0$) ergibt. Im Nenner wird $df_{m}$ mit der Stichprobengrösse minus Eins multipliziert. Dies führt dazu, dass der \gls{rmsea} bei Modellen mit vielen Freiheitsgraden und grossen Stichproben kleiner wird. Ein \gls{rmsea} $\leq.08$ deutet laut \citet{Browne1993} auf einen guten Modell-Fit hin.

\subsubsection*{\gls{srmr}}
Das \gls{srmr} ist ein Mass dafür, wie hoch die durchschnittlichen Korrelationsresiduen der manifesten Variablen sind \citep{Kline2011}. Anders formuliert gibt das \gls{srmr} den durchschnittlichen Zusammenhang der manifesten Variablen wieder, welcher nicht durch das Modell erklärt werden kann. Das \gls{srmr} sollte möglichst nahe bei Null zu liegen kommen, was bedeutet, dass das theoretische Modell die empirische Var\-ianz-Ko\-var\-ianz\-ma\-trix angemessen abbildet. Gemäss \citet{Hu1999} kann ein \gls{srmr} $\leq.08$ als guter Modell-Fit interpretiert werden.











% =================================================================
% R E S U L T S
% =================================================================
\chapter{Resultate \label{cha:Resultate}}

\section{Deskriptiv- und Inferenzstatistik \label{sec:Deskriptive_Statistik}}

\subsection{Spatial-Suppression-Aufgabe \label{subsec:SSres}}

Mittelwerte, Verteilungsangaben und die Reliabilitäten der Bedingungen sind in \autoref{tab:spatial_suppression_descriptives} abgetragen. 
Die Splithalf-Reliabilitäten der vier Bedingungen fielen mit $r_{tt}=.96$ sehr hoch aus und bestätigten die von \citet{Melnick2013} berichteten Reliabilitäten.
Die Streudiagramme der $82\,\%$-Er\-ken\-nungs\-schwel\-len sind in \autoref{fig:spatial_suppression_scatterplot} zu sehen.

\begin{table}[b]
	\centering
%	\captionsetup{font = small}
	\caption[Deskriptive Angaben zu den $82\,\%$-Erkennungsschwellen in der \gls{ssauf}]{\newline \textit{Deskriptive Angaben zu den $82\,\%$-Erkennungsschwellen der \gls{ssauf} in Millisekunden (Mittelwert, Standardabweichung, Minimum, Maximum) sowie Kennwerte zur Verteilungsform und der Reliabilität der Daten} \vspace{.2cm}}
	\label{tab:spatial_suppression_descriptives}
	\begin{adjustbox}{width=1\textwidth}
		\begin{threeparttable}
			\begin{tabular}{
					l
					S[table-format = 3.0]
					S[table-format = 2.0]
					S[table-format = 2.0]
					S[table-format = 3.0]
					S[table-format = 1.2]
					S[table-format = 1.2]
					S[table-format = <0.3, add-integer-zero=false]
					S[table-format = 0.2, add-integer-zero=false]
				}
				\hline
				\multicolumn{1}{c}{Bedingung} 		&	{\textit{M}}	&	\textit{SD}	&	{Min}	&	Max 	&	\textnormal{Schiefe}	&	\textnormal{Kurtosis}  &{S-W \textit{p}-Wert}& {$r_{tt}$}\\
				\hline
				$1.8^{\circ}$	&	82			&	28			&	31		&	216		&	-0.25	&	0.19	& 		.39		&	.96	\\
				$3.6^{\circ}$	&	89			&	31			&	37		&	282		&	0.02	&	0.80	& 		.03		&	.96	\\
				$5.4^{\circ}$	&	109			&	40			&	45		&	422		&	0.73	&	1.78	& 		<.001	&	.96	\\
				$7.2^{\circ}$	&	136			&	60			&	61		&	705		&	1.14	&	1.86	& 		<.001	&	.96	\\
				\hline
			\end{tabular}%
			%}
			\begin{tablenotes}[flushleft]
				\footnotesize				% font size
				\setlength\labelsep{0pt}	% no indent on second line
				\item \textit{Anmerkungen.} Min~=~Minimum; Max~=~Maximum; S-W~=~Shapiro-Wilk-Test; $r_{tt}$~=~nach der Odd-Even-Methode berechnete, mit der Spearman-Brown-Formel \citep[Spearman 1910; Brown 1910; zitiert nach][S. 123]{Schermelleh-Engel2007} korrigierte Splithalf-Reliabilität.
			\end{tablenotes}
		\end{threeparttable}
	\end{adjustbox}
\end{table}


\begin{figure}[p]
	\centering
	\begin{adjustbox}{width=1\textwidth} 
		% Created by tikzDevice version 0.10.1 on 2016-06-28 13:44:24
% !TEX encoding = UTF-8 Unicode
\begin{tikzpicture}[x=1pt,y=1pt]
\definecolor{fillColor}{RGB}{255,255,255}
\path[use as bounding box,fill=fillColor,fill opacity=0.00] (0,0) rectangle (505.89,505.89);
\begin{scope}
\path[clip] ( 40.39, 35.64) rectangle (121.72,494.80);
\definecolor{drawColor}{RGB}{0,0,0}
\definecolor{fillColor}{RGB}{0,0,0}

\path[draw=drawColor,line width= 0.4pt,line join=round,line cap=round,fill=fillColor] ( 43.83,131.60) circle (  0.99);

\path[draw=drawColor,line width= 0.4pt,line join=round,line cap=round,fill=fillColor] ( 44.26,133.42) circle (  0.99);

\path[draw=drawColor,line width= 0.4pt,line join=round,line cap=round,fill=fillColor] ( 44.68,104.27) circle (  0.99);

\path[draw=drawColor,line width= 0.4pt,line join=round,line cap=round,fill=fillColor] ( 45.11,131.60) circle (  0.99);

\path[draw=drawColor,line width= 0.4pt,line join=round,line cap=round,fill=fillColor] ( 45.53,119.46) circle (  0.99);

\path[draw=drawColor,line width= 0.4pt,line join=round,line cap=round,fill=fillColor] ( 45.96, 93.34) circle (  0.99);

\path[draw=drawColor,line width= 0.4pt,line join=round,line cap=round,fill=fillColor] ( 46.38,113.38) circle (  0.99);

\path[draw=drawColor,line width= 0.4pt,line join=round,line cap=round,fill=fillColor] ( 46.81,115.20) circle (  0.99);

\path[draw=drawColor,line width= 0.4pt,line join=round,line cap=round,fill=fillColor] ( 47.23,121.28) circle (  0.99);

\path[draw=drawColor,line width= 0.4pt,line join=round,line cap=round,fill=fillColor] ( 47.66, 97.59) circle (  0.99);

\path[draw=drawColor,line width= 0.4pt,line join=round,line cap=round,fill=fillColor] ( 48.08,126.74) circle (  0.99);

\path[draw=drawColor,line width= 0.4pt,line join=round,line cap=round,fill=fillColor] ( 48.51,114.60) circle (  0.99);

\path[draw=drawColor,line width= 0.4pt,line join=round,line cap=round,fill=fillColor] ( 48.93,115.20) circle (  0.99);

\path[draw=drawColor,line width= 0.4pt,line join=round,line cap=round,fill=fillColor] ( 49.36,104.88) circle (  0.99);

\path[draw=drawColor,line width= 0.4pt,line join=round,line cap=round,fill=fillColor] ( 49.79,120.06) circle (  0.99);

\path[draw=drawColor,line width= 0.4pt,line join=round,line cap=round,fill=fillColor] ( 50.21, 95.77) circle (  0.99);

\path[draw=drawColor,line width= 0.4pt,line join=round,line cap=round,fill=fillColor] ( 50.64, 84.84) circle (  0.99);

\path[draw=drawColor,line width= 0.4pt,line join=round,line cap=round,fill=fillColor] ( 51.06,120.06) circle (  0.99);

\path[draw=drawColor,line width= 0.4pt,line join=round,line cap=round,fill=fillColor] ( 51.49,117.03) circle (  0.99);

\path[draw=drawColor,line width= 0.4pt,line join=round,line cap=round,fill=fillColor] ( 51.91, 87.87) circle (  0.99);

\path[draw=drawColor,line width= 0.4pt,line join=round,line cap=round,fill=fillColor] ( 52.34,146.18) circle (  0.99);

\path[draw=drawColor,line width= 0.4pt,line join=round,line cap=round,fill=fillColor] ( 52.76,120.06) circle (  0.99);

\path[draw=drawColor,line width= 0.4pt,line join=round,line cap=round,fill=fillColor] ( 53.19, 91.52) circle (  0.99);

\path[draw=drawColor,line width= 0.4pt,line join=round,line cap=round,fill=fillColor] ( 53.61, 95.77) circle (  0.99);

\path[draw=drawColor,line width= 0.4pt,line join=round,line cap=round,fill=fillColor] ( 54.04, 90.91) circle (  0.99);

\path[draw=drawColor,line width= 0.4pt,line join=round,line cap=round,fill=fillColor] ( 54.47,104.88) circle (  0.99);

\path[draw=drawColor,line width= 0.4pt,line join=round,line cap=round,fill=fillColor] ( 54.89,112.17) circle (  0.99);

\path[draw=drawColor,line width= 0.4pt,line join=round,line cap=round,fill=fillColor] ( 55.32,109.74) circle (  0.99);

\path[draw=drawColor,line width= 0.4pt,line join=round,line cap=round,fill=fillColor] ( 55.74,100.02) circle (  0.99);

\path[draw=drawColor,line width= 0.4pt,line join=round,line cap=round,fill=fillColor] ( 56.17, 90.91) circle (  0.99);

\path[draw=drawColor,line width= 0.4pt,line join=round,line cap=round,fill=fillColor] ( 56.59, 85.44) circle (  0.99);

\path[draw=drawColor,line width= 0.4pt,line join=round,line cap=round,fill=fillColor] ( 57.02,121.28) circle (  0.99);

\path[draw=drawColor,line width= 0.4pt,line join=round,line cap=round,fill=fillColor] ( 57.44,101.23) circle (  0.99);

\path[draw=drawColor,line width= 0.4pt,line join=round,line cap=round,fill=fillColor] ( 57.87,107.31) circle (  0.99);

\path[draw=drawColor,line width= 0.4pt,line join=round,line cap=round,fill=fillColor] ( 58.29,100.63) circle (  0.99);

\path[draw=drawColor,line width= 0.4pt,line join=round,line cap=round,fill=fillColor] ( 58.72, 95.16) circle (  0.99);

\path[draw=drawColor,line width= 0.4pt,line join=round,line cap=round,fill=fillColor] ( 59.15, 84.23) circle (  0.99);

\path[draw=drawColor,line width= 0.4pt,line join=round,line cap=round,fill=fillColor] ( 59.57,101.23) circle (  0.99);

\path[draw=drawColor,line width= 0.4pt,line join=round,line cap=round,fill=fillColor] ( 60.00,120.06) circle (  0.99);

\path[draw=drawColor,line width= 0.4pt,line join=round,line cap=round,fill=fillColor] ( 60.42,131.00) circle (  0.99);

\path[draw=drawColor,line width= 0.4pt,line join=round,line cap=round,fill=fillColor] ( 60.85,130.39) circle (  0.99);

\path[draw=drawColor,line width= 0.4pt,line join=round,line cap=round,fill=fillColor] ( 61.27,103.06) circle (  0.99);

\path[draw=drawColor,line width= 0.4pt,line join=round,line cap=round,fill=fillColor] ( 61.70,129.17) circle (  0.99);

\path[draw=drawColor,line width= 0.4pt,line join=round,line cap=round,fill=fillColor] ( 62.12,133.42) circle (  0.99);

\path[draw=drawColor,line width= 0.4pt,line join=round,line cap=round,fill=fillColor] ( 62.55, 72.08) circle (  0.99);

\path[draw=drawColor,line width= 0.4pt,line join=round,line cap=round,fill=fillColor] ( 62.97,100.02) circle (  0.99);

\path[draw=drawColor,line width= 0.4pt,line join=round,line cap=round,fill=fillColor] ( 63.40,131.00) circle (  0.99);

\path[draw=drawColor,line width= 0.4pt,line join=round,line cap=round,fill=fillColor] ( 63.83, 83.62) circle (  0.99);

\path[draw=drawColor,line width= 0.4pt,line join=round,line cap=round,fill=fillColor] ( 64.25,113.38) circle (  0.99);

\path[draw=drawColor,line width= 0.4pt,line join=round,line cap=round,fill=fillColor] ( 64.68, 90.30) circle (  0.99);

\path[draw=drawColor,line width= 0.4pt,line join=round,line cap=round,fill=fillColor] ( 65.10, 79.98) circle (  0.99);

\path[draw=drawColor,line width= 0.4pt,line join=round,line cap=round,fill=fillColor] ( 65.53,106.09) circle (  0.99);

\path[draw=drawColor,line width= 0.4pt,line join=round,line cap=round,fill=fillColor] ( 65.95,132.21) circle (  0.99);

\path[draw=drawColor,line width= 0.4pt,line join=round,line cap=round,fill=fillColor] ( 66.38, 90.91) circle (  0.99);

\path[draw=drawColor,line width= 0.4pt,line join=round,line cap=round,fill=fillColor] ( 66.80,109.13) circle (  0.99);

\path[draw=drawColor,line width= 0.4pt,line join=round,line cap=round,fill=fillColor] ( 67.23,106.09) circle (  0.99);

\path[draw=drawColor,line width= 0.4pt,line join=round,line cap=round,fill=fillColor] ( 67.65,100.63) circle (  0.99);

\path[draw=drawColor,line width= 0.4pt,line join=round,line cap=round,fill=fillColor] ( 68.08,103.66) circle (  0.99);

\path[draw=drawColor,line width= 0.4pt,line join=round,line cap=round,fill=fillColor] ( 68.51, 76.33) circle (  0.99);

\path[draw=drawColor,line width= 0.4pt,line join=round,line cap=round,fill=fillColor] ( 68.93, 81.80) circle (  0.99);

\path[draw=drawColor,line width= 0.4pt,line join=round,line cap=round,fill=fillColor] ( 69.36,104.27) circle (  0.99);

\path[draw=drawColor,line width= 0.4pt,line join=round,line cap=round,fill=fillColor] ( 69.78,146.79) circle (  0.99);

\path[draw=drawColor,line width= 0.4pt,line join=round,line cap=round,fill=fillColor] ( 70.21,104.88) circle (  0.99);

\path[draw=drawColor,line width= 0.4pt,line join=round,line cap=round,fill=fillColor] ( 70.63,121.88) circle (  0.99);

\path[draw=drawColor,line width= 0.4pt,line join=round,line cap=round,fill=fillColor] ( 71.06,117.63) circle (  0.99);

\path[draw=drawColor,line width= 0.4pt,line join=round,line cap=round,fill=fillColor] ( 71.48, 95.16) circle (  0.99);

\path[draw=drawColor,line width= 0.4pt,line join=round,line cap=round,fill=fillColor] ( 71.91,127.96) circle (  0.99);

\path[draw=drawColor,line width= 0.4pt,line join=round,line cap=round,fill=fillColor] ( 72.33,117.03) circle (  0.99);

\path[draw=drawColor,line width= 0.4pt,line join=round,line cap=round,fill=fillColor] ( 72.76,112.77) circle (  0.99);

\path[draw=drawColor,line width= 0.4pt,line join=round,line cap=round,fill=fillColor] ( 73.19,111.56) circle (  0.99);

\path[draw=drawColor,line width= 0.4pt,line join=round,line cap=round,fill=fillColor] ( 73.61, 91.52) circle (  0.99);

\path[draw=drawColor,line width= 0.4pt,line join=round,line cap=round,fill=fillColor] ( 74.04,102.45) circle (  0.99);

\path[draw=drawColor,line width= 0.4pt,line join=round,line cap=round,fill=fillColor] ( 74.46,108.52) circle (  0.99);

\path[draw=drawColor,line width= 0.4pt,line join=round,line cap=round,fill=fillColor] ( 74.89,100.63) circle (  0.99);

\path[draw=drawColor,line width= 0.4pt,line join=round,line cap=round,fill=fillColor] ( 75.31, 94.55) circle (  0.99);

\path[draw=drawColor,line width= 0.4pt,line join=round,line cap=round,fill=fillColor] ( 75.74,114.60) circle (  0.99);

\path[draw=drawColor,line width= 0.4pt,line join=round,line cap=round,fill=fillColor] ( 76.16,117.03) circle (  0.99);

\path[draw=drawColor,line width= 0.4pt,line join=round,line cap=round,fill=fillColor] ( 76.59, 88.48) circle (  0.99);

\path[draw=drawColor,line width= 0.4pt,line join=round,line cap=round,fill=fillColor] ( 77.01,117.63) circle (  0.99);

\path[draw=drawColor,line width= 0.4pt,line join=round,line cap=round,fill=fillColor] ( 77.44,113.99) circle (  0.99);

\path[draw=drawColor,line width= 0.4pt,line join=round,line cap=round,fill=fillColor] ( 77.87,117.03) circle (  0.99);

\path[draw=drawColor,line width= 0.4pt,line join=round,line cap=round,fill=fillColor] ( 78.29, 99.41) circle (  0.99);

\path[draw=drawColor,line width= 0.4pt,line join=round,line cap=round,fill=fillColor] ( 78.72,104.27) circle (  0.99);

\path[draw=drawColor,line width= 0.4pt,line join=round,line cap=round,fill=fillColor] ( 79.14,114.60) circle (  0.99);

\path[draw=drawColor,line width= 0.4pt,line join=round,line cap=round,fill=fillColor] ( 79.57,124.31) circle (  0.99);

\path[draw=drawColor,line width= 0.4pt,line join=round,line cap=round,fill=fillColor] ( 79.99, 99.41) circle (  0.99);

\path[draw=drawColor,line width= 0.4pt,line join=round,line cap=round,fill=fillColor] ( 80.42, 99.41) circle (  0.99);

\path[draw=drawColor,line width= 0.4pt,line join=round,line cap=round,fill=fillColor] ( 80.84, 90.30) circle (  0.99);

\path[draw=drawColor,line width= 0.4pt,line join=round,line cap=round,fill=fillColor] ( 81.27, 81.80) circle (  0.99);

\path[draw=drawColor,line width= 0.4pt,line join=round,line cap=round,fill=fillColor] ( 81.69, 92.73) circle (  0.99);

\path[draw=drawColor,line width= 0.4pt,line join=round,line cap=round,fill=fillColor] ( 82.12, 98.81) circle (  0.99);

\path[draw=drawColor,line width= 0.4pt,line join=round,line cap=round,fill=fillColor] ( 82.55,112.77) circle (  0.99);

\path[draw=drawColor,line width= 0.4pt,line join=round,line cap=round,fill=fillColor] ( 82.97, 87.27) circle (  0.99);

\path[draw=drawColor,line width= 0.4pt,line join=round,line cap=round,fill=fillColor] ( 83.40, 84.23) circle (  0.99);

\path[draw=drawColor,line width= 0.4pt,line join=round,line cap=round,fill=fillColor] ( 83.82,110.34) circle (  0.99);

\path[draw=drawColor,line width= 0.4pt,line join=round,line cap=round,fill=fillColor] ( 84.25,112.17) circle (  0.99);

\path[draw=drawColor,line width= 0.4pt,line join=round,line cap=round,fill=fillColor] ( 84.67,135.25) circle (  0.99);

\path[draw=drawColor,line width= 0.4pt,line join=round,line cap=round,fill=fillColor] ( 85.10,104.27) circle (  0.99);

\path[draw=drawColor,line width= 0.4pt,line join=round,line cap=round,fill=fillColor] ( 85.52,102.45) circle (  0.99);

\path[draw=drawColor,line width= 0.4pt,line join=round,line cap=round,fill=fillColor] ( 85.95, 72.69) circle (  0.99);

\path[draw=drawColor,line width= 0.4pt,line join=round,line cap=round,fill=fillColor] ( 86.37, 90.91) circle (  0.99);

\path[draw=drawColor,line width= 0.4pt,line join=round,line cap=round,fill=fillColor] ( 86.80, 92.73) circle (  0.99);

\path[draw=drawColor,line width= 0.4pt,line join=round,line cap=round,fill=fillColor] ( 87.23, 92.12) circle (  0.99);

\path[draw=drawColor,line width= 0.4pt,line join=round,line cap=round,fill=fillColor] ( 87.65,103.66) circle (  0.99);

\path[draw=drawColor,line width= 0.4pt,line join=round,line cap=round,fill=fillColor] ( 88.08, 87.87) circle (  0.99);

\path[draw=drawColor,line width= 0.4pt,line join=round,line cap=round,fill=fillColor] ( 88.50, 99.41) circle (  0.99);

\path[draw=drawColor,line width= 0.4pt,line join=round,line cap=round,fill=fillColor] ( 88.93, 76.33) circle (  0.99);

\path[draw=drawColor,line width= 0.4pt,line join=round,line cap=round,fill=fillColor] ( 89.35,135.25) circle (  0.99);

\path[draw=drawColor,line width= 0.4pt,line join=round,line cap=round,fill=fillColor] ( 89.78,120.06) circle (  0.99);

\path[draw=drawColor,line width= 0.4pt,line join=round,line cap=round,fill=fillColor] ( 90.20,101.23) circle (  0.99);

\path[draw=drawColor,line width= 0.4pt,line join=round,line cap=round,fill=fillColor] ( 90.63, 83.62) circle (  0.99);

\path[draw=drawColor,line width= 0.4pt,line join=round,line cap=round,fill=fillColor] ( 91.05,117.03) circle (  0.99);

\path[draw=drawColor,line width= 0.4pt,line join=round,line cap=round,fill=fillColor] ( 91.48,120.06) circle (  0.99);

\path[draw=drawColor,line width= 0.4pt,line join=round,line cap=round,fill=fillColor] ( 91.91,106.70) circle (  0.99);

\path[draw=drawColor,line width= 0.4pt,line join=round,line cap=round,fill=fillColor] ( 92.33, 94.55) circle (  0.99);

\path[draw=drawColor,line width= 0.4pt,line join=round,line cap=round,fill=fillColor] ( 92.76, 83.01) circle (  0.99);

\path[draw=drawColor,line width= 0.4pt,line join=round,line cap=round,fill=fillColor] ( 93.18, 82.41) circle (  0.99);

\path[draw=drawColor,line width= 0.4pt,line join=round,line cap=round,fill=fillColor] ( 93.61, 99.41) circle (  0.99);

\path[draw=drawColor,line width= 0.4pt,line join=round,line cap=round,fill=fillColor] ( 94.03,107.31) circle (  0.99);

\path[draw=drawColor,line width= 0.4pt,line join=round,line cap=round,fill=fillColor] ( 94.46,102.45) circle (  0.99);

\path[draw=drawColor,line width= 0.4pt,line join=round,line cap=round,fill=fillColor] ( 94.88, 93.95) circle (  0.99);

\path[draw=drawColor,line width= 0.4pt,line join=round,line cap=round,fill=fillColor] ( 95.31, 89.09) circle (  0.99);

\path[draw=drawColor,line width= 0.4pt,line join=round,line cap=round,fill=fillColor] ( 95.73,116.42) circle (  0.99);

\path[draw=drawColor,line width= 0.4pt,line join=round,line cap=round,fill=fillColor] ( 96.16, 87.87) circle (  0.99);

\path[draw=drawColor,line width= 0.4pt,line join=round,line cap=round,fill=fillColor] ( 96.59,143.14) circle (  0.99);

\path[draw=drawColor,line width= 0.4pt,line join=round,line cap=round,fill=fillColor] ( 97.01, 92.73) circle (  0.99);

\path[draw=drawColor,line width= 0.4pt,line join=round,line cap=round,fill=fillColor] ( 97.44,145.57) circle (  0.99);

\path[draw=drawColor,line width= 0.4pt,line join=round,line cap=round,fill=fillColor] ( 97.86,115.81) circle (  0.99);

\path[draw=drawColor,line width= 0.4pt,line join=round,line cap=round,fill=fillColor] ( 98.29, 80.58) circle (  0.99);

\path[draw=drawColor,line width= 0.4pt,line join=round,line cap=round,fill=fillColor] ( 98.71, 93.34) circle (  0.99);

\path[draw=drawColor,line width= 0.4pt,line join=round,line cap=round,fill=fillColor] ( 99.14,119.46) circle (  0.99);

\path[draw=drawColor,line width= 0.4pt,line join=round,line cap=round,fill=fillColor] ( 99.56,103.66) circle (  0.99);

\path[draw=drawColor,line width= 0.4pt,line join=round,line cap=round,fill=fillColor] ( 99.99,101.23) circle (  0.99);

\path[draw=drawColor,line width= 0.4pt,line join=round,line cap=round,fill=fillColor] (100.41,100.63) circle (  0.99);

\path[draw=drawColor,line width= 0.4pt,line join=round,line cap=round,fill=fillColor] (100.84,101.84) circle (  0.99);

\path[draw=drawColor,line width= 0.4pt,line join=round,line cap=round,fill=fillColor] (101.26,107.31) circle (  0.99);

\path[draw=drawColor,line width= 0.4pt,line join=round,line cap=round,fill=fillColor] (101.69,116.42) circle (  0.99);

\path[draw=drawColor,line width= 0.4pt,line join=round,line cap=round,fill=fillColor] (102.12, 80.58) circle (  0.99);

\path[draw=drawColor,line width= 0.4pt,line join=round,line cap=round,fill=fillColor] (102.54, 99.41) circle (  0.99);

\path[draw=drawColor,line width= 0.4pt,line join=round,line cap=round,fill=fillColor] (102.97, 96.98) circle (  0.99);

\path[draw=drawColor,line width= 0.4pt,line join=round,line cap=round,fill=fillColor] (103.39,110.34) circle (  0.99);

\path[draw=drawColor,line width= 0.4pt,line join=round,line cap=round,fill=fillColor] (103.82,103.66) circle (  0.99);

\path[draw=drawColor,line width= 0.4pt,line join=round,line cap=round,fill=fillColor] (104.24,150.43) circle (  0.99);

\path[draw=drawColor,line width= 0.4pt,line join=round,line cap=round,fill=fillColor] (104.67,121.88) circle (  0.99);

\path[draw=drawColor,line width= 0.4pt,line join=round,line cap=round,fill=fillColor] (105.09,183.84) circle (  0.99);

\path[draw=drawColor,line width= 0.4pt,line join=round,line cap=round,fill=fillColor] (105.52,105.49) circle (  0.99);

\path[draw=drawColor,line width= 0.4pt,line join=round,line cap=round,fill=fillColor] (105.94, 93.95) circle (  0.99);

\path[draw=drawColor,line width= 0.4pt,line join=round,line cap=round,fill=fillColor] (106.37,117.03) circle (  0.99);

\path[draw=drawColor,line width= 0.4pt,line join=round,line cap=round,fill=fillColor] (106.80,101.84) circle (  0.99);

\path[draw=drawColor,line width= 0.4pt,line join=round,line cap=round,fill=fillColor] (107.22, 95.16) circle (  0.99);

\path[draw=drawColor,line width= 0.4pt,line join=round,line cap=round,fill=fillColor] (107.65,129.17) circle (  0.99);

\path[draw=drawColor,line width= 0.4pt,line join=round,line cap=round,fill=fillColor] (108.07,101.84) circle (  0.99);

\path[draw=drawColor,line width= 0.4pt,line join=round,line cap=round,fill=fillColor] (108.50, 95.77) circle (  0.99);

\path[draw=drawColor,line width= 0.4pt,line join=round,line cap=round,fill=fillColor] (108.92, 90.30) circle (  0.99);

\path[draw=drawColor,line width= 0.4pt,line join=round,line cap=round,fill=fillColor] (109.35, 89.09) circle (  0.99);

\path[draw=drawColor,line width= 0.4pt,line join=round,line cap=round,fill=fillColor] (109.77, 96.98) circle (  0.99);

\path[draw=drawColor,line width= 0.4pt,line join=round,line cap=round,fill=fillColor] (110.20, 79.98) circle (  0.99);

\path[draw=drawColor,line width= 0.4pt,line join=round,line cap=round,fill=fillColor] (110.62,106.09) circle (  0.99);

\path[draw=drawColor,line width= 0.4pt,line join=round,line cap=round,fill=fillColor] (111.05,106.09) circle (  0.99);

\path[draw=drawColor,line width= 0.4pt,line join=round,line cap=round,fill=fillColor] (111.48, 83.01) circle (  0.99);

\path[draw=drawColor,line width= 0.4pt,line join=round,line cap=round,fill=fillColor] (111.90, 95.16) circle (  0.99);

\path[draw=drawColor,line width= 0.4pt,line join=round,line cap=round,fill=fillColor] (112.33, 71.47) circle (  0.99);

\path[draw=drawColor,line width= 0.4pt,line join=round,line cap=round,fill=fillColor] (112.75, 96.98) circle (  0.99);

\path[draw=drawColor,line width= 0.4pt,line join=round,line cap=round,fill=fillColor] (113.18,120.67) circle (  0.99);

\path[draw=drawColor,line width= 0.4pt,line join=round,line cap=round,fill=fillColor] (113.60,126.14) circle (  0.99);

\path[draw=drawColor,line width= 0.4pt,line join=round,line cap=round,fill=fillColor] (114.03,103.06) circle (  0.99);

\path[draw=drawColor,line width= 0.4pt,line join=round,line cap=round,fill=fillColor] (114.45, 95.16) circle (  0.99);

\path[draw=drawColor,line width= 0.4pt,line join=round,line cap=round,fill=fillColor] (114.88,112.77) circle (  0.99);

\path[draw=drawColor,line width= 0.4pt,line join=round,line cap=round,fill=fillColor] (115.30, 93.34) circle (  0.99);

\path[draw=drawColor,line width= 0.4pt,line join=round,line cap=round,fill=fillColor] (115.73, 87.87) circle (  0.99);

\path[draw=drawColor,line width= 0.4pt,line join=round,line cap=round,fill=fillColor] (116.16, 90.91) circle (  0.99);

\path[draw=drawColor,line width= 0.4pt,line join=round,line cap=round,fill=fillColor] (116.58, 87.87) circle (  0.99);

\path[draw=drawColor,line width= 0.4pt,line join=round,line cap=round,fill=fillColor] (117.01,101.23) circle (  0.99);

\path[draw=drawColor,line width= 0.4pt,line join=round,line cap=round,fill=fillColor] (117.43, 93.95) circle (  0.99);

\path[draw=drawColor,line width= 0.4pt,line join=round,line cap=round,fill=fillColor] (117.86, 90.91) circle (  0.99);

\path[draw=drawColor,line width= 0.4pt,line join=round,line cap=round,fill=fillColor] (118.28, 95.16) circle (  0.99);

\path[draw=drawColor,line width= 0.4pt,line join=round,line cap=round,fill=fillColor] (118.71, 84.84) circle (  0.99);
\end{scope}
\begin{scope}
\path[clip] (  0.00,  0.00) rectangle (126.47,505.89);
\definecolor{drawColor}{RGB}{0,0,0}

\node[text=drawColor,anchor=base,inner sep=0pt, outer sep=0pt, scale=  1.32] at ( 81.06,495.79) {\bfseries \textsf{1.8}$^\circ$};

\node[text=drawColor,anchor=base,inner sep=0pt, outer sep=0pt, scale=  1.32] at ( 81.06,  5.54) {Vp};
\end{scope}
\begin{scope}
\path[clip] (  0.00,  0.00) rectangle (505.89,505.89);
\definecolor{drawColor}{RGB}{0,0,0}

\path[draw=drawColor,line width= 0.4pt,line join=round,line cap=round] ( 43.83, 35.64) -- (118.71, 35.64);

\path[draw=drawColor,line width= 0.4pt,line join=round,line cap=round] ( 43.83, 35.64) -- ( 43.83, 31.68);

\path[draw=drawColor,line width= 0.4pt,line join=round,line cap=round] ( 68.93, 35.64) -- ( 68.93, 31.68);

\path[draw=drawColor,line width= 0.4pt,line join=round,line cap=round] ( 94.46, 35.64) -- ( 94.46, 31.68);

\path[draw=drawColor,line width= 0.4pt,line join=round,line cap=round] (118.71, 35.64) -- (118.71, 31.68);

\node[text=drawColor,anchor=base,inner sep=0pt, outer sep=0pt, scale=  0.99] at ( 43.83, 21.38) {1};

\node[text=drawColor,anchor=base,inner sep=0pt, outer sep=0pt, scale=  0.99] at ( 68.93, 21.38) {60};

\node[text=drawColor,anchor=base,inner sep=0pt, outer sep=0pt, scale=  0.99] at ( 94.46, 21.38) {120};

\node[text=drawColor,anchor=base,inner sep=0pt, outer sep=0pt, scale=  0.99] at (118.71, 21.38) {177};

\path[draw=drawColor,line width= 0.4pt,line join=round,line cap=round] ( 40.39, 52.65) -- ( 40.39,477.80);

\path[draw=drawColor,line width= 0.4pt,line join=round,line cap=round] ( 40.39, 52.65) -- ( 36.43, 52.65);

\path[draw=drawColor,line width= 0.4pt,line join=round,line cap=round] ( 40.39,113.38) -- ( 36.43,113.38);

\path[draw=drawColor,line width= 0.4pt,line join=round,line cap=round] ( 40.39,174.12) -- ( 36.43,174.12);

\path[draw=drawColor,line width= 0.4pt,line join=round,line cap=round] ( 40.39,234.85) -- ( 36.43,234.85);

\path[draw=drawColor,line width= 0.4pt,line join=round,line cap=round] ( 40.39,295.59) -- ( 36.43,295.59);

\path[draw=drawColor,line width= 0.4pt,line join=round,line cap=round] ( 40.39,356.32) -- ( 36.43,356.32);

\path[draw=drawColor,line width= 0.4pt,line join=round,line cap=round] ( 40.39,417.06) -- ( 36.43,417.06);

\path[draw=drawColor,line width= 0.4pt,line join=round,line cap=round] ( 40.39,477.80) -- ( 36.43,477.80);

\node[text=drawColor,anchor=base east,inner sep=0pt, outer sep=0pt, scale=  0.99] at ( 32.47, 49.24) {0};

\node[text=drawColor,anchor=base east,inner sep=0pt, outer sep=0pt, scale=  0.99] at ( 32.47,109.97) {100};

\node[text=drawColor,anchor=base east,inner sep=0pt, outer sep=0pt, scale=  0.99] at ( 32.47,170.71) {200};

\node[text=drawColor,anchor=base east,inner sep=0pt, outer sep=0pt, scale=  0.99] at ( 32.47,231.44) {300};

\node[text=drawColor,anchor=base east,inner sep=0pt, outer sep=0pt, scale=  0.99] at ( 32.47,292.18) {400};

\node[text=drawColor,anchor=base east,inner sep=0pt, outer sep=0pt, scale=  0.99] at ( 32.47,352.92) {500};

\node[text=drawColor,anchor=base east,inner sep=0pt, outer sep=0pt, scale=  0.99] at ( 32.47,413.65) {600};

\node[text=drawColor,anchor=base east,inner sep=0pt, outer sep=0pt, scale=  0.99] at ( 32.47,474.39) {700};
\end{scope}
\begin{scope}
\path[clip] ( 40.39, 35.64) rectangle (121.72,494.80);
\definecolor{drawColor}{RGB}{0,0,0}

\path[draw=drawColor,line width= 0.4pt,line join=round,line cap=round] ( 40.39,102.24) -- (121.72,102.24);
\end{scope}
\begin{scope}
\path[clip] (  0.00,  0.00) rectangle (126.47,505.89);
\definecolor{drawColor}{RGB}{0,0,0}

\node[text=drawColor,rotate= 90.00,anchor=base,inner sep=0pt, outer sep=0pt, scale=  1.32] at ( 11.09,265.22) {Schwellensch{"a}tzungen 82 \% korrekt (ms)};
\end{scope}
\begin{scope}
\path[clip] (166.86, 35.64) rectangle (248.19,494.80);
\definecolor{drawColor}{RGB}{0,0,0}
\definecolor{fillColor}{RGB}{0,0,0}

\path[draw=drawColor,line width= 0.4pt,line join=round,line cap=round,fill=fillColor] (170.30,155.29) circle (  0.99);

\path[draw=drawColor,line width= 0.4pt,line join=round,line cap=round,fill=fillColor] (170.73,144.96) circle (  0.99);

\path[draw=drawColor,line width= 0.4pt,line join=round,line cap=round,fill=fillColor] (171.15,110.95) circle (  0.99);

\path[draw=drawColor,line width= 0.4pt,line join=round,line cap=round,fill=fillColor] (171.58,130.39) circle (  0.99);

\path[draw=drawColor,line width= 0.4pt,line join=round,line cap=round,fill=fillColor] (172.00,123.71) circle (  0.99);

\path[draw=drawColor,line width= 0.4pt,line join=round,line cap=round,fill=fillColor] (172.43,101.23) circle (  0.99);

\path[draw=drawColor,line width= 0.4pt,line join=round,line cap=round,fill=fillColor] (172.85,127.35) circle (  0.99);

\path[draw=drawColor,line width= 0.4pt,line join=round,line cap=round,fill=fillColor] (173.28,123.71) circle (  0.99);

\path[draw=drawColor,line width= 0.4pt,line join=round,line cap=round,fill=fillColor] (173.71,120.67) circle (  0.99);

\path[draw=drawColor,line width= 0.4pt,line join=round,line cap=round,fill=fillColor] (174.13,104.27) circle (  0.99);

\path[draw=drawColor,line width= 0.4pt,line join=round,line cap=round,fill=fillColor] (174.56,144.36) circle (  0.99);

\path[draw=drawColor,line width= 0.4pt,line join=round,line cap=round,fill=fillColor] (174.98,112.17) circle (  0.99);

\path[draw=drawColor,line width= 0.4pt,line join=round,line cap=round,fill=fillColor] (175.41,127.35) circle (  0.99);

\path[draw=drawColor,line width= 0.4pt,line join=round,line cap=round,fill=fillColor] (175.83,120.67) circle (  0.99);

\path[draw=drawColor,line width= 0.4pt,line join=round,line cap=round,fill=fillColor] (176.26,115.81) circle (  0.99);

\path[draw=drawColor,line width= 0.4pt,line join=round,line cap=round,fill=fillColor] (176.68,100.02) circle (  0.99);

\path[draw=drawColor,line width= 0.4pt,line join=round,line cap=round,fill=fillColor] (177.11, 86.66) circle (  0.99);

\path[draw=drawColor,line width= 0.4pt,line join=round,line cap=round,fill=fillColor] (177.53,121.28) circle (  0.99);

\path[draw=drawColor,line width= 0.4pt,line join=round,line cap=round,fill=fillColor] (177.96,120.06) circle (  0.99);

\path[draw=drawColor,line width= 0.4pt,line join=round,line cap=round,fill=fillColor] (178.39, 95.77) circle (  0.99);

\path[draw=drawColor,line width= 0.4pt,line join=round,line cap=round,fill=fillColor] (178.81,145.57) circle (  0.99);

\path[draw=drawColor,line width= 0.4pt,line join=round,line cap=round,fill=fillColor] (179.24,114.60) circle (  0.99);

\path[draw=drawColor,line width= 0.4pt,line join=round,line cap=round,fill=fillColor] (179.66, 89.09) circle (  0.99);

\path[draw=drawColor,line width= 0.4pt,line join=round,line cap=round,fill=fillColor] (180.09,104.27) circle (  0.99);

\path[draw=drawColor,line width= 0.4pt,line join=round,line cap=round,fill=fillColor] (180.51, 85.44) circle (  0.99);

\path[draw=drawColor,line width= 0.4pt,line join=round,line cap=round,fill=fillColor] (180.94,120.06) circle (  0.99);

\path[draw=drawColor,line width= 0.4pt,line join=round,line cap=round,fill=fillColor] (181.36,118.24) circle (  0.99);

\path[draw=drawColor,line width= 0.4pt,line join=round,line cap=round,fill=fillColor] (181.79,110.34) circle (  0.99);

\path[draw=drawColor,line width= 0.4pt,line join=round,line cap=round,fill=fillColor] (182.21,104.88) circle (  0.99);

\path[draw=drawColor,line width= 0.4pt,line join=round,line cap=round,fill=fillColor] (182.64,101.23) circle (  0.99);

\path[draw=drawColor,line width= 0.4pt,line join=round,line cap=round,fill=fillColor] (183.07, 88.48) circle (  0.99);

\path[draw=drawColor,line width= 0.4pt,line join=round,line cap=round,fill=fillColor] (183.49,103.06) circle (  0.99);

\path[draw=drawColor,line width= 0.4pt,line join=round,line cap=round,fill=fillColor] (183.92,110.34) circle (  0.99);

\path[draw=drawColor,line width= 0.4pt,line join=round,line cap=round,fill=fillColor] (184.34,118.24) circle (  0.99);

\path[draw=drawColor,line width= 0.4pt,line join=round,line cap=round,fill=fillColor] (184.77,113.99) circle (  0.99);

\path[draw=drawColor,line width= 0.4pt,line join=round,line cap=round,fill=fillColor] (185.19, 93.34) circle (  0.99);

\path[draw=drawColor,line width= 0.4pt,line join=round,line cap=round,fill=fillColor] (185.62, 79.98) circle (  0.99);

\path[draw=drawColor,line width= 0.4pt,line join=round,line cap=round,fill=fillColor] (186.04,105.49) circle (  0.99);

\path[draw=drawColor,line width= 0.4pt,line join=round,line cap=round,fill=fillColor] (186.47,117.63) circle (  0.99);

\path[draw=drawColor,line width= 0.4pt,line join=round,line cap=round,fill=fillColor] (186.89,128.57) circle (  0.99);

\path[draw=drawColor,line width= 0.4pt,line join=round,line cap=round,fill=fillColor] (187.32,121.88) circle (  0.99);

\path[draw=drawColor,line width= 0.4pt,line join=round,line cap=round,fill=fillColor] (187.75,124.92) circle (  0.99);

\path[draw=drawColor,line width= 0.4pt,line join=round,line cap=round,fill=fillColor] (188.17,120.67) circle (  0.99);

\path[draw=drawColor,line width= 0.4pt,line join=round,line cap=round,fill=fillColor] (188.60,123.10) circle (  0.99);

\path[draw=drawColor,line width= 0.4pt,line join=round,line cap=round,fill=fillColor] (189.02, 75.12) circle (  0.99);

\path[draw=drawColor,line width= 0.4pt,line join=round,line cap=round,fill=fillColor] (189.45,103.06) circle (  0.99);

\path[draw=drawColor,line width= 0.4pt,line join=round,line cap=round,fill=fillColor] (189.87,145.57) circle (  0.99);

\path[draw=drawColor,line width= 0.4pt,line join=round,line cap=round,fill=fillColor] (190.30, 87.27) circle (  0.99);

\path[draw=drawColor,line width= 0.4pt,line join=round,line cap=round,fill=fillColor] (190.72,122.49) circle (  0.99);

\path[draw=drawColor,line width= 0.4pt,line join=round,line cap=round,fill=fillColor] (191.15, 92.73) circle (  0.99);

\path[draw=drawColor,line width= 0.4pt,line join=round,line cap=round,fill=fillColor] (191.57, 89.09) circle (  0.99);

\path[draw=drawColor,line width= 0.4pt,line join=round,line cap=round,fill=fillColor] (192.00,113.99) circle (  0.99);

\path[draw=drawColor,line width= 0.4pt,line join=round,line cap=round,fill=fillColor] (192.43,132.82) circle (  0.99);

\path[draw=drawColor,line width= 0.4pt,line join=round,line cap=round,fill=fillColor] (192.85,107.92) circle (  0.99);

\path[draw=drawColor,line width= 0.4pt,line join=round,line cap=round,fill=fillColor] (193.28,120.06) circle (  0.99);

\path[draw=drawColor,line width= 0.4pt,line join=round,line cap=round,fill=fillColor] (193.70,126.14) circle (  0.99);

\path[draw=drawColor,line width= 0.4pt,line join=round,line cap=round,fill=fillColor] (194.13, 95.16) circle (  0.99);

\path[draw=drawColor,line width= 0.4pt,line join=round,line cap=round,fill=fillColor] (194.55,113.99) circle (  0.99);

\path[draw=drawColor,line width= 0.4pt,line join=round,line cap=round,fill=fillColor] (194.98, 75.73) circle (  0.99);

\path[draw=drawColor,line width= 0.4pt,line join=round,line cap=round,fill=fillColor] (195.40, 99.41) circle (  0.99);

\path[draw=drawColor,line width= 0.4pt,line join=round,line cap=round,fill=fillColor] (195.83,121.88) circle (  0.99);

\path[draw=drawColor,line width= 0.4pt,line join=round,line cap=round,fill=fillColor] (196.25,138.89) circle (  0.99);

\path[draw=drawColor,line width= 0.4pt,line join=round,line cap=round,fill=fillColor] (196.68,100.63) circle (  0.99);

\path[draw=drawColor,line width= 0.4pt,line join=round,line cap=round,fill=fillColor] (197.11,120.06) circle (  0.99);

\path[draw=drawColor,line width= 0.4pt,line join=round,line cap=round,fill=fillColor] (197.53,124.92) circle (  0.99);

\path[draw=drawColor,line width= 0.4pt,line join=round,line cap=round,fill=fillColor] (197.96, 92.73) circle (  0.99);

\path[draw=drawColor,line width= 0.4pt,line join=round,line cap=round,fill=fillColor] (198.38,145.57) circle (  0.99);

\path[draw=drawColor,line width= 0.4pt,line join=round,line cap=round,fill=fillColor] (198.81,111.56) circle (  0.99);

\path[draw=drawColor,line width= 0.4pt,line join=round,line cap=round,fill=fillColor] (199.23,110.34) circle (  0.99);

\path[draw=drawColor,line width= 0.4pt,line join=round,line cap=round,fill=fillColor] (199.66,113.38) circle (  0.99);

\path[draw=drawColor,line width= 0.4pt,line join=round,line cap=round,fill=fillColor] (200.08, 96.98) circle (  0.99);

\path[draw=drawColor,line width= 0.4pt,line join=round,line cap=round,fill=fillColor] (200.51,106.09) circle (  0.99);

\path[draw=drawColor,line width= 0.4pt,line join=round,line cap=round,fill=fillColor] (200.93,101.23) circle (  0.99);

\path[draw=drawColor,line width= 0.4pt,line join=round,line cap=round,fill=fillColor] (201.36, 98.81) circle (  0.99);

\path[draw=drawColor,line width= 0.4pt,line join=round,line cap=round,fill=fillColor] (201.79,105.49) circle (  0.99);

\path[draw=drawColor,line width= 0.4pt,line join=round,line cap=round,fill=fillColor] (202.21,123.10) circle (  0.99);

\path[draw=drawColor,line width= 0.4pt,line join=round,line cap=round,fill=fillColor] (202.64,110.95) circle (  0.99);

\path[draw=drawColor,line width= 0.4pt,line join=round,line cap=round,fill=fillColor] (203.06,101.84) circle (  0.99);

\path[draw=drawColor,line width= 0.4pt,line join=round,line cap=round,fill=fillColor] (203.49,115.20) circle (  0.99);

\path[draw=drawColor,line width= 0.4pt,line join=round,line cap=round,fill=fillColor] (203.91,110.95) circle (  0.99);

\path[draw=drawColor,line width= 0.4pt,line join=round,line cap=round,fill=fillColor] (204.34,106.09) circle (  0.99);

\path[draw=drawColor,line width= 0.4pt,line join=round,line cap=round,fill=fillColor] (204.76,112.17) circle (  0.99);

\path[draw=drawColor,line width= 0.4pt,line join=round,line cap=round,fill=fillColor] (205.19,103.06) circle (  0.99);

\path[draw=drawColor,line width= 0.4pt,line join=round,line cap=round,fill=fillColor] (205.61,126.14) circle (  0.99);

\path[draw=drawColor,line width= 0.4pt,line join=round,line cap=round,fill=fillColor] (206.04,113.99) circle (  0.99);

\path[draw=drawColor,line width= 0.4pt,line join=round,line cap=round,fill=fillColor] (206.47, 99.41) circle (  0.99);

\path[draw=drawColor,line width= 0.4pt,line join=round,line cap=round,fill=fillColor] (206.89,106.09) circle (  0.99);

\path[draw=drawColor,line width= 0.4pt,line join=round,line cap=round,fill=fillColor] (207.32, 96.98) circle (  0.99);

\path[draw=drawColor,line width= 0.4pt,line join=round,line cap=round,fill=fillColor] (207.74, 93.95) circle (  0.99);

\path[draw=drawColor,line width= 0.4pt,line join=round,line cap=round,fill=fillColor] (208.17, 98.81) circle (  0.99);

\path[draw=drawColor,line width= 0.4pt,line join=round,line cap=round,fill=fillColor] (208.59, 92.12) circle (  0.99);

\path[draw=drawColor,line width= 0.4pt,line join=round,line cap=round,fill=fillColor] (209.02,108.52) circle (  0.99);

\path[draw=drawColor,line width= 0.4pt,line join=round,line cap=round,fill=fillColor] (209.44,106.09) circle (  0.99);

\path[draw=drawColor,line width= 0.4pt,line join=round,line cap=round,fill=fillColor] (209.87, 81.80) circle (  0.99);

\path[draw=drawColor,line width= 0.4pt,line join=round,line cap=round,fill=fillColor] (210.29,118.24) circle (  0.99);

\path[draw=drawColor,line width= 0.4pt,line join=round,line cap=round,fill=fillColor] (210.72,126.14) circle (  0.99);

\path[draw=drawColor,line width= 0.4pt,line join=round,line cap=round,fill=fillColor] (211.15,117.63) circle (  0.99);

\path[draw=drawColor,line width= 0.4pt,line join=round,line cap=round,fill=fillColor] (211.57,104.27) circle (  0.99);

\path[draw=drawColor,line width= 0.4pt,line join=round,line cap=round,fill=fillColor] (212.00,107.31) circle (  0.99);

\path[draw=drawColor,line width= 0.4pt,line join=round,line cap=round,fill=fillColor] (212.42, 81.80) circle (  0.99);

\path[draw=drawColor,line width= 0.4pt,line join=round,line cap=round,fill=fillColor] (212.85,103.66) circle (  0.99);

\path[draw=drawColor,line width= 0.4pt,line join=round,line cap=round,fill=fillColor] (213.27, 95.16) circle (  0.99);

\path[draw=drawColor,line width= 0.4pt,line join=round,line cap=round,fill=fillColor] (213.70, 89.69) circle (  0.99);

\path[draw=drawColor,line width= 0.4pt,line join=round,line cap=round,fill=fillColor] (214.12,106.70) circle (  0.99);

\path[draw=drawColor,line width= 0.4pt,line join=round,line cap=round,fill=fillColor] (214.55, 96.38) circle (  0.99);

\path[draw=drawColor,line width= 0.4pt,line join=round,line cap=round,fill=fillColor] (214.97,100.63) circle (  0.99);

\path[draw=drawColor,line width= 0.4pt,line join=round,line cap=round,fill=fillColor] (215.40, 79.98) circle (  0.99);

\path[draw=drawColor,line width= 0.4pt,line join=round,line cap=round,fill=fillColor] (215.82,149.82) circle (  0.99);

\path[draw=drawColor,line width= 0.4pt,line join=round,line cap=round,fill=fillColor] (216.25,100.63) circle (  0.99);

\path[draw=drawColor,line width= 0.4pt,line join=round,line cap=round,fill=fillColor] (216.68,101.23) circle (  0.99);

\path[draw=drawColor,line width= 0.4pt,line join=round,line cap=round,fill=fillColor] (217.10, 82.41) circle (  0.99);

\path[draw=drawColor,line width= 0.4pt,line join=round,line cap=round,fill=fillColor] (217.53,103.66) circle (  0.99);

\path[draw=drawColor,line width= 0.4pt,line join=round,line cap=round,fill=fillColor] (217.95,131.60) circle (  0.99);

\path[draw=drawColor,line width= 0.4pt,line join=round,line cap=round,fill=fillColor] (218.38,113.38) circle (  0.99);

\path[draw=drawColor,line width= 0.4pt,line join=round,line cap=round,fill=fillColor] (218.80, 94.55) circle (  0.99);

\path[draw=drawColor,line width= 0.4pt,line join=round,line cap=round,fill=fillColor] (219.23, 77.55) circle (  0.99);

\path[draw=drawColor,line width= 0.4pt,line join=round,line cap=round,fill=fillColor] (219.65, 98.81) circle (  0.99);

\path[draw=drawColor,line width= 0.4pt,line join=round,line cap=round,fill=fillColor] (220.08, 99.41) circle (  0.99);

\path[draw=drawColor,line width= 0.4pt,line join=round,line cap=round,fill=fillColor] (220.50,103.66) circle (  0.99);

\path[draw=drawColor,line width= 0.4pt,line join=round,line cap=round,fill=fillColor] (220.93,110.95) circle (  0.99);

\path[draw=drawColor,line width= 0.4pt,line join=round,line cap=round,fill=fillColor] (221.36, 98.81) circle (  0.99);

\path[draw=drawColor,line width= 0.4pt,line join=round,line cap=round,fill=fillColor] (221.78, 86.05) circle (  0.99);

\path[draw=drawColor,line width= 0.4pt,line join=round,line cap=round,fill=fillColor] (222.21,121.88) circle (  0.99);

\path[draw=drawColor,line width= 0.4pt,line join=round,line cap=round,fill=fillColor] (222.63, 92.73) circle (  0.99);

\path[draw=drawColor,line width= 0.4pt,line join=round,line cap=round,fill=fillColor] (223.06,191.73) circle (  0.99);

\path[draw=drawColor,line width= 0.4pt,line join=round,line cap=round,fill=fillColor] (223.48, 94.55) circle (  0.99);

\path[draw=drawColor,line width= 0.4pt,line join=round,line cap=round,fill=fillColor] (223.91,223.92) circle (  0.99);

\path[draw=drawColor,line width= 0.4pt,line join=round,line cap=round,fill=fillColor] (224.33,115.20) circle (  0.99);

\path[draw=drawColor,line width= 0.4pt,line join=round,line cap=round,fill=fillColor] (224.76, 90.91) circle (  0.99);

\path[draw=drawColor,line width= 0.4pt,line join=round,line cap=round,fill=fillColor] (225.18, 90.30) circle (  0.99);

\path[draw=drawColor,line width= 0.4pt,line join=round,line cap=round,fill=fillColor] (225.61,121.88) circle (  0.99);

\path[draw=drawColor,line width= 0.4pt,line join=round,line cap=round,fill=fillColor] (226.04, 94.55) circle (  0.99);

\path[draw=drawColor,line width= 0.4pt,line join=round,line cap=round,fill=fillColor] (226.46,100.63) circle (  0.99);

\path[draw=drawColor,line width= 0.4pt,line join=round,line cap=round,fill=fillColor] (226.89,103.06) circle (  0.99);

\path[draw=drawColor,line width= 0.4pt,line join=round,line cap=round,fill=fillColor] (227.31,116.42) circle (  0.99);

\path[draw=drawColor,line width= 0.4pt,line join=round,line cap=round,fill=fillColor] (227.74,114.60) circle (  0.99);

\path[draw=drawColor,line width= 0.4pt,line join=round,line cap=round,fill=fillColor] (228.16,132.82) circle (  0.99);

\path[draw=drawColor,line width= 0.4pt,line join=round,line cap=round,fill=fillColor] (228.59, 78.15) circle (  0.99);

\path[draw=drawColor,line width= 0.4pt,line join=round,line cap=round,fill=fillColor] (229.01,107.92) circle (  0.99);

\path[draw=drawColor,line width= 0.4pt,line join=round,line cap=round,fill=fillColor] (229.44,101.23) circle (  0.99);

\path[draw=drawColor,line width= 0.4pt,line join=round,line cap=round,fill=fillColor] (229.86,112.17) circle (  0.99);

\path[draw=drawColor,line width= 0.4pt,line join=round,line cap=round,fill=fillColor] (230.29,106.70) circle (  0.99);

\path[draw=drawColor,line width= 0.4pt,line join=round,line cap=round,fill=fillColor] (230.72,169.87) circle (  0.99);

\path[draw=drawColor,line width= 0.4pt,line join=round,line cap=round,fill=fillColor] (231.14,121.88) circle (  0.99);

\path[draw=drawColor,line width= 0.4pt,line join=round,line cap=round,fill=fillColor] (231.57,177.76) circle (  0.99);

\path[draw=drawColor,line width= 0.4pt,line join=round,line cap=round,fill=fillColor] (231.99,104.88) circle (  0.99);

\path[draw=drawColor,line width= 0.4pt,line join=round,line cap=round,fill=fillColor] (232.42, 99.41) circle (  0.99);

\path[draw=drawColor,line width= 0.4pt,line join=round,line cap=round,fill=fillColor] (232.84,122.49) circle (  0.99);

\path[draw=drawColor,line width= 0.4pt,line join=round,line cap=round,fill=fillColor] (233.27,114.60) circle (  0.99);

\path[draw=drawColor,line width= 0.4pt,line join=round,line cap=round,fill=fillColor] (233.69,107.31) circle (  0.99);

\path[draw=drawColor,line width= 0.4pt,line join=round,line cap=round,fill=fillColor] (234.12,119.46) circle (  0.99);

\path[draw=drawColor,line width= 0.4pt,line join=round,line cap=round,fill=fillColor] (234.54, 99.41) circle (  0.99);

\path[draw=drawColor,line width= 0.4pt,line join=round,line cap=round,fill=fillColor] (234.97,104.27) circle (  0.99);

\path[draw=drawColor,line width= 0.4pt,line join=round,line cap=round,fill=fillColor] (235.40,123.10) circle (  0.99);

\path[draw=drawColor,line width= 0.4pt,line join=round,line cap=round,fill=fillColor] (235.82, 87.87) circle (  0.99);

\path[draw=drawColor,line width= 0.4pt,line join=round,line cap=round,fill=fillColor] (236.25, 96.98) circle (  0.99);

\path[draw=drawColor,line width= 0.4pt,line join=round,line cap=round,fill=fillColor] (236.67,113.38) circle (  0.99);

\path[draw=drawColor,line width= 0.4pt,line join=round,line cap=round,fill=fillColor] (237.10,110.34) circle (  0.99);

\path[draw=drawColor,line width= 0.4pt,line join=round,line cap=round,fill=fillColor] (237.52,117.03) circle (  0.99);

\path[draw=drawColor,line width= 0.4pt,line join=round,line cap=round,fill=fillColor] (237.95, 78.76) circle (  0.99);

\path[draw=drawColor,line width= 0.4pt,line join=round,line cap=round,fill=fillColor] (238.37, 98.81) circle (  0.99);

\path[draw=drawColor,line width= 0.4pt,line join=round,line cap=round,fill=fillColor] (238.80, 81.19) circle (  0.99);

\path[draw=drawColor,line width= 0.4pt,line join=round,line cap=round,fill=fillColor] (239.22, 95.16) circle (  0.99);

\path[draw=drawColor,line width= 0.4pt,line join=round,line cap=round,fill=fillColor] (239.65,148.61) circle (  0.99);

\path[draw=drawColor,line width= 0.4pt,line join=round,line cap=round,fill=fillColor] (240.08,148.00) circle (  0.99);

\path[draw=drawColor,line width= 0.4pt,line join=round,line cap=round,fill=fillColor] (240.50,109.13) circle (  0.99);

\path[draw=drawColor,line width= 0.4pt,line join=round,line cap=round,fill=fillColor] (240.93,117.03) circle (  0.99);

\path[draw=drawColor,line width= 0.4pt,line join=round,line cap=round,fill=fillColor] (241.35,111.56) circle (  0.99);

\path[draw=drawColor,line width= 0.4pt,line join=round,line cap=round,fill=fillColor] (241.78,103.06) circle (  0.99);

\path[draw=drawColor,line width= 0.4pt,line join=round,line cap=round,fill=fillColor] (242.20,102.45) circle (  0.99);

\path[draw=drawColor,line width= 0.4pt,line join=round,line cap=round,fill=fillColor] (242.63, 95.77) circle (  0.99);

\path[draw=drawColor,line width= 0.4pt,line join=round,line cap=round,fill=fillColor] (243.05, 83.01) circle (  0.99);

\path[draw=drawColor,line width= 0.4pt,line join=round,line cap=round,fill=fillColor] (243.48,110.95) circle (  0.99);

\path[draw=drawColor,line width= 0.4pt,line join=round,line cap=round,fill=fillColor] (243.90,102.45) circle (  0.99);

\path[draw=drawColor,line width= 0.4pt,line join=round,line cap=round,fill=fillColor] (244.33, 92.12) circle (  0.99);

\path[draw=drawColor,line width= 0.4pt,line join=round,line cap=round,fill=fillColor] (244.76, 90.30) circle (  0.99);

\path[draw=drawColor,line width= 0.4pt,line join=round,line cap=round,fill=fillColor] (245.18,109.74) circle (  0.99);
\end{scope}
\begin{scope}
\path[clip] (126.47,  0.00) rectangle (252.94,505.89);
\definecolor{drawColor}{RGB}{0,0,0}

\node[text=drawColor,anchor=base,inner sep=0pt, outer sep=0pt, scale=  1.32] at (207.53,495.79) {\bfseries \textsf{3.6}$^\circ$};

\node[text=drawColor,anchor=base,inner sep=0pt, outer sep=0pt, scale=  1.32] at (207.53,  5.54) {Vp};
\end{scope}
\begin{scope}
\path[clip] (  0.00,  0.00) rectangle (505.89,505.89);
\definecolor{drawColor}{RGB}{0,0,0}

\path[draw=drawColor,line width= 0.4pt,line join=round,line cap=round] (170.30, 35.64) -- (245.18, 35.64);

\path[draw=drawColor,line width= 0.4pt,line join=round,line cap=round] (170.30, 35.64) -- (170.30, 31.68);

\path[draw=drawColor,line width= 0.4pt,line join=round,line cap=round] (195.40, 35.64) -- (195.40, 31.68);

\path[draw=drawColor,line width= 0.4pt,line join=round,line cap=round] (220.93, 35.64) -- (220.93, 31.68);

\path[draw=drawColor,line width= 0.4pt,line join=round,line cap=round] (245.18, 35.64) -- (245.18, 31.68);

\node[text=drawColor,anchor=base,inner sep=0pt, outer sep=0pt, scale=  0.99] at (170.30, 21.38) {1};

\node[text=drawColor,anchor=base,inner sep=0pt, outer sep=0pt, scale=  0.99] at (195.40, 21.38) {60};

\node[text=drawColor,anchor=base,inner sep=0pt, outer sep=0pt, scale=  0.99] at (220.93, 21.38) {120};

\node[text=drawColor,anchor=base,inner sep=0pt, outer sep=0pt, scale=  0.99] at (245.18, 21.38) {177};

\path[draw=drawColor,line width= 0.4pt,line join=round,line cap=round] (166.86, 52.65) -- (166.86,477.80);

\path[draw=drawColor,line width= 0.4pt,line join=round,line cap=round] (166.86, 52.65) -- (162.90, 52.65);

\path[draw=drawColor,line width= 0.4pt,line join=round,line cap=round] (166.86,113.38) -- (162.90,113.38);

\path[draw=drawColor,line width= 0.4pt,line join=round,line cap=round] (166.86,174.12) -- (162.90,174.12);

\path[draw=drawColor,line width= 0.4pt,line join=round,line cap=round] (166.86,234.85) -- (162.90,234.85);

\path[draw=drawColor,line width= 0.4pt,line join=round,line cap=round] (166.86,295.59) -- (162.90,295.59);

\path[draw=drawColor,line width= 0.4pt,line join=round,line cap=round] (166.86,356.32) -- (162.90,356.32);

\path[draw=drawColor,line width= 0.4pt,line join=round,line cap=round] (166.86,417.06) -- (162.90,417.06);

\path[draw=drawColor,line width= 0.4pt,line join=round,line cap=round] (166.86,477.80) -- (162.90,477.80);

\node[text=drawColor,anchor=base east,inner sep=0pt, outer sep=0pt, scale=  0.99] at (158.94, 49.24) {0};

\node[text=drawColor,anchor=base east,inner sep=0pt, outer sep=0pt, scale=  0.99] at (158.94,109.97) {100};

\node[text=drawColor,anchor=base east,inner sep=0pt, outer sep=0pt, scale=  0.99] at (158.94,170.71) {200};

\node[text=drawColor,anchor=base east,inner sep=0pt, outer sep=0pt, scale=  0.99] at (158.94,231.44) {300};

\node[text=drawColor,anchor=base east,inner sep=0pt, outer sep=0pt, scale=  0.99] at (158.94,292.18) {400};

\node[text=drawColor,anchor=base east,inner sep=0pt, outer sep=0pt, scale=  0.99] at (158.94,352.92) {500};

\node[text=drawColor,anchor=base east,inner sep=0pt, outer sep=0pt, scale=  0.99] at (158.94,413.65) {600};

\node[text=drawColor,anchor=base east,inner sep=0pt, outer sep=0pt, scale=  0.99] at (158.94,474.39) {700};
\end{scope}
\begin{scope}
\path[clip] (166.86, 35.64) rectangle (248.19,494.80);
\definecolor{drawColor}{RGB}{0,0,0}

\path[draw=drawColor,line width= 0.4pt,line join=round,line cap=round] (166.86,107.00) -- (248.19,107.00);
\end{scope}
\begin{scope}
\path[clip] (293.34, 35.64) rectangle (374.67,494.80);
\definecolor{drawColor}{RGB}{0,0,0}
\definecolor{fillColor}{RGB}{0,0,0}

\path[draw=drawColor,line width= 0.4pt,line join=round,line cap=round,fill=fillColor] (296.77,234.25) circle (  0.99);

\path[draw=drawColor,line width= 0.4pt,line join=round,line cap=round,fill=fillColor] (297.20,193.55) circle (  0.99);

\path[draw=drawColor,line width= 0.4pt,line join=round,line cap=round,fill=fillColor] (297.63,117.63) circle (  0.99);

\path[draw=drawColor,line width= 0.4pt,line join=round,line cap=round,fill=fillColor] (298.05,148.61) circle (  0.99);

\path[draw=drawColor,line width= 0.4pt,line join=round,line cap=round,fill=fillColor] (298.48,139.50) circle (  0.99);

\path[draw=drawColor,line width= 0.4pt,line join=round,line cap=round,fill=fillColor] (298.90,114.60) circle (  0.99);

\path[draw=drawColor,line width= 0.4pt,line join=round,line cap=round,fill=fillColor] (299.33,144.96) circle (  0.99);

\path[draw=drawColor,line width= 0.4pt,line join=round,line cap=round,fill=fillColor] (299.75,114.60) circle (  0.99);

\path[draw=drawColor,line width= 0.4pt,line join=round,line cap=round,fill=fillColor] (300.18,126.14) circle (  0.99);

\path[draw=drawColor,line width= 0.4pt,line join=round,line cap=round,fill=fillColor] (300.60,128.57) circle (  0.99);

\path[draw=drawColor,line width= 0.4pt,line join=round,line cap=round,fill=fillColor] (301.03,228.78) circle (  0.99);

\path[draw=drawColor,line width= 0.4pt,line join=round,line cap=round,fill=fillColor] (301.45,128.57) circle (  0.99);

\path[draw=drawColor,line width= 0.4pt,line join=round,line cap=round,fill=fillColor] (301.88,152.25) circle (  0.99);

\path[draw=drawColor,line width= 0.4pt,line join=round,line cap=round,fill=fillColor] (302.31,152.25) circle (  0.99);

\path[draw=drawColor,line width= 0.4pt,line join=round,line cap=round,fill=fillColor] (302.73,154.07) circle (  0.99);

\path[draw=drawColor,line width= 0.4pt,line join=round,line cap=round,fill=fillColor] (303.16,111.56) circle (  0.99);

\path[draw=drawColor,line width= 0.4pt,line join=round,line cap=round,fill=fillColor] (303.58, 93.34) circle (  0.99);

\path[draw=drawColor,line width= 0.4pt,line join=round,line cap=round,fill=fillColor] (304.01,126.14) circle (  0.99);

\path[draw=drawColor,line width= 0.4pt,line join=round,line cap=round,fill=fillColor] (304.43,111.56) circle (  0.99);

\path[draw=drawColor,line width= 0.4pt,line join=round,line cap=round,fill=fillColor] (304.86,103.66) circle (  0.99);

\path[draw=drawColor,line width= 0.4pt,line join=round,line cap=round,fill=fillColor] (305.28,271.29) circle (  0.99);

\path[draw=drawColor,line width= 0.4pt,line join=round,line cap=round,fill=fillColor] (305.71,115.81) circle (  0.99);

\path[draw=drawColor,line width= 0.4pt,line join=round,line cap=round,fill=fillColor] (306.13, 93.95) circle (  0.99);

\path[draw=drawColor,line width= 0.4pt,line join=round,line cap=round,fill=fillColor] (306.56,113.99) circle (  0.99);

\path[draw=drawColor,line width= 0.4pt,line join=round,line cap=round,fill=fillColor] (306.99, 98.20) circle (  0.99);

\path[draw=drawColor,line width= 0.4pt,line join=round,line cap=round,fill=fillColor] (307.41,107.31) circle (  0.99);

\path[draw=drawColor,line width= 0.4pt,line join=round,line cap=round,fill=fillColor] (307.84,121.88) circle (  0.99);

\path[draw=drawColor,line width= 0.4pt,line join=round,line cap=round,fill=fillColor] (308.26,146.79) circle (  0.99);

\path[draw=drawColor,line width= 0.4pt,line join=round,line cap=round,fill=fillColor] (308.69,120.06) circle (  0.99);

\path[draw=drawColor,line width= 0.4pt,line join=round,line cap=round,fill=fillColor] (309.11, 98.81) circle (  0.99);

\path[draw=drawColor,line width= 0.4pt,line join=round,line cap=round,fill=fillColor] (309.54,103.06) circle (  0.99);

\path[draw=drawColor,line width= 0.4pt,line join=round,line cap=round,fill=fillColor] (309.96,101.84) circle (  0.99);

\path[draw=drawColor,line width= 0.4pt,line join=round,line cap=round,fill=fillColor] (310.39,112.77) circle (  0.99);

\path[draw=drawColor,line width= 0.4pt,line join=round,line cap=round,fill=fillColor] (310.81,131.00) circle (  0.99);

\path[draw=drawColor,line width= 0.4pt,line join=round,line cap=round,fill=fillColor] (311.24,129.17) circle (  0.99);

\path[draw=drawColor,line width= 0.4pt,line join=round,line cap=round,fill=fillColor] (311.67,106.09) circle (  0.99);

\path[draw=drawColor,line width= 0.4pt,line join=round,line cap=round,fill=fillColor] (312.09, 81.80) circle (  0.99);

\path[draw=drawColor,line width= 0.4pt,line join=round,line cap=round,fill=fillColor] (312.52,101.84) circle (  0.99);

\path[draw=drawColor,line width= 0.4pt,line join=round,line cap=round,fill=fillColor] (312.94,121.28) circle (  0.99);

\path[draw=drawColor,line width= 0.4pt,line join=round,line cap=round,fill=fillColor] (313.37,131.00) circle (  0.99);

\path[draw=drawColor,line width= 0.4pt,line join=round,line cap=round,fill=fillColor] (313.79,131.60) circle (  0.99);

\path[draw=drawColor,line width= 0.4pt,line join=round,line cap=round,fill=fillColor] (314.22,146.79) circle (  0.99);

\path[draw=drawColor,line width= 0.4pt,line join=round,line cap=round,fill=fillColor] (314.64,128.57) circle (  0.99);

\path[draw=drawColor,line width= 0.4pt,line join=round,line cap=round,fill=fillColor] (315.07,125.53) circle (  0.99);

\path[draw=drawColor,line width= 0.4pt,line join=round,line cap=round,fill=fillColor] (315.49, 85.44) circle (  0.99);

\path[draw=drawColor,line width= 0.4pt,line join=round,line cap=round,fill=fillColor] (315.92,114.60) circle (  0.99);

\path[draw=drawColor,line width= 0.4pt,line join=round,line cap=round,fill=fillColor] (316.35,146.18) circle (  0.99);

\path[draw=drawColor,line width= 0.4pt,line join=round,line cap=round,fill=fillColor] (316.77, 99.41) circle (  0.99);

\path[draw=drawColor,line width= 0.4pt,line join=round,line cap=round,fill=fillColor] (317.20,200.84) circle (  0.99);

\path[draw=drawColor,line width= 0.4pt,line join=round,line cap=round,fill=fillColor] (317.62,104.88) circle (  0.99);

\path[draw=drawColor,line width= 0.4pt,line join=round,line cap=round,fill=fillColor] (318.05, 96.38) circle (  0.99);

\path[draw=drawColor,line width= 0.4pt,line join=round,line cap=round,fill=fillColor] (318.47,109.74) circle (  0.99);

\path[draw=drawColor,line width= 0.4pt,line join=round,line cap=round,fill=fillColor] (318.90,148.00) circle (  0.99);

\path[draw=drawColor,line width= 0.4pt,line join=round,line cap=round,fill=fillColor] (319.32,108.52) circle (  0.99);

\path[draw=drawColor,line width= 0.4pt,line join=round,line cap=round,fill=fillColor] (319.75,134.03) circle (  0.99);

\path[draw=drawColor,line width= 0.4pt,line join=round,line cap=round,fill=fillColor] (320.17,130.39) circle (  0.99);

\path[draw=drawColor,line width= 0.4pt,line join=round,line cap=round,fill=fillColor] (320.60,110.34) circle (  0.99);

\path[draw=drawColor,line width= 0.4pt,line join=round,line cap=round,fill=fillColor] (321.03,147.39) circle (  0.99);

\path[draw=drawColor,line width= 0.4pt,line join=round,line cap=round,fill=fillColor] (321.45, 92.12) circle (  0.99);

\path[draw=drawColor,line width= 0.4pt,line join=round,line cap=round,fill=fillColor] (321.88,107.31) circle (  0.99);

\path[draw=drawColor,line width= 0.4pt,line join=round,line cap=round,fill=fillColor] (322.30,127.35) circle (  0.99);

\path[draw=drawColor,line width= 0.4pt,line join=round,line cap=round,fill=fillColor] (322.73,168.65) circle (  0.99);

\path[draw=drawColor,line width= 0.4pt,line join=round,line cap=round,fill=fillColor] (323.15,110.34) circle (  0.99);

\path[draw=drawColor,line width= 0.4pt,line join=round,line cap=round,fill=fillColor] (323.58,115.81) circle (  0.99);

\path[draw=drawColor,line width= 0.4pt,line join=round,line cap=round,fill=fillColor] (324.00,139.50) circle (  0.99);

\path[draw=drawColor,line width= 0.4pt,line join=round,line cap=round,fill=fillColor] (324.43,101.84) circle (  0.99);

\path[draw=drawColor,line width= 0.4pt,line join=round,line cap=round,fill=fillColor] (324.85,148.61) circle (  0.99);

\path[draw=drawColor,line width= 0.4pt,line join=round,line cap=round,fill=fillColor] (325.28,140.71) circle (  0.99);

\path[draw=drawColor,line width= 0.4pt,line join=round,line cap=round,fill=fillColor] (325.71,112.17) circle (  0.99);

\path[draw=drawColor,line width= 0.4pt,line join=round,line cap=round,fill=fillColor] (326.13,124.31) circle (  0.99);

\path[draw=drawColor,line width= 0.4pt,line join=round,line cap=round,fill=fillColor] (326.56,110.95) circle (  0.99);

\path[draw=drawColor,line width= 0.4pt,line join=round,line cap=round,fill=fillColor] (326.98,117.63) circle (  0.99);

\path[draw=drawColor,line width= 0.4pt,line join=round,line cap=round,fill=fillColor] (327.41,109.74) circle (  0.99);

\path[draw=drawColor,line width= 0.4pt,line join=round,line cap=round,fill=fillColor] (327.83,113.38) circle (  0.99);

\path[draw=drawColor,line width= 0.4pt,line join=round,line cap=round,fill=fillColor] (328.26,114.60) circle (  0.99);

\path[draw=drawColor,line width= 0.4pt,line join=round,line cap=round,fill=fillColor] (328.68,167.44) circle (  0.99);

\path[draw=drawColor,line width= 0.4pt,line join=round,line cap=round,fill=fillColor] (329.11,118.85) circle (  0.99);

\path[draw=drawColor,line width= 0.4pt,line join=round,line cap=round,fill=fillColor] (329.53, 93.95) circle (  0.99);

\path[draw=drawColor,line width= 0.4pt,line join=round,line cap=round,fill=fillColor] (329.96,126.14) circle (  0.99);

\path[draw=drawColor,line width= 0.4pt,line join=round,line cap=round,fill=fillColor] (330.38,118.85) circle (  0.99);

\path[draw=drawColor,line width= 0.4pt,line join=round,line cap=round,fill=fillColor] (330.81,106.70) circle (  0.99);

\path[draw=drawColor,line width= 0.4pt,line join=round,line cap=round,fill=fillColor] (331.24,105.49) circle (  0.99);

\path[draw=drawColor,line width= 0.4pt,line join=round,line cap=round,fill=fillColor] (331.66,107.92) circle (  0.99);

\path[draw=drawColor,line width= 0.4pt,line join=round,line cap=round,fill=fillColor] (332.09,141.93) circle (  0.99);

\path[draw=drawColor,line width= 0.4pt,line join=round,line cap=round,fill=fillColor] (332.51,123.71) circle (  0.99);

\path[draw=drawColor,line width= 0.4pt,line join=round,line cap=round,fill=fillColor] (332.94,107.31) circle (  0.99);

\path[draw=drawColor,line width= 0.4pt,line join=round,line cap=round,fill=fillColor] (333.36,118.24) circle (  0.99);

\path[draw=drawColor,line width= 0.4pt,line join=round,line cap=round,fill=fillColor] (333.79,105.49) circle (  0.99);

\path[draw=drawColor,line width= 0.4pt,line join=round,line cap=round,fill=fillColor] (334.21,118.24) circle (  0.99);

\path[draw=drawColor,line width= 0.4pt,line join=round,line cap=round,fill=fillColor] (334.64,109.74) circle (  0.99);

\path[draw=drawColor,line width= 0.4pt,line join=round,line cap=round,fill=fillColor] (335.06, 96.38) circle (  0.99);

\path[draw=drawColor,line width= 0.4pt,line join=round,line cap=round,fill=fillColor] (335.49,114.60) circle (  0.99);

\path[draw=drawColor,line width= 0.4pt,line join=round,line cap=round,fill=fillColor] (335.92,124.31) circle (  0.99);

\path[draw=drawColor,line width= 0.4pt,line join=round,line cap=round,fill=fillColor] (336.34, 94.55) circle (  0.99);

\path[draw=drawColor,line width= 0.4pt,line join=round,line cap=round,fill=fillColor] (336.77,134.03) circle (  0.99);

\path[draw=drawColor,line width= 0.4pt,line join=round,line cap=round,fill=fillColor] (337.19,138.28) circle (  0.99);

\path[draw=drawColor,line width= 0.4pt,line join=round,line cap=round,fill=fillColor] (337.62,120.06) circle (  0.99);

\path[draw=drawColor,line width= 0.4pt,line join=round,line cap=round,fill=fillColor] (338.04,111.56) circle (  0.99);

\path[draw=drawColor,line width= 0.4pt,line join=round,line cap=round,fill=fillColor] (338.47,105.49) circle (  0.99);

\path[draw=drawColor,line width= 0.4pt,line join=round,line cap=round,fill=fillColor] (338.89,100.02) circle (  0.99);

\path[draw=drawColor,line width= 0.4pt,line join=round,line cap=round,fill=fillColor] (339.32,118.85) circle (  0.99);

\path[draw=drawColor,line width= 0.4pt,line join=round,line cap=round,fill=fillColor] (339.74,114.60) circle (  0.99);

\path[draw=drawColor,line width= 0.4pt,line join=round,line cap=round,fill=fillColor] (340.17, 98.20) circle (  0.99);

\path[draw=drawColor,line width= 0.4pt,line join=round,line cap=round,fill=fillColor] (340.60,129.17) circle (  0.99);

\path[draw=drawColor,line width= 0.4pt,line join=round,line cap=round,fill=fillColor] (341.02,110.95) circle (  0.99);

\path[draw=drawColor,line width= 0.4pt,line join=round,line cap=round,fill=fillColor] (341.45,113.99) circle (  0.99);

\path[draw=drawColor,line width= 0.4pt,line join=round,line cap=round,fill=fillColor] (341.87, 90.30) circle (  0.99);

\path[draw=drawColor,line width= 0.4pt,line join=round,line cap=round,fill=fillColor] (342.30,159.54) circle (  0.99);

\path[draw=drawColor,line width= 0.4pt,line join=round,line cap=round,fill=fillColor] (342.72,109.13) circle (  0.99);

\path[draw=drawColor,line width= 0.4pt,line join=round,line cap=round,fill=fillColor] (343.15,113.38) circle (  0.99);

\path[draw=drawColor,line width= 0.4pt,line join=round,line cap=round,fill=fillColor] (343.57, 89.09) circle (  0.99);

\path[draw=drawColor,line width= 0.4pt,line join=round,line cap=round,fill=fillColor] (344.00,133.42) circle (  0.99);

\path[draw=drawColor,line width= 0.4pt,line join=round,line cap=round,fill=fillColor] (344.42,135.25) circle (  0.99);

\path[draw=drawColor,line width= 0.4pt,line join=round,line cap=round,fill=fillColor] (344.85,151.65) circle (  0.99);

\path[draw=drawColor,line width= 0.4pt,line join=round,line cap=round,fill=fillColor] (345.28,110.34) circle (  0.99);

\path[draw=drawColor,line width= 0.4pt,line join=round,line cap=round,fill=fillColor] (345.70, 81.80) circle (  0.99);

\path[draw=drawColor,line width= 0.4pt,line join=round,line cap=round,fill=fillColor] (346.13,118.85) circle (  0.99);

\path[draw=drawColor,line width= 0.4pt,line join=round,line cap=round,fill=fillColor] (346.55,106.09) circle (  0.99);

\path[draw=drawColor,line width= 0.4pt,line join=round,line cap=round,fill=fillColor] (346.98,103.06) circle (  0.99);

\path[draw=drawColor,line width= 0.4pt,line join=round,line cap=round,fill=fillColor] (347.40,120.06) circle (  0.99);

\path[draw=drawColor,line width= 0.4pt,line join=round,line cap=round,fill=fillColor] (347.83,118.85) circle (  0.99);

\path[draw=drawColor,line width= 0.4pt,line join=round,line cap=round,fill=fillColor] (348.25, 98.20) circle (  0.99);

\path[draw=drawColor,line width= 0.4pt,line join=round,line cap=round,fill=fillColor] (348.68,139.50) circle (  0.99);

\path[draw=drawColor,line width= 0.4pt,line join=round,line cap=round,fill=fillColor] (349.10,109.74) circle (  0.99);

\path[draw=drawColor,line width= 0.4pt,line join=round,line cap=round,fill=fillColor] (349.53,164.40) circle (  0.99);

\path[draw=drawColor,line width= 0.4pt,line join=round,line cap=round,fill=fillColor] (349.96, 90.30) circle (  0.99);

\path[draw=drawColor,line width= 0.4pt,line join=round,line cap=round,fill=fillColor] (350.38,271.29) circle (  0.99);

\path[draw=drawColor,line width= 0.4pt,line join=round,line cap=round,fill=fillColor] (350.81,112.17) circle (  0.99);

\path[draw=drawColor,line width= 0.4pt,line join=round,line cap=round,fill=fillColor] (351.23, 96.98) circle (  0.99);

\path[draw=drawColor,line width= 0.4pt,line join=round,line cap=round,fill=fillColor] (351.66, 96.98) circle (  0.99);

\path[draw=drawColor,line width= 0.4pt,line join=round,line cap=round,fill=fillColor] (352.08,132.82) circle (  0.99);

\path[draw=drawColor,line width= 0.4pt,line join=round,line cap=round,fill=fillColor] (352.51,104.27) circle (  0.99);

\path[draw=drawColor,line width= 0.4pt,line join=round,line cap=round,fill=fillColor] (352.93,113.99) circle (  0.99);

\path[draw=drawColor,line width= 0.4pt,line join=round,line cap=round,fill=fillColor] (353.36,110.95) circle (  0.99);

\path[draw=drawColor,line width= 0.4pt,line join=round,line cap=round,fill=fillColor] (353.78,119.46) circle (  0.99);

\path[draw=drawColor,line width= 0.4pt,line join=round,line cap=round,fill=fillColor] (354.21,110.34) circle (  0.99);

\path[draw=drawColor,line width= 0.4pt,line join=round,line cap=round,fill=fillColor] (354.64,155.29) circle (  0.99);

\path[draw=drawColor,line width= 0.4pt,line join=round,line cap=round,fill=fillColor] (355.06, 90.30) circle (  0.99);

\path[draw=drawColor,line width= 0.4pt,line join=round,line cap=round,fill=fillColor] (355.49,113.38) circle (  0.99);

\path[draw=drawColor,line width= 0.4pt,line join=round,line cap=round,fill=fillColor] (355.91,107.92) circle (  0.99);

\path[draw=drawColor,line width= 0.4pt,line join=round,line cap=round,fill=fillColor] (356.34,129.17) circle (  0.99);

\path[draw=drawColor,line width= 0.4pt,line join=round,line cap=round,fill=fillColor] (356.76,121.88) circle (  0.99);

\path[draw=drawColor,line width= 0.4pt,line join=round,line cap=round,fill=fillColor] (357.19,147.39) circle (  0.99);

\path[draw=drawColor,line width= 0.4pt,line join=round,line cap=round,fill=fillColor] (357.61,141.93) circle (  0.99);

\path[draw=drawColor,line width= 0.4pt,line join=round,line cap=round,fill=fillColor] (358.04,205.09) circle (  0.99);

\path[draw=drawColor,line width= 0.4pt,line join=round,line cap=round,fill=fillColor] (358.46,117.03) circle (  0.99);

\path[draw=drawColor,line width= 0.4pt,line join=round,line cap=round,fill=fillColor] (358.89, 98.20) circle (  0.99);

\path[draw=drawColor,line width= 0.4pt,line join=round,line cap=round,fill=fillColor] (359.32,143.14) circle (  0.99);

\path[draw=drawColor,line width= 0.4pt,line join=round,line cap=round,fill=fillColor] (359.74,129.17) circle (  0.99);

\path[draw=drawColor,line width= 0.4pt,line join=round,line cap=round,fill=fillColor] (360.17,108.52) circle (  0.99);

\path[draw=drawColor,line width= 0.4pt,line join=round,line cap=round,fill=fillColor] (360.59,126.14) circle (  0.99);

\path[draw=drawColor,line width= 0.4pt,line join=round,line cap=round,fill=fillColor] (361.02,106.09) circle (  0.99);

\path[draw=drawColor,line width= 0.4pt,line join=round,line cap=round,fill=fillColor] (361.44,134.03) circle (  0.99);

\path[draw=drawColor,line width= 0.4pt,line join=round,line cap=round,fill=fillColor] (361.87,128.57) circle (  0.99);

\path[draw=drawColor,line width= 0.4pt,line join=round,line cap=round,fill=fillColor] (362.29,107.31) circle (  0.99);

\path[draw=drawColor,line width= 0.4pt,line join=round,line cap=round,fill=fillColor] (362.72,117.03) circle (  0.99);

\path[draw=drawColor,line width= 0.4pt,line join=round,line cap=round,fill=fillColor] (363.14,109.13) circle (  0.99);

\path[draw=drawColor,line width= 0.4pt,line join=round,line cap=round,fill=fillColor] (363.57,115.81) circle (  0.99);

\path[draw=drawColor,line width= 0.4pt,line join=round,line cap=round,fill=fillColor] (364.00,154.07) circle (  0.99);

\path[draw=drawColor,line width= 0.4pt,line join=round,line cap=round,fill=fillColor] (364.42, 87.87) circle (  0.99);

\path[draw=drawColor,line width= 0.4pt,line join=round,line cap=round,fill=fillColor] (364.85,113.38) circle (  0.99);

\path[draw=drawColor,line width= 0.4pt,line join=round,line cap=round,fill=fillColor] (365.27, 79.98) circle (  0.99);

\path[draw=drawColor,line width= 0.4pt,line join=round,line cap=round,fill=fillColor] (365.70,100.63) circle (  0.99);

\path[draw=drawColor,line width= 0.4pt,line join=round,line cap=round,fill=fillColor] (366.12,308.95) circle (  0.99);

\path[draw=drawColor,line width= 0.4pt,line join=round,line cap=round,fill=fillColor] (366.55,161.97) circle (  0.99);

\path[draw=drawColor,line width= 0.4pt,line join=round,line cap=round,fill=fillColor] (366.97,113.38) circle (  0.99);

\path[draw=drawColor,line width= 0.4pt,line join=round,line cap=round,fill=fillColor] (367.40,129.17) circle (  0.99);

\path[draw=drawColor,line width= 0.4pt,line join=round,line cap=round,fill=fillColor] (367.82,129.78) circle (  0.99);

\path[draw=drawColor,line width= 0.4pt,line join=round,line cap=round,fill=fillColor] (368.25,111.56) circle (  0.99);

\path[draw=drawColor,line width= 0.4pt,line join=round,line cap=round,fill=fillColor] (368.68,110.95) circle (  0.99);

\path[draw=drawColor,line width= 0.4pt,line join=round,line cap=round,fill=fillColor] (369.10,104.27) circle (  0.99);

\path[draw=drawColor,line width= 0.4pt,line join=round,line cap=round,fill=fillColor] (369.53, 84.84) circle (  0.99);

\path[draw=drawColor,line width= 0.4pt,line join=round,line cap=round,fill=fillColor] (369.95,133.42) circle (  0.99);

\path[draw=drawColor,line width= 0.4pt,line join=round,line cap=round,fill=fillColor] (370.38,127.96) circle (  0.99);

\path[draw=drawColor,line width= 0.4pt,line join=round,line cap=round,fill=fillColor] (370.80,112.17) circle (  0.99);

\path[draw=drawColor,line width= 0.4pt,line join=round,line cap=round,fill=fillColor] (371.23, 90.91) circle (  0.99);

\path[draw=drawColor,line width= 0.4pt,line join=round,line cap=round,fill=fillColor] (371.65,124.31) circle (  0.99);
\end{scope}
\begin{scope}
\path[clip] (252.94,  0.00) rectangle (379.42,505.89);
\definecolor{drawColor}{RGB}{0,0,0}

\node[text=drawColor,anchor=base,inner sep=0pt, outer sep=0pt, scale=  1.32] at (334.00,495.79) {\bfseries \textsf{5.4}$^\circ$};

\node[text=drawColor,anchor=base,inner sep=0pt, outer sep=0pt, scale=  1.32] at (334.00,  5.54) {Vp};
\end{scope}
\begin{scope}
\path[clip] (  0.00,  0.00) rectangle (505.89,505.89);
\definecolor{drawColor}{RGB}{0,0,0}

\path[draw=drawColor,line width= 0.4pt,line join=round,line cap=round] (296.77, 35.64) -- (371.65, 35.64);

\path[draw=drawColor,line width= 0.4pt,line join=round,line cap=round] (296.77, 35.64) -- (296.77, 31.68);

\path[draw=drawColor,line width= 0.4pt,line join=round,line cap=round] (321.88, 35.64) -- (321.88, 31.68);

\path[draw=drawColor,line width= 0.4pt,line join=round,line cap=round] (347.40, 35.64) -- (347.40, 31.68);

\path[draw=drawColor,line width= 0.4pt,line join=round,line cap=round] (371.65, 35.64) -- (371.65, 31.68);

\node[text=drawColor,anchor=base,inner sep=0pt, outer sep=0pt, scale=  0.99] at (296.77, 21.38) {1};

\node[text=drawColor,anchor=base,inner sep=0pt, outer sep=0pt, scale=  0.99] at (321.88, 21.38) {60};

\node[text=drawColor,anchor=base,inner sep=0pt, outer sep=0pt, scale=  0.99] at (347.40, 21.38) {120};

\node[text=drawColor,anchor=base,inner sep=0pt, outer sep=0pt, scale=  0.99] at (371.65, 21.38) {177};

\path[draw=drawColor,line width= 0.4pt,line join=round,line cap=round] (293.34, 52.65) -- (293.34,477.80);

\path[draw=drawColor,line width= 0.4pt,line join=round,line cap=round] (293.34, 52.65) -- (289.38, 52.65);

\path[draw=drawColor,line width= 0.4pt,line join=round,line cap=round] (293.34,113.38) -- (289.38,113.38);

\path[draw=drawColor,line width= 0.4pt,line join=round,line cap=round] (293.34,174.12) -- (289.38,174.12);

\path[draw=drawColor,line width= 0.4pt,line join=round,line cap=round] (293.34,234.85) -- (289.38,234.85);

\path[draw=drawColor,line width= 0.4pt,line join=round,line cap=round] (293.34,295.59) -- (289.38,295.59);

\path[draw=drawColor,line width= 0.4pt,line join=round,line cap=round] (293.34,356.32) -- (289.38,356.32);

\path[draw=drawColor,line width= 0.4pt,line join=round,line cap=round] (293.34,417.06) -- (289.38,417.06);

\path[draw=drawColor,line width= 0.4pt,line join=round,line cap=round] (293.34,477.80) -- (289.38,477.80);

\node[text=drawColor,anchor=base east,inner sep=0pt, outer sep=0pt, scale=  0.99] at (285.42, 49.24) {0};

\node[text=drawColor,anchor=base east,inner sep=0pt, outer sep=0pt, scale=  0.99] at (285.42,109.97) {100};

\node[text=drawColor,anchor=base east,inner sep=0pt, outer sep=0pt, scale=  0.99] at (285.42,170.71) {200};

\node[text=drawColor,anchor=base east,inner sep=0pt, outer sep=0pt, scale=  0.99] at (285.42,231.44) {300};

\node[text=drawColor,anchor=base east,inner sep=0pt, outer sep=0pt, scale=  0.99] at (285.42,292.18) {400};

\node[text=drawColor,anchor=base east,inner sep=0pt, outer sep=0pt, scale=  0.99] at (285.42,352.92) {500};

\node[text=drawColor,anchor=base east,inner sep=0pt, outer sep=0pt, scale=  0.99] at (285.42,413.65) {600};

\node[text=drawColor,anchor=base east,inner sep=0pt, outer sep=0pt, scale=  0.99] at (285.42,474.39) {700};
\end{scope}
\begin{scope}
\path[clip] (293.34, 35.64) rectangle (374.67,494.80);
\definecolor{drawColor}{RGB}{0,0,0}

\path[draw=drawColor,line width= 0.4pt,line join=round,line cap=round] (293.34,118.71) -- (374.67,118.71);
\end{scope}
\begin{scope}
\path[clip] (419.81, 35.64) rectangle (501.14,494.80);
\definecolor{drawColor}{RGB}{0,0,0}
\definecolor{fillColor}{RGB}{0,0,0}

\path[draw=drawColor,line width= 0.4pt,line join=round,line cap=round,fill=fillColor] (423.25,215.42) circle (  0.99);

\path[draw=drawColor,line width= 0.4pt,line join=round,line cap=round,fill=fillColor] (423.67,355.72) circle (  0.99);

\path[draw=drawColor,line width= 0.4pt,line join=round,line cap=round,fill=fillColor] (424.10,138.89) circle (  0.99);

\path[draw=drawColor,line width= 0.4pt,line join=round,line cap=round,fill=fillColor] (424.52,183.84) circle (  0.99);

\path[draw=drawColor,line width= 0.4pt,line join=round,line cap=round,fill=fillColor] (424.95,149.22) circle (  0.99);

\path[draw=drawColor,line width= 0.4pt,line join=round,line cap=round,fill=fillColor] (425.37,122.49) circle (  0.99);

\path[draw=drawColor,line width= 0.4pt,line join=round,line cap=round,fill=fillColor] (425.80,172.90) circle (  0.99);

\path[draw=drawColor,line width= 0.4pt,line join=round,line cap=round,fill=fillColor] (426.23,141.93) circle (  0.99);

\path[draw=drawColor,line width= 0.4pt,line join=round,line cap=round,fill=fillColor] (426.65,126.74) circle (  0.99);

\path[draw=drawColor,line width= 0.4pt,line join=round,line cap=round,fill=fillColor] (427.08,170.47) circle (  0.99);

\path[draw=drawColor,line width= 0.4pt,line join=round,line cap=round,fill=fillColor] (427.50,288.91) circle (  0.99);

\path[draw=drawColor,line width= 0.4pt,line join=round,line cap=round,fill=fillColor] (427.93,137.68) circle (  0.99);

\path[draw=drawColor,line width= 0.4pt,line join=round,line cap=round,fill=fillColor] (428.35,175.33) circle (  0.99);

\path[draw=drawColor,line width= 0.4pt,line join=round,line cap=round,fill=fillColor] (428.78,176.55) circle (  0.99);

\path[draw=drawColor,line width= 0.4pt,line join=round,line cap=round,fill=fillColor] (429.20,215.42) circle (  0.99);

\path[draw=drawColor,line width= 0.4pt,line join=round,line cap=round,fill=fillColor] (429.63,131.00) circle (  0.99);

\path[draw=drawColor,line width= 0.4pt,line join=round,line cap=round,fill=fillColor] (430.05,105.49) circle (  0.99);

\path[draw=drawColor,line width= 0.4pt,line join=round,line cap=round,fill=fillColor] (430.48,140.71) circle (  0.99);

\path[draw=drawColor,line width= 0.4pt,line join=round,line cap=round,fill=fillColor] (430.91,114.60) circle (  0.99);

\path[draw=drawColor,line width= 0.4pt,line join=round,line cap=round,fill=fillColor] (431.33,110.95) circle (  0.99);

\path[draw=drawColor,line width= 0.4pt,line join=round,line cap=round,fill=fillColor] (431.76,348.43) circle (  0.99);

\path[draw=drawColor,line width= 0.4pt,line join=round,line cap=round,fill=fillColor] (432.18,137.68) circle (  0.99);

\path[draw=drawColor,line width= 0.4pt,line join=round,line cap=round,fill=fillColor] (432.61,110.95) circle (  0.99);

\path[draw=drawColor,line width= 0.4pt,line join=round,line cap=round,fill=fillColor] (433.03,155.90) circle (  0.99);

\path[draw=drawColor,line width= 0.4pt,line join=round,line cap=round,fill=fillColor] (433.46,104.88) circle (  0.99);

\path[draw=drawColor,line width= 0.4pt,line join=round,line cap=round,fill=fillColor] (433.88,133.42) circle (  0.99);

\path[draw=drawColor,line width= 0.4pt,line join=round,line cap=round,fill=fillColor] (434.31,134.03) circle (  0.99);

\path[draw=drawColor,line width= 0.4pt,line join=round,line cap=round,fill=fillColor] (434.73,132.21) circle (  0.99);

\path[draw=drawColor,line width= 0.4pt,line join=round,line cap=round,fill=fillColor] (435.16,131.00) circle (  0.99);

\path[draw=drawColor,line width= 0.4pt,line join=round,line cap=round,fill=fillColor] (435.59,112.77) circle (  0.99);

\path[draw=drawColor,line width= 0.4pt,line join=round,line cap=round,fill=fillColor] (436.01,121.28) circle (  0.99);

\path[draw=drawColor,line width= 0.4pt,line join=round,line cap=round,fill=fillColor] (436.44, 97.59) circle (  0.99);

\path[draw=drawColor,line width= 0.4pt,line join=round,line cap=round,fill=fillColor] (436.86,120.67) circle (  0.99);

\path[draw=drawColor,line width= 0.4pt,line join=round,line cap=round,fill=fillColor] (437.29,138.28) circle (  0.99);

\path[draw=drawColor,line width= 0.4pt,line join=round,line cap=round,fill=fillColor] (437.71,169.26) circle (  0.99);

\path[draw=drawColor,line width= 0.4pt,line join=round,line cap=round,fill=fillColor] (438.14,118.85) circle (  0.99);

\path[draw=drawColor,line width= 0.4pt,line join=round,line cap=round,fill=fillColor] (438.56, 89.69) circle (  0.99);

\path[draw=drawColor,line width= 0.4pt,line join=round,line cap=round,fill=fillColor] (438.99,111.56) circle (  0.99);

\path[draw=drawColor,line width= 0.4pt,line join=round,line cap=round,fill=fillColor] (439.41,146.18) circle (  0.99);

\path[draw=drawColor,line width= 0.4pt,line join=round,line cap=round,fill=fillColor] (439.84,120.67) circle (  0.99);

\path[draw=drawColor,line width= 0.4pt,line join=round,line cap=round,fill=fillColor] (440.27,163.18) circle (  0.99);

\path[draw=drawColor,line width= 0.4pt,line join=round,line cap=round,fill=fillColor] (440.69,199.63) circle (  0.99);

\path[draw=drawColor,line width= 0.4pt,line join=round,line cap=round,fill=fillColor] (441.12,145.57) circle (  0.99);

\path[draw=drawColor,line width= 0.4pt,line join=round,line cap=round,fill=fillColor] (441.54,115.20) circle (  0.99);

\path[draw=drawColor,line width= 0.4pt,line join=round,line cap=round,fill=fillColor] (441.97,100.63) circle (  0.99);

\path[draw=drawColor,line width= 0.4pt,line join=round,line cap=round,fill=fillColor] (442.39,131.60) circle (  0.99);

\path[draw=drawColor,line width= 0.4pt,line join=round,line cap=round,fill=fillColor] (442.82,188.09) circle (  0.99);

\path[draw=drawColor,line width= 0.4pt,line join=round,line cap=round,fill=fillColor] (443.24,112.17) circle (  0.99);

\path[draw=drawColor,line width= 0.4pt,line join=round,line cap=round,fill=fillColor] (443.67,291.94) circle (  0.99);

\path[draw=drawColor,line width= 0.4pt,line join=round,line cap=round,fill=fillColor] (444.09,114.60) circle (  0.99);

\path[draw=drawColor,line width= 0.4pt,line join=round,line cap=round,fill=fillColor] (444.52,107.31) circle (  0.99);

\path[draw=drawColor,line width= 0.4pt,line join=round,line cap=round,fill=fillColor] (444.94,122.49) circle (  0.99);

\path[draw=drawColor,line width= 0.4pt,line join=round,line cap=round,fill=fillColor] (445.37,154.68) circle (  0.99);

\path[draw=drawColor,line width= 0.4pt,line join=round,line cap=round,fill=fillColor] (445.80,127.96) circle (  0.99);

\path[draw=drawColor,line width= 0.4pt,line join=round,line cap=round,fill=fillColor] (446.22,169.87) circle (  0.99);

\path[draw=drawColor,line width= 0.4pt,line join=round,line cap=round,fill=fillColor] (446.65,138.28) circle (  0.99);

\path[draw=drawColor,line width= 0.4pt,line join=round,line cap=round,fill=fillColor] (447.07,110.95) circle (  0.99);

\path[draw=drawColor,line width= 0.4pt,line join=round,line cap=round,fill=fillColor] (447.50,163.18) circle (  0.99);

\path[draw=drawColor,line width= 0.4pt,line join=round,line cap=round,fill=fillColor] (447.92, 99.41) circle (  0.99);

\path[draw=drawColor,line width= 0.4pt,line join=round,line cap=round,fill=fillColor] (448.35,150.43) circle (  0.99);

\path[draw=drawColor,line width= 0.4pt,line join=round,line cap=round,fill=fillColor] (448.77,163.18) circle (  0.99);

\path[draw=drawColor,line width= 0.4pt,line join=round,line cap=round,fill=fillColor] (449.20,195.98) circle (  0.99);

\path[draw=drawColor,line width= 0.4pt,line join=round,line cap=round,fill=fillColor] (449.62,118.24) circle (  0.99);

\path[draw=drawColor,line width= 0.4pt,line join=round,line cap=round,fill=fillColor] (450.05,117.03) circle (  0.99);

\path[draw=drawColor,line width= 0.4pt,line join=round,line cap=round,fill=fillColor] (450.48,157.11) circle (  0.99);

\path[draw=drawColor,line width= 0.4pt,line join=round,line cap=round,fill=fillColor] (450.90,109.74) circle (  0.99);

\path[draw=drawColor,line width= 0.4pt,line join=round,line cap=round,fill=fillColor] (451.33,154.68) circle (  0.99);

\path[draw=drawColor,line width= 0.4pt,line join=round,line cap=round,fill=fillColor] (451.75,151.04) circle (  0.99);

\path[draw=drawColor,line width= 0.4pt,line join=round,line cap=round,fill=fillColor] (452.18,121.88) circle (  0.99);

\path[draw=drawColor,line width= 0.4pt,line join=round,line cap=round,fill=fillColor] (452.60,138.89) circle (  0.99);

\path[draw=drawColor,line width= 0.4pt,line join=round,line cap=round,fill=fillColor] (453.03,121.88) circle (  0.99);

\path[draw=drawColor,line width= 0.4pt,line join=round,line cap=round,fill=fillColor] (453.45,128.57) circle (  0.99);

\path[draw=drawColor,line width= 0.4pt,line join=round,line cap=round,fill=fillColor] (453.88,119.46) circle (  0.99);

\path[draw=drawColor,line width= 0.4pt,line join=round,line cap=round,fill=fillColor] (454.30,134.64) circle (  0.99);

\path[draw=drawColor,line width= 0.4pt,line join=round,line cap=round,fill=fillColor] (454.73,130.39) circle (  0.99);

\path[draw=drawColor,line width= 0.4pt,line join=round,line cap=round,fill=fillColor] (455.16,225.14) circle (  0.99);

\path[draw=drawColor,line width= 0.4pt,line join=round,line cap=round,fill=fillColor] (455.58,122.49) circle (  0.99);

\path[draw=drawColor,line width= 0.4pt,line join=round,line cap=round,fill=fillColor] (456.01,103.06) circle (  0.99);

\path[draw=drawColor,line width= 0.4pt,line join=round,line cap=round,fill=fillColor] (456.43,138.89) circle (  0.99);

\path[draw=drawColor,line width= 0.4pt,line join=round,line cap=round,fill=fillColor] (456.86,146.79) circle (  0.99);

\path[draw=drawColor,line width= 0.4pt,line join=round,line cap=round,fill=fillColor] (457.28,139.50) circle (  0.99);

\path[draw=drawColor,line width= 0.4pt,line join=round,line cap=round,fill=fillColor] (457.71,119.46) circle (  0.99);

\path[draw=drawColor,line width= 0.4pt,line join=round,line cap=round,fill=fillColor] (458.13,126.74) circle (  0.99);

\path[draw=drawColor,line width= 0.4pt,line join=round,line cap=round,fill=fillColor] (458.56,185.05) circle (  0.99);

\path[draw=drawColor,line width= 0.4pt,line join=round,line cap=round,fill=fillColor] (458.98,118.24) circle (  0.99);

\path[draw=drawColor,line width= 0.4pt,line join=round,line cap=round,fill=fillColor] (459.41,118.24) circle (  0.99);

\path[draw=drawColor,line width= 0.4pt,line join=round,line cap=round,fill=fillColor] (459.84,127.35) circle (  0.99);

\path[draw=drawColor,line width= 0.4pt,line join=round,line cap=round,fill=fillColor] (460.26,103.66) circle (  0.99);

\path[draw=drawColor,line width= 0.4pt,line join=round,line cap=round,fill=fillColor] (460.69,186.87) circle (  0.99);

\path[draw=drawColor,line width= 0.4pt,line join=round,line cap=round,fill=fillColor] (461.11,111.56) circle (  0.99);

\path[draw=drawColor,line width= 0.4pt,line join=round,line cap=round,fill=fillColor] (461.54,106.70) circle (  0.99);

\path[draw=drawColor,line width= 0.4pt,line join=round,line cap=round,fill=fillColor] (461.96,131.00) circle (  0.99);

\path[draw=drawColor,line width= 0.4pt,line join=round,line cap=round,fill=fillColor] (462.39,160.15) circle (  0.99);

\path[draw=drawColor,line width= 0.4pt,line join=round,line cap=round,fill=fillColor] (462.81,100.63) circle (  0.99);

\path[draw=drawColor,line width= 0.4pt,line join=round,line cap=round,fill=fillColor] (463.24,152.86) circle (  0.99);

\path[draw=drawColor,line width= 0.4pt,line join=round,line cap=round,fill=fillColor] (463.66,205.70) circle (  0.99);

\path[draw=drawColor,line width= 0.4pt,line join=round,line cap=round,fill=fillColor] (464.09,127.96) circle (  0.99);

\path[draw=drawColor,line width= 0.4pt,line join=round,line cap=round,fill=fillColor] (464.52,115.81) circle (  0.99);

\path[draw=drawColor,line width= 0.4pt,line join=round,line cap=round,fill=fillColor] (464.94,115.81) circle (  0.99);

\path[draw=drawColor,line width= 0.4pt,line join=round,line cap=round,fill=fillColor] (465.37,115.81) circle (  0.99);

\path[draw=drawColor,line width= 0.4pt,line join=round,line cap=round,fill=fillColor] (465.79,133.42) circle (  0.99);

\path[draw=drawColor,line width= 0.4pt,line join=round,line cap=round,fill=fillColor] (466.22,143.75) circle (  0.99);

\path[draw=drawColor,line width= 0.4pt,line join=round,line cap=round,fill=fillColor] (466.64,107.31) circle (  0.99);

\path[draw=drawColor,line width= 0.4pt,line join=round,line cap=round,fill=fillColor] (467.07,140.71) circle (  0.99);

\path[draw=drawColor,line width= 0.4pt,line join=round,line cap=round,fill=fillColor] (467.49,137.68) circle (  0.99);

\path[draw=drawColor,line width= 0.4pt,line join=round,line cap=round,fill=fillColor] (467.92,140.11) circle (  0.99);

\path[draw=drawColor,line width= 0.4pt,line join=round,line cap=round,fill=fillColor] (468.34,101.84) circle (  0.99);

\path[draw=drawColor,line width= 0.4pt,line join=round,line cap=round,fill=fillColor] (468.77,155.29) circle (  0.99);

\path[draw=drawColor,line width= 0.4pt,line join=round,line cap=round,fill=fillColor] (469.20,118.85) circle (  0.99);

\path[draw=drawColor,line width= 0.4pt,line join=round,line cap=round,fill=fillColor] (469.62,115.81) circle (  0.99);

\path[draw=drawColor,line width= 0.4pt,line join=round,line cap=round,fill=fillColor] (470.05,101.23) circle (  0.99);

\path[draw=drawColor,line width= 0.4pt,line join=round,line cap=round,fill=fillColor] (470.47,126.74) circle (  0.99);

\path[draw=drawColor,line width= 0.4pt,line join=round,line cap=round,fill=fillColor] (470.90,155.90) circle (  0.99);

\path[draw=drawColor,line width= 0.4pt,line join=round,line cap=round,fill=fillColor] (471.32,188.09) circle (  0.99);

\path[draw=drawColor,line width= 0.4pt,line join=round,line cap=round,fill=fillColor] (471.75,118.24) circle (  0.99);

\path[draw=drawColor,line width= 0.4pt,line join=round,line cap=round,fill=fillColor] (472.17, 94.55) circle (  0.99);

\path[draw=drawColor,line width= 0.4pt,line join=round,line cap=round,fill=fillColor] (472.60,126.74) circle (  0.99);

\path[draw=drawColor,line width= 0.4pt,line join=round,line cap=round,fill=fillColor] (473.02,113.99) circle (  0.99);

\path[draw=drawColor,line width= 0.4pt,line join=round,line cap=round,fill=fillColor] (473.45,109.13) circle (  0.99);

\path[draw=drawColor,line width= 0.4pt,line join=round,line cap=round,fill=fillColor] (473.88,132.21) circle (  0.99);

\path[draw=drawColor,line width= 0.4pt,line join=round,line cap=round,fill=fillColor] (474.30,121.28) circle (  0.99);

\path[draw=drawColor,line width= 0.4pt,line join=round,line cap=round,fill=fillColor] (474.73,121.28) circle (  0.99);

\path[draw=drawColor,line width= 0.4pt,line join=round,line cap=round,fill=fillColor] (475.15,151.65) circle (  0.99);

\path[draw=drawColor,line width= 0.4pt,line join=round,line cap=round,fill=fillColor] (475.58,135.85) circle (  0.99);

\path[draw=drawColor,line width= 0.4pt,line join=round,line cap=round,fill=fillColor] (476.00,178.98) circle (  0.99);

\path[draw=drawColor,line width= 0.4pt,line join=round,line cap=round,fill=fillColor] (476.43, 93.34) circle (  0.99);

\path[draw=drawColor,line width= 0.4pt,line join=round,line cap=round,fill=fillColor] (476.85,200.23) circle (  0.99);

\path[draw=drawColor,line width= 0.4pt,line join=round,line cap=round,fill=fillColor] (477.28,120.67) circle (  0.99);

\path[draw=drawColor,line width= 0.4pt,line join=round,line cap=round,fill=fillColor] (477.70,106.70) circle (  0.99);

\path[draw=drawColor,line width= 0.4pt,line join=round,line cap=round,fill=fillColor] (478.13, 98.81) circle (  0.99);

\path[draw=drawColor,line width= 0.4pt,line join=round,line cap=round,fill=fillColor] (478.56,123.10) circle (  0.99);

\path[draw=drawColor,line width= 0.4pt,line join=round,line cap=round,fill=fillColor] (478.98,112.77) circle (  0.99);

\path[draw=drawColor,line width= 0.4pt,line join=round,line cap=round,fill=fillColor] (479.41,122.49) circle (  0.99);

\path[draw=drawColor,line width= 0.4pt,line join=round,line cap=round,fill=fillColor] (479.83,135.85) circle (  0.99);

\path[draw=drawColor,line width= 0.4pt,line join=round,line cap=round,fill=fillColor] (480.26,127.96) circle (  0.99);

\path[draw=drawColor,line width= 0.4pt,line join=round,line cap=round,fill=fillColor] (480.68,113.99) circle (  0.99);

\path[draw=drawColor,line width= 0.4pt,line join=round,line cap=round,fill=fillColor] (481.11,256.72) circle (  0.99);

\path[draw=drawColor,line width= 0.4pt,line join=round,line cap=round,fill=fillColor] (481.53,100.02) circle (  0.99);

\path[draw=drawColor,line width= 0.4pt,line join=round,line cap=round,fill=fillColor] (481.96,115.81) circle (  0.99);

\path[draw=drawColor,line width= 0.4pt,line join=round,line cap=round,fill=fillColor] (482.38,121.28) circle (  0.99);

\path[draw=drawColor,line width= 0.4pt,line join=round,line cap=round,fill=fillColor] (482.81,148.61) circle (  0.99);

\path[draw=drawColor,line width= 0.4pt,line join=round,line cap=round,fill=fillColor] (483.24,121.88) circle (  0.99);

\path[draw=drawColor,line width= 0.4pt,line join=round,line cap=round,fill=fillColor] (483.66,234.25) circle (  0.99);

\path[draw=drawColor,line width= 0.4pt,line join=round,line cap=round,fill=fillColor] (484.09,129.17) circle (  0.99);

\path[draw=drawColor,line width= 0.4pt,line join=round,line cap=round,fill=fillColor] (484.51,352.68) circle (  0.99);

\path[draw=drawColor,line width= 0.4pt,line join=round,line cap=round,fill=fillColor] (484.94,134.64) circle (  0.99);

\path[draw=drawColor,line width= 0.4pt,line join=round,line cap=round,fill=fillColor] (485.36,109.13) circle (  0.99);

\path[draw=drawColor,line width= 0.4pt,line join=round,line cap=round,fill=fillColor] (485.79,183.84) circle (  0.99);

\path[draw=drawColor,line width= 0.4pt,line join=round,line cap=round,fill=fillColor] (486.21,200.84) circle (  0.99);

\path[draw=drawColor,line width= 0.4pt,line join=round,line cap=round,fill=fillColor] (486.64,135.25) circle (  0.99);

\path[draw=drawColor,line width= 0.4pt,line join=round,line cap=round,fill=fillColor] (487.06,134.03) circle (  0.99);

\path[draw=drawColor,line width= 0.4pt,line join=round,line cap=round,fill=fillColor] (487.49,114.60) circle (  0.99);

\path[draw=drawColor,line width= 0.4pt,line join=round,line cap=round,fill=fillColor] (487.92,162.58) circle (  0.99);

\path[draw=drawColor,line width= 0.4pt,line join=round,line cap=round,fill=fillColor] (488.34,151.65) circle (  0.99);

\path[draw=drawColor,line width= 0.4pt,line join=round,line cap=round,fill=fillColor] (488.77,117.63) circle (  0.99);

\path[draw=drawColor,line width= 0.4pt,line join=round,line cap=round,fill=fillColor] (489.19,118.24) circle (  0.99);

\path[draw=drawColor,line width= 0.4pt,line join=round,line cap=round,fill=fillColor] (489.62,127.35) circle (  0.99);

\path[draw=drawColor,line width= 0.4pt,line join=round,line cap=round,fill=fillColor] (490.04,120.06) circle (  0.99);

\path[draw=drawColor,line width= 0.4pt,line join=round,line cap=round,fill=fillColor] (490.47,156.50) circle (  0.99);

\path[draw=drawColor,line width= 0.4pt,line join=round,line cap=round,fill=fillColor] (490.89, 93.95) circle (  0.99);

\path[draw=drawColor,line width= 0.4pt,line join=round,line cap=round,fill=fillColor] (491.32,129.17) circle (  0.99);

\path[draw=drawColor,line width= 0.4pt,line join=round,line cap=round,fill=fillColor] (491.74,101.84) circle (  0.99);

\path[draw=drawColor,line width= 0.4pt,line join=round,line cap=round,fill=fillColor] (492.17,113.38) circle (  0.99);

\path[draw=drawColor,line width= 0.4pt,line join=round,line cap=round,fill=fillColor] (492.60,480.83) circle (  0.99);

\path[draw=drawColor,line width= 0.4pt,line join=round,line cap=round,fill=fillColor] (493.02,146.18) circle (  0.99);

\path[draw=drawColor,line width= 0.4pt,line join=round,line cap=round,fill=fillColor] (493.45,120.06) circle (  0.99);

\path[draw=drawColor,line width= 0.4pt,line join=round,line cap=round,fill=fillColor] (493.87,181.41) circle (  0.99);

\path[draw=drawColor,line width= 0.4pt,line join=round,line cap=round,fill=fillColor] (494.30,138.28) circle (  0.99);

\path[draw=drawColor,line width= 0.4pt,line join=round,line cap=round,fill=fillColor] (494.72,123.10) circle (  0.99);

\path[draw=drawColor,line width= 0.4pt,line join=round,line cap=round,fill=fillColor] (495.15,124.31) circle (  0.99);

\path[draw=drawColor,line width= 0.4pt,line join=round,line cap=round,fill=fillColor] (495.57,117.03) circle (  0.99);

\path[draw=drawColor,line width= 0.4pt,line join=round,line cap=round,fill=fillColor] (496.00,107.92) circle (  0.99);

\path[draw=drawColor,line width= 0.4pt,line join=round,line cap=round,fill=fillColor] (496.42,204.49) circle (  0.99);

\path[draw=drawColor,line width= 0.4pt,line join=round,line cap=round,fill=fillColor] (496.85,368.47) circle (  0.99);

\path[draw=drawColor,line width= 0.4pt,line join=round,line cap=round,fill=fillColor] (497.27,111.56) circle (  0.99);

\path[draw=drawColor,line width= 0.4pt,line join=round,line cap=round,fill=fillColor] (497.70, 93.34) circle (  0.99);

\path[draw=drawColor,line width= 0.4pt,line join=round,line cap=round,fill=fillColor] (498.13,165.61) circle (  0.99);
\end{scope}
\begin{scope}
\path[clip] (379.42,  0.00) rectangle (505.89,505.89);
\definecolor{drawColor}{RGB}{0,0,0}

\node[text=drawColor,anchor=base,inner sep=0pt, outer sep=0pt, scale=  1.32] at (460.47,495.79) {\bfseries \textsf{7.2}$^\circ$};

\node[text=drawColor,anchor=base,inner sep=0pt, outer sep=0pt, scale=  1.32] at (460.47,  5.54) {Vp};
\end{scope}
\begin{scope}
\path[clip] (  0.00,  0.00) rectangle (505.89,505.89);
\definecolor{drawColor}{RGB}{0,0,0}

\path[draw=drawColor,line width= 0.4pt,line join=round,line cap=round] (423.25, 35.64) -- (498.13, 35.64);

\path[draw=drawColor,line width= 0.4pt,line join=round,line cap=round] (423.25, 35.64) -- (423.25, 31.68);

\path[draw=drawColor,line width= 0.4pt,line join=round,line cap=round] (448.35, 35.64) -- (448.35, 31.68);

\path[draw=drawColor,line width= 0.4pt,line join=round,line cap=round] (473.88, 35.64) -- (473.88, 31.68);

\path[draw=drawColor,line width= 0.4pt,line join=round,line cap=round] (498.13, 35.64) -- (498.13, 31.68);

\node[text=drawColor,anchor=base,inner sep=0pt, outer sep=0pt, scale=  0.99] at (423.25, 21.38) {1};

\node[text=drawColor,anchor=base,inner sep=0pt, outer sep=0pt, scale=  0.99] at (448.35, 21.38) {60};

\node[text=drawColor,anchor=base,inner sep=0pt, outer sep=0pt, scale=  0.99] at (473.88, 21.38) {120};

\node[text=drawColor,anchor=base,inner sep=0pt, outer sep=0pt, scale=  0.99] at (498.13, 21.38) {177};

\path[draw=drawColor,line width= 0.4pt,line join=round,line cap=round] (419.81, 52.65) -- (419.81,477.80);

\path[draw=drawColor,line width= 0.4pt,line join=round,line cap=round] (419.81, 52.65) -- (415.85, 52.65);

\path[draw=drawColor,line width= 0.4pt,line join=round,line cap=round] (419.81,113.38) -- (415.85,113.38);

\path[draw=drawColor,line width= 0.4pt,line join=round,line cap=round] (419.81,174.12) -- (415.85,174.12);

\path[draw=drawColor,line width= 0.4pt,line join=round,line cap=round] (419.81,234.85) -- (415.85,234.85);

\path[draw=drawColor,line width= 0.4pt,line join=round,line cap=round] (419.81,295.59) -- (415.85,295.59);

\path[draw=drawColor,line width= 0.4pt,line join=round,line cap=round] (419.81,356.32) -- (415.85,356.32);

\path[draw=drawColor,line width= 0.4pt,line join=round,line cap=round] (419.81,417.06) -- (415.85,417.06);

\path[draw=drawColor,line width= 0.4pt,line join=round,line cap=round] (419.81,477.80) -- (415.85,477.80);

\node[text=drawColor,anchor=base east,inner sep=0pt, outer sep=0pt, scale=  0.99] at (411.89, 49.24) {0};

\node[text=drawColor,anchor=base east,inner sep=0pt, outer sep=0pt, scale=  0.99] at (411.89,109.97) {100};

\node[text=drawColor,anchor=base east,inner sep=0pt, outer sep=0pt, scale=  0.99] at (411.89,170.71) {200};

\node[text=drawColor,anchor=base east,inner sep=0pt, outer sep=0pt, scale=  0.99] at (411.89,231.44) {300};

\node[text=drawColor,anchor=base east,inner sep=0pt, outer sep=0pt, scale=  0.99] at (411.89,292.18) {400};

\node[text=drawColor,anchor=base east,inner sep=0pt, outer sep=0pt, scale=  0.99] at (411.89,352.92) {500};

\node[text=drawColor,anchor=base east,inner sep=0pt, outer sep=0pt, scale=  0.99] at (411.89,413.65) {600};

\node[text=drawColor,anchor=base east,inner sep=0pt, outer sep=0pt, scale=  0.99] at (411.89,474.39) {700};
\end{scope}
\begin{scope}
\path[clip] (419.81, 35.64) rectangle (501.14,494.80);
\definecolor{drawColor}{RGB}{0,0,0}

\path[draw=drawColor,line width= 0.4pt,line join=round,line cap=round] (419.81,135.32) -- (501.14,135.32);
\end{scope}
\end{tikzpicture}

	\end{adjustbox}
	\caption[Streudiagramme der $82\,\%$-Erkennungsschwellen in der \gls{ssauf}]{Streudiagramme der $82\,\%$-Erkennungsschwellen für horizontale Bewegung in der \gls{ssauf}. Die horizontale Linie kennzeichnet jeweils den Mittelwert innerhalb einer Bedingung (siehe \autoref{tab:spatial_suppression_descriptives}). Siehe \autoref{cha:Anhang_A} für eine Beschreibung der Ausreisserkontrolle. Vp = Versuchsperson.}
	\label{fig:spatial_suppression_scatterplot}
\end{figure}

Als erstes wurde geprüft, ob die experimentelle Manipulation (die Mustergrösse) einen Einfluss auf die abhängige Variable (die $82\,\%$-Er\-ken\-nungs\-schwel\-le) ausübte. Dafür wurde eine einfaktorielle Varianzanalyse mit Messwiederholung\footnote{Die Abweichung der Daten von der Normalverteilung (siehe Kennwerte zur Verteilung in \autoref{tab:spatial_suppression_descriptives}) erforderten eigentlich verteilungsfreien Analyseverfahren. Da die Ergebnisse dieser nonparametrischen Analyseverfahren aber nicht bedeutend von den mit parametrischen Verfahren ermittelten Ergebnissen abwichen, werden im Folgenden die Ergebnisse der traditionellen (parametrischen) Verfahren berichtet. Siehe \autoref{cha:Anhang_B} für die Analyse der Aufgaben mittels nonparametrischer Verfahren.}
gerechnet. Weil Sphärizität gemäss einem Mauchly-Test nicht gegeben war, $\upchi^2(5)=202.12$, $p<.001$, wurden die Freiheitsgrade des \textit{F}-Tests mit der Greenhouse-Geisser-Methode korrigiert ($\hat{\varepsilon}=.55$).
Der \textit{F}-Test hat ergeben, dass die Unterschiede zwischen den Bedingungsmittelwerten signifikant von 0 abwichen, $F(1.65,\,290.40)=275.26$, $p<.001$, $\eta_{G}^2=.27$. Der Effekt der Mustergrösse auf die $82\,\%$-Erkennungsschwelle konnte dabei gemäss generalisiertem $\eta_{G}^2$ \citep{Olejnik2003} als gross bezeichnet werden \citep[S. 383]{Bakeman2005}.
Um zu erfahren, ob sich alle oder nur bestimmte Mittelwertpaare signifikant voneinander unterschieden, wurden post hoc alle Mittelwerte miteinander verglichen.
Tukey-Tests haben gezeigt, dass sich alle Mittelwertpaare signifikant voneinander unterschieden (alle \textit{p}s $<.001$).
Die $82\,\%$-Er\-ken\-nungs\-schwel\-len der \glspl{vp} verschlechterte sich also mit zunehmender Mustergrösse signifikant.
Die Effektstärken für die Mittelwertsunterschiede wurden mit Cohens \textit{d} für abhängige Stichproben \citep{Gibbons1993} bestimmt. 
Dabei hat sich gezeigt, dass die Effektstärken im mittleren bis hohen Bereich \citep[][S. 40]{Cohen1988} lagen (siehe \autoref{tab:spatial_suppression_effect_sizes}). 

\begin{table}[htbp]
	\centering
	\setlength{\tabcolsep}{10pt}
	\captionsetup{labelsep = none}
	\caption[Effektstärken für die Mittelwertsunterschiede in der \gls{ssauf}]{\newline \textit{Effektstärken (Cohens \textit{d} für abhängige Stichproben) der Mittelwertunterschiede in der \gls{ssauf}} \vspace{.2cm}}
	\label{tab:spatial_suppression_effect_sizes}
	\sisetup{table-number-alignment = center}
	\begin{threeparttable}
		\begin{tabular}{
				l
				S[table-format = 1.2]
				S[table-format = 1.2]
				S[table-format = 1.2]
				>{\centering\arraybackslash}p{1.2cm}
			}
			\hline
			
			\multicolumn{1}{c}{Bedingung}		&	\(1.8^{\circ}\)		&	\(3.6^{\circ}\)		&	\(5.4^{\circ}\)		\\
			\hline
			$1.8^{\circ}$	&						&						&						\\
			$3.6^{\circ}$	&	0.51				&						&						\\
			$5.4^{\circ}$	&	1.12				&	1.07				&						\\
			$7.2^{\circ}$	&	1.39				&	1.42				&	1.08					\\

			\hline
			
		\end{tabular}%
		%}
		\begin{tablenotes}[flushleft]
			\footnotesize				% font size
			\setlength\labelsep{0pt}	% no indent on second line
			\item \textit{Anmerkung}. Alle Mittelwertsunterschiede waren statistisch signifikant ($p<.001$).
		\end{tablenotes}
		
	\end{threeparttable}
\end{table}

\clearpage
Produkt-Moment-Korrelationen zwischen den vier Bedingungen der \gls{ssauf} sind in \autoref{tab:product_moment_correlations_manifest} abgetragen. Sie deuteten ausnahmslos auf stark positive Zusammenhänge zwischen den Bedingungen hin.

Der \gls{si} wies einen Mittelwert $\pm$ Standardabweichung von $0.222\,\pm\,0.160$ auf (Minimum $= -0.185$, Maximum $= 0.886$). 
Die Verteilung des \gls{si} (siehe \autoref{fig:suppression_index_density}) hatte eine Schiefe von $0.91$ und eine Kurtosis von $1.80$ und wich damit gemäss einem Shapiro-Wilk-Test signifikant von der Normalverteilung ab ($p<.001$).

\begin{figure}[htbp]
	\centering
%	\captionsetup{font = small}
	\begin{adjustbox}{width=1\textwidth}
		% Created by tikzDevice version 0.10.1 on 2016-07-21 17:26:20
% !TEX encoding = UTF-8 Unicode
\begin{tikzpicture}[x=1pt,y=1pt]
\definecolor{fillColor}{RGB}{255,255,255}
\path[use as bounding box,fill=fillColor,fill opacity=0.00] (0,0) rectangle (361.35,144.54);
\begin{scope}
\path[clip] ( 48.00, 48.00) rectangle (361.35,120.54);
\definecolor{drawColor}{RGB}{0,0,0}

\path[draw=drawColor,line width= 0.4pt,line join=round,line cap=round] ( 91.73, 50.71) --
	( 92.21, 50.72) --
	( 92.69, 50.73) --
	( 93.18, 50.74) --
	( 93.66, 50.76) --
	( 94.14, 50.79) --
	( 94.63, 50.82) --
	( 95.11, 50.86) --
	( 95.59, 50.91) --
	( 96.08, 50.97) --
	( 96.56, 51.04) --
	( 97.04, 51.12) --
	( 97.52, 51.21) --
	( 98.01, 51.32) --
	( 98.49, 51.43) --
	( 98.97, 51.55) --
	( 99.46, 51.68) --
	( 99.94, 51.81) --
	(100.42, 51.95) --
	(100.91, 52.08) --
	(101.39, 52.20) --
	(101.87, 52.31) --
	(102.35, 52.40) --
	(102.84, 52.48) --
	(103.32, 52.54) --
	(103.80, 52.57) --
	(104.29, 52.58) --
	(104.77, 52.56) --
	(105.25, 52.52) --
	(105.74, 52.45) --
	(106.22, 52.37) --
	(106.70, 52.27) --
	(107.18, 52.16) --
	(107.67, 52.04) --
	(108.15, 51.92) --
	(108.63, 51.80) --
	(109.12, 51.69) --
	(109.60, 51.59) --
	(110.08, 51.50) --
	(110.57, 51.43) --
	(111.05, 51.38) --
	(111.53, 51.35) --
	(112.02, 51.35) --
	(112.50, 51.36) --
	(112.98, 51.40) --
	(113.46, 51.45) --
	(113.95, 51.53) --
	(114.43, 51.62) --
	(114.91, 51.73) --
	(115.40, 51.85) --
	(115.88, 51.97) --
	(116.36, 52.09) --
	(116.85, 52.21) --
	(117.33, 52.33) --
	(117.81, 52.44) --
	(118.29, 52.53) --
	(118.78, 52.60) --
	(119.26, 52.66) --
	(119.74, 52.71) --
	(120.23, 52.73) --
	(120.71, 52.75) --
	(121.19, 52.75) --
	(121.68, 52.74) --
	(122.16, 52.74) --
	(122.64, 52.74) --
	(123.12, 52.74) --
	(123.61, 52.77) --
	(124.09, 52.81) --
	(124.57, 52.88) --
	(125.06, 52.97) --
	(125.54, 53.10) --
	(126.02, 53.27) --
	(126.51, 53.47) --
	(126.99, 53.71) --
	(127.47, 53.98) --
	(127.95, 54.29) --
	(128.44, 54.63) --
	(128.92, 55.01) --
	(129.40, 55.40) --
	(129.89, 55.83) --
	(130.37, 56.28) --
	(130.85, 56.74) --
	(131.34, 57.21) --
	(131.82, 57.69) --
	(132.30, 58.17) --
	(132.79, 58.64) --
	(133.27, 59.10) --
	(133.75, 59.55) --
	(134.23, 59.96) --
	(134.72, 60.35) --
	(135.20, 60.71) --
	(135.68, 61.02) --
	(136.17, 61.30) --
	(136.65, 61.54) --
	(137.13, 61.74) --
	(137.62, 61.91) --
	(138.10, 62.03) --
	(138.58, 62.13) --
	(139.06, 62.20) --
	(139.55, 62.26) --
	(140.03, 62.31) --
	(140.51, 62.36) --
	(141.00, 62.42) --
	(141.48, 62.50) --
	(141.96, 62.61) --
	(142.45, 62.76) --
	(142.93, 62.95) --
	(143.41, 63.18) --
	(143.89, 63.46) --
	(144.38, 63.79) --
	(144.86, 64.17) --
	(145.34, 64.60) --
	(145.83, 65.07) --
	(146.31, 65.58) --
	(146.79, 66.12) --
	(147.28, 66.68) --
	(147.76, 67.25) --
	(148.24, 67.84) --
	(148.72, 68.42) --
	(149.21, 69.01) --
	(149.69, 69.59) --
	(150.17, 70.16) --
	(150.66, 70.72) --
	(151.14, 71.26) --
	(151.62, 71.80) --
	(152.11, 72.33) --
	(152.59, 72.86) --
	(153.07, 73.38) --
	(153.56, 73.90) --
	(154.04, 74.43) --
	(154.52, 74.97) --
	(155.00, 75.52) --
	(155.49, 76.10) --
	(155.97, 76.69) --
	(156.45, 77.31) --
	(156.94, 77.97) --
	(157.42, 78.65) --
	(157.90, 79.38) --
	(158.39, 80.15) --
	(158.87, 80.97) --
	(159.35, 81.84) --
	(159.83, 82.77) --
	(160.32, 83.74) --
	(160.80, 84.77) --
	(161.28, 85.86) --
	(161.77, 87.01) --
	(162.25, 88.23) --
	(162.73, 89.51) --
	(163.22, 90.86) --
	(163.70, 92.27) --
	(164.18, 93.73) --
	(164.66, 95.26) --
	(165.15, 96.82) --
	(165.63, 98.42) --
	(166.11,100.03) --
	(166.60,101.64) --
	(167.08,103.21) --
	(167.56,104.72) --
	(168.05,106.14) --
	(168.53,107.43) --
	(169.01,108.57) --
	(169.49,109.54) --
	(169.98,110.31) --
	(170.46,110.84) --
	(170.94,111.12) --
	(171.43,111.17) --
	(171.91,111.01) --
	(172.39,110.65) --
	(172.88,110.12) --
	(173.36,109.45) --
	(173.84,108.69) --
	(174.33,107.87) --
	(174.81,107.06) --
	(175.29,106.30) --
	(175.77,105.63) --
	(176.26,105.09) --
	(176.74,104.69) --
	(177.22,104.45) --
	(177.71,104.38) --
	(178.19,104.47) --
	(178.67,104.71) --
	(179.16,105.07) --
	(179.64,105.51) --
	(180.12,105.99) --
	(180.60,106.47) --
	(181.09,106.91) --
	(181.57,107.29) --
	(182.05,107.57) --
	(182.54,107.74) --
	(183.02,107.75) --
	(183.50,107.62) --
	(183.99,107.36) --
	(184.47,106.96) --
	(184.95,106.43) --
	(185.43,105.79) --
	(185.92,105.05) --
	(186.40,104.23) --
	(186.88,103.31) --
	(187.37,102.32) --
	(187.85,101.27) --
	(188.33,100.15) --
	(188.82, 98.98) --
	(189.30, 97.75) --
	(189.78, 96.47) --
	(190.26, 95.14) --
	(190.75, 93.76) --
	(191.23, 92.34) --
	(191.71, 90.91) --
	(192.20, 89.47) --
	(192.68, 88.04) --
	(193.16, 86.65) --
	(193.65, 85.30) --
	(194.13, 84.03) --
	(194.61, 82.86) --
	(195.10, 81.80) --
	(195.58, 80.88) --
	(196.06, 80.10) --
	(196.54, 79.46) --
	(197.03, 78.97) --
	(197.51, 78.62) --
	(197.99, 78.41) --
	(198.48, 78.33) --
	(198.96, 78.36) --
	(199.44, 78.49) --
	(199.93, 78.70) --
	(200.41, 78.96) --
	(200.89, 79.24) --
	(201.37, 79.54) --
	(201.86, 79.83) --
	(202.34, 80.09) --
	(202.82, 80.31) --
	(203.31, 80.47) --
	(203.79, 80.56) --
	(204.27, 80.56) --
	(204.76, 80.48) --
	(205.24, 80.31) --
	(205.72, 80.05) --
	(206.20, 79.69) --
	(206.69, 79.23) --
	(207.17, 78.68) --
	(207.65, 78.03) --
	(208.14, 77.30) --
	(208.62, 76.49) --
	(209.10, 75.62) --
	(209.59, 74.70) --
	(210.07, 73.75) --
	(210.55, 72.77) --
	(211.04, 71.80) --
	(211.52, 70.85) --
	(212.00, 69.94) --
	(212.48, 69.10) --
	(212.97, 68.34) --
	(213.45, 67.68) --
	(213.93, 67.13) --
	(214.42, 66.69) --
	(214.90, 66.39) --
	(215.38, 66.23) --
	(215.87, 66.20) --
	(216.35, 66.30) --
	(216.83, 66.50) --
	(217.31, 66.80) --
	(217.80, 67.18) --
	(218.28, 67.63) --
	(218.76, 68.12) --
	(219.25, 68.63) --
	(219.73, 69.15) --
	(220.21, 69.65) --
	(220.70, 70.11) --
	(221.18, 70.52) --
	(221.66, 70.88) --
	(222.14, 71.16) --
	(222.63, 71.37) --
	(223.11, 71.49) --
	(223.59, 71.52) --
	(224.08, 71.46) --
	(224.56, 71.31) --
	(225.04, 71.08) --
	(225.53, 70.77) --
	(226.01, 70.39) --
	(226.49, 69.94) --
	(226.97, 69.44) --
	(227.46, 68.88) --
	(227.94, 68.27) --
	(228.42, 67.62) --
	(228.91, 66.96) --
	(229.39, 66.27) --
	(229.87, 65.59) --
	(230.36, 64.90) --
	(230.84, 64.22) --
	(231.32, 63.57) --
	(231.81, 62.94) --
	(232.29, 62.36) --
	(232.77, 61.81) --
	(233.25, 61.32) --
	(233.74, 60.89) --
	(234.22, 60.52) --
	(234.70, 60.21) --
	(235.19, 59.98) --
	(235.67, 59.82) --
	(236.15, 59.76) --
	(236.64, 59.77) --
	(237.12, 59.87) --
	(237.60, 60.04) --
	(238.08, 60.29) --
	(238.57, 60.62) --
	(239.05, 61.01) --
	(239.53, 61.46) --
	(240.02, 61.96) --
	(240.50, 62.49) --
	(240.98, 63.03) --
	(241.47, 63.56) --
	(241.95, 64.07) --
	(242.43, 64.53) --
	(242.91, 64.94) --
	(243.40, 65.28) --
	(243.88, 65.52) --
	(244.36, 65.65) --
	(244.85, 65.68) --
	(245.33, 65.59) --
	(245.81, 65.40) --
	(246.30, 65.10) --
	(246.78, 64.71) --
	(247.26, 64.24) --
	(247.74, 63.70) --
	(248.23, 63.09) --
	(248.71, 62.45) --
	(249.19, 61.80) --
	(249.68, 61.14) --
	(250.16, 60.48) --
	(250.64, 59.85) --
	(251.13, 59.25) --
	(251.61, 58.68) --
	(252.09, 58.15) --
	(252.58, 57.67) --
	(253.06, 57.22) --
	(253.54, 56.82) --
	(254.02, 56.46) --
	(254.51, 56.14) --
	(254.99, 55.85) --
	(255.47, 55.59) --
	(255.96, 55.37) --
	(256.44, 55.18) --
	(256.92, 55.01) --
	(257.41, 54.88) --
	(257.89, 54.76) --
	(258.37, 54.67) --
	(258.85, 54.61) --
	(259.34, 54.56) --
	(259.82, 54.52) --
	(260.30, 54.50) --
	(260.79, 54.48) --
	(261.27, 54.47) --
	(261.75, 54.45) --
	(262.24, 54.43) --
	(262.72, 54.40) --
	(263.20, 54.35) --
	(263.68, 54.28) --
	(264.17, 54.19) --
	(264.65, 54.09) --
	(265.13, 53.96) --
	(265.62, 53.81) --
	(266.10, 53.64) --
	(266.58, 53.46) --
	(267.07, 53.27) --
	(267.55, 53.08) --
	(268.03, 52.88) --
	(268.51, 52.68) --
	(269.00, 52.49) --
	(269.48, 52.31) --
	(269.96, 52.15) --
	(270.45, 52.01) --
	(270.93, 51.89) --
	(271.41, 51.79) --
	(271.90, 51.72) --
	(272.38, 51.68) --
	(272.86, 51.67) --
	(273.35, 51.69) --
	(273.83, 51.73) --
	(274.31, 51.78) --
	(274.79, 51.86) --
	(275.28, 51.95) --
	(275.76, 52.05) --
	(276.24, 52.15) --
	(276.73, 52.26) --
	(277.21, 52.35) --
	(277.69, 52.44) --
	(278.18, 52.50) --
	(278.66, 52.55) --
	(279.14, 52.58) --
	(279.62, 52.58) --
	(280.11, 52.56) --
	(280.59, 52.51) --
	(281.07, 52.45) --
	(281.56, 52.36) --
	(282.04, 52.25) --
	(282.52, 52.13) --
	(283.01, 52.01) --
	(283.49, 51.88) --
	(283.97, 51.74) --
	(284.45, 51.61) --
	(284.94, 51.49) --
	(285.42, 51.37) --
	(285.90, 51.26) --
	(286.39, 51.16) --
	(286.87, 51.08) --
	(287.35, 51.00) --
	(287.84, 50.94) --
	(288.32, 50.88) --
	(288.80, 50.84) --
	(289.28, 50.80) --
	(289.77, 50.78) --
	(290.25, 50.75) --
	(290.73, 50.74) --
	(291.22, 50.72) --
	(291.70, 50.71) --
	(292.18, 50.71) --
	(292.67, 50.70) --
	(293.15, 50.70) --
	(293.63, 50.70) --
	(294.12, 50.70) --
	(294.60, 50.70) --
	(295.08, 50.70) --
	(295.56, 50.71) --
	(296.05, 50.71) --
	(296.53, 50.72) --
	(297.01, 50.73) --
	(297.50, 50.75) --
	(297.98, 50.77) --
	(298.46, 50.80) --
	(298.95, 50.84) --
	(299.43, 50.88) --
	(299.91, 50.94) --
	(300.39, 51.00) --
	(300.88, 51.07) --
	(301.36, 51.16) --
	(301.84, 51.25) --
	(302.33, 51.36) --
	(302.81, 51.48) --
	(303.29, 51.60) --
	(303.78, 51.73) --
	(304.26, 51.86) --
	(304.74, 52.00) --
	(305.22, 52.12) --
	(305.71, 52.24) --
	(306.19, 52.35) --
	(306.67, 52.43) --
	(307.16, 52.50) --
	(307.64, 52.55) --
	(308.12, 52.57) --
	(308.61, 52.57) --
	(309.09, 52.54) --
	(309.57, 52.49) --
	(310.05, 52.41) --
	(310.54, 52.32) --
	(311.02, 52.21) --
	(311.50, 52.09) --
	(311.99, 51.97) --
	(312.47, 51.84) --
	(312.95, 51.71) --
	(313.44, 51.58) --
	(313.92, 51.47) --
	(314.40, 51.37) --
	(314.89, 51.28) --
	(315.37, 51.20) --
	(315.85, 51.14) --
	(316.33, 51.10) --
	(316.82, 51.07) --
	(317.30, 51.07) --
	(317.78, 51.08) --
	(318.27, 51.11) --
	(318.75, 51.16) --
	(319.23, 51.23) --
	(319.72, 51.31) --
	(320.20, 51.40) --
	(320.68, 51.51) --
	(321.16, 51.63) --
	(321.65, 51.76) --
	(322.13, 51.89) --
	(322.61, 52.02) --
	(323.10, 52.14) --
	(323.58, 52.26) --
	(324.06, 52.36) --
	(324.55, 52.45) --
	(325.03, 52.51) --
	(325.51, 52.56) --
	(325.99, 52.58) --
	(326.48, 52.57) --
	(326.96, 52.54) --
	(327.44, 52.48) --
	(327.93, 52.40) --
	(328.41, 52.31) --
	(328.89, 52.20) --
	(329.38, 52.08) --
	(329.86, 51.95) --
	(330.34, 51.81) --
	(330.82, 51.68) --
	(331.31, 51.55) --
	(331.79, 51.43) --
	(332.27, 51.32) --
	(332.76, 51.21) --
	(333.24, 51.12) --
	(333.72, 51.04) --
	(334.21, 50.97) --
	(334.69, 50.91) --
	(335.17, 50.86) --
	(335.66, 50.82) --
	(336.14, 50.79) --
	(336.62, 50.76) --
	(337.10, 50.74) --
	(337.59, 50.73) --
	(338.07, 50.72) --
	(338.55, 50.71);
\end{scope}
\begin{scope}
\path[clip] (  0.00,  0.00) rectangle (361.35,144.54);
\definecolor{drawColor}{RGB}{0,0,0}

\node[text=drawColor,anchor=base,inner sep=0pt, outer sep=0pt, scale=  1.00] at (204.67,  8.40) {Suppression-Index};
\end{scope}
\begin{scope}
\path[clip] (  0.00,  0.00) rectangle (361.35,144.54);
\definecolor{drawColor}{RGB}{0,0,0}

\path[draw=drawColor,line width= 0.4pt,line join=round,line cap=round] ( 59.61, 48.00) -- (349.74, 48.00);

\path[draw=drawColor,line width= 0.4pt,line join=round,line cap=round] ( 59.61, 48.00) -- ( 59.61, 42.00);

\path[draw=drawColor,line width= 0.4pt,line join=round,line cap=round] (101.05, 48.00) -- (101.05, 42.00);

\path[draw=drawColor,line width= 0.4pt,line join=round,line cap=round] (142.50, 48.00) -- (142.50, 42.00);

\path[draw=drawColor,line width= 0.4pt,line join=round,line cap=round] (183.95, 48.00) -- (183.95, 42.00);

\path[draw=drawColor,line width= 0.4pt,line join=round,line cap=round] (225.40, 48.00) -- (225.40, 42.00);

\path[draw=drawColor,line width= 0.4pt,line join=round,line cap=round] (266.85, 48.00) -- (266.85, 42.00);

\path[draw=drawColor,line width= 0.4pt,line join=round,line cap=round] (308.30, 48.00) -- (308.30, 42.00);

\path[draw=drawColor,line width= 0.4pt,line join=round,line cap=round] (349.74, 48.00) -- (349.74, 42.00);

\node[text=drawColor,anchor=base,inner sep=0pt, outer sep=0pt, scale=  1.00] at ( 59.61, 30.00) {-0.4};

\node[text=drawColor,anchor=base,inner sep=0pt, outer sep=0pt, scale=  1.00] at (101.05, 30.00) {-0.2};

\node[text=drawColor,anchor=base,inner sep=0pt, outer sep=0pt, scale=  1.00] at (142.50, 30.00) {0.0};

\node[text=drawColor,anchor=base,inner sep=0pt, outer sep=0pt, scale=  1.00] at (183.95, 30.00) {0.2};

\node[text=drawColor,anchor=base,inner sep=0pt, outer sep=0pt, scale=  1.00] at (225.40, 30.00) {0.4};

\node[text=drawColor,anchor=base,inner sep=0pt, outer sep=0pt, scale=  1.00] at (266.85, 30.00) {0.6};

\node[text=drawColor,anchor=base,inner sep=0pt, outer sep=0pt, scale=  1.00] at (308.30, 30.00) {0.8};

\node[text=drawColor,anchor=base,inner sep=0pt, outer sep=0pt, scale=  1.00] at (349.74, 30.00) {1.0};

\path[draw=drawColor,line width= 0.2pt,line join=round,line cap=round] (104.16, 48.00) -- (104.16, 55.25);

\path[draw=drawColor,line width= 0.2pt,line join=round,line cap=round] (119.71, 48.00) -- (119.71, 55.25);

\path[draw=drawColor,line width= 0.2pt,line join=round,line cap=round] (129.45, 48.00) -- (129.45, 55.25);

\path[draw=drawColor,line width= 0.2pt,line join=round,line cap=round] (134.21, 48.00) -- (134.21, 55.25);

\path[draw=drawColor,line width= 0.2pt,line join=round,line cap=round] (134.63, 48.00) -- (134.63, 55.25);

\path[draw=drawColor,line width= 0.2pt,line join=round,line cap=round] (135.87, 48.00) -- (135.87, 55.25);

\path[draw=drawColor,line width= 0.2pt,line join=round,line cap=round] (138.98, 48.00) -- (138.98, 55.25);

\path[draw=drawColor,line width= 0.2pt,line join=round,line cap=round] (139.19, 48.00) -- (139.19, 55.25);

\path[draw=drawColor,line width= 0.2pt,line join=round,line cap=round] (140.43, 48.00) -- (140.43, 55.25);

\path[draw=drawColor,line width= 0.2pt,line join=round,line cap=round] (143.33, 48.00) -- (143.33, 55.25);

\path[draw=drawColor,line width= 0.2pt,line join=round,line cap=round] (145.40, 48.00) -- (145.40, 55.25);

\path[draw=drawColor,line width= 0.2pt,line join=round,line cap=round] (147.06, 48.00) -- (147.06, 55.25);

\path[draw=drawColor,line width= 0.2pt,line join=round,line cap=round] (148.31, 48.00) -- (148.31, 55.25);

\path[draw=drawColor,line width= 0.2pt,line join=round,line cap=round] (149.55, 48.00) -- (149.55, 55.25);

\path[draw=drawColor,line width= 0.2pt,line join=round,line cap=round] (149.76, 48.00) -- (149.76, 55.25);

\path[draw=drawColor,line width= 0.2pt,line join=round,line cap=round] (150.38, 48.00) -- (150.38, 55.25);

\path[draw=drawColor,line width= 0.2pt,line join=round,line cap=round] (151.62, 48.00) -- (151.62, 55.25);

\path[draw=drawColor,line width= 0.2pt,line join=round,line cap=round] (152.86, 48.00) -- (152.86, 55.25);

\path[draw=drawColor,line width= 0.2pt,line join=round,line cap=round] (153.90, 48.00) -- (153.90, 55.25);

\path[draw=drawColor,line width= 0.2pt,line join=round,line cap=round] (155.14, 48.00) -- (155.14, 55.25);

\path[draw=drawColor,line width= 0.2pt,line join=round,line cap=round] (155.35, 48.00) -- (155.35, 55.25);

\path[draw=drawColor,line width= 0.2pt,line join=round,line cap=round] (156.18, 48.00) -- (156.18, 55.25);

\path[draw=drawColor,line width= 0.2pt,line join=round,line cap=round] (156.18, 48.00) -- (156.18, 55.25);

\path[draw=drawColor,line width= 0.2pt,line join=round,line cap=round] (157.01, 48.00) -- (157.01, 55.25);

\path[draw=drawColor,line width= 0.2pt,line join=round,line cap=round] (158.46, 48.00) -- (158.46, 55.25);

\path[draw=drawColor,line width= 0.2pt,line join=round,line cap=round] (159.50, 48.00) -- (159.50, 55.25);

\path[draw=drawColor,line width= 0.2pt,line join=round,line cap=round] (160.12, 48.00) -- (160.12, 55.25);

\path[draw=drawColor,line width= 0.2pt,line join=round,line cap=round] (160.12, 48.00) -- (160.12, 55.25);

\path[draw=drawColor,line width= 0.2pt,line join=round,line cap=round] (160.74, 48.00) -- (160.74, 55.25);

\path[draw=drawColor,line width= 0.2pt,line join=round,line cap=round] (162.19, 48.00) -- (162.19, 55.25);

\path[draw=drawColor,line width= 0.2pt,line join=round,line cap=round] (162.40, 48.00) -- (162.40, 55.25);

\path[draw=drawColor,line width= 0.2pt,line join=round,line cap=round] (163.02, 48.00) -- (163.02, 55.25);

\path[draw=drawColor,line width= 0.2pt,line join=round,line cap=round] (163.43, 48.00) -- (163.43, 55.25);

\path[draw=drawColor,line width= 0.2pt,line join=round,line cap=round] (163.64, 48.00) -- (163.64, 55.25);

\path[draw=drawColor,line width= 0.2pt,line join=round,line cap=round] (163.64, 48.00) -- (163.64, 55.25);

\path[draw=drawColor,line width= 0.2pt,line join=round,line cap=round] (163.85, 48.00) -- (163.85, 55.25);

\path[draw=drawColor,line width= 0.2pt,line join=round,line cap=round] (164.68, 48.00) -- (164.68, 55.25);

\path[draw=drawColor,line width= 0.2pt,line join=round,line cap=round] (166.13, 48.00) -- (166.13, 55.25);

\path[draw=drawColor,line width= 0.2pt,line join=round,line cap=round] (166.54, 48.00) -- (166.54, 55.25);

\path[draw=drawColor,line width= 0.2pt,line join=round,line cap=round] (166.54, 48.00) -- (166.54, 55.25);

\path[draw=drawColor,line width= 0.2pt,line join=round,line cap=round] (166.54, 48.00) -- (166.54, 55.25);

\path[draw=drawColor,line width= 0.2pt,line join=round,line cap=round] (167.37, 48.00) -- (167.37, 55.25);

\path[draw=drawColor,line width= 0.2pt,line join=round,line cap=round] (168.20, 48.00) -- (168.20, 55.25);

\path[draw=drawColor,line width= 0.2pt,line join=round,line cap=round] (168.20, 48.00) -- (168.20, 55.25);

\path[draw=drawColor,line width= 0.2pt,line join=round,line cap=round] (169.03, 48.00) -- (169.03, 55.25);

\path[draw=drawColor,line width= 0.2pt,line join=round,line cap=round] (169.24, 48.00) -- (169.24, 55.25);

\path[draw=drawColor,line width= 0.2pt,line join=round,line cap=round] (169.24, 48.00) -- (169.24, 55.25);

\path[draw=drawColor,line width= 0.2pt,line join=round,line cap=round] (169.65, 48.00) -- (169.65, 55.25);

\path[draw=drawColor,line width= 0.2pt,line join=round,line cap=round] (169.65, 48.00) -- (169.65, 55.25);

\path[draw=drawColor,line width= 0.2pt,line join=round,line cap=round] (170.07, 48.00) -- (170.07, 55.25);

\path[draw=drawColor,line width= 0.2pt,line join=round,line cap=round] (170.48, 48.00) -- (170.48, 55.25);

\path[draw=drawColor,line width= 0.2pt,line join=round,line cap=round] (170.48, 48.00) -- (170.48, 55.25);

\path[draw=drawColor,line width= 0.2pt,line join=round,line cap=round] (170.48, 48.00) -- (170.48, 55.25);

\path[draw=drawColor,line width= 0.2pt,line join=round,line cap=round] (170.89, 48.00) -- (170.89, 55.25);

\path[draw=drawColor,line width= 0.2pt,line join=round,line cap=round] (170.89, 48.00) -- (170.89, 55.25);

\path[draw=drawColor,line width= 0.2pt,line join=round,line cap=round] (171.31, 48.00) -- (171.31, 55.25);

\path[draw=drawColor,line width= 0.2pt,line join=round,line cap=round] (171.31, 48.00) -- (171.31, 55.25);

\path[draw=drawColor,line width= 0.2pt,line join=round,line cap=round] (171.93, 48.00) -- (171.93, 55.25);

\path[draw=drawColor,line width= 0.2pt,line join=round,line cap=round] (171.93, 48.00) -- (171.93, 55.25);

\path[draw=drawColor,line width= 0.2pt,line join=round,line cap=round] (172.55, 48.00) -- (172.55, 55.25);

\path[draw=drawColor,line width= 0.2pt,line join=round,line cap=round] (172.55, 48.00) -- (172.55, 55.25);

\path[draw=drawColor,line width= 0.2pt,line join=round,line cap=round] (172.76, 48.00) -- (172.76, 55.25);

\path[draw=drawColor,line width= 0.2pt,line join=round,line cap=round] (172.76, 48.00) -- (172.76, 55.25);

\path[draw=drawColor,line width= 0.2pt,line join=round,line cap=round] (172.97, 48.00) -- (172.97, 55.25);

\path[draw=drawColor,line width= 0.2pt,line join=round,line cap=round] (174.21, 48.00) -- (174.21, 55.25);

\path[draw=drawColor,line width= 0.2pt,line join=round,line cap=round] (174.21, 48.00) -- (174.21, 55.25);

\path[draw=drawColor,line width= 0.2pt,line join=round,line cap=round] (174.21, 48.00) -- (174.21, 55.25);

\path[draw=drawColor,line width= 0.2pt,line join=round,line cap=round] (174.42, 48.00) -- (174.42, 55.25);

\path[draw=drawColor,line width= 0.2pt,line join=round,line cap=round] (174.62, 48.00) -- (174.62, 55.25);

\path[draw=drawColor,line width= 0.2pt,line join=round,line cap=round] (175.25, 48.00) -- (175.25, 55.25);

\path[draw=drawColor,line width= 0.2pt,line join=round,line cap=round] (175.87, 48.00) -- (175.87, 55.25);

\path[draw=drawColor,line width= 0.2pt,line join=round,line cap=round] (176.28, 48.00) -- (176.28, 55.25);

\path[draw=drawColor,line width= 0.2pt,line join=round,line cap=round] (176.70, 48.00) -- (176.70, 55.25);

\path[draw=drawColor,line width= 0.2pt,line join=round,line cap=round] (178.15, 48.00) -- (178.15, 55.25);

\path[draw=drawColor,line width= 0.2pt,line join=round,line cap=round] (178.77, 48.00) -- (178.77, 55.25);

\path[draw=drawColor,line width= 0.2pt,line join=round,line cap=round] (180.01, 48.00) -- (180.01, 55.25);

\path[draw=drawColor,line width= 0.2pt,line join=round,line cap=round] (180.01, 48.00) -- (180.01, 55.25);

\path[draw=drawColor,line width= 0.2pt,line join=round,line cap=round] (180.43, 48.00) -- (180.43, 55.25);

\path[draw=drawColor,line width= 0.2pt,line join=round,line cap=round] (180.63, 48.00) -- (180.63, 55.25);

\path[draw=drawColor,line width= 0.2pt,line join=round,line cap=round] (180.63, 48.00) -- (180.63, 55.25);

\path[draw=drawColor,line width= 0.2pt,line join=round,line cap=round] (181.05, 48.00) -- (181.05, 55.25);

\path[draw=drawColor,line width= 0.2pt,line join=round,line cap=round] (181.46, 48.00) -- (181.46, 55.25);

\path[draw=drawColor,line width= 0.2pt,line join=round,line cap=round] (181.46, 48.00) -- (181.46, 55.25);

\path[draw=drawColor,line width= 0.2pt,line join=round,line cap=round] (181.67, 48.00) -- (181.67, 55.25);

\path[draw=drawColor,line width= 0.2pt,line join=round,line cap=round] (181.88, 48.00) -- (181.88, 55.25);

\path[draw=drawColor,line width= 0.2pt,line join=round,line cap=round] (182.09, 48.00) -- (182.09, 55.25);

\path[draw=drawColor,line width= 0.2pt,line join=round,line cap=round] (182.29, 48.00) -- (182.29, 55.25);

\path[draw=drawColor,line width= 0.2pt,line join=round,line cap=round] (182.50, 48.00) -- (182.50, 55.25);

\path[draw=drawColor,line width= 0.2pt,line join=round,line cap=round] (182.50, 48.00) -- (182.50, 55.25);

\path[draw=drawColor,line width= 0.2pt,line join=round,line cap=round] (182.50, 48.00) -- (182.50, 55.25);

\path[draw=drawColor,line width= 0.2pt,line join=round,line cap=round] (182.91, 48.00) -- (182.91, 55.25);

\path[draw=drawColor,line width= 0.2pt,line join=round,line cap=round] (183.12, 48.00) -- (183.12, 55.25);

\path[draw=drawColor,line width= 0.2pt,line join=round,line cap=round] (183.12, 48.00) -- (183.12, 55.25);

\path[draw=drawColor,line width= 0.2pt,line join=round,line cap=round] (184.37, 48.00) -- (184.37, 55.25);

\path[draw=drawColor,line width= 0.2pt,line join=round,line cap=round] (184.37, 48.00) -- (184.37, 55.25);

\path[draw=drawColor,line width= 0.2pt,line join=round,line cap=round] (184.57, 48.00) -- (184.57, 55.25);

\path[draw=drawColor,line width= 0.2pt,line join=round,line cap=round] (185.19, 48.00) -- (185.19, 55.25);

\path[draw=drawColor,line width= 0.2pt,line join=round,line cap=round] (185.40, 48.00) -- (185.40, 55.25);

\path[draw=drawColor,line width= 0.2pt,line join=round,line cap=round] (187.27, 48.00) -- (187.27, 55.25);

\path[draw=drawColor,line width= 0.2pt,line join=round,line cap=round] (187.47, 48.00) -- (187.47, 55.25);

\path[draw=drawColor,line width= 0.2pt,line join=round,line cap=round] (187.68, 48.00) -- (187.68, 55.25);

\path[draw=drawColor,line width= 0.2pt,line join=round,line cap=round] (187.89, 48.00) -- (187.89, 55.25);

\path[draw=drawColor,line width= 0.2pt,line join=round,line cap=round] (187.89, 48.00) -- (187.89, 55.25);

\path[draw=drawColor,line width= 0.2pt,line join=round,line cap=round] (187.89, 48.00) -- (187.89, 55.25);

\path[draw=drawColor,line width= 0.2pt,line join=round,line cap=round] (188.10, 48.00) -- (188.10, 55.25);

\path[draw=drawColor,line width= 0.2pt,line join=round,line cap=round] (188.30, 48.00) -- (188.30, 55.25);

\path[draw=drawColor,line width= 0.2pt,line join=round,line cap=round] (189.13, 48.00) -- (189.13, 55.25);

\path[draw=drawColor,line width= 0.2pt,line join=round,line cap=round] (189.75, 48.00) -- (189.75, 55.25);

\path[draw=drawColor,line width= 0.2pt,line join=round,line cap=round] (189.75, 48.00) -- (189.75, 55.25);

\path[draw=drawColor,line width= 0.2pt,line join=round,line cap=round] (190.79, 48.00) -- (190.79, 55.25);

\path[draw=drawColor,line width= 0.2pt,line join=round,line cap=round] (191.62, 48.00) -- (191.62, 55.25);

\path[draw=drawColor,line width= 0.2pt,line join=round,line cap=round] (191.62, 48.00) -- (191.62, 55.25);

\path[draw=drawColor,line width= 0.2pt,line join=round,line cap=round] (191.62, 48.00) -- (191.62, 55.25);

\path[draw=drawColor,line width= 0.2pt,line join=round,line cap=round] (191.62, 48.00) -- (191.62, 55.25);

\path[draw=drawColor,line width= 0.2pt,line join=round,line cap=round] (191.83, 48.00) -- (191.83, 55.25);

\path[draw=drawColor,line width= 0.2pt,line join=round,line cap=round] (192.03, 48.00) -- (192.03, 55.25);

\path[draw=drawColor,line width= 0.2pt,line join=round,line cap=round] (194.11, 48.00) -- (194.11, 55.25);

\path[draw=drawColor,line width= 0.2pt,line join=round,line cap=round] (194.31, 48.00) -- (194.31, 55.25);

\path[draw=drawColor,line width= 0.2pt,line join=round,line cap=round] (194.93, 48.00) -- (194.93, 55.25);

\path[draw=drawColor,line width= 0.2pt,line join=round,line cap=round] (196.39, 48.00) -- (196.39, 55.25);

\path[draw=drawColor,line width= 0.2pt,line join=round,line cap=round] (198.04, 48.00) -- (198.04, 55.25);

\path[draw=drawColor,line width= 0.2pt,line join=round,line cap=round] (198.46, 48.00) -- (198.46, 55.25);

\path[draw=drawColor,line width= 0.2pt,line join=round,line cap=round] (198.66, 48.00) -- (198.66, 55.25);

\path[draw=drawColor,line width= 0.2pt,line join=round,line cap=round] (200.32, 48.00) -- (200.32, 55.25);

\path[draw=drawColor,line width= 0.2pt,line join=round,line cap=round] (201.15, 48.00) -- (201.15, 55.25);

\path[draw=drawColor,line width= 0.2pt,line join=round,line cap=round] (201.36, 48.00) -- (201.36, 55.25);

\path[draw=drawColor,line width= 0.2pt,line join=round,line cap=round] (202.40, 48.00) -- (202.40, 55.25);

\path[draw=drawColor,line width= 0.2pt,line join=round,line cap=round] (202.40, 48.00) -- (202.40, 55.25);

\path[draw=drawColor,line width= 0.2pt,line join=round,line cap=round] (203.02, 48.00) -- (203.02, 55.25);

\path[draw=drawColor,line width= 0.2pt,line join=round,line cap=round] (203.43, 48.00) -- (203.43, 55.25);

\path[draw=drawColor,line width= 0.2pt,line join=round,line cap=round] (204.05, 48.00) -- (204.05, 55.25);

\path[draw=drawColor,line width= 0.2pt,line join=round,line cap=round] (204.67, 48.00) -- (204.67, 55.25);

\path[draw=drawColor,line width= 0.2pt,line join=round,line cap=round] (205.09, 48.00) -- (205.09, 55.25);

\path[draw=drawColor,line width= 0.2pt,line join=round,line cap=round] (206.33, 48.00) -- (206.33, 55.25);

\path[draw=drawColor,line width= 0.2pt,line join=round,line cap=round] (206.54, 48.00) -- (206.54, 55.25);

\path[draw=drawColor,line width= 0.2pt,line join=round,line cap=round] (207.78, 48.00) -- (207.78, 55.25);

\path[draw=drawColor,line width= 0.2pt,line join=round,line cap=round] (207.99, 48.00) -- (207.99, 55.25);

\path[draw=drawColor,line width= 0.2pt,line join=round,line cap=round] (209.03, 48.00) -- (209.03, 55.25);

\path[draw=drawColor,line width= 0.2pt,line join=round,line cap=round] (209.23, 48.00) -- (209.23, 55.25);

\path[draw=drawColor,line width= 0.2pt,line join=round,line cap=round] (209.23, 48.00) -- (209.23, 55.25);

\path[draw=drawColor,line width= 0.2pt,line join=round,line cap=round] (210.69, 48.00) -- (210.69, 55.25);

\path[draw=drawColor,line width= 0.2pt,line join=round,line cap=round] (210.89, 48.00) -- (210.89, 55.25);

\path[draw=drawColor,line width= 0.2pt,line join=round,line cap=round] (212.14, 48.00) -- (212.14, 55.25);

\path[draw=drawColor,line width= 0.2pt,line join=round,line cap=round] (216.07, 48.00) -- (216.07, 55.25);

\path[draw=drawColor,line width= 0.2pt,line join=round,line cap=round] (217.11, 48.00) -- (217.11, 55.25);

\path[draw=drawColor,line width= 0.2pt,line join=round,line cap=round] (218.97, 48.00) -- (218.97, 55.25);

\path[draw=drawColor,line width= 0.2pt,line join=round,line cap=round] (220.63, 48.00) -- (220.63, 55.25);

\path[draw=drawColor,line width= 0.2pt,line join=round,line cap=round] (220.84, 48.00) -- (220.84, 55.25);

\path[draw=drawColor,line width= 0.2pt,line join=round,line cap=round] (221.25, 48.00) -- (221.25, 55.25);

\path[draw=drawColor,line width= 0.2pt,line join=round,line cap=round] (221.88, 48.00) -- (221.88, 55.25);

\path[draw=drawColor,line width= 0.2pt,line join=round,line cap=round] (222.50, 48.00) -- (222.50, 55.25);

\path[draw=drawColor,line width= 0.2pt,line join=round,line cap=round] (224.16, 48.00) -- (224.16, 55.25);

\path[draw=drawColor,line width= 0.2pt,line join=round,line cap=round] (224.57, 48.00) -- (224.57, 55.25);

\path[draw=drawColor,line width= 0.2pt,line join=round,line cap=round] (225.19, 48.00) -- (225.19, 55.25);

\path[draw=drawColor,line width= 0.2pt,line join=round,line cap=round] (226.44, 48.00) -- (226.44, 55.25);

\path[draw=drawColor,line width= 0.2pt,line join=round,line cap=round] (227.89, 48.00) -- (227.89, 55.25);

\path[draw=drawColor,line width= 0.2pt,line join=round,line cap=round] (227.89, 48.00) -- (227.89, 55.25);

\path[draw=drawColor,line width= 0.2pt,line join=round,line cap=round] (229.54, 48.00) -- (229.54, 55.25);

\path[draw=drawColor,line width= 0.2pt,line join=round,line cap=round] (229.75, 48.00) -- (229.75, 55.25);

\path[draw=drawColor,line width= 0.2pt,line join=round,line cap=round] (233.90, 48.00) -- (233.90, 55.25);

\path[draw=drawColor,line width= 0.2pt,line join=round,line cap=round] (234.31, 48.00) -- (234.31, 55.25);

\path[draw=drawColor,line width= 0.2pt,line join=round,line cap=round] (238.66, 48.00) -- (238.66, 55.25);

\path[draw=drawColor,line width= 0.2pt,line join=round,line cap=round] (242.19, 48.00) -- (242.19, 55.25);

\path[draw=drawColor,line width= 0.2pt,line join=round,line cap=round] (242.19, 48.00) -- (242.19, 55.25);

\path[draw=drawColor,line width= 0.2pt,line join=round,line cap=round] (244.05, 48.00) -- (244.05, 55.25);

\path[draw=drawColor,line width= 0.2pt,line join=round,line cap=round] (244.88, 48.00) -- (244.88, 55.25);

\path[draw=drawColor,line width= 0.2pt,line join=round,line cap=round] (245.09, 48.00) -- (245.09, 55.25);

\path[draw=drawColor,line width= 0.2pt,line join=round,line cap=round] (246.12, 48.00) -- (246.12, 55.25);

\path[draw=drawColor,line width= 0.2pt,line join=round,line cap=round] (246.75, 48.00) -- (246.75, 55.25);

\path[draw=drawColor,line width= 0.2pt,line join=round,line cap=round] (247.57, 48.00) -- (247.57, 55.25);

\path[draw=drawColor,line width= 0.2pt,line join=round,line cap=round] (252.13, 48.00) -- (252.13, 55.25);

\path[draw=drawColor,line width= 0.2pt,line join=round,line cap=round] (255.03, 48.00) -- (255.03, 55.25);

\path[draw=drawColor,line width= 0.2pt,line join=round,line cap=round] (261.46, 48.00) -- (261.46, 55.25);

\path[draw=drawColor,line width= 0.2pt,line join=round,line cap=round] (265.60, 48.00) -- (265.60, 55.25);

\path[draw=drawColor,line width= 0.2pt,line join=round,line cap=round] (279.49, 48.00) -- (279.49, 55.25);

\path[draw=drawColor,line width= 0.2pt,line join=round,line cap=round] (308.30, 48.00) -- (308.30, 55.25);

\path[draw=drawColor,line width= 0.2pt,line join=round,line cap=round] (326.12, 48.00) -- (326.12, 55.25);
\end{scope}
\end{tikzpicture}

	\end{adjustbox}
	\caption[Dichtefunktion des \gls{si}]{Dichtefunktion des \gls{si}. Der \gls{si} wurde als Differenz zwischen der $82\,\%$-$\log_{10}$-Er\-ken\-nungs\-schwel\-le für die Mustergrösse $7.2^{\circ}$ und der $82\,\%$-$\log_{10}$-Er\-ken\-nungs\-schwel\-le für die Mustergrösse $1.8^{\circ}$ gebildet. Alle Datenpunkte sind auf der x-Achse mit vertikalen Strichen markiert.}
	\label{fig:suppression_index_density}
\end{figure}

\subsection{Hick-Aufgabe}

In \autoref{fig:hick_scatterplot} sind die mittleren Reaktionszeiten aller \glspl{vp} als Streudiagramme abgebildet.
%Die mittleren Reaktionszeiten aller \glspl{vp} sind als Streudiagramme in  \autoref{fig:hick_scatterplot} zu finden.
Mittelwerte, Verteilungsangaben und die Reliabilitäten der Bedingungen lassen sich in \autoref{tab:hick_descriptives} finden. 

\begin{figure}[p]
	\centering
	\begin{adjustbox}{width=1\textwidth}
		% Created by tikzDevice version 0.10.1 on 2016-06-28 08:12:54
% !TEX encoding = UTF-8 Unicode
\begin{tikzpicture}[x=1pt,y=1pt]
\definecolor{fillColor}{RGB}{255,255,255}
\path[use as bounding box,fill=fillColor,fill opacity=0.00] (0,0) rectangle (505.89,505.89);
\begin{scope}
\path[clip] ( 40.39, 35.64) rectangle (121.72,494.80);
\definecolor{drawColor}{RGB}{0,0,0}
\definecolor{fillColor}{RGB}{0,0,0}

\path[draw=drawColor,line width= 0.4pt,line join=round,line cap=round,fill=fillColor] ( 43.83,291.94) circle (  0.99);

\path[draw=drawColor,line width= 0.4pt,line join=round,line cap=round,fill=fillColor] ( 44.26,203.88) circle (  0.99);

\path[draw=drawColor,line width= 0.4pt,line join=round,line cap=round,fill=fillColor] ( 44.69,196.59) circle (  0.99);

\path[draw=drawColor,line width= 0.4pt,line join=round,line cap=round,fill=fillColor] ( 45.12,205.70) circle (  0.99);

\path[draw=drawColor,line width= 0.4pt,line join=round,line cap=round,fill=fillColor] ( 45.54,186.26) circle (  0.99);

\path[draw=drawColor,line width= 0.4pt,line join=round,line cap=round,fill=fillColor] ( 45.97,187.48) circle (  0.99);

\path[draw=drawColor,line width= 0.4pt,line join=round,line cap=round,fill=fillColor] ( 46.40,188.09) circle (  0.99);

\path[draw=drawColor,line width= 0.4pt,line join=round,line cap=round,fill=fillColor] ( 46.83,185.05) circle (  0.99);

\path[draw=drawColor,line width= 0.4pt,line join=round,line cap=round,fill=fillColor] ( 47.25,192.34) circle (  0.99);

\path[draw=drawColor,line width= 0.4pt,line join=round,line cap=round,fill=fillColor] ( 47.68,193.55) circle (  0.99);

\path[draw=drawColor,line width= 0.4pt,line join=round,line cap=round,fill=fillColor] ( 48.11,221.49) circle (  0.99);

\path[draw=drawColor,line width= 0.4pt,line join=round,line cap=round,fill=fillColor] ( 48.54,172.30) circle (  0.99);

\path[draw=drawColor,line width= 0.4pt,line join=round,line cap=round,fill=fillColor] ( 48.97,206.91) circle (  0.99);

\path[draw=drawColor,line width= 0.4pt,line join=round,line cap=round,fill=fillColor] ( 49.39,208.13) circle (  0.99);

\path[draw=drawColor,line width= 0.4pt,line join=round,line cap=round,fill=fillColor] ( 49.82,192.34) circle (  0.99);

\path[draw=drawColor,line width= 0.4pt,line join=round,line cap=round,fill=fillColor] ( 50.25,202.06) circle (  0.99);

\path[draw=drawColor,line width= 0.4pt,line join=round,line cap=round,fill=fillColor] ( 50.68,180.80) circle (  0.99);

\path[draw=drawColor,line width= 0.4pt,line join=round,line cap=round,fill=fillColor] ( 51.11,203.27) circle (  0.99);

\path[draw=drawColor,line width= 0.4pt,line join=round,line cap=round,fill=fillColor] ( 51.53,183.23) circle (  0.99);

\path[draw=drawColor,line width= 0.4pt,line join=round,line cap=round,fill=fillColor] ( 51.96,189.30) circle (  0.99);

\path[draw=drawColor,line width= 0.4pt,line join=round,line cap=round,fill=fillColor] ( 52.39,219.06) circle (  0.99);

\path[draw=drawColor,line width= 0.4pt,line join=round,line cap=round,fill=fillColor] ( 52.82,191.12) circle (  0.99);

\path[draw=drawColor,line width= 0.4pt,line join=round,line cap=round,fill=fillColor] ( 53.25,191.73) circle (  0.99);

\path[draw=drawColor,line width= 0.4pt,line join=round,line cap=round,fill=fillColor] ( 53.67,203.27) circle (  0.99);

\path[draw=drawColor,line width= 0.4pt,line join=round,line cap=round,fill=fillColor] ( 54.10,211.17) circle (  0.99);

\path[draw=drawColor,line width= 0.4pt,line join=round,line cap=round,fill=fillColor] ( 54.53,202.66) circle (  0.99);

\path[draw=drawColor,line width= 0.4pt,line join=round,line cap=round,fill=fillColor] ( 54.96,187.48) circle (  0.99);

\path[draw=drawColor,line width= 0.4pt,line join=round,line cap=round,fill=fillColor] ( 55.38,188.69) circle (  0.99);

\path[draw=drawColor,line width= 0.4pt,line join=round,line cap=round,fill=fillColor] ( 55.81,176.55) circle (  0.99);

\path[draw=drawColor,line width= 0.4pt,line join=round,line cap=round,fill=fillColor] ( 56.24,193.55) circle (  0.99);

\path[draw=drawColor,line width= 0.4pt,line join=round,line cap=round,fill=fillColor] ( 56.67,182.62) circle (  0.99);

\path[draw=drawColor,line width= 0.4pt,line join=round,line cap=round,fill=fillColor] ( 57.10,201.45) circle (  0.99);

\path[draw=drawColor,line width= 0.4pt,line join=round,line cap=round,fill=fillColor] ( 57.52,177.15) circle (  0.99);

\path[draw=drawColor,line width= 0.4pt,line join=round,line cap=round,fill=fillColor] ( 57.95,195.37) circle (  0.99);

\path[draw=drawColor,line width= 0.4pt,line join=round,line cap=round,fill=fillColor] ( 58.38,186.26) circle (  0.99);

\path[draw=drawColor,line width= 0.4pt,line join=round,line cap=round,fill=fillColor] ( 58.81,207.52) circle (  0.99);

\path[draw=drawColor,line width= 0.4pt,line join=round,line cap=round,fill=fillColor] ( 59.24,187.48) circle (  0.99);

\path[draw=drawColor,line width= 0.4pt,line join=round,line cap=round,fill=fillColor] ( 59.66,187.48) circle (  0.99);

\path[draw=drawColor,line width= 0.4pt,line join=round,line cap=round,fill=fillColor] ( 60.09,219.67) circle (  0.99);

\path[draw=drawColor,line width= 0.4pt,line join=round,line cap=round,fill=fillColor] ( 60.52,222.10) circle (  0.99);

\path[draw=drawColor,line width= 0.4pt,line join=round,line cap=round,fill=fillColor] ( 60.95,193.55) circle (  0.99);

\path[draw=drawColor,line width= 0.4pt,line join=round,line cap=round,fill=fillColor] ( 61.37,177.15) circle (  0.99);

\path[draw=drawColor,line width= 0.4pt,line join=round,line cap=round,fill=fillColor] ( 61.80,196.59) circle (  0.99);

\path[draw=drawColor,line width= 0.4pt,line join=round,line cap=round,fill=fillColor] ( 62.23,193.55) circle (  0.99);

\path[draw=drawColor,line width= 0.4pt,line join=round,line cap=round,fill=fillColor] ( 62.66,178.98) circle (  0.99);

\path[draw=drawColor,line width= 0.4pt,line join=round,line cap=round,fill=fillColor] ( 63.09,195.37) circle (  0.99);

\path[draw=drawColor,line width= 0.4pt,line join=round,line cap=round,fill=fillColor] ( 63.51,199.02) circle (  0.99);

\path[draw=drawColor,line width= 0.4pt,line join=round,line cap=round,fill=fillColor] ( 63.94,212.99) circle (  0.99);

\path[draw=drawColor,line width= 0.4pt,line join=round,line cap=round,fill=fillColor] ( 64.37,189.30) circle (  0.99);

\path[draw=drawColor,line width= 0.4pt,line join=round,line cap=round,fill=fillColor] ( 64.80,192.34) circle (  0.99);

\path[draw=drawColor,line width= 0.4pt,line join=round,line cap=round,fill=fillColor] ( 65.23,183.84) circle (  0.99);

\path[draw=drawColor,line width= 0.4pt,line join=round,line cap=round,fill=fillColor] ( 65.65,203.88) circle (  0.99);

\path[draw=drawColor,line width= 0.4pt,line join=round,line cap=round,fill=fillColor] ( 66.08,191.73) circle (  0.99);

\path[draw=drawColor,line width= 0.4pt,line join=round,line cap=round,fill=fillColor] ( 66.51,197.80) circle (  0.99);

\path[draw=drawColor,line width= 0.4pt,line join=round,line cap=round,fill=fillColor] ( 66.94,225.14) circle (  0.99);

\path[draw=drawColor,line width= 0.4pt,line join=round,line cap=round,fill=fillColor] ( 67.36,253.07) circle (  0.99);

\path[draw=drawColor,line width= 0.4pt,line join=round,line cap=round,fill=fillColor] ( 67.79,207.52) circle (  0.99);

\path[draw=drawColor,line width= 0.4pt,line join=round,line cap=round,fill=fillColor] ( 68.22,195.98) circle (  0.99);

\path[draw=drawColor,line width= 0.4pt,line join=round,line cap=round,fill=fillColor] ( 68.65,194.16) circle (  0.99);

\path[draw=drawColor,line width= 0.4pt,line join=round,line cap=round,fill=fillColor] ( 69.08,172.30) circle (  0.99);

\path[draw=drawColor,line width= 0.4pt,line join=round,line cap=round,fill=fillColor] ( 69.50,219.67) circle (  0.99);

\path[draw=drawColor,line width= 0.4pt,line join=round,line cap=round,fill=fillColor] ( 69.93,229.39) circle (  0.99);

\path[draw=drawColor,line width= 0.4pt,line join=round,line cap=round,fill=fillColor] ( 70.36,175.94) circle (  0.99);

\path[draw=drawColor,line width= 0.4pt,line join=round,line cap=round,fill=fillColor] ( 70.79,204.49) circle (  0.99);

\path[draw=drawColor,line width= 0.4pt,line join=round,line cap=round,fill=fillColor] ( 71.22,191.12) circle (  0.99);

\path[draw=drawColor,line width= 0.4pt,line join=round,line cap=round,fill=fillColor] ( 71.64,201.45) circle (  0.99);

\path[draw=drawColor,line width= 0.4pt,line join=round,line cap=round,fill=fillColor] ( 72.07,189.91) circle (  0.99);

\path[draw=drawColor,line width= 0.4pt,line join=round,line cap=round,fill=fillColor] ( 72.50,225.14) circle (  0.99);

\path[draw=drawColor,line width= 0.4pt,line join=round,line cap=round,fill=fillColor] ( 72.93,198.41) circle (  0.99);

\path[draw=drawColor,line width= 0.4pt,line join=round,line cap=round,fill=fillColor] ( 73.35,197.20) circle (  0.99);

\path[draw=drawColor,line width= 0.4pt,line join=round,line cap=round,fill=fillColor] ( 73.78,187.48) circle (  0.99);

\path[draw=drawColor,line width= 0.4pt,line join=round,line cap=round,fill=fillColor] ( 74.21,198.41) circle (  0.99);

\path[draw=drawColor,line width= 0.4pt,line join=round,line cap=round,fill=fillColor] ( 74.64,193.55) circle (  0.99);

\path[draw=drawColor,line width= 0.4pt,line join=round,line cap=round,fill=fillColor] ( 75.07,188.69) circle (  0.99);

\path[draw=drawColor,line width= 0.4pt,line join=round,line cap=round,fill=fillColor] ( 75.49,210.56) circle (  0.99);

\path[draw=drawColor,line width= 0.4pt,line join=round,line cap=round,fill=fillColor] ( 75.92,175.94) circle (  0.99);

\path[draw=drawColor,line width= 0.4pt,line join=round,line cap=round,fill=fillColor] ( 76.35,208.74) circle (  0.99);

\path[draw=drawColor,line width= 0.4pt,line join=round,line cap=round,fill=fillColor] ( 76.78,198.41) circle (  0.99);

\path[draw=drawColor,line width= 0.4pt,line join=round,line cap=round,fill=fillColor] ( 77.21,184.44) circle (  0.99);

\path[draw=drawColor,line width= 0.4pt,line join=round,line cap=round,fill=fillColor] ( 77.63,191.73) circle (  0.99);

\path[draw=drawColor,line width= 0.4pt,line join=round,line cap=round,fill=fillColor] ( 78.06,196.59) circle (  0.99);

\path[draw=drawColor,line width= 0.4pt,line join=round,line cap=round,fill=fillColor] ( 78.49,219.67) circle (  0.99);

\path[draw=drawColor,line width= 0.4pt,line join=round,line cap=round,fill=fillColor] ( 78.92,209.34) circle (  0.99);

\path[draw=drawColor,line width= 0.4pt,line join=round,line cap=round,fill=fillColor] ( 79.34,267.04) circle (  0.99);

\path[draw=drawColor,line width= 0.4pt,line join=round,line cap=round,fill=fillColor] ( 79.77,193.55) circle (  0.99);

\path[draw=drawColor,line width= 0.4pt,line join=round,line cap=round,fill=fillColor] ( 80.20,208.74) circle (  0.99);

\path[draw=drawColor,line width= 0.4pt,line join=round,line cap=round,fill=fillColor] ( 80.63,178.98) circle (  0.99);

\path[draw=drawColor,line width= 0.4pt,line join=round,line cap=round,fill=fillColor] ( 81.06,212.38) circle (  0.99);

\path[draw=drawColor,line width= 0.4pt,line join=round,line cap=round,fill=fillColor] ( 81.48,206.31) circle (  0.99);

\path[draw=drawColor,line width= 0.4pt,line join=round,line cap=round,fill=fillColor] ( 81.91,177.76) circle (  0.99);

\path[draw=drawColor,line width= 0.4pt,line join=round,line cap=round,fill=fillColor] ( 82.34,184.44) circle (  0.99);

\path[draw=drawColor,line width= 0.4pt,line join=round,line cap=round,fill=fillColor] ( 82.77,196.59) circle (  0.99);

\path[draw=drawColor,line width= 0.4pt,line join=round,line cap=round,fill=fillColor] ( 83.20,178.37) circle (  0.99);

\path[draw=drawColor,line width= 0.4pt,line join=round,line cap=round,fill=fillColor] ( 83.62,207.52) circle (  0.99);

\path[draw=drawColor,line width= 0.4pt,line join=round,line cap=round,fill=fillColor] ( 84.05,206.91) circle (  0.99);

\path[draw=drawColor,line width= 0.4pt,line join=round,line cap=round,fill=fillColor] ( 84.48,166.83) circle (  0.99);

\path[draw=drawColor,line width= 0.4pt,line join=round,line cap=round,fill=fillColor] ( 84.91,194.16) circle (  0.99);

\path[draw=drawColor,line width= 0.4pt,line join=round,line cap=round,fill=fillColor] ( 85.33,229.99) circle (  0.99);

\path[draw=drawColor,line width= 0.4pt,line join=round,line cap=round,fill=fillColor] ( 85.76,197.80) circle (  0.99);

\path[draw=drawColor,line width= 0.4pt,line join=round,line cap=round,fill=fillColor] ( 86.19,186.26) circle (  0.99);

\path[draw=drawColor,line width= 0.4pt,line join=round,line cap=round,fill=fillColor] ( 86.62,216.63) circle (  0.99);

\path[draw=drawColor,line width= 0.4pt,line join=round,line cap=round,fill=fillColor] ( 87.05,189.30) circle (  0.99);

\path[draw=drawColor,line width= 0.4pt,line join=round,line cap=round,fill=fillColor] ( 87.47,199.63) circle (  0.99);

\path[draw=drawColor,line width= 0.4pt,line join=round,line cap=round,fill=fillColor] ( 87.90,190.52) circle (  0.99);

\path[draw=drawColor,line width= 0.4pt,line join=round,line cap=round,fill=fillColor] ( 88.33,196.59) circle (  0.99);

\path[draw=drawColor,line width= 0.4pt,line join=round,line cap=round,fill=fillColor] ( 88.76,191.73) circle (  0.99);

\path[draw=drawColor,line width= 0.4pt,line join=round,line cap=round,fill=fillColor] ( 89.19,181.41) circle (  0.99);

\path[draw=drawColor,line width= 0.4pt,line join=round,line cap=round,fill=fillColor] ( 89.61,199.02) circle (  0.99);

\path[draw=drawColor,line width= 0.4pt,line join=round,line cap=round,fill=fillColor] ( 90.04,186.87) circle (  0.99);

\path[draw=drawColor,line width= 0.4pt,line join=round,line cap=round,fill=fillColor] ( 90.47,202.06) circle (  0.99);

\path[draw=drawColor,line width= 0.4pt,line join=round,line cap=round,fill=fillColor] ( 90.90,230.60) circle (  0.99);

\path[draw=drawColor,line width= 0.4pt,line join=round,line cap=round,fill=fillColor] ( 91.32,233.03) circle (  0.99);

\path[draw=drawColor,line width= 0.4pt,line join=round,line cap=round,fill=fillColor] ( 91.75,194.77) circle (  0.99);

\path[draw=drawColor,line width= 0.4pt,line join=round,line cap=round,fill=fillColor] ( 92.18,194.16) circle (  0.99);

\path[draw=drawColor,line width= 0.4pt,line join=round,line cap=round,fill=fillColor] ( 92.61,182.01) circle (  0.99);

\path[draw=drawColor,line width= 0.4pt,line join=round,line cap=round,fill=fillColor] ( 93.04,186.26) circle (  0.99);

\path[draw=drawColor,line width= 0.4pt,line join=round,line cap=round,fill=fillColor] ( 93.46,195.98) circle (  0.99);

\path[draw=drawColor,line width= 0.4pt,line join=round,line cap=round,fill=fillColor] ( 93.89,177.76) circle (  0.99);

\path[draw=drawColor,line width= 0.4pt,line join=round,line cap=round,fill=fillColor] ( 94.32,172.90) circle (  0.99);

\path[draw=drawColor,line width= 0.4pt,line join=round,line cap=round,fill=fillColor] ( 94.75,202.66) circle (  0.99);

\path[draw=drawColor,line width= 0.4pt,line join=round,line cap=round,fill=fillColor] ( 95.18,203.88) circle (  0.99);

\path[draw=drawColor,line width= 0.4pt,line join=round,line cap=round,fill=fillColor] ( 95.60,198.41) circle (  0.99);

\path[draw=drawColor,line width= 0.4pt,line join=round,line cap=round,fill=fillColor] ( 96.03,226.96) circle (  0.99);

\path[draw=drawColor,line width= 0.4pt,line join=round,line cap=round,fill=fillColor] ( 96.46,200.84) circle (  0.99);

\path[draw=drawColor,line width= 0.4pt,line join=round,line cap=round,fill=fillColor] ( 96.89,246.39) circle (  0.99);

\path[draw=drawColor,line width= 0.4pt,line join=round,line cap=round,fill=fillColor] ( 97.32,183.84) circle (  0.99);

\path[draw=drawColor,line width= 0.4pt,line join=round,line cap=round,fill=fillColor] ( 97.74,197.80) circle (  0.99);

\path[draw=drawColor,line width= 0.4pt,line join=round,line cap=round,fill=fillColor] ( 98.17,195.37) circle (  0.99);

\path[draw=drawColor,line width= 0.4pt,line join=round,line cap=round,fill=fillColor] ( 98.60,189.91) circle (  0.99);

\path[draw=drawColor,line width= 0.4pt,line join=round,line cap=round,fill=fillColor] ( 99.03,184.44) circle (  0.99);

\path[draw=drawColor,line width= 0.4pt,line join=round,line cap=round,fill=fillColor] ( 99.45,221.49) circle (  0.99);

\path[draw=drawColor,line width= 0.4pt,line join=round,line cap=round,fill=fillColor] ( 99.88,186.26) circle (  0.99);

\path[draw=drawColor,line width= 0.4pt,line join=round,line cap=round,fill=fillColor] (100.31,195.98) circle (  0.99);

\path[draw=drawColor,line width= 0.4pt,line join=round,line cap=round,fill=fillColor] (100.74,182.62) circle (  0.99);

\path[draw=drawColor,line width= 0.4pt,line join=round,line cap=round,fill=fillColor] (101.17,195.98) circle (  0.99);

\path[draw=drawColor,line width= 0.4pt,line join=round,line cap=round,fill=fillColor] (101.59,213.60) circle (  0.99);

\path[draw=drawColor,line width= 0.4pt,line join=round,line cap=round,fill=fillColor] (102.02,190.52) circle (  0.99);

\path[draw=drawColor,line width= 0.4pt,line join=round,line cap=round,fill=fillColor] (102.45,199.63) circle (  0.99);

\path[draw=drawColor,line width= 0.4pt,line join=round,line cap=round,fill=fillColor] (102.88,203.88) circle (  0.99);

\path[draw=drawColor,line width= 0.4pt,line join=round,line cap=round,fill=fillColor] (103.31,195.37) circle (  0.99);

\path[draw=drawColor,line width= 0.4pt,line join=round,line cap=round,fill=fillColor] (103.73,209.34) circle (  0.99);

\path[draw=drawColor,line width= 0.4pt,line join=round,line cap=round,fill=fillColor] (104.16,189.30) circle (  0.99);

\path[draw=drawColor,line width= 0.4pt,line join=round,line cap=round,fill=fillColor] (104.59,168.04) circle (  0.99);

\path[draw=drawColor,line width= 0.4pt,line join=round,line cap=round,fill=fillColor] (105.02,185.05) circle (  0.99);

\path[draw=drawColor,line width= 0.4pt,line join=round,line cap=round,fill=fillColor] (105.44,187.48) circle (  0.99);

\path[draw=drawColor,line width= 0.4pt,line join=round,line cap=round,fill=fillColor] (105.87,174.72) circle (  0.99);

\path[draw=drawColor,line width= 0.4pt,line join=round,line cap=round,fill=fillColor] (106.30,200.23) circle (  0.99);

\path[draw=drawColor,line width= 0.4pt,line join=round,line cap=round,fill=fillColor] (106.73,208.13) circle (  0.99);

\path[draw=drawColor,line width= 0.4pt,line join=round,line cap=round,fill=fillColor] (107.16,238.50) circle (  0.99);

\path[draw=drawColor,line width= 0.4pt,line join=round,line cap=round,fill=fillColor] (107.58,210.56) circle (  0.99);

\path[draw=drawColor,line width= 0.4pt,line join=round,line cap=round,fill=fillColor] (108.01,185.66) circle (  0.99);

\path[draw=drawColor,line width= 0.4pt,line join=round,line cap=round,fill=fillColor] (108.44,197.80) circle (  0.99);

\path[draw=drawColor,line width= 0.4pt,line join=round,line cap=round,fill=fillColor] (108.87,200.23) circle (  0.99);

\path[draw=drawColor,line width= 0.4pt,line join=round,line cap=round,fill=fillColor] (109.30,224.53) circle (  0.99);

\path[draw=drawColor,line width= 0.4pt,line join=round,line cap=round,fill=fillColor] (109.72,223.31) circle (  0.99);

\path[draw=drawColor,line width= 0.4pt,line join=round,line cap=round,fill=fillColor] (110.15,206.31) circle (  0.99);

\path[draw=drawColor,line width= 0.4pt,line join=round,line cap=round,fill=fillColor] (110.58,209.34) circle (  0.99);

\path[draw=drawColor,line width= 0.4pt,line join=round,line cap=round,fill=fillColor] (111.01,209.95) circle (  0.99);

\path[draw=drawColor,line width= 0.4pt,line join=round,line cap=round,fill=fillColor] (111.43,195.98) circle (  0.99);

\path[draw=drawColor,line width= 0.4pt,line join=round,line cap=round,fill=fillColor] (111.86,199.02) circle (  0.99);

\path[draw=drawColor,line width= 0.4pt,line join=round,line cap=round,fill=fillColor] (112.29,198.41) circle (  0.99);

\path[draw=drawColor,line width= 0.4pt,line join=round,line cap=round,fill=fillColor] (112.72,188.09) circle (  0.99);

\path[draw=drawColor,line width= 0.4pt,line join=round,line cap=round,fill=fillColor] (113.15,193.55) circle (  0.99);

\path[draw=drawColor,line width= 0.4pt,line join=round,line cap=round,fill=fillColor] (113.57,194.77) circle (  0.99);

\path[draw=drawColor,line width= 0.4pt,line join=round,line cap=round,fill=fillColor] (114.00,199.02) circle (  0.99);

\path[draw=drawColor,line width= 0.4pt,line join=round,line cap=round,fill=fillColor] (114.43,236.07) circle (  0.99);

\path[draw=drawColor,line width= 0.4pt,line join=round,line cap=round,fill=fillColor] (114.86,205.09) circle (  0.99);

\path[draw=drawColor,line width= 0.4pt,line join=round,line cap=round,fill=fillColor] (115.29,189.30) circle (  0.99);

\path[draw=drawColor,line width= 0.4pt,line join=round,line cap=round,fill=fillColor] (115.71,196.59) circle (  0.99);

\path[draw=drawColor,line width= 0.4pt,line join=round,line cap=round,fill=fillColor] (116.14,195.98) circle (  0.99);

\path[draw=drawColor,line width= 0.4pt,line join=round,line cap=round,fill=fillColor] (116.57,174.12) circle (  0.99);

\path[draw=drawColor,line width= 0.4pt,line join=round,line cap=round,fill=fillColor] (117.00,195.98) circle (  0.99);

\path[draw=drawColor,line width= 0.4pt,line join=round,line cap=round,fill=fillColor] (117.42,183.23) circle (  0.99);

\path[draw=drawColor,line width= 0.4pt,line join=round,line cap=round,fill=fillColor] (117.85,177.15) circle (  0.99);

\path[draw=drawColor,line width= 0.4pt,line join=round,line cap=round,fill=fillColor] (118.28,182.62) circle (  0.99);

\path[draw=drawColor,line width= 0.4pt,line join=round,line cap=round,fill=fillColor] (118.71,167.44) circle (  0.99);

\path[draw=drawColor,line width= 0.4pt,line join=round,line cap=round,fill=fillColor] (119.14,214.20) circle (  0.99);
\end{scope}
\begin{scope}
\path[clip] (  0.00,  0.00) rectangle (126.47,505.89);
\definecolor{drawColor}{RGB}{0,0,0}

\node[text=drawColor,anchor=base,inner sep=0pt, outer sep=0pt, scale=  1.32] at ( 81.06,495.79) {\bfseries \textsf{0-bit}};

\node[text=drawColor,anchor=base,inner sep=0pt, outer sep=0pt, scale=  1.32] at ( 81.06,  5.54) {Vp};
\end{scope}
\begin{scope}
\path[clip] (  0.00,  0.00) rectangle (505.89,505.89);
\definecolor{drawColor}{RGB}{0,0,0}

\path[draw=drawColor,line width= 0.4pt,line join=round,line cap=round] ( 43.83, 35.64) -- (118.71, 35.64);

\path[draw=drawColor,line width= 0.4pt,line join=round,line cap=round] ( 43.83, 35.64) -- ( 43.83, 31.68);

\path[draw=drawColor,line width= 0.4pt,line join=round,line cap=round] ( 69.08, 35.64) -- ( 69.08, 31.68);

\path[draw=drawColor,line width= 0.4pt,line join=round,line cap=round] ( 94.75, 35.64) -- ( 94.75, 31.68);

\path[draw=drawColor,line width= 0.4pt,line join=round,line cap=round] (118.71, 35.64) -- (118.71, 31.68);

\node[text=drawColor,anchor=base,inner sep=0pt, outer sep=0pt, scale=  0.99] at ( 43.83, 21.38) {1};

\node[text=drawColor,anchor=base,inner sep=0pt, outer sep=0pt, scale=  0.99] at ( 69.08, 21.38) {60};

\node[text=drawColor,anchor=base,inner sep=0pt, outer sep=0pt, scale=  0.99] at ( 94.75, 21.38) {120};

\node[text=drawColor,anchor=base,inner sep=0pt, outer sep=0pt, scale=  0.99] at (118.71, 21.38) {176};

\path[draw=drawColor,line width= 0.4pt,line join=round,line cap=round] ( 40.39, 52.65) -- ( 40.39,477.80);

\path[draw=drawColor,line width= 0.4pt,line join=round,line cap=round] ( 40.39, 52.65) -- ( 36.43, 52.65);

\path[draw=drawColor,line width= 0.4pt,line join=round,line cap=round] ( 40.39,113.38) -- ( 36.43,113.38);

\path[draw=drawColor,line width= 0.4pt,line join=round,line cap=round] ( 40.39,174.12) -- ( 36.43,174.12);

\path[draw=drawColor,line width= 0.4pt,line join=round,line cap=round] ( 40.39,234.85) -- ( 36.43,234.85);

\path[draw=drawColor,line width= 0.4pt,line join=round,line cap=round] ( 40.39,295.59) -- ( 36.43,295.59);

\path[draw=drawColor,line width= 0.4pt,line join=round,line cap=round] ( 40.39,356.32) -- ( 36.43,356.32);

\path[draw=drawColor,line width= 0.4pt,line join=round,line cap=round] ( 40.39,417.06) -- ( 36.43,417.06);

\path[draw=drawColor,line width= 0.4pt,line join=round,line cap=round] ( 40.39,477.80) -- ( 36.43,477.80);

\node[text=drawColor,anchor=base east,inner sep=0pt, outer sep=0pt, scale=  0.99] at ( 32.47, 49.24) {0};

\node[text=drawColor,anchor=base east,inner sep=0pt, outer sep=0pt, scale=  0.99] at ( 32.47,109.97) {100};

\node[text=drawColor,anchor=base east,inner sep=0pt, outer sep=0pt, scale=  0.99] at ( 32.47,170.71) {200};

\node[text=drawColor,anchor=base east,inner sep=0pt, outer sep=0pt, scale=  0.99] at ( 32.47,231.44) {300};

\node[text=drawColor,anchor=base east,inner sep=0pt, outer sep=0pt, scale=  0.99] at ( 32.47,292.18) {400};

\node[text=drawColor,anchor=base east,inner sep=0pt, outer sep=0pt, scale=  0.99] at ( 32.47,352.92) {500};

\node[text=drawColor,anchor=base east,inner sep=0pt, outer sep=0pt, scale=  0.99] at ( 32.47,413.65) {600};

\node[text=drawColor,anchor=base east,inner sep=0pt, outer sep=0pt, scale=  0.99] at ( 32.47,474.39) {700};
\end{scope}
\begin{scope}
\path[clip] ( 40.39, 35.64) rectangle (121.72,494.80);
\definecolor{drawColor}{RGB}{0,0,0}

\path[draw=drawColor,line width= 0.4pt,line join=round,line cap=round] ( 40.39,198.42) -- (121.72,198.42);
\end{scope}
\begin{scope}
\path[clip] (  0.00,  0.00) rectangle (126.47,505.89);
\definecolor{drawColor}{RGB}{0,0,0}

\node[text=drawColor,rotate= 90.00,anchor=base,inner sep=0pt, outer sep=0pt, scale=  1.32] at ( 11.09,265.22) {Mittlere Reaktionszeit (ms)};
\end{scope}
\begin{scope}
\path[clip] (166.86, 35.64) rectangle (248.19,494.80);
\definecolor{drawColor}{RGB}{0,0,0}
\definecolor{fillColor}{RGB}{0,0,0}

\path[draw=drawColor,line width= 0.4pt,line join=round,line cap=round,fill=fillColor] (170.30,273.72) circle (  0.99);

\path[draw=drawColor,line width= 0.4pt,line join=round,line cap=round,fill=fillColor] (170.73,223.31) circle (  0.99);

\path[draw=drawColor,line width= 0.4pt,line join=round,line cap=round,fill=fillColor] (171.16,226.35) circle (  0.99);

\path[draw=drawColor,line width= 0.4pt,line join=round,line cap=round,fill=fillColor] (171.59,236.68) circle (  0.99);

\path[draw=drawColor,line width= 0.4pt,line join=round,line cap=round,fill=fillColor] (172.02,216.03) circle (  0.99);

\path[draw=drawColor,line width= 0.4pt,line join=round,line cap=round,fill=fillColor] (172.44,200.84) circle (  0.99);

\path[draw=drawColor,line width= 0.4pt,line join=round,line cap=round,fill=fillColor] (172.87,213.60) circle (  0.99);

\path[draw=drawColor,line width= 0.4pt,line join=round,line cap=round,fill=fillColor] (173.30,211.17) circle (  0.99);

\path[draw=drawColor,line width= 0.4pt,line join=round,line cap=round,fill=fillColor] (173.73,229.39) circle (  0.99);

\path[draw=drawColor,line width= 0.4pt,line join=round,line cap=round,fill=fillColor] (174.16,247.00) circle (  0.99);

\path[draw=drawColor,line width= 0.4pt,line join=round,line cap=round,fill=fillColor] (174.58,251.25) circle (  0.99);

\path[draw=drawColor,line width= 0.4pt,line join=round,line cap=round,fill=fillColor] (175.01,207.52) circle (  0.99);

\path[draw=drawColor,line width= 0.4pt,line join=round,line cap=round,fill=fillColor] (175.44,220.88) circle (  0.99);

\path[draw=drawColor,line width= 0.4pt,line join=round,line cap=round,fill=fillColor] (175.87,227.56) circle (  0.99);

\path[draw=drawColor,line width= 0.4pt,line join=round,line cap=round,fill=fillColor] (176.29,224.53) circle (  0.99);

\path[draw=drawColor,line width= 0.4pt,line join=round,line cap=round,fill=fillColor] (176.72,236.68) circle (  0.99);

\path[draw=drawColor,line width= 0.4pt,line join=round,line cap=round,fill=fillColor] (177.15,209.34) circle (  0.99);

\path[draw=drawColor,line width= 0.4pt,line join=round,line cap=round,fill=fillColor] (177.58,228.17) circle (  0.99);

\path[draw=drawColor,line width= 0.4pt,line join=round,line cap=round,fill=fillColor] (178.01,223.92) circle (  0.99);

\path[draw=drawColor,line width= 0.4pt,line join=round,line cap=round,fill=fillColor] (178.43,214.81) circle (  0.99);

\path[draw=drawColor,line width= 0.4pt,line join=round,line cap=round,fill=fillColor] (178.86,281.01) circle (  0.99);

\path[draw=drawColor,line width= 0.4pt,line join=round,line cap=round,fill=fillColor] (179.29,219.67) circle (  0.99);

\path[draw=drawColor,line width= 0.4pt,line join=round,line cap=round,fill=fillColor] (179.72,214.81) circle (  0.99);

\path[draw=drawColor,line width= 0.4pt,line join=round,line cap=round,fill=fillColor] (180.15,227.56) circle (  0.99);

\path[draw=drawColor,line width= 0.4pt,line join=round,line cap=round,fill=fillColor] (180.57,236.07) circle (  0.99);

\path[draw=drawColor,line width= 0.4pt,line join=round,line cap=round,fill=fillColor] (181.00,220.28) circle (  0.99);

\path[draw=drawColor,line width= 0.4pt,line join=round,line cap=round,fill=fillColor] (181.43,226.35) circle (  0.99);

\path[draw=drawColor,line width= 0.4pt,line join=round,line cap=round,fill=fillColor] (181.86,222.71) circle (  0.99);

\path[draw=drawColor,line width= 0.4pt,line join=round,line cap=round,fill=fillColor] (182.28,194.77) circle (  0.99);

\path[draw=drawColor,line width= 0.4pt,line join=round,line cap=round,fill=fillColor] (182.71,239.10) circle (  0.99);

\path[draw=drawColor,line width= 0.4pt,line join=round,line cap=round,fill=fillColor] (183.14,239.10) circle (  0.99);

\path[draw=drawColor,line width= 0.4pt,line join=round,line cap=round,fill=fillColor] (183.57,262.79) circle (  0.99);

\path[draw=drawColor,line width= 0.4pt,line join=round,line cap=round,fill=fillColor] (184.00,205.70) circle (  0.99);

\path[draw=drawColor,line width= 0.4pt,line join=round,line cap=round,fill=fillColor] (184.42,222.10) circle (  0.99);

\path[draw=drawColor,line width= 0.4pt,line join=round,line cap=round,fill=fillColor] (184.85,201.45) circle (  0.99);

\path[draw=drawColor,line width= 0.4pt,line join=round,line cap=round,fill=fillColor] (185.28,228.78) circle (  0.99);

\path[draw=drawColor,line width= 0.4pt,line join=round,line cap=round,fill=fillColor] (185.71,236.68) circle (  0.99);

\path[draw=drawColor,line width= 0.4pt,line join=round,line cap=round,fill=fillColor] (186.14,224.53) circle (  0.99);

\path[draw=drawColor,line width= 0.4pt,line join=round,line cap=round,fill=fillColor] (186.56,237.28) circle (  0.99);

\path[draw=drawColor,line width= 0.4pt,line join=round,line cap=round,fill=fillColor] (186.99,237.28) circle (  0.99);

\path[draw=drawColor,line width= 0.4pt,line join=round,line cap=round,fill=fillColor] (187.42,225.74) circle (  0.99);

\path[draw=drawColor,line width= 0.4pt,line join=round,line cap=round,fill=fillColor] (187.85,214.20) circle (  0.99);

\path[draw=drawColor,line width= 0.4pt,line join=round,line cap=round,fill=fillColor] (188.27,223.92) circle (  0.99);

\path[draw=drawColor,line width= 0.4pt,line join=round,line cap=round,fill=fillColor] (188.70,225.74) circle (  0.99);

\path[draw=drawColor,line width= 0.4pt,line join=round,line cap=round,fill=fillColor] (189.13,196.59) circle (  0.99);

\path[draw=drawColor,line width= 0.4pt,line join=round,line cap=round,fill=fillColor] (189.56,237.28) circle (  0.99);

\path[draw=drawColor,line width= 0.4pt,line join=round,line cap=round,fill=fillColor] (189.99,228.17) circle (  0.99);

\path[draw=drawColor,line width= 0.4pt,line join=round,line cap=round,fill=fillColor] (190.41,236.07) circle (  0.99);

\path[draw=drawColor,line width= 0.4pt,line join=round,line cap=round,fill=fillColor] (190.84,233.64) circle (  0.99);

\path[draw=drawColor,line width= 0.4pt,line join=round,line cap=round,fill=fillColor] (191.27,234.85) circle (  0.99);

\path[draw=drawColor,line width= 0.4pt,line join=round,line cap=round,fill=fillColor] (191.70,225.14) circle (  0.99);

\path[draw=drawColor,line width= 0.4pt,line join=round,line cap=round,fill=fillColor] (192.13,237.28) circle (  0.99);

\path[draw=drawColor,line width= 0.4pt,line join=round,line cap=round,fill=fillColor] (192.55,245.18) circle (  0.99);

\path[draw=drawColor,line width= 0.4pt,line join=round,line cap=round,fill=fillColor] (192.98,233.03) circle (  0.99);

\path[draw=drawColor,line width= 0.4pt,line join=round,line cap=round,fill=fillColor] (193.41,268.87) circle (  0.99);

\path[draw=drawColor,line width= 0.4pt,line join=round,line cap=round,fill=fillColor] (193.84,300.45) circle (  0.99);

\path[draw=drawColor,line width= 0.4pt,line join=round,line cap=round,fill=fillColor] (194.26,264.01) circle (  0.99);

\path[draw=drawColor,line width= 0.4pt,line join=round,line cap=round,fill=fillColor] (194.69,233.03) circle (  0.99);

\path[draw=drawColor,line width= 0.4pt,line join=round,line cap=round,fill=fillColor] (195.12,225.74) circle (  0.99);

\path[draw=drawColor,line width= 0.4pt,line join=round,line cap=round,fill=fillColor] (195.55,212.99) circle (  0.99);

\path[draw=drawColor,line width= 0.4pt,line join=round,line cap=round,fill=fillColor] (195.98,274.94) circle (  0.99);

\path[draw=drawColor,line width= 0.4pt,line join=round,line cap=round,fill=fillColor] (196.40,261.58) circle (  0.99);

\path[draw=drawColor,line width= 0.4pt,line join=round,line cap=round,fill=fillColor] (196.83,207.52) circle (  0.99);

\path[draw=drawColor,line width= 0.4pt,line join=round,line cap=round,fill=fillColor] (197.26,229.39) circle (  0.99);

\path[draw=drawColor,line width= 0.4pt,line join=round,line cap=round,fill=fillColor] (197.69,214.20) circle (  0.99);

\path[draw=drawColor,line width= 0.4pt,line join=round,line cap=round,fill=fillColor] (198.12,243.96) circle (  0.99);

\path[draw=drawColor,line width= 0.4pt,line join=round,line cap=round,fill=fillColor] (198.54,228.17) circle (  0.99);

\path[draw=drawColor,line width= 0.4pt,line join=round,line cap=round,fill=fillColor] (198.97,257.33) circle (  0.99);

\path[draw=drawColor,line width= 0.4pt,line join=round,line cap=round,fill=fillColor] (199.40,251.86) circle (  0.99);

\path[draw=drawColor,line width= 0.4pt,line join=round,line cap=round,fill=fillColor] (199.83,225.14) circle (  0.99);

\path[draw=drawColor,line width= 0.4pt,line join=round,line cap=round,fill=fillColor] (200.26,214.20) circle (  0.99);

\path[draw=drawColor,line width= 0.4pt,line join=round,line cap=round,fill=fillColor] (200.68,224.53) circle (  0.99);

\path[draw=drawColor,line width= 0.4pt,line join=round,line cap=round,fill=fillColor] (201.11,243.96) circle (  0.99);

\path[draw=drawColor,line width= 0.4pt,line join=round,line cap=round,fill=fillColor] (201.54,221.49) circle (  0.99);

\path[draw=drawColor,line width= 0.4pt,line join=round,line cap=round,fill=fillColor] (201.97,236.68) circle (  0.99);

\path[draw=drawColor,line width= 0.4pt,line join=round,line cap=round,fill=fillColor] (202.39,209.95) circle (  0.99);

\path[draw=drawColor,line width= 0.4pt,line join=round,line cap=round,fill=fillColor] (202.82,261.58) circle (  0.99);

\path[draw=drawColor,line width= 0.4pt,line join=round,line cap=round,fill=fillColor] (203.25,227.56) circle (  0.99);

\path[draw=drawColor,line width= 0.4pt,line join=round,line cap=round,fill=fillColor] (203.68,223.31) circle (  0.99);

\path[draw=drawColor,line width= 0.4pt,line join=round,line cap=round,fill=fillColor] (204.11,225.74) circle (  0.99);

\path[draw=drawColor,line width= 0.4pt,line join=round,line cap=round,fill=fillColor] (204.53,223.92) circle (  0.99);

\path[draw=drawColor,line width= 0.4pt,line join=round,line cap=round,fill=fillColor] (204.96,250.64) circle (  0.99);

\path[draw=drawColor,line width= 0.4pt,line join=round,line cap=round,fill=fillColor] (205.39,228.17) circle (  0.99);

\path[draw=drawColor,line width= 0.4pt,line join=round,line cap=round,fill=fillColor] (205.82,282.23) circle (  0.99);

\path[draw=drawColor,line width= 0.4pt,line join=round,line cap=round,fill=fillColor] (206.25,231.21) circle (  0.99);

\path[draw=drawColor,line width= 0.4pt,line join=round,line cap=round,fill=fillColor] (206.67,264.01) circle (  0.99);

\path[draw=drawColor,line width= 0.4pt,line join=round,line cap=round,fill=fillColor] (207.10,219.06) circle (  0.99);

\path[draw=drawColor,line width= 0.4pt,line join=round,line cap=round,fill=fillColor] (207.53,233.03) circle (  0.99);

\path[draw=drawColor,line width= 0.4pt,line join=round,line cap=round,fill=fillColor] (207.96,251.86) circle (  0.99);

\path[draw=drawColor,line width= 0.4pt,line join=round,line cap=round,fill=fillColor] (208.38,209.95) circle (  0.99);

\path[draw=drawColor,line width= 0.4pt,line join=round,line cap=round,fill=fillColor] (208.81,208.74) circle (  0.99);

\path[draw=drawColor,line width= 0.4pt,line join=round,line cap=round,fill=fillColor] (209.24,224.53) circle (  0.99);

\path[draw=drawColor,line width= 0.4pt,line join=round,line cap=round,fill=fillColor] (209.67,224.53) circle (  0.99);

\path[draw=drawColor,line width= 0.4pt,line join=round,line cap=round,fill=fillColor] (210.10,256.11) circle (  0.99);

\path[draw=drawColor,line width= 0.4pt,line join=round,line cap=round,fill=fillColor] (210.52,223.31) circle (  0.99);

\path[draw=drawColor,line width= 0.4pt,line join=round,line cap=round,fill=fillColor] (210.95,206.91) circle (  0.99);

\path[draw=drawColor,line width= 0.4pt,line join=round,line cap=round,fill=fillColor] (211.38,238.50) circle (  0.99);

\path[draw=drawColor,line width= 0.4pt,line join=round,line cap=round,fill=fillColor] (211.81,229.39) circle (  0.99);

\path[draw=drawColor,line width= 0.4pt,line join=round,line cap=round,fill=fillColor] (212.24,227.56) circle (  0.99);

\path[draw=drawColor,line width= 0.4pt,line join=round,line cap=round,fill=fillColor] (212.66,224.53) circle (  0.99);

\path[draw=drawColor,line width= 0.4pt,line join=round,line cap=round,fill=fillColor] (213.09,251.86) circle (  0.99);

\path[draw=drawColor,line width= 0.4pt,line join=round,line cap=round,fill=fillColor] (213.52,226.96) circle (  0.99);

\path[draw=drawColor,line width= 0.4pt,line join=round,line cap=round,fill=fillColor] (213.95,243.96) circle (  0.99);

\path[draw=drawColor,line width= 0.4pt,line join=round,line cap=round,fill=fillColor] (214.37,220.88) circle (  0.99);

\path[draw=drawColor,line width= 0.4pt,line join=round,line cap=round,fill=fillColor] (214.80,223.31) circle (  0.99);

\path[draw=drawColor,line width= 0.4pt,line join=round,line cap=round,fill=fillColor] (215.23,217.85) circle (  0.99);

\path[draw=drawColor,line width= 0.4pt,line join=round,line cap=round,fill=fillColor] (215.66,230.60) circle (  0.99);

\path[draw=drawColor,line width= 0.4pt,line join=round,line cap=round,fill=fillColor] (216.09,229.99) circle (  0.99);

\path[draw=drawColor,line width= 0.4pt,line join=round,line cap=round,fill=fillColor] (216.51,224.53) circle (  0.99);

\path[draw=drawColor,line width= 0.4pt,line join=round,line cap=round,fill=fillColor] (216.94,242.75) circle (  0.99);

\path[draw=drawColor,line width= 0.4pt,line join=round,line cap=round,fill=fillColor] (217.37,265.83) circle (  0.99);

\path[draw=drawColor,line width= 0.4pt,line join=round,line cap=round,fill=fillColor] (217.80,260.36) circle (  0.99);

\path[draw=drawColor,line width= 0.4pt,line join=round,line cap=round,fill=fillColor] (218.23,229.99) circle (  0.99);

\path[draw=drawColor,line width= 0.4pt,line join=round,line cap=round,fill=fillColor] (218.65,239.10) circle (  0.99);

\path[draw=drawColor,line width= 0.4pt,line join=round,line cap=round,fill=fillColor] (219.08,218.45) circle (  0.99);

\path[draw=drawColor,line width= 0.4pt,line join=round,line cap=round,fill=fillColor] (219.51,227.56) circle (  0.99);

\path[draw=drawColor,line width= 0.4pt,line join=round,line cap=round,fill=fillColor] (219.94,264.01) circle (  0.99);

\path[draw=drawColor,line width= 0.4pt,line join=round,line cap=round,fill=fillColor] (220.36,230.60) circle (  0.99);

\path[draw=drawColor,line width= 0.4pt,line join=round,line cap=round,fill=fillColor] (220.79,209.34) circle (  0.99);

\path[draw=drawColor,line width= 0.4pt,line join=round,line cap=round,fill=fillColor] (221.22,211.17) circle (  0.99);

\path[draw=drawColor,line width= 0.4pt,line join=round,line cap=round,fill=fillColor] (221.65,219.06) circle (  0.99);

\path[draw=drawColor,line width= 0.4pt,line join=round,line cap=round,fill=fillColor] (222.08,243.36) circle (  0.99);

\path[draw=drawColor,line width= 0.4pt,line join=round,line cap=round,fill=fillColor] (222.50,248.82) circle (  0.99);

\path[draw=drawColor,line width= 0.4pt,line join=round,line cap=round,fill=fillColor] (222.93,218.45) circle (  0.99);

\path[draw=drawColor,line width= 0.4pt,line join=round,line cap=round,fill=fillColor] (223.36,250.64) circle (  0.99);

\path[draw=drawColor,line width= 0.4pt,line join=round,line cap=round,fill=fillColor] (223.79,234.25) circle (  0.99);

\path[draw=drawColor,line width= 0.4pt,line join=round,line cap=round,fill=fillColor] (224.22,237.89) circle (  0.99);

\path[draw=drawColor,line width= 0.4pt,line join=round,line cap=round,fill=fillColor] (224.64,240.93) circle (  0.99);

\path[draw=drawColor,line width= 0.4pt,line join=round,line cap=round,fill=fillColor] (225.07,234.25) circle (  0.99);

\path[draw=drawColor,line width= 0.4pt,line join=round,line cap=round,fill=fillColor] (225.50,212.99) circle (  0.99);

\path[draw=drawColor,line width= 0.4pt,line join=round,line cap=round,fill=fillColor] (225.93,263.40) circle (  0.99);

\path[draw=drawColor,line width= 0.4pt,line join=round,line cap=round,fill=fillColor] (226.35,229.99) circle (  0.99);

\path[draw=drawColor,line width= 0.4pt,line join=round,line cap=round,fill=fillColor] (226.78,226.96) circle (  0.99);

\path[draw=drawColor,line width= 0.4pt,line join=round,line cap=round,fill=fillColor] (227.21,250.04) circle (  0.99);

\path[draw=drawColor,line width= 0.4pt,line join=round,line cap=round,fill=fillColor] (227.64,226.96) circle (  0.99);

\path[draw=drawColor,line width= 0.4pt,line join=round,line cap=round,fill=fillColor] (228.07,257.93) circle (  0.99);

\path[draw=drawColor,line width= 0.4pt,line join=round,line cap=round,fill=fillColor] (228.49,222.10) circle (  0.99);

\path[draw=drawColor,line width= 0.4pt,line join=round,line cap=round,fill=fillColor] (228.92,241.53) circle (  0.99);

\path[draw=drawColor,line width= 0.4pt,line join=round,line cap=round,fill=fillColor] (229.35,242.75) circle (  0.99);

\path[draw=drawColor,line width= 0.4pt,line join=round,line cap=round,fill=fillColor] (229.78,222.10) circle (  0.99);

\path[draw=drawColor,line width= 0.4pt,line join=round,line cap=round,fill=fillColor] (230.21,274.33) circle (  0.99);

\path[draw=drawColor,line width= 0.4pt,line join=round,line cap=round,fill=fillColor] (230.63,221.49) circle (  0.99);

\path[draw=drawColor,line width= 0.4pt,line join=round,line cap=round,fill=fillColor] (231.06,206.31) circle (  0.99);

\path[draw=drawColor,line width= 0.4pt,line join=round,line cap=round,fill=fillColor] (231.49,212.99) circle (  0.99);

\path[draw=drawColor,line width= 0.4pt,line join=round,line cap=round,fill=fillColor] (231.92,219.67) circle (  0.99);

\path[draw=drawColor,line width= 0.4pt,line join=round,line cap=round,fill=fillColor] (232.34,198.41) circle (  0.99);

\path[draw=drawColor,line width= 0.4pt,line join=round,line cap=round,fill=fillColor] (232.77,237.89) circle (  0.99);

\path[draw=drawColor,line width= 0.4pt,line join=round,line cap=round,fill=fillColor] (233.20,232.42) circle (  0.99);

\path[draw=drawColor,line width= 0.4pt,line join=round,line cap=round,fill=fillColor] (233.63,305.31) circle (  0.99);

\path[draw=drawColor,line width= 0.4pt,line join=round,line cap=round,fill=fillColor] (234.06,250.64) circle (  0.99);

\path[draw=drawColor,line width= 0.4pt,line join=round,line cap=round,fill=fillColor] (234.48,228.78) circle (  0.99);

\path[draw=drawColor,line width= 0.4pt,line join=round,line cap=round,fill=fillColor] (234.91,256.72) circle (  0.99);

\path[draw=drawColor,line width= 0.4pt,line join=round,line cap=round,fill=fillColor] (235.34,236.68) circle (  0.99);

\path[draw=drawColor,line width= 0.4pt,line join=round,line cap=round,fill=fillColor] (235.77,264.61) circle (  0.99);

\path[draw=drawColor,line width= 0.4pt,line join=round,line cap=round,fill=fillColor] (236.20,245.18) circle (  0.99);

\path[draw=drawColor,line width= 0.4pt,line join=round,line cap=round,fill=fillColor] (236.62,229.99) circle (  0.99);

\path[draw=drawColor,line width= 0.4pt,line join=round,line cap=round,fill=fillColor] (237.05,237.28) circle (  0.99);

\path[draw=drawColor,line width= 0.4pt,line join=round,line cap=round,fill=fillColor] (237.48,225.14) circle (  0.99);

\path[draw=drawColor,line width= 0.4pt,line join=round,line cap=round,fill=fillColor] (237.91,234.25) circle (  0.99);

\path[draw=drawColor,line width= 0.4pt,line join=round,line cap=round,fill=fillColor] (238.33,242.75) circle (  0.99);

\path[draw=drawColor,line width= 0.4pt,line join=round,line cap=round,fill=fillColor] (238.76,238.50) circle (  0.99);

\path[draw=drawColor,line width= 0.4pt,line join=round,line cap=round,fill=fillColor] (239.19,217.85) circle (  0.99);

\path[draw=drawColor,line width= 0.4pt,line join=round,line cap=round,fill=fillColor] (239.62,240.32) circle (  0.99);

\path[draw=drawColor,line width= 0.4pt,line join=round,line cap=round,fill=fillColor] (240.05,225.74) circle (  0.99);

\path[draw=drawColor,line width= 0.4pt,line join=round,line cap=round,fill=fillColor] (240.47,237.28) circle (  0.99);

\path[draw=drawColor,line width= 0.4pt,line join=round,line cap=round,fill=fillColor] (240.90,281.01) circle (  0.99);

\path[draw=drawColor,line width= 0.4pt,line join=round,line cap=round,fill=fillColor] (241.33,240.93) circle (  0.99);

\path[draw=drawColor,line width= 0.4pt,line join=round,line cap=round,fill=fillColor] (241.76,221.49) circle (  0.99);

\path[draw=drawColor,line width= 0.4pt,line join=round,line cap=round,fill=fillColor] (242.19,217.24) circle (  0.99);

\path[draw=drawColor,line width= 0.4pt,line join=round,line cap=round,fill=fillColor] (242.61,245.79) circle (  0.99);

\path[draw=drawColor,line width= 0.4pt,line join=round,line cap=round,fill=fillColor] (243.04,202.06) circle (  0.99);

\path[draw=drawColor,line width= 0.4pt,line join=round,line cap=round,fill=fillColor] (243.47,231.21) circle (  0.99);

\path[draw=drawColor,line width= 0.4pt,line join=round,line cap=round,fill=fillColor] (243.90,212.38) circle (  0.99);

\path[draw=drawColor,line width= 0.4pt,line join=round,line cap=round,fill=fillColor] (244.33,218.45) circle (  0.99);

\path[draw=drawColor,line width= 0.4pt,line join=round,line cap=round,fill=fillColor] (244.75,208.74) circle (  0.99);

\path[draw=drawColor,line width= 0.4pt,line join=round,line cap=round,fill=fillColor] (245.18,207.52) circle (  0.99);

\path[draw=drawColor,line width= 0.4pt,line join=round,line cap=round,fill=fillColor] (245.61,234.25) circle (  0.99);
\end{scope}
\begin{scope}
\path[clip] (126.47,  0.00) rectangle (252.94,505.89);
\definecolor{drawColor}{RGB}{0,0,0}

\node[text=drawColor,anchor=base,inner sep=0pt, outer sep=0pt, scale=  1.32] at (207.53,495.79) {\bfseries \textsf{1-bit}};

\node[text=drawColor,anchor=base,inner sep=0pt, outer sep=0pt, scale=  1.32] at (207.53,  5.54) {Vp};
\end{scope}
\begin{scope}
\path[clip] (  0.00,  0.00) rectangle (505.89,505.89);
\definecolor{drawColor}{RGB}{0,0,0}

\path[draw=drawColor,line width= 0.4pt,line join=round,line cap=round] (170.30, 35.64) -- (245.18, 35.64);

\path[draw=drawColor,line width= 0.4pt,line join=round,line cap=round] (170.30, 35.64) -- (170.30, 31.68);

\path[draw=drawColor,line width= 0.4pt,line join=round,line cap=round] (195.55, 35.64) -- (195.55, 31.68);

\path[draw=drawColor,line width= 0.4pt,line join=round,line cap=round] (221.22, 35.64) -- (221.22, 31.68);

\path[draw=drawColor,line width= 0.4pt,line join=round,line cap=round] (245.18, 35.64) -- (245.18, 31.68);

\node[text=drawColor,anchor=base,inner sep=0pt, outer sep=0pt, scale=  0.99] at (170.30, 21.38) {1};

\node[text=drawColor,anchor=base,inner sep=0pt, outer sep=0pt, scale=  0.99] at (195.55, 21.38) {60};

\node[text=drawColor,anchor=base,inner sep=0pt, outer sep=0pt, scale=  0.99] at (221.22, 21.38) {120};

\node[text=drawColor,anchor=base,inner sep=0pt, outer sep=0pt, scale=  0.99] at (245.18, 21.38) {176};

\path[draw=drawColor,line width= 0.4pt,line join=round,line cap=round] (166.86, 52.65) -- (166.86,477.80);

\path[draw=drawColor,line width= 0.4pt,line join=round,line cap=round] (166.86, 52.65) -- (162.90, 52.65);

\path[draw=drawColor,line width= 0.4pt,line join=round,line cap=round] (166.86,113.38) -- (162.90,113.38);

\path[draw=drawColor,line width= 0.4pt,line join=round,line cap=round] (166.86,174.12) -- (162.90,174.12);

\path[draw=drawColor,line width= 0.4pt,line join=round,line cap=round] (166.86,234.85) -- (162.90,234.85);

\path[draw=drawColor,line width= 0.4pt,line join=round,line cap=round] (166.86,295.59) -- (162.90,295.59);

\path[draw=drawColor,line width= 0.4pt,line join=round,line cap=round] (166.86,356.32) -- (162.90,356.32);

\path[draw=drawColor,line width= 0.4pt,line join=round,line cap=round] (166.86,417.06) -- (162.90,417.06);

\path[draw=drawColor,line width= 0.4pt,line join=round,line cap=round] (166.86,477.80) -- (162.90,477.80);

\node[text=drawColor,anchor=base east,inner sep=0pt, outer sep=0pt, scale=  0.99] at (158.94, 49.24) {0};

\node[text=drawColor,anchor=base east,inner sep=0pt, outer sep=0pt, scale=  0.99] at (158.94,109.97) {100};

\node[text=drawColor,anchor=base east,inner sep=0pt, outer sep=0pt, scale=  0.99] at (158.94,170.71) {200};

\node[text=drawColor,anchor=base east,inner sep=0pt, outer sep=0pt, scale=  0.99] at (158.94,231.44) {300};

\node[text=drawColor,anchor=base east,inner sep=0pt, outer sep=0pt, scale=  0.99] at (158.94,292.18) {400};

\node[text=drawColor,anchor=base east,inner sep=0pt, outer sep=0pt, scale=  0.99] at (158.94,352.92) {500};

\node[text=drawColor,anchor=base east,inner sep=0pt, outer sep=0pt, scale=  0.99] at (158.94,413.65) {600};

\node[text=drawColor,anchor=base east,inner sep=0pt, outer sep=0pt, scale=  0.99] at (158.94,474.39) {700};
\end{scope}
\begin{scope}
\path[clip] (166.86, 35.64) rectangle (248.19,494.80);
\definecolor{drawColor}{RGB}{0,0,0}

\path[draw=drawColor,line width= 0.4pt,line join=round,line cap=round] (166.86,232.65) -- (248.19,232.65);
\end{scope}
\begin{scope}
\path[clip] (293.34, 35.64) rectangle (374.67,494.80);
\definecolor{drawColor}{RGB}{0,0,0}
\definecolor{fillColor}{RGB}{0,0,0}

\path[draw=drawColor,line width= 0.4pt,line join=round,line cap=round,fill=fillColor] (296.78,367.86) circle (  0.99);

\path[draw=drawColor,line width= 0.4pt,line join=round,line cap=round,fill=fillColor] (297.20,261.58) circle (  0.99);

\path[draw=drawColor,line width= 0.4pt,line join=round,line cap=round,fill=fillColor] (297.63,271.29) circle (  0.99);

\path[draw=drawColor,line width= 0.4pt,line join=round,line cap=round,fill=fillColor] (298.06,301.06) circle (  0.99);

\path[draw=drawColor,line width= 0.4pt,line join=round,line cap=round,fill=fillColor] (298.49,254.29) circle (  0.99);

\path[draw=drawColor,line width= 0.4pt,line join=round,line cap=round,fill=fillColor] (298.92,228.17) circle (  0.99);

\path[draw=drawColor,line width= 0.4pt,line join=round,line cap=round,fill=fillColor] (299.34,257.33) circle (  0.99);

\path[draw=drawColor,line width= 0.4pt,line join=round,line cap=round,fill=fillColor] (299.77,268.26) circle (  0.99);

\path[draw=drawColor,line width= 0.4pt,line join=round,line cap=round,fill=fillColor] (300.20,276.76) circle (  0.99);

\path[draw=drawColor,line width= 0.4pt,line join=round,line cap=round,fill=fillColor] (300.63,308.34) circle (  0.99);

\path[draw=drawColor,line width= 0.4pt,line join=round,line cap=round,fill=fillColor] (301.06,296.80) circle (  0.99);

\path[draw=drawColor,line width= 0.4pt,line join=round,line cap=round,fill=fillColor] (301.48,257.93) circle (  0.99);

\path[draw=drawColor,line width= 0.4pt,line join=round,line cap=round,fill=fillColor] (301.91,253.68) circle (  0.99);

\path[draw=drawColor,line width= 0.4pt,line join=round,line cap=round,fill=fillColor] (302.34,276.76) circle (  0.99);

\path[draw=drawColor,line width= 0.4pt,line join=round,line cap=round,fill=fillColor] (302.77,250.64) circle (  0.99);

\path[draw=drawColor,line width= 0.4pt,line join=round,line cap=round,fill=fillColor] (303.19,285.26) circle (  0.99);

\path[draw=drawColor,line width= 0.4pt,line join=round,line cap=round,fill=fillColor] (303.62,241.53) circle (  0.99);

\path[draw=drawColor,line width= 0.4pt,line join=round,line cap=round,fill=fillColor] (304.05,262.79) circle (  0.99);

\path[draw=drawColor,line width= 0.4pt,line join=round,line cap=round,fill=fillColor] (304.48,290.12) circle (  0.99);

\path[draw=drawColor,line width= 0.4pt,line join=round,line cap=round,fill=fillColor] (304.91,250.64) circle (  0.99);

\path[draw=drawColor,line width= 0.4pt,line join=round,line cap=round,fill=fillColor] (305.33,332.64) circle (  0.99);

\path[draw=drawColor,line width= 0.4pt,line join=round,line cap=round,fill=fillColor] (305.76,262.18) circle (  0.99);

\path[draw=drawColor,line width= 0.4pt,line join=round,line cap=round,fill=fillColor] (306.19,296.80) circle (  0.99);

\path[draw=drawColor,line width= 0.4pt,line join=round,line cap=round,fill=fillColor] (306.62,279.80) circle (  0.99);

\path[draw=drawColor,line width= 0.4pt,line join=round,line cap=round,fill=fillColor] (307.05,281.01) circle (  0.99);

\path[draw=drawColor,line width= 0.4pt,line join=round,line cap=round,fill=fillColor] (307.47,253.68) circle (  0.99);

\path[draw=drawColor,line width= 0.4pt,line join=round,line cap=round,fill=fillColor] (307.90,284.05) circle (  0.99);

\path[draw=drawColor,line width= 0.4pt,line join=round,line cap=round,fill=fillColor] (308.33,251.86) circle (  0.99);

\path[draw=drawColor,line width= 0.4pt,line join=round,line cap=round,fill=fillColor] (308.76,242.75) circle (  0.99);

\path[draw=drawColor,line width= 0.4pt,line join=round,line cap=round,fill=fillColor] (309.19,285.87) circle (  0.99);

\path[draw=drawColor,line width= 0.4pt,line join=round,line cap=round,fill=fillColor] (309.61,277.98) circle (  0.99);

\path[draw=drawColor,line width= 0.4pt,line join=round,line cap=round,fill=fillColor] (310.04,329.60) circle (  0.99);

\path[draw=drawColor,line width= 0.4pt,line join=round,line cap=round,fill=fillColor] (310.47,240.32) circle (  0.99);

\path[draw=drawColor,line width= 0.4pt,line join=round,line cap=round,fill=fillColor] (310.90,251.25) circle (  0.99);

\path[draw=drawColor,line width= 0.4pt,line join=round,line cap=round,fill=fillColor] (311.32,222.71) circle (  0.99);

\path[draw=drawColor,line width= 0.4pt,line join=round,line cap=round,fill=fillColor] (311.75,288.91) circle (  0.99);

\path[draw=drawColor,line width= 0.4pt,line join=round,line cap=round,fill=fillColor] (312.18,296.20) circle (  0.99);

\path[draw=drawColor,line width= 0.4pt,line join=round,line cap=round,fill=fillColor] (312.61,256.72) circle (  0.99);

\path[draw=drawColor,line width= 0.4pt,line join=round,line cap=round,fill=fillColor] (313.04,342.96) circle (  0.99);

\path[draw=drawColor,line width= 0.4pt,line join=round,line cap=round,fill=fillColor] (313.46,294.98) circle (  0.99);

\path[draw=drawColor,line width= 0.4pt,line join=round,line cap=round,fill=fillColor] (313.89,287.69) circle (  0.99);

\path[draw=drawColor,line width= 0.4pt,line join=round,line cap=round,fill=fillColor] (314.32,272.51) circle (  0.99);

\path[draw=drawColor,line width= 0.4pt,line join=round,line cap=round,fill=fillColor] (314.75,280.40) circle (  0.99);

\path[draw=drawColor,line width= 0.4pt,line join=round,line cap=round,fill=fillColor] (315.18,263.40) circle (  0.99);

\path[draw=drawColor,line width= 0.4pt,line join=round,line cap=round,fill=fillColor] (315.60,240.32) circle (  0.99);

\path[draw=drawColor,line width= 0.4pt,line join=round,line cap=round,fill=fillColor] (316.03,322.31) circle (  0.99);

\path[draw=drawColor,line width= 0.4pt,line join=round,line cap=round,fill=fillColor] (316.46,231.82) circle (  0.99);

\path[draw=drawColor,line width= 0.4pt,line join=round,line cap=round,fill=fillColor] (316.89,290.73) circle (  0.99);

\path[draw=drawColor,line width= 0.4pt,line join=round,line cap=round,fill=fillColor] (317.31,288.30) circle (  0.99);

\path[draw=drawColor,line width= 0.4pt,line join=round,line cap=round,fill=fillColor] (317.74,284.05) circle (  0.99);

\path[draw=drawColor,line width= 0.4pt,line join=round,line cap=round,fill=fillColor] (318.17,270.08) circle (  0.99);

\path[draw=drawColor,line width= 0.4pt,line join=round,line cap=round,fill=fillColor] (318.60,267.04) circle (  0.99);

\path[draw=drawColor,line width= 0.4pt,line join=round,line cap=round,fill=fillColor] (319.03,291.34) circle (  0.99);

\path[draw=drawColor,line width= 0.4pt,line join=round,line cap=round,fill=fillColor] (319.45,278.58) circle (  0.99);

\path[draw=drawColor,line width= 0.4pt,line join=round,line cap=round,fill=fillColor] (319.88,282.23) circle (  0.99);

\path[draw=drawColor,line width= 0.4pt,line join=round,line cap=round,fill=fillColor] (320.31,300.45) circle (  0.99);

\path[draw=drawColor,line width= 0.4pt,line join=round,line cap=round,fill=fillColor] (320.74,410.99) circle (  0.99);

\path[draw=drawColor,line width= 0.4pt,line join=round,line cap=round,fill=fillColor] (321.17,290.12) circle (  0.99);

\path[draw=drawColor,line width= 0.4pt,line join=round,line cap=round,fill=fillColor] (321.59,260.36) circle (  0.99);

\path[draw=drawColor,line width= 0.4pt,line join=round,line cap=round,fill=fillColor] (322.02,240.32) circle (  0.99);

\path[draw=drawColor,line width= 0.4pt,line join=round,line cap=round,fill=fillColor] (322.45,332.03) circle (  0.99);

\path[draw=drawColor,line width= 0.4pt,line join=round,line cap=round,fill=fillColor] (322.88,333.85) circle (  0.99);

\path[draw=drawColor,line width= 0.4pt,line join=round,line cap=round,fill=fillColor] (323.30,249.43) circle (  0.99);

\path[draw=drawColor,line width= 0.4pt,line join=round,line cap=round,fill=fillColor] (323.73,291.34) circle (  0.99);

\path[draw=drawColor,line width= 0.4pt,line join=round,line cap=round,fill=fillColor] (324.16,254.29) circle (  0.99);

\path[draw=drawColor,line width= 0.4pt,line join=round,line cap=round,fill=fillColor] (324.59,264.61) circle (  0.99);

\path[draw=drawColor,line width= 0.4pt,line join=round,line cap=round,fill=fillColor] (325.02,279.80) circle (  0.99);

\path[draw=drawColor,line width= 0.4pt,line join=round,line cap=round,fill=fillColor] (325.44,323.53) circle (  0.99);

\path[draw=drawColor,line width= 0.4pt,line join=round,line cap=round,fill=fillColor] (325.87,319.28) circle (  0.99);

\path[draw=drawColor,line width= 0.4pt,line join=round,line cap=round,fill=fillColor] (326.30,249.43) circle (  0.99);

\path[draw=drawColor,line width= 0.4pt,line join=round,line cap=round,fill=fillColor] (326.73,233.64) circle (  0.99);

\path[draw=drawColor,line width= 0.4pt,line join=round,line cap=round,fill=fillColor] (327.16,264.01) circle (  0.99);

\path[draw=drawColor,line width= 0.4pt,line join=round,line cap=round,fill=fillColor] (327.58,276.76) circle (  0.99);

\path[draw=drawColor,line width= 0.4pt,line join=round,line cap=round,fill=fillColor] (328.01,250.04) circle (  0.99);

\path[draw=drawColor,line width= 0.4pt,line join=round,line cap=round,fill=fillColor] (328.44,259.15) circle (  0.99);

\path[draw=drawColor,line width= 0.4pt,line join=round,line cap=round,fill=fillColor] (328.87,271.90) circle (  0.99);

\path[draw=drawColor,line width= 0.4pt,line join=round,line cap=round,fill=fillColor] (329.29,298.02) circle (  0.99);

\path[draw=drawColor,line width= 0.4pt,line join=round,line cap=round,fill=fillColor] (329.72,284.05) circle (  0.99);

\path[draw=drawColor,line width= 0.4pt,line join=round,line cap=round,fill=fillColor] (330.15,259.15) circle (  0.99);

\path[draw=drawColor,line width= 0.4pt,line join=round,line cap=round,fill=fillColor] (330.58,288.30) circle (  0.99);

\path[draw=drawColor,line width= 0.4pt,line join=round,line cap=round,fill=fillColor] (331.01,265.22) circle (  0.99);

\path[draw=drawColor,line width= 0.4pt,line join=round,line cap=round,fill=fillColor] (331.43,342.96) circle (  0.99);

\path[draw=drawColor,line width= 0.4pt,line join=round,line cap=round,fill=fillColor] (331.86,304.09) circle (  0.99);

\path[draw=drawColor,line width= 0.4pt,line join=round,line cap=round,fill=fillColor] (332.29,353.29) circle (  0.99);

\path[draw=drawColor,line width= 0.4pt,line join=round,line cap=round,fill=fillColor] (332.72,288.91) circle (  0.99);

\path[draw=drawColor,line width= 0.4pt,line join=round,line cap=round,fill=fillColor] (333.15,347.21) circle (  0.99);

\path[draw=drawColor,line width= 0.4pt,line join=round,line cap=round,fill=fillColor] (333.57,282.23) circle (  0.99);

\path[draw=drawColor,line width= 0.4pt,line join=round,line cap=round,fill=fillColor] (334.00,298.02) circle (  0.99);

\path[draw=drawColor,line width= 0.4pt,line join=round,line cap=round,fill=fillColor] (334.43,268.26) circle (  0.99);

\path[draw=drawColor,line width= 0.4pt,line join=round,line cap=round,fill=fillColor] (334.86,245.79) circle (  0.99);

\path[draw=drawColor,line width= 0.4pt,line join=round,line cap=round,fill=fillColor] (335.28,234.25) circle (  0.99);

\path[draw=drawColor,line width= 0.4pt,line join=round,line cap=round,fill=fillColor] (335.71,273.12) circle (  0.99);

\path[draw=drawColor,line width= 0.4pt,line join=round,line cap=round,fill=fillColor] (336.14,290.12) circle (  0.99);

\path[draw=drawColor,line width= 0.4pt,line join=round,line cap=round,fill=fillColor] (336.57,280.40) circle (  0.99);

\path[draw=drawColor,line width= 0.4pt,line join=round,line cap=round,fill=fillColor] (337.00,243.96) circle (  0.99);

\path[draw=drawColor,line width= 0.4pt,line join=round,line cap=round,fill=fillColor] (337.42,308.95) circle (  0.99);

\path[draw=drawColor,line width= 0.4pt,line join=round,line cap=round,fill=fillColor] (337.85,251.86) circle (  0.99);

\path[draw=drawColor,line width= 0.4pt,line join=round,line cap=round,fill=fillColor] (338.28,341.75) circle (  0.99);

\path[draw=drawColor,line width= 0.4pt,line join=round,line cap=round,fill=fillColor] (338.71,304.70) circle (  0.99);

\path[draw=drawColor,line width= 0.4pt,line join=round,line cap=round,fill=fillColor] (339.14,265.83) circle (  0.99);

\path[draw=drawColor,line width= 0.4pt,line join=round,line cap=round,fill=fillColor] (339.56,260.97) circle (  0.99);

\path[draw=drawColor,line width= 0.4pt,line join=round,line cap=round,fill=fillColor] (339.99,256.11) circle (  0.99);

\path[draw=drawColor,line width= 0.4pt,line join=round,line cap=round,fill=fillColor] (340.42,296.20) circle (  0.99);

\path[draw=drawColor,line width= 0.4pt,line join=round,line cap=round,fill=fillColor] (340.85,253.07) circle (  0.99);

\path[draw=drawColor,line width= 0.4pt,line join=round,line cap=round,fill=fillColor] (341.27,248.82) circle (  0.99);

\path[draw=drawColor,line width= 0.4pt,line join=round,line cap=round,fill=fillColor] (341.70,250.64) circle (  0.99);

\path[draw=drawColor,line width= 0.4pt,line join=round,line cap=round,fill=fillColor] (342.13,280.40) circle (  0.99);

\path[draw=drawColor,line width= 0.4pt,line join=round,line cap=round,fill=fillColor] (342.56,298.02) circle (  0.99);

\path[draw=drawColor,line width= 0.4pt,line join=round,line cap=round,fill=fillColor] (342.99,279.80) circle (  0.99);

\path[draw=drawColor,line width= 0.4pt,line join=round,line cap=round,fill=fillColor] (343.41,292.55) circle (  0.99);

\path[draw=drawColor,line width= 0.4pt,line join=round,line cap=round,fill=fillColor] (343.84,323.53) circle (  0.99);

\path[draw=drawColor,line width= 0.4pt,line join=round,line cap=round,fill=fillColor] (344.27,316.24) circle (  0.99);

\path[draw=drawColor,line width= 0.4pt,line join=round,line cap=round,fill=fillColor] (344.70,325.96) circle (  0.99);

\path[draw=drawColor,line width= 0.4pt,line join=round,line cap=round,fill=fillColor] (345.13,292.55) circle (  0.99);

\path[draw=drawColor,line width= 0.4pt,line join=round,line cap=round,fill=fillColor] (345.55,250.64) circle (  0.99);

\path[draw=drawColor,line width= 0.4pt,line join=round,line cap=round,fill=fillColor] (345.98,289.52) circle (  0.99);

\path[draw=drawColor,line width= 0.4pt,line join=round,line cap=round,fill=fillColor] (346.41,324.13) circle (  0.99);

\path[draw=drawColor,line width= 0.4pt,line join=round,line cap=round,fill=fillColor] (346.84,254.29) circle (  0.99);

\path[draw=drawColor,line width= 0.4pt,line join=round,line cap=round,fill=fillColor] (347.27,249.43) circle (  0.99);

\path[draw=drawColor,line width= 0.4pt,line join=round,line cap=round,fill=fillColor] (347.69,236.68) circle (  0.99);

\path[draw=drawColor,line width= 0.4pt,line join=round,line cap=round,fill=fillColor] (348.12,256.11) circle (  0.99);

\path[draw=drawColor,line width= 0.4pt,line join=round,line cap=round,fill=fillColor] (348.55,291.34) circle (  0.99);

\path[draw=drawColor,line width= 0.4pt,line join=round,line cap=round,fill=fillColor] (348.98,305.91) circle (  0.99);

\path[draw=drawColor,line width= 0.4pt,line join=round,line cap=round,fill=fillColor] (349.40,293.16) circle (  0.99);

\path[draw=drawColor,line width= 0.4pt,line join=round,line cap=round,fill=fillColor] (349.83,296.80) circle (  0.99);

\path[draw=drawColor,line width= 0.4pt,line join=round,line cap=round,fill=fillColor] (350.26,272.51) circle (  0.99);

\path[draw=drawColor,line width= 0.4pt,line join=round,line cap=round,fill=fillColor] (350.69,268.87) circle (  0.99);

\path[draw=drawColor,line width= 0.4pt,line join=round,line cap=round,fill=fillColor] (351.12,291.34) circle (  0.99);

\path[draw=drawColor,line width= 0.4pt,line join=round,line cap=round,fill=fillColor] (351.54,300.45) circle (  0.99);

\path[draw=drawColor,line width= 0.4pt,line join=round,line cap=round,fill=fillColor] (351.97,233.03) circle (  0.99);

\path[draw=drawColor,line width= 0.4pt,line join=round,line cap=round,fill=fillColor] (352.40,377.58) circle (  0.99);

\path[draw=drawColor,line width= 0.4pt,line join=round,line cap=round,fill=fillColor] (352.83,260.97) circle (  0.99);

\path[draw=drawColor,line width= 0.4pt,line join=round,line cap=round,fill=fillColor] (353.26,260.97) circle (  0.99);

\path[draw=drawColor,line width= 0.4pt,line join=round,line cap=round,fill=fillColor] (353.68,330.21) circle (  0.99);

\path[draw=drawColor,line width= 0.4pt,line join=round,line cap=round,fill=fillColor] (354.11,281.01) circle (  0.99);

\path[draw=drawColor,line width= 0.4pt,line join=round,line cap=round,fill=fillColor] (354.54,331.42) circle (  0.99);

\path[draw=drawColor,line width= 0.4pt,line join=round,line cap=round,fill=fillColor] (354.97,264.01) circle (  0.99);

\path[draw=drawColor,line width= 0.4pt,line join=round,line cap=round,fill=fillColor] (355.39,308.95) circle (  0.99);

\path[draw=drawColor,line width= 0.4pt,line join=round,line cap=round,fill=fillColor] (355.82,284.66) circle (  0.99);

\path[draw=drawColor,line width= 0.4pt,line join=round,line cap=round,fill=fillColor] (356.25,307.74) circle (  0.99);

\path[draw=drawColor,line width= 0.4pt,line join=round,line cap=round,fill=fillColor] (356.68,282.83) circle (  0.99);

\path[draw=drawColor,line width= 0.4pt,line join=round,line cap=round,fill=fillColor] (357.11,270.69) circle (  0.99);

\path[draw=drawColor,line width= 0.4pt,line join=round,line cap=round,fill=fillColor] (357.53,255.50) circle (  0.99);

\path[draw=drawColor,line width= 0.4pt,line join=round,line cap=round,fill=fillColor] (357.96,237.28) circle (  0.99);

\path[draw=drawColor,line width= 0.4pt,line join=round,line cap=round,fill=fillColor] (358.39,281.62) circle (  0.99);

\path[draw=drawColor,line width= 0.4pt,line join=round,line cap=round,fill=fillColor] (358.82,248.21) circle (  0.99);

\path[draw=drawColor,line width= 0.4pt,line join=round,line cap=round,fill=fillColor] (359.25,311.38) circle (  0.99);

\path[draw=drawColor,line width= 0.4pt,line join=round,line cap=round,fill=fillColor] (359.67,246.39) circle (  0.99);

\path[draw=drawColor,line width= 0.4pt,line join=round,line cap=round,fill=fillColor] (360.10,362.40) circle (  0.99);

\path[draw=drawColor,line width= 0.4pt,line join=round,line cap=round,fill=fillColor] (360.53,290.12) circle (  0.99);

\path[draw=drawColor,line width= 0.4pt,line join=round,line cap=round,fill=fillColor] (360.96,310.17) circle (  0.99);

\path[draw=drawColor,line width= 0.4pt,line join=round,line cap=round,fill=fillColor] (361.38,260.36) circle (  0.99);

\path[draw=drawColor,line width= 0.4pt,line join=round,line cap=round,fill=fillColor] (361.81,296.20) circle (  0.99);

\path[draw=drawColor,line width= 0.4pt,line join=round,line cap=round,fill=fillColor] (362.24,330.82) circle (  0.99);

\path[draw=drawColor,line width= 0.4pt,line join=round,line cap=round,fill=fillColor] (362.67,302.88) circle (  0.99);

\path[draw=drawColor,line width= 0.4pt,line join=round,line cap=round,fill=fillColor] (363.10,260.97) circle (  0.99);

\path[draw=drawColor,line width= 0.4pt,line join=round,line cap=round,fill=fillColor] (363.52,241.53) circle (  0.99);

\path[draw=drawColor,line width= 0.4pt,line join=round,line cap=round,fill=fillColor] (363.95,287.09) circle (  0.99);

\path[draw=drawColor,line width= 0.4pt,line join=round,line cap=round,fill=fillColor] (364.38,293.16) circle (  0.99);

\path[draw=drawColor,line width= 0.4pt,line join=round,line cap=round,fill=fillColor] (364.81,358.15) circle (  0.99);

\path[draw=drawColor,line width= 0.4pt,line join=round,line cap=round,fill=fillColor] (365.24,278.58) circle (  0.99);

\path[draw=drawColor,line width= 0.4pt,line join=round,line cap=round,fill=fillColor] (365.66,243.96) circle (  0.99);

\path[draw=drawColor,line width= 0.4pt,line join=round,line cap=round,fill=fillColor] (366.09,316.85) circle (  0.99);

\path[draw=drawColor,line width= 0.4pt,line join=round,line cap=round,fill=fillColor] (366.52,264.61) circle (  0.99);

\path[draw=drawColor,line width= 0.4pt,line join=round,line cap=round,fill=fillColor] (366.95,274.94) circle (  0.99);

\path[draw=drawColor,line width= 0.4pt,line join=round,line cap=round,fill=fillColor] (367.37,356.32) circle (  0.99);

\path[draw=drawColor,line width= 0.4pt,line join=round,line cap=round,fill=fillColor] (367.80,287.09) circle (  0.99);

\path[draw=drawColor,line width= 0.4pt,line join=round,line cap=round,fill=fillColor] (368.23,281.62) circle (  0.99);

\path[draw=drawColor,line width= 0.4pt,line join=round,line cap=round,fill=fillColor] (368.66,257.33) circle (  0.99);

\path[draw=drawColor,line width= 0.4pt,line join=round,line cap=round,fill=fillColor] (369.09,279.80) circle (  0.99);

\path[draw=drawColor,line width= 0.4pt,line join=round,line cap=round,fill=fillColor] (369.51,226.35) circle (  0.99);

\path[draw=drawColor,line width= 0.4pt,line join=round,line cap=round,fill=fillColor] (369.94,260.97) circle (  0.99);

\path[draw=drawColor,line width= 0.4pt,line join=round,line cap=round,fill=fillColor] (370.37,264.01) circle (  0.99);

\path[draw=drawColor,line width= 0.4pt,line join=round,line cap=round,fill=fillColor] (370.80,296.80) circle (  0.99);

\path[draw=drawColor,line width= 0.4pt,line join=round,line cap=round,fill=fillColor] (371.23,234.25) circle (  0.99);

\path[draw=drawColor,line width= 0.4pt,line join=round,line cap=round,fill=fillColor] (371.65,250.04) circle (  0.99);

\path[draw=drawColor,line width= 0.4pt,line join=round,line cap=round,fill=fillColor] (372.08,290.12) circle (  0.99);
\end{scope}
\begin{scope}
\path[clip] (252.94,  0.00) rectangle (379.42,505.89);
\definecolor{drawColor}{RGB}{0,0,0}

\node[text=drawColor,anchor=base,inner sep=0pt, outer sep=0pt, scale=  1.32] at (334.00,495.79) {\bfseries \textsf{2-bit}};

\node[text=drawColor,anchor=base,inner sep=0pt, outer sep=0pt, scale=  1.32] at (334.00,  5.54) {Vp};
\end{scope}
\begin{scope}
\path[clip] (  0.00,  0.00) rectangle (505.89,505.89);
\definecolor{drawColor}{RGB}{0,0,0}

\path[draw=drawColor,line width= 0.4pt,line join=round,line cap=round] (296.78, 35.64) -- (371.65, 35.64);

\path[draw=drawColor,line width= 0.4pt,line join=round,line cap=round] (296.78, 35.64) -- (296.78, 31.68);

\path[draw=drawColor,line width= 0.4pt,line join=round,line cap=round] (322.02, 35.64) -- (322.02, 31.68);

\path[draw=drawColor,line width= 0.4pt,line join=round,line cap=round] (347.69, 35.64) -- (347.69, 31.68);

\path[draw=drawColor,line width= 0.4pt,line join=round,line cap=round] (371.65, 35.64) -- (371.65, 31.68);

\node[text=drawColor,anchor=base,inner sep=0pt, outer sep=0pt, scale=  0.99] at (296.78, 21.38) {1};

\node[text=drawColor,anchor=base,inner sep=0pt, outer sep=0pt, scale=  0.99] at (322.02, 21.38) {60};

\node[text=drawColor,anchor=base,inner sep=0pt, outer sep=0pt, scale=  0.99] at (347.69, 21.38) {120};

\node[text=drawColor,anchor=base,inner sep=0pt, outer sep=0pt, scale=  0.99] at (371.65, 21.38) {176};

\path[draw=drawColor,line width= 0.4pt,line join=round,line cap=round] (293.34, 52.65) -- (293.34,477.80);

\path[draw=drawColor,line width= 0.4pt,line join=round,line cap=round] (293.34, 52.65) -- (289.38, 52.65);

\path[draw=drawColor,line width= 0.4pt,line join=round,line cap=round] (293.34,113.38) -- (289.38,113.38);

\path[draw=drawColor,line width= 0.4pt,line join=round,line cap=round] (293.34,174.12) -- (289.38,174.12);

\path[draw=drawColor,line width= 0.4pt,line join=round,line cap=round] (293.34,234.85) -- (289.38,234.85);

\path[draw=drawColor,line width= 0.4pt,line join=round,line cap=round] (293.34,295.59) -- (289.38,295.59);

\path[draw=drawColor,line width= 0.4pt,line join=round,line cap=round] (293.34,356.32) -- (289.38,356.32);

\path[draw=drawColor,line width= 0.4pt,line join=round,line cap=round] (293.34,417.06) -- (289.38,417.06);

\path[draw=drawColor,line width= 0.4pt,line join=round,line cap=round] (293.34,477.80) -- (289.38,477.80);

\node[text=drawColor,anchor=base east,inner sep=0pt, outer sep=0pt, scale=  0.99] at (285.42, 49.24) {0};

\node[text=drawColor,anchor=base east,inner sep=0pt, outer sep=0pt, scale=  0.99] at (285.42,109.97) {100};

\node[text=drawColor,anchor=base east,inner sep=0pt, outer sep=0pt, scale=  0.99] at (285.42,170.71) {200};

\node[text=drawColor,anchor=base east,inner sep=0pt, outer sep=0pt, scale=  0.99] at (285.42,231.44) {300};

\node[text=drawColor,anchor=base east,inner sep=0pt, outer sep=0pt, scale=  0.99] at (285.42,292.18) {400};

\node[text=drawColor,anchor=base east,inner sep=0pt, outer sep=0pt, scale=  0.99] at (285.42,352.92) {500};

\node[text=drawColor,anchor=base east,inner sep=0pt, outer sep=0pt, scale=  0.99] at (285.42,413.65) {600};

\node[text=drawColor,anchor=base east,inner sep=0pt, outer sep=0pt, scale=  0.99] at (285.42,474.39) {700};
\end{scope}
\begin{scope}
\path[clip] (293.34, 35.64) rectangle (374.67,494.80);
\definecolor{drawColor}{RGB}{0,0,0}

\path[draw=drawColor,line width= 0.4pt,line join=round,line cap=round] (293.34,281.63) -- (374.67,281.63);
\end{scope}
\begin{scope}
\path[clip] (419.81, 35.64) rectangle (501.14,494.80);
\definecolor{drawColor}{RGB}{0,0,0}
\definecolor{fillColor}{RGB}{0,0,0}

\path[draw=drawColor,line width= 0.4pt,line join=round,line cap=round,fill=fillColor] (423.25,447.43) circle (  0.99);

\path[draw=drawColor,line width= 0.4pt,line join=round,line cap=round,fill=fillColor] (423.68,284.66) circle (  0.99);

\path[draw=drawColor,line width= 0.4pt,line join=round,line cap=round,fill=fillColor] (424.11,275.55) circle (  0.99);

\path[draw=drawColor,line width= 0.4pt,line join=round,line cap=round,fill=fillColor] (424.53,296.80) circle (  0.99);

\path[draw=drawColor,line width= 0.4pt,line join=round,line cap=round,fill=fillColor] (424.96,301.06) circle (  0.99);

\path[draw=drawColor,line width= 0.4pt,line join=round,line cap=round,fill=fillColor] (425.39,282.23) circle (  0.99);

\path[draw=drawColor,line width= 0.4pt,line join=round,line cap=round,fill=fillColor] (425.82,290.73) circle (  0.99);

\path[draw=drawColor,line width= 0.4pt,line join=round,line cap=round,fill=fillColor] (426.24,296.80) circle (  0.99);

\path[draw=drawColor,line width= 0.4pt,line join=round,line cap=round,fill=fillColor] (426.67,336.28) circle (  0.99);

\path[draw=drawColor,line width= 0.4pt,line join=round,line cap=round,fill=fillColor] (427.10,319.88) circle (  0.99);

\path[draw=drawColor,line width= 0.4pt,line join=round,line cap=round,fill=fillColor] (427.53,344.78) circle (  0.99);

\path[draw=drawColor,line width= 0.4pt,line join=round,line cap=round,fill=fillColor] (427.96,277.98) circle (  0.99);

\path[draw=drawColor,line width= 0.4pt,line join=round,line cap=round,fill=fillColor] (428.38,281.62) circle (  0.99);

\path[draw=drawColor,line width= 0.4pt,line join=round,line cap=round,fill=fillColor] (428.81,286.48) circle (  0.99);

\path[draw=drawColor,line width= 0.4pt,line join=round,line cap=round,fill=fillColor] (429.24,291.94) circle (  0.99);

\path[draw=drawColor,line width= 0.4pt,line join=round,line cap=round,fill=fillColor] (429.67,311.99) circle (  0.99);

\path[draw=drawColor,line width= 0.4pt,line join=round,line cap=round,fill=fillColor] (430.10,269.47) circle (  0.99);

\path[draw=drawColor,line width= 0.4pt,line join=round,line cap=round,fill=fillColor] (430.52,293.77) circle (  0.99);

\path[draw=drawColor,line width= 0.4pt,line join=round,line cap=round,fill=fillColor] (430.95,321.10) circle (  0.99);

\path[draw=drawColor,line width= 0.4pt,line join=round,line cap=round,fill=fillColor] (431.38,255.50) circle (  0.99);

\path[draw=drawColor,line width= 0.4pt,line join=round,line cap=round,fill=fillColor] (431.81,440.14) circle (  0.99);

\path[draw=drawColor,line width= 0.4pt,line join=round,line cap=round,fill=fillColor] (432.23,272.51) circle (  0.99);

\path[draw=drawColor,line width= 0.4pt,line join=round,line cap=round,fill=fillColor] (432.66,349.64) circle (  0.99);

\path[draw=drawColor,line width= 0.4pt,line join=round,line cap=round,fill=fillColor] (433.09,293.16) circle (  0.99);

\path[draw=drawColor,line width= 0.4pt,line join=round,line cap=round,fill=fillColor] (433.52,329.60) circle (  0.99);

\path[draw=drawColor,line width= 0.4pt,line join=round,line cap=round,fill=fillColor] (433.95,273.12) circle (  0.99);

\path[draw=drawColor,line width= 0.4pt,line join=round,line cap=round,fill=fillColor] (434.37,328.39) circle (  0.99);

\path[draw=drawColor,line width= 0.4pt,line join=round,line cap=round,fill=fillColor] (434.80,298.02) circle (  0.99);

\path[draw=drawColor,line width= 0.4pt,line join=round,line cap=round,fill=fillColor] (435.23,251.86) circle (  0.99);

\path[draw=drawColor,line width= 0.4pt,line join=round,line cap=round,fill=fillColor] (435.66,333.24) circle (  0.99);

\path[draw=drawColor,line width= 0.4pt,line join=round,line cap=round,fill=fillColor] (436.09,294.98) circle (  0.99);

\path[draw=drawColor,line width= 0.4pt,line join=round,line cap=round,fill=fillColor] (436.51,384.26) circle (  0.99);

\path[draw=drawColor,line width= 0.4pt,line join=round,line cap=round,fill=fillColor] (436.94,274.94) circle (  0.99);

\path[draw=drawColor,line width= 0.4pt,line join=round,line cap=round,fill=fillColor] (437.37,279.19) circle (  0.99);

\path[draw=drawColor,line width= 0.4pt,line join=round,line cap=round,fill=fillColor] (437.80,255.50) circle (  0.99);

\path[draw=drawColor,line width= 0.4pt,line join=round,line cap=round,fill=fillColor] (438.22,349.64) circle (  0.99);

\path[draw=drawColor,line width= 0.4pt,line join=round,line cap=round,fill=fillColor] (438.65,344.78) circle (  0.99);

\path[draw=drawColor,line width= 0.4pt,line join=round,line cap=round,fill=fillColor] (439.08,309.56) circle (  0.99);

\path[draw=drawColor,line width= 0.4pt,line join=round,line cap=round,fill=fillColor] (439.51,420.70) circle (  0.99);

\path[draw=drawColor,line width= 0.4pt,line join=round,line cap=round,fill=fillColor] (439.94,321.10) circle (  0.99);

\path[draw=drawColor,line width= 0.4pt,line join=round,line cap=round,fill=fillColor] (440.36,318.06) circle (  0.99);

\path[draw=drawColor,line width= 0.4pt,line join=round,line cap=round,fill=fillColor] (440.79,312.59) circle (  0.99);

\path[draw=drawColor,line width= 0.4pt,line join=round,line cap=round,fill=fillColor] (441.22,358.15) circle (  0.99);

\path[draw=drawColor,line width= 0.4pt,line join=round,line cap=round,fill=fillColor] (441.65,286.48) circle (  0.99);

\path[draw=drawColor,line width= 0.4pt,line join=round,line cap=round,fill=fillColor] (442.08,256.72) circle (  0.99);

\path[draw=drawColor,line width= 0.4pt,line join=round,line cap=round,fill=fillColor] (442.50,379.40) circle (  0.99);

\path[draw=drawColor,line width= 0.4pt,line join=round,line cap=round,fill=fillColor] (442.93,259.15) circle (  0.99);

\path[draw=drawColor,line width= 0.4pt,line join=round,line cap=round,fill=fillColor] (443.36,314.42) circle (  0.99);

\path[draw=drawColor,line width= 0.4pt,line join=round,line cap=round,fill=fillColor] (443.79,344.78) circle (  0.99);

\path[draw=drawColor,line width= 0.4pt,line join=round,line cap=round,fill=fillColor] (444.21,342.36) circle (  0.99);

\path[draw=drawColor,line width= 0.4pt,line join=round,line cap=round,fill=fillColor] (444.64,314.42) circle (  0.99);

\path[draw=drawColor,line width= 0.4pt,line join=round,line cap=round,fill=fillColor] (445.07,335.67) circle (  0.99);

\path[draw=drawColor,line width= 0.4pt,line join=round,line cap=round,fill=fillColor] (445.50,364.83) circle (  0.99);

\path[draw=drawColor,line width= 0.4pt,line join=round,line cap=round,fill=fillColor] (445.93,325.35) circle (  0.99);

\path[draw=drawColor,line width= 0.4pt,line join=round,line cap=round,fill=fillColor] (446.35,364.83) circle (  0.99);

\path[draw=drawColor,line width= 0.4pt,line join=round,line cap=round,fill=fillColor] (446.78,309.56) circle (  0.99);

\path[draw=drawColor,line width= 0.4pt,line join=round,line cap=round,fill=fillColor] (447.21,404.31) circle (  0.99);

\path[draw=drawColor,line width= 0.4pt,line join=round,line cap=round,fill=fillColor] (447.64,365.43) circle (  0.99);

\path[draw=drawColor,line width= 0.4pt,line join=round,line cap=round,fill=fillColor] (448.07,285.87) circle (  0.99);

\path[draw=drawColor,line width= 0.4pt,line join=round,line cap=round,fill=fillColor] (448.49,275.55) circle (  0.99);

\path[draw=drawColor,line width= 0.4pt,line join=round,line cap=round,fill=fillColor] (448.92,435.28) circle (  0.99);

\path[draw=drawColor,line width= 0.4pt,line join=round,line cap=round,fill=fillColor] (449.35,387.91) circle (  0.99);

\path[draw=drawColor,line width= 0.4pt,line join=round,line cap=round,fill=fillColor] (449.78,276.15) circle (  0.99);

\path[draw=drawColor,line width= 0.4pt,line join=round,line cap=round,fill=fillColor] (450.21,288.91) circle (  0.99);

\path[draw=drawColor,line width= 0.4pt,line join=round,line cap=round,fill=fillColor] (450.63,260.97) circle (  0.99);

\path[draw=drawColor,line width= 0.4pt,line join=round,line cap=round,fill=fillColor] (451.06,304.70) circle (  0.99);

\path[draw=drawColor,line width= 0.4pt,line join=round,line cap=round,fill=fillColor] (451.49,289.52) circle (  0.99);

\path[draw=drawColor,line width= 0.4pt,line join=round,line cap=round,fill=fillColor] (451.92,353.29) circle (  0.99);

\path[draw=drawColor,line width= 0.4pt,line join=round,line cap=round,fill=fillColor] (452.34,304.70) circle (  0.99);

\path[draw=drawColor,line width= 0.4pt,line join=round,line cap=round,fill=fillColor] (452.77,300.45) circle (  0.99);

\path[draw=drawColor,line width= 0.4pt,line join=round,line cap=round,fill=fillColor] (453.20,282.23) circle (  0.99);

\path[draw=drawColor,line width= 0.4pt,line join=round,line cap=round,fill=fillColor] (453.63,284.05) circle (  0.99);

\path[draw=drawColor,line width= 0.4pt,line join=round,line cap=round,fill=fillColor] (454.06,328.99) circle (  0.99);

\path[draw=drawColor,line width= 0.4pt,line join=round,line cap=round,fill=fillColor] (454.48,298.02) circle (  0.99);

\path[draw=drawColor,line width= 0.4pt,line join=round,line cap=round,fill=fillColor] (454.91,304.70) circle (  0.99);

\path[draw=drawColor,line width= 0.4pt,line join=round,line cap=round,fill=fillColor] (455.34,310.77) circle (  0.99);

\path[draw=drawColor,line width= 0.4pt,line join=round,line cap=round,fill=fillColor] (455.77,321.71) circle (  0.99);

\path[draw=drawColor,line width= 0.4pt,line join=round,line cap=round,fill=fillColor] (456.20,318.06) circle (  0.99);

\path[draw=drawColor,line width= 0.4pt,line join=round,line cap=round,fill=fillColor] (456.62,325.96) circle (  0.99);

\path[draw=drawColor,line width= 0.4pt,line join=round,line cap=round,fill=fillColor] (457.05,285.87) circle (  0.99);

\path[draw=drawColor,line width= 0.4pt,line join=round,line cap=round,fill=fillColor] (457.48,288.91) circle (  0.99);

\path[draw=drawColor,line width= 0.4pt,line join=round,line cap=round,fill=fillColor] (457.91,361.79) circle (  0.99);

\path[draw=drawColor,line width= 0.4pt,line join=round,line cap=round,fill=fillColor] (458.33,384.26) circle (  0.99);

\path[draw=drawColor,line width= 0.4pt,line join=round,line cap=round,fill=fillColor] (458.76,381.83) circle (  0.99);

\path[draw=drawColor,line width= 0.4pt,line join=round,line cap=round,fill=fillColor] (459.19,325.96) circle (  0.99);

\path[draw=drawColor,line width= 0.4pt,line join=round,line cap=round,fill=fillColor] (459.62,341.75) circle (  0.99);

\path[draw=drawColor,line width= 0.4pt,line join=round,line cap=round,fill=fillColor] (460.05,307.74) circle (  0.99);

\path[draw=drawColor,line width= 0.4pt,line join=round,line cap=round,fill=fillColor] (460.47,339.32) circle (  0.99);

\path[draw=drawColor,line width= 0.4pt,line join=round,line cap=round,fill=fillColor] (460.90,317.45) circle (  0.99);

\path[draw=drawColor,line width= 0.4pt,line join=round,line cap=round,fill=fillColor] (461.33,279.80) circle (  0.99);

\path[draw=drawColor,line width= 0.4pt,line join=round,line cap=round,fill=fillColor] (461.76,269.47) circle (  0.99);

\path[draw=drawColor,line width= 0.4pt,line join=round,line cap=round,fill=fillColor] (462.19,304.09) circle (  0.99);

\path[draw=drawColor,line width= 0.4pt,line join=round,line cap=round,fill=fillColor] (462.61,310.77) circle (  0.99);

\path[draw=drawColor,line width= 0.4pt,line join=round,line cap=round,fill=fillColor] (463.04,318.67) circle (  0.99);

\path[draw=drawColor,line width= 0.4pt,line join=round,line cap=round,fill=fillColor] (463.47,288.91) circle (  0.99);

\path[draw=drawColor,line width= 0.4pt,line join=round,line cap=round,fill=fillColor] (463.90,352.07) circle (  0.99);

\path[draw=drawColor,line width= 0.4pt,line join=round,line cap=round,fill=fillColor] (464.32,281.62) circle (  0.99);

\path[draw=drawColor,line width= 0.4pt,line join=round,line cap=round,fill=fillColor] (464.75,413.42) circle (  0.99);

\path[draw=drawColor,line width= 0.4pt,line join=round,line cap=round,fill=fillColor] (465.18,318.67) circle (  0.99);

\path[draw=drawColor,line width= 0.4pt,line join=round,line cap=round,fill=fillColor] (465.61,293.77) circle (  0.99);

\path[draw=drawColor,line width= 0.4pt,line join=round,line cap=round,fill=fillColor] (466.04,285.87) circle (  0.99);

\path[draw=drawColor,line width= 0.4pt,line join=round,line cap=round,fill=fillColor] (466.46,276.15) circle (  0.99);

\path[draw=drawColor,line width= 0.4pt,line join=round,line cap=round,fill=fillColor] (466.89,354.50) circle (  0.99);

\path[draw=drawColor,line width= 0.4pt,line join=round,line cap=round,fill=fillColor] (467.32,272.51) circle (  0.99);

\path[draw=drawColor,line width= 0.4pt,line join=round,line cap=round,fill=fillColor] (467.75,281.01) circle (  0.99);

\path[draw=drawColor,line width= 0.4pt,line join=round,line cap=round,fill=fillColor] (468.18,289.52) circle (  0.99);

\path[draw=drawColor,line width= 0.4pt,line join=round,line cap=round,fill=fillColor] (468.60,327.78) circle (  0.99);

\path[draw=drawColor,line width= 0.4pt,line join=round,line cap=round,fill=fillColor] (469.03,325.35) circle (  0.99);

\path[draw=drawColor,line width= 0.4pt,line join=round,line cap=round,fill=fillColor] (469.46,294.98) circle (  0.99);

\path[draw=drawColor,line width= 0.4pt,line join=round,line cap=round,fill=fillColor] (469.89,316.85) circle (  0.99);

\path[draw=drawColor,line width= 0.4pt,line join=round,line cap=round,fill=fillColor] (470.31,340.53) circle (  0.99);

\path[draw=drawColor,line width= 0.4pt,line join=round,line cap=round,fill=fillColor] (470.74,371.51) circle (  0.99);

\path[draw=drawColor,line width= 0.4pt,line join=round,line cap=round,fill=fillColor] (471.17,353.90) circle (  0.99);

\path[draw=drawColor,line width= 0.4pt,line join=round,line cap=round,fill=fillColor] (471.60,345.39) circle (  0.99);

\path[draw=drawColor,line width= 0.4pt,line join=round,line cap=round,fill=fillColor] (472.03,346.61) circle (  0.99);

\path[draw=drawColor,line width= 0.4pt,line join=round,line cap=round,fill=fillColor] (472.45,339.32) circle (  0.99);

\path[draw=drawColor,line width= 0.4pt,line join=round,line cap=round,fill=fillColor] (472.88,352.07) circle (  0.99);

\path[draw=drawColor,line width= 0.4pt,line join=round,line cap=round,fill=fillColor] (473.31,306.52) circle (  0.99);

\path[draw=drawColor,line width= 0.4pt,line join=round,line cap=round,fill=fillColor] (473.74,298.02) circle (  0.99);

\path[draw=drawColor,line width= 0.4pt,line join=round,line cap=round,fill=fillColor] (474.17,267.04) circle (  0.99);

\path[draw=drawColor,line width= 0.4pt,line join=round,line cap=round,fill=fillColor] (474.59,281.62) circle (  0.99);

\path[draw=drawColor,line width= 0.4pt,line join=round,line cap=round,fill=fillColor] (475.02,366.65) circle (  0.99);

\path[draw=drawColor,line width= 0.4pt,line join=round,line cap=round,fill=fillColor] (475.45,372.72) circle (  0.99);

\path[draw=drawColor,line width= 0.4pt,line join=round,line cap=round,fill=fillColor] (475.88,326.56) circle (  0.99);

\path[draw=drawColor,line width= 0.4pt,line join=round,line cap=round,fill=fillColor] (476.30,315.02) circle (  0.99);

\path[draw=drawColor,line width= 0.4pt,line join=round,line cap=round,fill=fillColor] (476.73,348.43) circle (  0.99);

\path[draw=drawColor,line width= 0.4pt,line join=round,line cap=round,fill=fillColor] (477.16,295.59) circle (  0.99);

\path[draw=drawColor,line width= 0.4pt,line join=round,line cap=round,fill=fillColor] (477.59,294.98) circle (  0.99);

\path[draw=drawColor,line width= 0.4pt,line join=round,line cap=round,fill=fillColor] (478.02,324.13) circle (  0.99);

\path[draw=drawColor,line width= 0.4pt,line join=round,line cap=round,fill=fillColor] (478.44,243.96) circle (  0.99);

\path[draw=drawColor,line width= 0.4pt,line join=round,line cap=round,fill=fillColor] (478.87,413.42) circle (  0.99);

\path[draw=drawColor,line width= 0.4pt,line join=round,line cap=round,fill=fillColor] (479.30,298.63) circle (  0.99);

\path[draw=drawColor,line width= 0.4pt,line join=round,line cap=round,fill=fillColor] (479.73,312.59) circle (  0.99);

\path[draw=drawColor,line width= 0.4pt,line join=round,line cap=round,fill=fillColor] (480.16,378.19) circle (  0.99);

\path[draw=drawColor,line width= 0.4pt,line join=round,line cap=round,fill=fillColor] (480.58,400.05) circle (  0.99);

\path[draw=drawColor,line width= 0.4pt,line join=round,line cap=round,fill=fillColor] (481.01,388.51) circle (  0.99);

\path[draw=drawColor,line width= 0.4pt,line join=round,line cap=round,fill=fillColor] (481.44,303.48) circle (  0.99);

\path[draw=drawColor,line width= 0.4pt,line join=round,line cap=round,fill=fillColor] (481.87,302.27) circle (  0.99);

\path[draw=drawColor,line width= 0.4pt,line join=round,line cap=round,fill=fillColor] (482.29,290.12) circle (  0.99);

\path[draw=drawColor,line width= 0.4pt,line join=round,line cap=round,fill=fillColor] (482.72,298.02) circle (  0.99);

\path[draw=drawColor,line width= 0.4pt,line join=round,line cap=round,fill=fillColor] (483.15,333.24) circle (  0.99);

\path[draw=drawColor,line width= 0.4pt,line join=round,line cap=round,fill=fillColor] (483.58,342.96) circle (  0.99);

\path[draw=drawColor,line width= 0.4pt,line join=round,line cap=round,fill=fillColor] (484.01,308.95) circle (  0.99);

\path[draw=drawColor,line width= 0.4pt,line join=round,line cap=round,fill=fillColor] (484.43,280.40) circle (  0.99);

\path[draw=drawColor,line width= 0.4pt,line join=round,line cap=round,fill=fillColor] (484.86,323.53) circle (  0.99);

\path[draw=drawColor,line width= 0.4pt,line join=round,line cap=round,fill=fillColor] (485.29,254.90) circle (  0.99);

\path[draw=drawColor,line width= 0.4pt,line join=round,line cap=round,fill=fillColor] (485.72,362.40) circle (  0.99);

\path[draw=drawColor,line width= 0.4pt,line join=round,line cap=round,fill=fillColor] (486.15,267.04) circle (  0.99);

\path[draw=drawColor,line width= 0.4pt,line join=round,line cap=round,fill=fillColor] (486.57,415.24) circle (  0.99);

\path[draw=drawColor,line width= 0.4pt,line join=round,line cap=round,fill=fillColor] (487.00,351.47) circle (  0.99);

\path[draw=drawColor,line width= 0.4pt,line join=round,line cap=round,fill=fillColor] (487.43,337.50) circle (  0.99);

\path[draw=drawColor,line width= 0.4pt,line join=round,line cap=round,fill=fillColor] (487.86,305.91) circle (  0.99);

\path[draw=drawColor,line width= 0.4pt,line join=round,line cap=round,fill=fillColor] (488.28,334.46) circle (  0.99);

\path[draw=drawColor,line width= 0.4pt,line join=round,line cap=round,fill=fillColor] (488.71,358.75) circle (  0.99);

\path[draw=drawColor,line width= 0.4pt,line join=round,line cap=round,fill=fillColor] (489.14,341.75) circle (  0.99);

\path[draw=drawColor,line width= 0.4pt,line join=round,line cap=round,fill=fillColor] (489.57,273.72) circle (  0.99);

\path[draw=drawColor,line width= 0.4pt,line join=round,line cap=round,fill=fillColor] (490.00,276.76) circle (  0.99);

\path[draw=drawColor,line width= 0.4pt,line join=round,line cap=round,fill=fillColor] (490.42,325.96) circle (  0.99);

\path[draw=drawColor,line width= 0.4pt,line join=round,line cap=round,fill=fillColor] (490.85,299.84) circle (  0.99);

\path[draw=drawColor,line width= 0.4pt,line join=round,line cap=round,fill=fillColor] (491.28,373.33) circle (  0.99);

\path[draw=drawColor,line width= 0.4pt,line join=round,line cap=round,fill=fillColor] (491.71,325.35) circle (  0.99);

\path[draw=drawColor,line width= 0.4pt,line join=round,line cap=round,fill=fillColor] (492.14,284.66) circle (  0.99);

\path[draw=drawColor,line width= 0.4pt,line join=round,line cap=round,fill=fillColor] (492.56,386.09) circle (  0.99);

\path[draw=drawColor,line width= 0.4pt,line join=round,line cap=round,fill=fillColor] (492.99,319.28) circle (  0.99);

\path[draw=drawColor,line width= 0.4pt,line join=round,line cap=round,fill=fillColor] (493.42,294.37) circle (  0.99);

\path[draw=drawColor,line width= 0.4pt,line join=round,line cap=round,fill=fillColor] (493.85,412.81) circle (  0.99);

\path[draw=drawColor,line width= 0.4pt,line join=round,line cap=round,fill=fillColor] (494.28,313.81) circle (  0.99);

\path[draw=drawColor,line width= 0.4pt,line join=round,line cap=round,fill=fillColor] (494.70,331.42) circle (  0.99);

\path[draw=drawColor,line width= 0.4pt,line join=round,line cap=round,fill=fillColor] (495.13,315.63) circle (  0.99);

\path[draw=drawColor,line width= 0.4pt,line join=round,line cap=round,fill=fillColor] (495.56,306.52) circle (  0.99);

\path[draw=drawColor,line width= 0.4pt,line join=round,line cap=round,fill=fillColor] (495.99,254.90) circle (  0.99);

\path[draw=drawColor,line width= 0.4pt,line join=round,line cap=round,fill=fillColor] (496.41,302.88) circle (  0.99);

\path[draw=drawColor,line width= 0.4pt,line join=round,line cap=round,fill=fillColor] (496.84,281.01) circle (  0.99);

\path[draw=drawColor,line width= 0.4pt,line join=round,line cap=round,fill=fillColor] (497.27,313.81) circle (  0.99);

\path[draw=drawColor,line width= 0.4pt,line join=round,line cap=round,fill=fillColor] (497.70,301.06) circle (  0.99);

\path[draw=drawColor,line width= 0.4pt,line join=round,line cap=round,fill=fillColor] (498.13,287.69) circle (  0.99);

\path[draw=drawColor,line width= 0.4pt,line join=round,line cap=round,fill=fillColor] (498.55,295.59) circle (  0.99);
\end{scope}
\begin{scope}
\path[clip] (379.42,  0.00) rectangle (505.89,505.89);
\definecolor{drawColor}{RGB}{0,0,0}

\node[text=drawColor,anchor=base,inner sep=0pt, outer sep=0pt, scale=  1.32] at (460.47,495.79) {\bfseries \textsf{2.58-bit}};

\node[text=drawColor,anchor=base,inner sep=0pt, outer sep=0pt, scale=  1.32] at (460.47,  5.54) {Vp};
\end{scope}
\begin{scope}
\path[clip] (  0.00,  0.00) rectangle (505.89,505.89);
\definecolor{drawColor}{RGB}{0,0,0}

\path[draw=drawColor,line width= 0.4pt,line join=round,line cap=round] (423.25, 35.64) -- (498.13, 35.64);

\path[draw=drawColor,line width= 0.4pt,line join=round,line cap=round] (423.25, 35.64) -- (423.25, 31.68);

\path[draw=drawColor,line width= 0.4pt,line join=round,line cap=round] (448.49, 35.64) -- (448.49, 31.68);

\path[draw=drawColor,line width= 0.4pt,line join=round,line cap=round] (474.17, 35.64) -- (474.17, 31.68);

\path[draw=drawColor,line width= 0.4pt,line join=round,line cap=round] (498.13, 35.64) -- (498.13, 31.68);

\node[text=drawColor,anchor=base,inner sep=0pt, outer sep=0pt, scale=  0.99] at (423.25, 21.38) {1};

\node[text=drawColor,anchor=base,inner sep=0pt, outer sep=0pt, scale=  0.99] at (448.49, 21.38) {60};

\node[text=drawColor,anchor=base,inner sep=0pt, outer sep=0pt, scale=  0.99] at (474.17, 21.38) {120};

\node[text=drawColor,anchor=base,inner sep=0pt, outer sep=0pt, scale=  0.99] at (498.13, 21.38) {176};

\path[draw=drawColor,line width= 0.4pt,line join=round,line cap=round] (419.81, 52.65) -- (419.81,477.80);

\path[draw=drawColor,line width= 0.4pt,line join=round,line cap=round] (419.81, 52.65) -- (415.85, 52.65);

\path[draw=drawColor,line width= 0.4pt,line join=round,line cap=round] (419.81,113.38) -- (415.85,113.38);

\path[draw=drawColor,line width= 0.4pt,line join=round,line cap=round] (419.81,174.12) -- (415.85,174.12);

\path[draw=drawColor,line width= 0.4pt,line join=round,line cap=round] (419.81,234.85) -- (415.85,234.85);

\path[draw=drawColor,line width= 0.4pt,line join=round,line cap=round] (419.81,295.59) -- (415.85,295.59);

\path[draw=drawColor,line width= 0.4pt,line join=round,line cap=round] (419.81,356.32) -- (415.85,356.32);

\path[draw=drawColor,line width= 0.4pt,line join=round,line cap=round] (419.81,417.06) -- (415.85,417.06);

\path[draw=drawColor,line width= 0.4pt,line join=round,line cap=round] (419.81,477.80) -- (415.85,477.80);

\node[text=drawColor,anchor=base east,inner sep=0pt, outer sep=0pt, scale=  0.99] at (411.89, 49.24) {0};

\node[text=drawColor,anchor=base east,inner sep=0pt, outer sep=0pt, scale=  0.99] at (411.89,109.97) {100};

\node[text=drawColor,anchor=base east,inner sep=0pt, outer sep=0pt, scale=  0.99] at (411.89,170.71) {200};

\node[text=drawColor,anchor=base east,inner sep=0pt, outer sep=0pt, scale=  0.99] at (411.89,231.44) {300};

\node[text=drawColor,anchor=base east,inner sep=0pt, outer sep=0pt, scale=  0.99] at (411.89,292.18) {400};

\node[text=drawColor,anchor=base east,inner sep=0pt, outer sep=0pt, scale=  0.99] at (411.89,352.92) {500};

\node[text=drawColor,anchor=base east,inner sep=0pt, outer sep=0pt, scale=  0.99] at (411.89,413.65) {600};

\node[text=drawColor,anchor=base east,inner sep=0pt, outer sep=0pt, scale=  0.99] at (411.89,474.39) {700};
\end{scope}
\begin{scope}
\path[clip] (419.81, 35.64) rectangle (501.14,494.80);
\definecolor{drawColor}{RGB}{0,0,0}

\path[draw=drawColor,line width= 0.4pt,line join=round,line cap=round] (419.81,318.79) -- (501.14,318.79);
\end{scope}
\end{tikzpicture}

	\end{adjustbox}
	\caption[Streudiagramme der Reaktionszeiten in der \gls{ha}]{Streudiagramme der mittleren Reaktionszeiten in der \gls{ha}. Die horizontale Linie kennzeichnet jeweils den Mittelwert innerhalb einer Bedingung (siehe \autoref{tab:hick_descriptives}). Siehe \autoref{subsec:hick_Versuchsablauf} für eine Beschreibung der Datenaufbereitung. Vp = Versuchsperson.}
	\label{fig:hick_scatterplot}
\end{figure}

\begin{table}[htb]
	\centering
	\captionsetup{labelsep = none}
	\caption[Deskriptive Angaben zu den Reaktionszeiten in der \gls{ha}]{\newline \textit{Deskriptive Angaben zu den mittleren Reaktionszeiten der \gls{ha} in Millisekunden (Mittelwert, Standardabweichung, Minimum, Maximum) sowie Kennwerte zur Verteilungsform und der Reliabilität der Daten} \vspace{.2cm}}
	\label{tab:hick_descriptives}
	\begin{adjustbox}{width=1\textwidth}
	\begin{threeparttable}
		
		\begin{tabular}{
				l
				S[table-format = 3.0, add-integer-zero=false]
				S[table-format = 2.0, add-integer-zero=false]
				S[table-format = 3.0, add-integer-zero=false]
				S[table-format = 3.0, add-integer-zero=false]
				S[table-format = 0.2, add-integer-zero=false]
				S[table-format = 1.2, add-integer-zero=false]
				S[table-format = <0.3, add-integer-zero=false]
				S[table-format = 0.2, add-integer-zero=false]
			}
			\hline
			\multicolumn{1}{c}{Bedingung}	& 	{\textit{M}}&{\textit{SD}}	&	{Min}	&	{Max} 	&	{\textnormal{Schiefe}}	&	{\textnormal{Kurtosis}} &	{S-W \textit{p}-Wert} & $r_{tt}$\\
			\hline
			0-bit		&	240			&	29		&	188		&	394		&	1.58	&	4.99	& 	<.001	&	.90	\\
			1-bit		&	296			&	32		&	234		&	416		&	0.94	&	1.33	& 	<.001	&	.93	\\
			2-bit		&	377			&	54		&	280		&	590		&	0.88	&	1.01	& 	<.001	&	.94	\\
			2.58-bit	&	438			&	67		&	315		&	650		&	0.82	&	0.41	& 	<.001	&	.93	\\
			\hline
		\end{tabular}%
		%}
		\begin{tablenotes}[flushleft]
			\footnotesize				% font size
			\setlength\labelsep{0pt}	% no indent on second line
			\item \textit{Anmerkungen.}  Min~=~Minimum; Max~=~Maximum; S-W~=~Shapiro-Wilk-Test; $r_{tt}$~=~nach der Odd-Even-Methode berechnete, mit der Spearman-Brown-Formel \citep[Spearman 1910; Brown 1910; zitiert nach][S. 123]{Schermelleh-Engel2007} korrigierte Splithalf-Reliabilität.
		\end{tablenotes}%
	\end{threeparttable}%
	\end{adjustbox}
\end{table}

Wie bei der \gls{ssauf} wurde bei der \gls{ha} als erstes geprüft, ob die experimentelle Manipulation (die Anzahl an Antwortalternativen) einen Einfluss auf die abhängige Variable (die Reaktionszeit) ausübte. Dafür wurde 
eine einfaktorielle Varianzanalyse mit Messwiederholung gerechnet.
Weil Sphärizität gemäss einem Mauchly-Test nicht gegeben war, $\upchi^2(5)=219.06$, $p<.001$, wurden die Freiheitsgrade des \textit{F}-Tests mit der Greenhouse-Geisser-Methode korrigiert ($\hat{\varepsilon}=.57$). Der \textit{F}-Test hat ergeben, dass die Unterschiede zwischen den Bedingungsmittelwerten signifikant von 0 abwichen, $F(1.71,\,300.96)=1434.32$, $p<.001$, $\eta_{G}^2=.71$. Der Effekt der Anzahl Antwortalternativen auf die Reaktionszeit konnte dabei gemäss generalisiertem $\eta_{G}^2$ \citep{Olejnik2003} als gross bezeichnet werden \citep[S. 383]{Bakeman2005}.
Um zu erfahren, ob sich alle oder nur bestimmte Mittelwertpaare signifikant voneinander unterschieden, wurden post hoc alle Mittelwerte miteinander verglichen.
Tukey-Tests haben gezeigt, dass sich alle Mittelwertpaare signifikant voneinander unterschieden (alle \textit{p}s $<.001$).
Die Reaktionszeit der \glspl{vp} erhöhte sich also mit zunehmender Anzahl Antwortalternativen signifikant.
Die Effektstärken für die Mittelwertsunterschiede wurden mit Cohens \textit{d} für abhängige Stichproben \citep{Gibbons1993} bestimmt und lagen alle im hohen Bereich \citep[][S. 40; siehe \autoref{tab:hick_effect_sizes}]{Cohen1988}.

\begin{table}[htbp]
	\centering
	\setlength{\tabcolsep}{10pt}
	\captionsetup{labelsep = none}
	\caption[Effektstärken für die Mittelwertsunterschiede in der \gls{ha}]{\newline \textit{Effektstärken (Cohens \textit{d} für abhängige Stichproben) der Mittelwertunterschiede in der \gls{ha}} \vspace{.2cm}}
	\label{tab:hick_effect_sizes}
	\sisetup{table-number-alignment = center}
	\begin{threeparttable}
		\begin{tabular}{
				l
				S[table-format = 1.2]
				S[table-format = 1.2]
				S[table-format = 1.2]
				>{\centering\arraybackslash}p{1.2cm}
			}
			\hline
			
			\multicolumn{1}{c}{Bedingung}	&	{0-bit}		&	{1-bit}		&	{2-bit}		\\
			\hline
			0-bit		&				&				&				\\
			1-bit		&	2.67		&				&				\\
			2-bit		&	3.13		&	2.13		&				\\
			2.58-bit	&	3.43		&	2.71		&	1.62		\\
			
			\hline
			
		\end{tabular}

		\begin{tablenotes}[flushleft]
			\footnotesize				% font size
			\setlength\labelsep{0pt}	% no indent on second line
			\item \textit{Anmerkung}. Alle Mittelwertsunterschiede waren statistisch signifikant ($p<.001$).
		\end{tablenotes}
	\end{threeparttable}
\end{table}

Produkt-Moment-Korrelationen zwischen den vier Bedingungen der \gls{ha} sind in \autoref{tab:product_moment_correlations_manifest} abgetragen. Sie deuteten auf stark positive Zusam\-men\-hänge zwischen den Bedingungen hin. 
%Die Zusammenhänge zwischen der 0-bit-, der 1-bit- und der 2-bit Bedingung waren dabei etwa gleich stark wie bei \citet{Schweizer2006a}.

\FloatBarrier
\subsection{BIS-Test}

Deskriptive Angaben zu den Subtests des \gls{bist} sind in \autoref{tab:bis_descriptives} zu finden.
Wie aufgrund der Modellannahmen des \gls{bist} zu erwarten war, liessen sich zwischen der Mehrheit der Subtests signifikante, positive Zusammenhänge beobachten (siehe \autoref{tab:bis_product_correlations}). Diese Zusammenhänge bildeten die Voraussetzung für die Beantwortung der dritten, vierten und fünften Fragestellung, bei welchen aus den Aufgaben des \gls{bist} mit Hilfe von Faktorenanalysen ein \gls{gfaktor} extrahiert wurde.

Der \gls{zwert} des \gls{bist}s, gebildet als Mittelwert aller $18$ \textit{z}-stand\-ard\-isier\-ter Subtests, wies einen Mittelwert $\pm$ Standardabweichung von $0.02\,\pm\,0.53$ auf (Minimum $= -1.60$, Maximum $= 1.40$). Die Verteilung der \gls{zwert}s (siehe \autoref{fig:bis_zscore_density}) hatte eine negative Schiefe ($-0.02$) und eine positive Kurtosis ($0.16$), wich damit aber gemäss einem Shapiro-Wilk-Test  nicht signifikant von der Normalverteilung ab ($p=.82$).

\begin{table}[!t]
	%\flushleft
	\centering
	\captionsetup{labelsep = none}
	\caption[Deskriptive Angaben zur Anzahl richtig gelöster Items der Subtests im \gls{bist}]{\newline  \textit{Deskriptive Angaben zur Anzahl richtig gelöster Items der Subtests im \gls{bist} (Mittelwert, Standardabweichung, Minimum, Maximum) und Kennwerte zur Verteilungsform der Daten} \vspace{.2cm}}
	\label{tab:bis_descriptives}
	\begin{adjustbox}{width=.82\textwidth,keepaspectratio} %totalheight=1\textheight,
	\begin{threeparttable}
		\begin{tabular}{
				l
				S[table-format = 2.2]
				S[table-format = 2.2]
				S[table-format = 1.0]
				S[table-format = 2.0]
				S[table-format = 1.2]
				S[table-format = 1.2]
				S[table-format = < 0.3, add-integer-zero=false]
				}
			\hline

			\multicolumn{1}{c}{Subtest} &  {\textit{M}}	& {\textit{SD}}	&	{Min}	&	{Max} 	& {\textnormal{Schiefe}}	& {\textnormal{Kurtosis}} & {S-W  \textit{p}-Wert}\\
			\hline
			OG		&	15.31		&	4.82		&	3		&	27		&	-0.35				&	-0.05					& 		.03			\\
			ZN		&	3.86		&	2.44		&	0		&	9		&	0.50				&	-0.83					& 		<.001			\\
			AN		&	3.23		&	1.62		&	0		&	7		&	0.08				&	-0.41					& 		<.001			\\
			XG		&	19.45		&	6.52		&	1		&	36		&	0.14				&	0.08					& 		.05			\\
			WA		&	3.23		&	1.87		&	0		&	7		&	0.10				&	-0.71					& 		<.001			\\
			ZP		&	5.95		&	2.28		&	1		&	12		&	0.27				&	-0.12					& 		.003			\\
			TM		&	9.25		&	3.62		&	1		&	16		&	-0.03				&	-0.83					& 		.002			\\
			BD		&	51.01		&	10.76		&	2		&	78		&	-1.46				&	5.86					& 		<.001			\\
			SC		&	3.16		&	1.97		&	0		&	7		&	0.06				&	-1.01					& 		<.001			\\
			ST		&	8.59		&	3.68		&	0		&	18		&	-0.04				&	-0.34					& 		.12			\\
			CH		&	2.76		&	1.66		&	0		&	6		&	-0.01				&	-0.81					& 		<.001			\\
			TG		&	11.72		&	3.20		&	1		&	20		&	-0.66				&	1.01					& 		<.001			\\
			RZ		&	10.80		&	4.02		&	1		&	20		&	-0.19				&	-0.49					& 		.06			\\
			WM		&	7.15		&	2.89		&	2		&	17		&	0.77				&	0.83					& 		<.001			\\
			KW		&	23.31		&	6.46		&	2		&	37		&	-0.24				&	0.13					& 		.04			\\
			ZZ		&	6.41		&	2.94		&	1		&	14		&	0.32				&	-0.33					& 		.002			\\
			OE		&	34.33		&	5.93		&	9		&	47		&	-0.46				&	1.08					& 		.007			\\
			WE		&	18.25		&	6.07		&	1		&	31		&	-0.25				&	-0.33					& 		.14			\\
			
			\hline
		\end{tabular}

		\begin{tablenotes}[flushleft]
			\footnotesize				% font size
			\setlength\labelsep{0pt}	% no indent on second line
			\item \textit{Anmerkungen.} Siehe \autoref{tab:bis_subtest_description} für eine Beschreibung der Subtests.
			Min~=~Minimum; Max~=~Maximum; S-W~=~Shapiro-Wilk-Test.
		\end{tablenotes}
	\end{threeparttable}
	\end{adjustbox}
\end{table}

\begin{figure}[!b]
	\centering
	%	\captionsetup{font = small}
	\begin{adjustbox}{width=1\textwidth}
		% Created by tikzDevice version 0.10.1 on 2016-07-22 08:13:35
% !TEX encoding = UTF-8 Unicode
\begin{tikzpicture}[x=1pt,y=1pt]
\definecolor{fillColor}{RGB}{255,255,255}
\path[use as bounding box,fill=fillColor,fill opacity=0.00] (0,0) rectangle (361.35,144.54);
\begin{scope}
\path[clip] (  0.00, 48.00) rectangle (361.35,144.54);
\definecolor{drawColor}{RGB}{0,0,0}

\path[draw=drawColor,line width= 0.4pt,line join=round,line cap=round] ( 26.91, 51.61) --
	( 27.48, 51.62) --
	( 28.05, 51.63) --
	( 28.62, 51.64) --
	( 29.19, 51.66) --
	( 29.76, 51.68) --
	( 30.33, 51.70) --
	( 30.90, 51.73) --
	( 31.47, 51.77) --
	( 32.04, 51.81) --
	( 32.61, 51.86) --
	( 33.18, 51.91) --
	( 33.75, 51.98) --
	( 34.32, 52.05) --
	( 34.89, 52.13) --
	( 35.46, 52.22) --
	( 36.03, 52.32) --
	( 36.60, 52.42) --
	( 37.17, 52.54) --
	( 37.74, 52.67) --
	( 38.31, 52.80) --
	( 38.88, 52.94) --
	( 39.45, 53.08) --
	( 40.02, 53.23) --
	( 40.59, 53.39) --
	( 41.16, 53.54) --
	( 41.73, 53.69) --
	( 42.30, 53.84) --
	( 42.87, 53.99) --
	( 43.43, 54.13) --
	( 44.00, 54.26) --
	( 44.57, 54.38) --
	( 45.14, 54.49) --
	( 45.71, 54.59) --
	( 46.28, 54.67) --
	( 46.85, 54.75) --
	( 47.42, 54.81) --
	( 47.99, 54.87) --
	( 48.56, 54.91) --
	( 49.13, 54.94) --
	( 49.70, 54.96) --
	( 50.27, 54.98) --
	( 50.84, 54.99) --
	( 51.41, 55.00) --
	( 51.98, 55.00) --
	( 52.55, 55.01) --
	( 53.12, 55.01) --
	( 53.69, 55.01) --
	( 54.26, 55.01) --
	( 54.83, 55.01) --
	( 55.40, 55.00) --
	( 55.97, 55.00) --
	( 56.54, 54.99) --
	( 57.11, 54.98) --
	( 57.68, 54.96) --
	( 58.25, 54.93) --
	( 58.82, 54.90) --
	( 59.39, 54.86) --
	( 59.96, 54.80) --
	( 60.53, 54.73) --
	( 61.10, 54.65) --
	( 61.67, 54.56) --
	( 62.24, 54.46) --
	( 62.81, 54.35) --
	( 63.38, 54.22) --
	( 63.95, 54.09) --
	( 64.52, 53.95) --
	( 65.09, 53.81) --
	( 65.66, 53.66) --
	( 66.23, 53.52) --
	( 66.80, 53.37) --
	( 67.37, 53.22) --
	( 67.94, 53.09) --
	( 68.51, 52.95) --
	( 69.08, 52.83) --
	( 69.65, 52.72) --
	( 70.22, 52.62) --
	( 70.79, 52.53) --
	( 71.36, 52.46) --
	( 71.93, 52.40) --
	( 72.50, 52.36) --
	( 73.07, 52.33) --
	( 73.64, 52.32) --
	( 74.21, 52.33) --
	( 74.78, 52.36) --
	( 75.35, 52.40) --
	( 75.92, 52.46) --
	( 76.49, 52.54) --
	( 77.06, 52.64) --
	( 77.63, 52.75) --
	( 78.20, 52.88) --
	( 78.77, 53.03) --
	( 79.34, 53.19) --
	( 79.91, 53.37) --
	( 80.48, 53.56) --
	( 81.05, 53.76) --
	( 81.62, 53.99) --
	( 82.19, 54.23) --
	( 82.76, 54.48) --
	( 83.33, 54.74) --
	( 83.90, 55.03) --
	( 84.46, 55.32) --
	( 85.03, 55.64) --
	( 85.60, 55.98) --
	( 86.17, 56.34) --
	( 86.74, 56.72) --
	( 87.31, 57.12) --
	( 87.88, 57.54) --
	( 88.45, 58.00) --
	( 89.02, 58.48) --
	( 89.59, 58.99) --
	( 90.16, 59.53) --
	( 90.73, 60.09) --
	( 91.30, 60.67) --
	( 91.87, 61.28) --
	( 92.44, 61.90) --
	( 93.01, 62.53) --
	( 93.58, 63.17) --
	( 94.15, 63.81) --
	( 94.72, 64.43) --
	( 95.29, 65.04) --
	( 95.86, 65.62) --
	( 96.43, 66.16) --
	( 97.00, 66.66) --
	( 97.57, 67.10) --
	( 98.14, 67.49) --
	( 98.71, 67.81) --
	( 99.28, 68.06) --
	( 99.85, 68.23) --
	(100.42, 68.32) --
	(100.99, 68.34) --
	(101.56, 68.29) --
	(102.13, 68.17) --
	(102.70, 67.98) --
	(103.27, 67.73) --
	(103.84, 67.43) --
	(104.41, 67.10) --
	(104.98, 66.75) --
	(105.55, 66.38) --
	(106.12, 66.01) --
	(106.69, 65.66) --
	(107.26, 65.34) --
	(107.83, 65.06) --
	(108.40, 64.83) --
	(108.97, 64.67) --
	(109.54, 64.58) --
	(110.11, 64.58) --
	(110.68, 64.70) --
	(111.25, 64.92) --
	(111.82, 65.26) --
	(112.39, 65.71) --
	(112.96, 66.27) --
	(113.53, 66.96) --
	(114.10, 67.78) --
	(114.67, 68.73) --
	(115.24, 69.79) --
	(115.81, 70.96) --
	(116.38, 72.22) --
	(116.95, 73.56) --
	(117.52, 74.98) --
	(118.09, 76.47) --
	(118.66, 78.00) --
	(119.23, 79.54) --
	(119.80, 81.10) --
	(120.37, 82.64) --
	(120.94, 84.15) --
	(121.51, 85.61) --
	(122.08, 86.99) --
	(122.65, 88.28) --
	(123.22, 89.46) --
	(123.79, 90.54) --
	(124.36, 91.48) --
	(124.93, 92.29) --
	(125.49, 92.92) --
	(126.06, 93.41) --
	(126.63, 93.75) --
	(127.20, 93.94) --
	(127.77, 93.98) --
	(128.34, 93.89) --
	(128.91, 93.65) --
	(129.48, 93.29) --
	(130.05, 92.82) --
	(130.62, 92.27) --
	(131.19, 91.64) --
	(131.76, 90.96) --
	(132.33, 90.23) --
	(132.90, 89.48) --
	(133.47, 88.73) --
	(134.04, 88.00) --
	(134.61, 87.29) --
	(135.18, 86.63) --
	(135.75, 86.02) --
	(136.32, 85.50) --
	(136.89, 85.06) --
	(137.46, 84.71) --
	(138.03, 84.45) --
	(138.60, 84.30) --
	(139.17, 84.25) --
	(139.74, 84.32) --
	(140.31, 84.52) --
	(140.88, 84.83) --
	(141.45, 85.24) --
	(142.02, 85.76) --
	(142.59, 86.39) --
	(143.16, 87.13) --
	(143.73, 87.99) --
	(144.30, 88.94) --
	(144.87, 89.99) --
	(145.44, 91.13) --
	(146.01, 92.36) --
	(146.58, 93.67) --
	(147.15, 95.08) --
	(147.72, 96.57) --
	(148.29, 98.13) --
	(148.86, 99.77) --
	(149.43,101.48) --
	(150.00,103.24) --
	(150.57,105.07) --
	(151.14,106.97) --
	(151.71,108.90) --
	(152.28,110.87) --
	(152.85,112.87) --
	(153.42,114.89) --
	(153.99,116.91) --
	(154.56,118.93) --
	(155.13,120.92) --
	(155.70,122.86) --
	(156.27,124.75) --
	(156.84,126.56) --
	(157.41,128.28) --
	(157.98,129.89) --
	(158.55,131.35) --
	(159.12,132.66) --
	(159.69,133.82) --
	(160.26,134.81) --
	(160.83,135.62) --
	(161.40,136.26) --
	(161.97,136.68) --
	(162.54,136.91) --
	(163.11,136.98) --
	(163.68,136.89) --
	(164.25,136.64) --
	(164.82,136.27) --
	(165.39,135.78) --
	(165.95,135.19) --
	(166.52,134.54) --
	(167.09,133.86) --
	(167.66,133.17) --
	(168.23,132.50) --
	(168.80,131.85) --
	(169.37,131.28) --
	(169.94,130.80) --
	(170.51,130.40) --
	(171.08,130.12) --
	(171.65,129.94) --
	(172.22,129.88) --
	(172.79,129.95) --
	(173.36,130.14) --
	(173.93,130.42) --
	(174.50,130.79) --
	(175.07,131.24) --
	(175.64,131.73) --
	(176.21,132.27) --
	(176.78,132.81) --
	(177.35,133.34) --
	(177.92,133.83) --
	(178.49,134.26) --
	(179.06,134.61) --
	(179.63,134.87) --
	(180.20,135.01) --
	(180.77,135.01) --
	(181.34,134.88) --
	(181.91,134.61) --
	(182.48,134.21) --
	(183.05,133.67) --
	(183.62,132.99) --
	(184.19,132.18) --
	(184.76,131.26) --
	(185.33,130.26) --
	(185.90,129.19) --
	(186.47,128.07) --
	(187.04,126.91) --
	(187.61,125.75) --
	(188.18,124.61) --
	(188.75,123.50) --
	(189.32,122.46) --
	(189.89,121.48) --
	(190.46,120.59) --
	(191.03,119.81) --
	(191.60,119.16) --
	(192.17,118.62) --
	(192.74,118.20) --
	(193.31,117.89) --
	(193.88,117.70) --
	(194.45,117.62) --
	(195.02,117.65) --
	(195.59,117.78) --
	(196.16,117.97) --
	(196.73,118.22) --
	(197.30,118.52) --
	(197.87,118.85) --
	(198.44,119.19) --
	(199.01,119.52) --
	(199.58,119.84) --
	(200.15,120.12) --
	(200.72,120.36) --
	(201.29,120.54) --
	(201.86,120.66) --
	(202.43,120.69) --
	(203.00,120.63) --
	(203.57,120.50) --
	(204.14,120.28) --
	(204.71,119.97) --
	(205.28,119.58) --
	(205.85,119.09) --
	(206.42,118.50) --
	(206.98,117.84) --
	(207.55,117.11) --
	(208.12,116.31) --
	(208.69,115.44) --
	(209.26,114.51) --
	(209.83,113.52) --
	(210.40,112.48) --
	(210.97,111.41) --
	(211.54,110.31) --
	(212.11,109.19) --
	(212.68,108.04) --
	(213.25,106.89) --
	(213.82,105.75) --
	(214.39,104.61) --
	(214.96,103.49) --
	(215.53,102.40) --
	(216.10,101.35) --
	(216.67,100.33) --
	(217.24, 99.38) --
	(217.81, 98.49) --
	(218.38, 97.67) --
	(218.95, 96.91) --
	(219.52, 96.22) --
	(220.09, 95.61) --
	(220.66, 95.09) --
	(221.23, 94.64) --
	(221.80, 94.26) --
	(222.37, 93.94) --
	(222.94, 93.67) --
	(223.51, 93.45) --
	(224.08, 93.27) --
	(224.65, 93.11) --
	(225.22, 92.95) --
	(225.79, 92.79) --
	(226.36, 92.60) --
	(226.93, 92.39) --
	(227.50, 92.13) --
	(228.07, 91.81) --
	(228.64, 91.42) --
	(229.21, 90.96) --
	(229.78, 90.43) --
	(230.35, 89.83) --
	(230.92, 89.15) --
	(231.49, 88.40) --
	(232.06, 87.59) --
	(232.63, 86.72) --
	(233.20, 85.81) --
	(233.77, 84.87) --
	(234.34, 83.91) --
	(234.91, 82.96) --
	(235.48, 82.01) --
	(236.05, 81.10) --
	(236.62, 80.23) --
	(237.19, 79.41) --
	(237.76, 78.66) --
	(238.33, 77.99) --
	(238.90, 77.40) --
	(239.47, 76.92) --
	(240.04, 76.53) --
	(240.61, 76.23) --
	(241.18, 76.02) --
	(241.75, 75.91) --
	(242.32, 75.88) --
	(242.89, 75.94) --
	(243.46, 76.07) --
	(244.03, 76.25) --
	(244.60, 76.47) --
	(245.17, 76.72) --
	(245.74, 77.00) --
	(246.31, 77.28) --
	(246.88, 77.54) --
	(247.45, 77.79) --
	(248.01, 78.00) --
	(248.58, 78.16) --
	(249.15, 78.27) --
	(249.72, 78.32) --
	(250.29, 78.29) --
	(250.86, 78.18) --
	(251.43, 78.01) --
	(252.00, 77.76) --
	(252.57, 77.45) --
	(253.14, 77.07) --
	(253.71, 76.61) --
	(254.28, 76.11) --
	(254.85, 75.55) --
	(255.42, 74.96) --
	(255.99, 74.35) --
	(256.56, 73.71) --
	(257.13, 73.05) --
	(257.70, 72.40) --
	(258.27, 71.75) --
	(258.84, 71.12) --
	(259.41, 70.50) --
	(259.98, 69.90) --
	(260.55, 69.33) --
	(261.12, 68.80) --
	(261.69, 68.29) --
	(262.26, 67.82) --
	(262.83, 67.37) --
	(263.40, 66.95) --
	(263.97, 66.56) --
	(264.54, 66.19) --
	(265.11, 65.84) --
	(265.68, 65.51) --
	(266.25, 65.19) --
	(266.82, 64.87) --
	(267.39, 64.56) --
	(267.96, 64.24) --
	(268.53, 63.93) --
	(269.10, 63.60) --
	(269.67, 63.27) --
	(270.24, 62.93) --
	(270.81, 62.58) --
	(271.38, 62.22) --
	(271.95, 61.85) --
	(272.52, 61.48) --
	(273.09, 61.10) --
	(273.66, 60.72) --
	(274.23, 60.35) --
	(274.80, 59.98) --
	(275.37, 59.62) --
	(275.94, 59.28) --
	(276.51, 58.96) --
	(277.08, 58.66) --
	(277.65, 58.38) --
	(278.22, 58.14) --
	(278.79, 57.92) --
	(279.36, 57.74) --
	(279.93, 57.59) --
	(280.50, 57.48) --
	(281.07, 57.40) --
	(281.64, 57.35) --
	(282.21, 57.32) --
	(282.78, 57.33) --
	(283.35, 57.36) --
	(283.92, 57.42) --
	(284.49, 57.48) --
	(285.06, 57.56) --
	(285.63, 57.65) --
	(286.20, 57.74) --
	(286.77, 57.84) --
	(287.34, 57.92) --
	(287.91, 58.01) --
	(288.48, 58.08) --
	(289.04, 58.14) --
	(289.61, 58.18) --
	(290.18, 58.21) --
	(290.75, 58.21) --
	(291.32, 58.20) --
	(291.89, 58.17) --
	(292.46, 58.12) --
	(293.03, 58.04) --
	(293.60, 57.95) --
	(294.17, 57.83) --
	(294.74, 57.70) --
	(295.31, 57.54) --
	(295.88, 57.37) --
	(296.45, 57.18) --
	(297.02, 56.98) --
	(297.59, 56.77) --
	(298.16, 56.54) --
	(298.73, 56.30) --
	(299.30, 56.06) --
	(299.87, 55.80) --
	(300.44, 55.55) --
	(301.01, 55.29) --
	(301.58, 55.03) --
	(302.15, 54.78) --
	(302.72, 54.53) --
	(303.29, 54.28) --
	(303.86, 54.04) --
	(304.43, 53.81) --
	(305.00, 53.59) --
	(305.57, 53.38) --
	(306.14, 53.19) --
	(306.71, 53.00) --
	(307.28, 52.83) --
	(307.85, 52.67) --
	(308.42, 52.53) --
	(308.99, 52.40) --
	(309.56, 52.29) --
	(310.13, 52.18) --
	(310.70, 52.09) --
	(311.27, 52.01) --
	(311.84, 51.94) --
	(312.41, 51.88) --
	(312.98, 51.82) --
	(313.55, 51.78) --
	(314.12, 51.74) --
	(314.69, 51.71) --
	(315.26, 51.68) --
	(315.83, 51.66) --
	(316.40, 51.64) --
	(316.97, 51.63) --
	(317.54, 51.62) --
	(318.11, 51.61);
\end{scope}
\begin{scope}
\path[clip] (  0.00,  0.00) rectangle (361.35,144.54);
\definecolor{drawColor}{RGB}{0,0,0}

\node[text=drawColor,anchor=base,inner sep=0pt, outer sep=0pt, scale=  1.00] at (180.68,  8.40) {BIS-Test \textit{z}-Wert};
\end{scope}
\begin{scope}
\path[clip] (  0.00,  0.00) rectangle (361.35,144.54);
\definecolor{drawColor}{RGB}{0,0,0}

\path[draw=drawColor,line width= 0.4pt,line join=round,line cap=round] ( 13.38, 48.00) -- (347.97, 48.00);

\path[draw=drawColor,line width= 0.4pt,line join=round,line cap=round] ( 13.38, 48.00) -- ( 13.38, 42.00);

\path[draw=drawColor,line width= 0.4pt,line join=round,line cap=round] ( 97.03, 48.00) -- ( 97.03, 42.00);

\path[draw=drawColor,line width= 0.4pt,line join=round,line cap=round] (180.67, 48.00) -- (180.67, 42.00);

\path[draw=drawColor,line width= 0.4pt,line join=round,line cap=round] (264.32, 48.00) -- (264.32, 42.00);

\path[draw=drawColor,line width= 0.4pt,line join=round,line cap=round] (347.97, 48.00) -- (347.97, 42.00);

\node[text=drawColor,anchor=base,inner sep=0pt, outer sep=0pt, scale=  1.00] at ( 13.38, 30.00) {-2};

\node[text=drawColor,anchor=base,inner sep=0pt, outer sep=0pt, scale=  1.00] at ( 97.03, 30.00) {-1};

\node[text=drawColor,anchor=base,inner sep=0pt, outer sep=0pt, scale=  1.00] at (180.68, 30.00) {0};

\node[text=drawColor,anchor=base,inner sep=0pt, outer sep=0pt, scale=  1.00] at (264.32, 30.00) {1};

\node[text=drawColor,anchor=base,inner sep=0pt, outer sep=0pt, scale=  1.00] at (347.97, 30.00) {2};

\path[draw=drawColor,line width= 0.2pt,line join=round,line cap=round] ( 46.98, 48.00) -- ( 46.98, 57.65);

\path[draw=drawColor,line width= 0.2pt,line join=round,line cap=round] ( 60.23, 48.00) -- ( 60.23, 57.65);

\path[draw=drawColor,line width= 0.2pt,line join=round,line cap=round] ( 87.23, 48.00) -- ( 87.23, 57.65);

\path[draw=drawColor,line width= 0.2pt,line join=round,line cap=round] ( 97.59, 48.00) -- ( 97.59, 57.65);

\path[draw=drawColor,line width= 0.2pt,line join=round,line cap=round] ( 99.08, 48.00) -- ( 99.08, 57.65);

\path[draw=drawColor,line width= 0.2pt,line join=round,line cap=round] ( 99.70, 48.00) -- ( 99.70, 57.65);

\path[draw=drawColor,line width= 0.2pt,line join=round,line cap=round] (100.01, 48.00) -- (100.01, 57.65);

\path[draw=drawColor,line width= 0.2pt,line join=round,line cap=round] (102.50, 48.00) -- (102.50, 57.65);

\path[draw=drawColor,line width= 0.2pt,line join=round,line cap=round] (104.94, 48.00) -- (104.94, 57.65);

\path[draw=drawColor,line width= 0.2pt,line join=round,line cap=round] (112.82, 48.00) -- (112.82, 57.65);

\path[draw=drawColor,line width= 0.2pt,line join=round,line cap=round] (120.20, 48.00) -- (120.20, 57.65);

\path[draw=drawColor,line width= 0.2pt,line join=round,line cap=round] (122.48, 48.00) -- (122.48, 57.65);

\path[draw=drawColor,line width= 0.2pt,line join=round,line cap=round] (122.80, 48.00) -- (122.80, 57.65);

\path[draw=drawColor,line width= 0.2pt,line join=round,line cap=round] (122.95, 48.00) -- (122.95, 57.65);

\path[draw=drawColor,line width= 0.2pt,line join=round,line cap=round] (123.58, 48.00) -- (123.58, 57.65);

\path[draw=drawColor,line width= 0.2pt,line join=round,line cap=round] (124.02, 48.00) -- (124.02, 57.65);

\path[draw=drawColor,line width= 0.2pt,line join=round,line cap=round] (124.66, 48.00) -- (124.66, 57.65);

\path[draw=drawColor,line width= 0.2pt,line join=round,line cap=round] (125.16, 48.00) -- (125.16, 57.65);

\path[draw=drawColor,line width= 0.2pt,line join=round,line cap=round] (126.43, 48.00) -- (126.43, 57.65);

\path[draw=drawColor,line width= 0.2pt,line join=round,line cap=round] (127.78, 48.00) -- (127.78, 57.65);

\path[draw=drawColor,line width= 0.2pt,line join=round,line cap=round] (128.72, 48.00) -- (128.72, 57.65);

\path[draw=drawColor,line width= 0.2pt,line join=round,line cap=round] (128.97, 48.00) -- (128.97, 57.65);

\path[draw=drawColor,line width= 0.2pt,line join=round,line cap=round] (130.40, 48.00) -- (130.40, 57.65);

\path[draw=drawColor,line width= 0.2pt,line join=round,line cap=round] (131.87, 48.00) -- (131.87, 57.65);

\path[draw=drawColor,line width= 0.2pt,line join=round,line cap=round] (132.11, 48.00) -- (132.11, 57.65);

\path[draw=drawColor,line width= 0.2pt,line join=round,line cap=round] (132.25, 48.00) -- (132.25, 57.65);

\path[draw=drawColor,line width= 0.2pt,line join=round,line cap=round] (133.15, 48.00) -- (133.15, 57.65);

\path[draw=drawColor,line width= 0.2pt,line join=round,line cap=round] (138.92, 48.00) -- (138.92, 57.65);

\path[draw=drawColor,line width= 0.2pt,line join=round,line cap=round] (140.24, 48.00) -- (140.24, 57.65);

\path[draw=drawColor,line width= 0.2pt,line join=round,line cap=round] (141.99, 48.00) -- (141.99, 57.65);

\path[draw=drawColor,line width= 0.2pt,line join=round,line cap=round] (142.92, 48.00) -- (142.92, 57.65);

\path[draw=drawColor,line width= 0.2pt,line join=round,line cap=round] (143.07, 48.00) -- (143.07, 57.65);

\path[draw=drawColor,line width= 0.2pt,line join=round,line cap=round] (145.10, 48.00) -- (145.10, 57.65);

\path[draw=drawColor,line width= 0.2pt,line join=round,line cap=round] (149.25, 48.00) -- (149.25, 57.65);

\path[draw=drawColor,line width= 0.2pt,line join=round,line cap=round] (149.95, 48.00) -- (149.95, 57.65);

\path[draw=drawColor,line width= 0.2pt,line join=round,line cap=round] (150.75, 48.00) -- (150.75, 57.65);

\path[draw=drawColor,line width= 0.2pt,line join=round,line cap=round] (150.98, 48.00) -- (150.98, 57.65);

\path[draw=drawColor,line width= 0.2pt,line join=round,line cap=round] (151.16, 48.00) -- (151.16, 57.65);

\path[draw=drawColor,line width= 0.2pt,line join=round,line cap=round] (151.26, 48.00) -- (151.26, 57.65);

\path[draw=drawColor,line width= 0.2pt,line join=round,line cap=round] (151.45, 48.00) -- (151.45, 57.65);

\path[draw=drawColor,line width= 0.2pt,line join=round,line cap=round] (151.68, 48.00) -- (151.68, 57.65);

\path[draw=drawColor,line width= 0.2pt,line join=round,line cap=round] (154.04, 48.00) -- (154.04, 57.65);

\path[draw=drawColor,line width= 0.2pt,line join=round,line cap=round] (156.54, 48.00) -- (156.54, 57.65);

\path[draw=drawColor,line width= 0.2pt,line join=round,line cap=round] (156.92, 48.00) -- (156.92, 57.65);

\path[draw=drawColor,line width= 0.2pt,line join=round,line cap=round] (157.83, 48.00) -- (157.83, 57.65);

\path[draw=drawColor,line width= 0.2pt,line join=round,line cap=round] (158.28, 48.00) -- (158.28, 57.65);

\path[draw=drawColor,line width= 0.2pt,line join=round,line cap=round] (158.86, 48.00) -- (158.86, 57.65);

\path[draw=drawColor,line width= 0.2pt,line join=round,line cap=round] (159.96, 48.00) -- (159.96, 57.65);

\path[draw=drawColor,line width= 0.2pt,line join=round,line cap=round] (159.98, 48.00) -- (159.98, 57.65);

\path[draw=drawColor,line width= 0.2pt,line join=round,line cap=round] (160.00, 48.00) -- (160.00, 57.65);

\path[draw=drawColor,line width= 0.2pt,line join=round,line cap=round] (160.16, 48.00) -- (160.16, 57.65);

\path[draw=drawColor,line width= 0.2pt,line join=round,line cap=round] (160.70, 48.00) -- (160.70, 57.65);

\path[draw=drawColor,line width= 0.2pt,line join=round,line cap=round] (160.79, 48.00) -- (160.79, 57.65);

\path[draw=drawColor,line width= 0.2pt,line join=round,line cap=round] (160.90, 48.00) -- (160.90, 57.65);

\path[draw=drawColor,line width= 0.2pt,line join=round,line cap=round] (161.28, 48.00) -- (161.28, 57.65);

\path[draw=drawColor,line width= 0.2pt,line join=round,line cap=round] (161.35, 48.00) -- (161.35, 57.65);

\path[draw=drawColor,line width= 0.2pt,line join=round,line cap=round] (161.63, 48.00) -- (161.63, 57.65);

\path[draw=drawColor,line width= 0.2pt,line join=round,line cap=round] (162.25, 48.00) -- (162.25, 57.65);

\path[draw=drawColor,line width= 0.2pt,line join=round,line cap=round] (162.46, 48.00) -- (162.46, 57.65);

\path[draw=drawColor,line width= 0.2pt,line join=round,line cap=round] (162.93, 48.00) -- (162.93, 57.65);

\path[draw=drawColor,line width= 0.2pt,line join=round,line cap=round] (164.39, 48.00) -- (164.39, 57.65);

\path[draw=drawColor,line width= 0.2pt,line join=round,line cap=round] (164.95, 48.00) -- (164.95, 57.65);

\path[draw=drawColor,line width= 0.2pt,line join=round,line cap=round] (165.51, 48.00) -- (165.51, 57.65);

\path[draw=drawColor,line width= 0.2pt,line join=round,line cap=round] (165.53, 48.00) -- (165.53, 57.65);

\path[draw=drawColor,line width= 0.2pt,line join=round,line cap=round] (165.99, 48.00) -- (165.99, 57.65);

\path[draw=drawColor,line width= 0.2pt,line join=round,line cap=round] (166.30, 48.00) -- (166.30, 57.65);

\path[draw=drawColor,line width= 0.2pt,line join=round,line cap=round] (166.40, 48.00) -- (166.40, 57.65);

\path[draw=drawColor,line width= 0.2pt,line join=round,line cap=round] (166.93, 48.00) -- (166.93, 57.65);

\path[draw=drawColor,line width= 0.2pt,line join=round,line cap=round] (167.14, 48.00) -- (167.14, 57.65);

\path[draw=drawColor,line width= 0.2pt,line join=round,line cap=round] (167.57, 48.00) -- (167.57, 57.65);

\path[draw=drawColor,line width= 0.2pt,line join=round,line cap=round] (170.98, 48.00) -- (170.98, 57.65);

\path[draw=drawColor,line width= 0.2pt,line join=round,line cap=round] (171.03, 48.00) -- (171.03, 57.65);

\path[draw=drawColor,line width= 0.2pt,line join=round,line cap=round] (171.67, 48.00) -- (171.67, 57.65);

\path[draw=drawColor,line width= 0.2pt,line join=round,line cap=round] (171.78, 48.00) -- (171.78, 57.65);

\path[draw=drawColor,line width= 0.2pt,line join=round,line cap=round] (173.32, 48.00) -- (173.32, 57.65);

\path[draw=drawColor,line width= 0.2pt,line join=round,line cap=round] (174.43, 48.00) -- (174.43, 57.65);

\path[draw=drawColor,line width= 0.2pt,line join=round,line cap=round] (175.26, 48.00) -- (175.26, 57.65);

\path[draw=drawColor,line width= 0.2pt,line join=round,line cap=round] (176.63, 48.00) -- (176.63, 57.65);

\path[draw=drawColor,line width= 0.2pt,line join=round,line cap=round] (176.85, 48.00) -- (176.85, 57.65);

\path[draw=drawColor,line width= 0.2pt,line join=round,line cap=round] (177.14, 48.00) -- (177.14, 57.65);

\path[draw=drawColor,line width= 0.2pt,line join=round,line cap=round] (177.16, 48.00) -- (177.16, 57.65);

\path[draw=drawColor,line width= 0.2pt,line join=round,line cap=round] (177.61, 48.00) -- (177.61, 57.65);

\path[draw=drawColor,line width= 0.2pt,line join=round,line cap=round] (178.13, 48.00) -- (178.13, 57.65);

\path[draw=drawColor,line width= 0.2pt,line join=round,line cap=round] (178.61, 48.00) -- (178.61, 57.65);

\path[draw=drawColor,line width= 0.2pt,line join=round,line cap=round] (179.04, 48.00) -- (179.04, 57.65);

\path[draw=drawColor,line width= 0.2pt,line join=round,line cap=round] (179.41, 48.00) -- (179.41, 57.65);

\path[draw=drawColor,line width= 0.2pt,line join=round,line cap=round] (179.95, 48.00) -- (179.95, 57.65);

\path[draw=drawColor,line width= 0.2pt,line join=round,line cap=round] (180.72, 48.00) -- (180.72, 57.65);

\path[draw=drawColor,line width= 0.2pt,line join=round,line cap=round] (181.14, 48.00) -- (181.14, 57.65);

\path[draw=drawColor,line width= 0.2pt,line join=round,line cap=round] (182.19, 48.00) -- (182.19, 57.65);

\path[draw=drawColor,line width= 0.2pt,line join=round,line cap=round] (182.29, 48.00) -- (182.29, 57.65);

\path[draw=drawColor,line width= 0.2pt,line join=round,line cap=round] (182.64, 48.00) -- (182.64, 57.65);

\path[draw=drawColor,line width= 0.2pt,line join=round,line cap=round] (182.76, 48.00) -- (182.76, 57.65);

\path[draw=drawColor,line width= 0.2pt,line join=round,line cap=round] (183.11, 48.00) -- (183.11, 57.65);

\path[draw=drawColor,line width= 0.2pt,line join=round,line cap=round] (183.69, 48.00) -- (183.69, 57.65);

\path[draw=drawColor,line width= 0.2pt,line join=round,line cap=round] (184.14, 48.00) -- (184.14, 57.65);

\path[draw=drawColor,line width= 0.2pt,line join=round,line cap=round] (184.28, 48.00) -- (184.28, 57.65);

\path[draw=drawColor,line width= 0.2pt,line join=round,line cap=round] (184.43, 48.00) -- (184.43, 57.65);

\path[draw=drawColor,line width= 0.2pt,line join=round,line cap=round] (184.89, 48.00) -- (184.89, 57.65);

\path[draw=drawColor,line width= 0.2pt,line join=round,line cap=round] (185.60, 48.00) -- (185.60, 57.65);

\path[draw=drawColor,line width= 0.2pt,line join=round,line cap=round] (186.12, 48.00) -- (186.12, 57.65);

\path[draw=drawColor,line width= 0.2pt,line join=round,line cap=round] (187.52, 48.00) -- (187.52, 57.65);

\path[draw=drawColor,line width= 0.2pt,line join=round,line cap=round] (187.90, 48.00) -- (187.90, 57.65);

\path[draw=drawColor,line width= 0.2pt,line join=round,line cap=round] (188.55, 48.00) -- (188.55, 57.65);

\path[draw=drawColor,line width= 0.2pt,line join=round,line cap=round] (189.78, 48.00) -- (189.78, 57.65);

\path[draw=drawColor,line width= 0.2pt,line join=round,line cap=round] (190.19, 48.00) -- (190.19, 57.65);

\path[draw=drawColor,line width= 0.2pt,line join=round,line cap=round] (190.76, 48.00) -- (190.76, 57.65);

\path[draw=drawColor,line width= 0.2pt,line join=round,line cap=round] (193.09, 48.00) -- (193.09, 57.65);

\path[draw=drawColor,line width= 0.2pt,line join=round,line cap=round] (195.04, 48.00) -- (195.04, 57.65);

\path[draw=drawColor,line width= 0.2pt,line join=round,line cap=round] (195.67, 48.00) -- (195.67, 57.65);

\path[draw=drawColor,line width= 0.2pt,line join=round,line cap=round] (196.05, 48.00) -- (196.05, 57.65);

\path[draw=drawColor,line width= 0.2pt,line join=round,line cap=round] (196.15, 48.00) -- (196.15, 57.65);

\path[draw=drawColor,line width= 0.2pt,line join=round,line cap=round] (197.25, 48.00) -- (197.25, 57.65);

\path[draw=drawColor,line width= 0.2pt,line join=round,line cap=round] (197.96, 48.00) -- (197.96, 57.65);

\path[draw=drawColor,line width= 0.2pt,line join=round,line cap=round] (198.19, 48.00) -- (198.19, 57.65);

\path[draw=drawColor,line width= 0.2pt,line join=round,line cap=round] (198.56, 48.00) -- (198.56, 57.65);

\path[draw=drawColor,line width= 0.2pt,line join=round,line cap=round] (199.17, 48.00) -- (199.17, 57.65);

\path[draw=drawColor,line width= 0.2pt,line join=round,line cap=round] (199.47, 48.00) -- (199.47, 57.65);

\path[draw=drawColor,line width= 0.2pt,line join=round,line cap=round] (200.82, 48.00) -- (200.82, 57.65);

\path[draw=drawColor,line width= 0.2pt,line join=round,line cap=round] (201.19, 48.00) -- (201.19, 57.65);

\path[draw=drawColor,line width= 0.2pt,line join=round,line cap=round] (201.86, 48.00) -- (201.86, 57.65);

\path[draw=drawColor,line width= 0.2pt,line join=round,line cap=round] (202.72, 48.00) -- (202.72, 57.65);

\path[draw=drawColor,line width= 0.2pt,line join=round,line cap=round] (203.12, 48.00) -- (203.12, 57.65);

\path[draw=drawColor,line width= 0.2pt,line join=round,line cap=round] (204.09, 48.00) -- (204.09, 57.65);

\path[draw=drawColor,line width= 0.2pt,line join=round,line cap=round] (205.11, 48.00) -- (205.11, 57.65);

\path[draw=drawColor,line width= 0.2pt,line join=round,line cap=round] (205.75, 48.00) -- (205.75, 57.65);

\path[draw=drawColor,line width= 0.2pt,line join=round,line cap=round] (206.07, 48.00) -- (206.07, 57.65);

\path[draw=drawColor,line width= 0.2pt,line join=round,line cap=round] (206.36, 48.00) -- (206.36, 57.65);

\path[draw=drawColor,line width= 0.2pt,line join=round,line cap=round] (206.41, 48.00) -- (206.41, 57.65);

\path[draw=drawColor,line width= 0.2pt,line join=round,line cap=round] (206.42, 48.00) -- (206.42, 57.65);

\path[draw=drawColor,line width= 0.2pt,line join=round,line cap=round] (206.73, 48.00) -- (206.73, 57.65);

\path[draw=drawColor,line width= 0.2pt,line join=round,line cap=round] (207.35, 48.00) -- (207.35, 57.65);

\path[draw=drawColor,line width= 0.2pt,line join=round,line cap=round] (210.57, 48.00) -- (210.57, 57.65);

\path[draw=drawColor,line width= 0.2pt,line join=round,line cap=round] (210.60, 48.00) -- (210.60, 57.65);

\path[draw=drawColor,line width= 0.2pt,line join=round,line cap=round] (210.92, 48.00) -- (210.92, 57.65);

\path[draw=drawColor,line width= 0.2pt,line join=round,line cap=round] (212.66, 48.00) -- (212.66, 57.65);

\path[draw=drawColor,line width= 0.2pt,line join=round,line cap=round] (212.88, 48.00) -- (212.88, 57.65);

\path[draw=drawColor,line width= 0.2pt,line join=round,line cap=round] (214.08, 48.00) -- (214.08, 57.65);

\path[draw=drawColor,line width= 0.2pt,line join=round,line cap=round] (214.23, 48.00) -- (214.23, 57.65);

\path[draw=drawColor,line width= 0.2pt,line join=round,line cap=round] (215.10, 48.00) -- (215.10, 57.65);

\path[draw=drawColor,line width= 0.2pt,line join=round,line cap=round] (216.15, 48.00) -- (216.15, 57.65);

\path[draw=drawColor,line width= 0.2pt,line join=round,line cap=round] (216.32, 48.00) -- (216.32, 57.65);

\path[draw=drawColor,line width= 0.2pt,line join=round,line cap=round] (216.62, 48.00) -- (216.62, 57.65);

\path[draw=drawColor,line width= 0.2pt,line join=round,line cap=round] (220.19, 48.00) -- (220.19, 57.65);

\path[draw=drawColor,line width= 0.2pt,line join=round,line cap=round] (220.94, 48.00) -- (220.94, 57.65);

\path[draw=drawColor,line width= 0.2pt,line join=round,line cap=round] (221.07, 48.00) -- (221.07, 57.65);

\path[draw=drawColor,line width= 0.2pt,line join=round,line cap=round] (227.02, 48.00) -- (227.02, 57.65);

\path[draw=drawColor,line width= 0.2pt,line join=round,line cap=round] (227.20, 48.00) -- (227.20, 57.65);

\path[draw=drawColor,line width= 0.2pt,line join=round,line cap=round] (227.32, 48.00) -- (227.32, 57.65);

\path[draw=drawColor,line width= 0.2pt,line join=round,line cap=round] (227.54, 48.00) -- (227.54, 57.65);

\path[draw=drawColor,line width= 0.2pt,line join=round,line cap=round] (228.23, 48.00) -- (228.23, 57.65);

\path[draw=drawColor,line width= 0.2pt,line join=round,line cap=round] (228.25, 48.00) -- (228.25, 57.65);

\path[draw=drawColor,line width= 0.2pt,line join=round,line cap=round] (228.31, 48.00) -- (228.31, 57.65);

\path[draw=drawColor,line width= 0.2pt,line join=round,line cap=round] (230.39, 48.00) -- (230.39, 57.65);

\path[draw=drawColor,line width= 0.2pt,line join=round,line cap=round] (231.66, 48.00) -- (231.66, 57.65);

\path[draw=drawColor,line width= 0.2pt,line join=round,line cap=round] (232.54, 48.00) -- (232.54, 57.65);

\path[draw=drawColor,line width= 0.2pt,line join=round,line cap=round] (232.82, 48.00) -- (232.82, 57.65);

\path[draw=drawColor,line width= 0.2pt,line join=round,line cap=round] (235.35, 48.00) -- (235.35, 57.65);

\path[draw=drawColor,line width= 0.2pt,line join=round,line cap=round] (238.67, 48.00) -- (238.67, 57.65);

\path[draw=drawColor,line width= 0.2pt,line join=round,line cap=round] (241.30, 48.00) -- (241.30, 57.65);

\path[draw=drawColor,line width= 0.2pt,line join=round,line cap=round] (247.23, 48.00) -- (247.23, 57.65);

\path[draw=drawColor,line width= 0.2pt,line join=round,line cap=round] (247.26, 48.00) -- (247.26, 57.65);

\path[draw=drawColor,line width= 0.2pt,line join=round,line cap=round] (248.06, 48.00) -- (248.06, 57.65);

\path[draw=drawColor,line width= 0.2pt,line join=round,line cap=round] (248.85, 48.00) -- (248.85, 57.65);

\path[draw=drawColor,line width= 0.2pt,line join=round,line cap=round] (249.68, 48.00) -- (249.68, 57.65);

\path[draw=drawColor,line width= 0.2pt,line join=round,line cap=round] (252.55, 48.00) -- (252.55, 57.65);

\path[draw=drawColor,line width= 0.2pt,line join=round,line cap=round] (253.37, 48.00) -- (253.37, 57.65);

\path[draw=drawColor,line width= 0.2pt,line join=round,line cap=round] (253.78, 48.00) -- (253.78, 57.65);

\path[draw=drawColor,line width= 0.2pt,line join=round,line cap=round] (255.08, 48.00) -- (255.08, 57.65);

\path[draw=drawColor,line width= 0.2pt,line join=round,line cap=round] (260.44, 48.00) -- (260.44, 57.65);

\path[draw=drawColor,line width= 0.2pt,line join=round,line cap=round] (264.07, 48.00) -- (264.07, 57.65);

\path[draw=drawColor,line width= 0.2pt,line join=round,line cap=round] (266.35, 48.00) -- (266.35, 57.65);

\path[draw=drawColor,line width= 0.2pt,line join=round,line cap=round] (271.82, 48.00) -- (271.82, 57.65);

\path[draw=drawColor,line width= 0.2pt,line join=round,line cap=round] (272.37, 48.00) -- (272.37, 57.65);

\path[draw=drawColor,line width= 0.2pt,line join=round,line cap=round] (286.02, 48.00) -- (286.02, 57.65);

\path[draw=drawColor,line width= 0.2pt,line join=round,line cap=round] (291.18, 48.00) -- (291.18, 57.65);

\path[draw=drawColor,line width= 0.2pt,line join=round,line cap=round] (298.03, 48.00) -- (298.03, 57.65);
\end{scope}
\end{tikzpicture}

	\end{adjustbox}
	\caption[Dichtefunktion des \gls{zwert}s aus dem \gls{bist}]{Dichtefunktion des \gls{bist} \gls{zwert}s. Der \gls{zwert} wurde als Mittelwert aller 18 \textit{z}-standardisierter Subtests gebildet. Alle Datenpunkte sind auf der x-Achse mit vertikalen Strichen markiert.}
	\label{fig:bis_zscore_density}
\end{figure}

\begin{sidewaystable}
	\captionsetup{labelsep = none}
	\caption[Produkt-Moment-Korrelationen zwischen den Subtests des \gls{bist}s]{\newline  \textit{Produkt-Moment-Korrelationen zwischen den Subtests des \gls{bist}s} \vspace{.2cm}}
	\label{tab:bis_product_correlations}
	\sisetup{table-space-text-post = $^{*ab}$}
	\begin{adjustbox}{width=1\textwidth,totalheight=1\textheight,keepaspectratio}
		\begin{threeparttable}
			\begin{tabular}{
				l
				l
				S[table-format = 0.2, add-integer-zero=false]
				S[table-format = 0.2, add-integer-zero=false]
				S[table-format = 0.2, add-integer-zero=false]
				S[table-format = 0.2, add-integer-zero=false]
				S[table-format = 0.2, add-integer-zero=false]
				S[table-format = 0.2, add-integer-zero=false]
				S[table-format = 0.2, add-integer-zero=false]
				S[table-format = 0.2, add-integer-zero=false]
				S[table-format = 0.2, add-integer-zero=false]
				S[table-format = 0.2, add-integer-zero=false]
				S[table-format = 0.2, add-integer-zero=false]
				S[table-format = 0.2, add-integer-zero=false]
				S[table-format = 0.2, add-integer-zero=false]
				S[table-format = 0.2, add-integer-zero=false]
				S[table-format = 0.2, add-integer-zero=false]
				S[table-format = 0.2, add-integer-zero=false]
				S[table-format = 0.2, add-integer-zero=false]
				>{\centering\arraybackslash}p{1.2cm}
			}
			\hline
			&	\multicolumn{1}{c}{Subtest}	&	{1}	&	{2}	&	{3}	&	{4}	&	{5}	&	{6}	&	{7}	&	{8}	&	{9}	&	{10}&	{11}&	{12}&	{13}&	{14}&	{15}&	{16}&	{17}	\\
			\hline
			
1	&	OG	&	&	&	&	&	&	&	&	&	&	&	&	&	&	&	&	&	\\
2	&	ZN	&	.27{$^{***}$}	&	&	&	&	&	&	&	&	&	&	&	&	&	&	&	&	\\
3	&	AN	&	.31{$^{***}$}	&	.42{$^{***}$}	&	&	&	&	&	&	&	&	&	&	&	&	&	&	&	\\
4	&	XG	&	.21{$^{**}$}	&	.56{$^{***}$}	&	.32{$^{***}$}	&	&	&	&	&	&	&	&	&	&	&	&	&	&	\\
5	&	WA	&	.34{$^{***}$}	&	.41{$^{***}$}	&	.49{$^{***}$}	&	.34{$^{***}$}	&	&	&	&	&	&	&	&	&	&	&	&	&	\\
6	&	ZP	&	.22{$^{**}$}	&	.17{$^{*}$}		&	.13				&	.31{$^{***}$}	&	.17{$^{*}$}		&	&	&	&	&	&	&	&	&	&	&	&	\\
7	&	TM	&	.30{$^{***}$}	&	.26{$^{***}$}	&	.44{$^{***}$}	&	.33{$^{***}$}	&	.51{$^{***}$}	&	.22{$^{**}$}	&	&	&	&	&	&	&	&	&	&	&	\\
8	&	BD	&	.07				&	.08				&	.05				&	.11				&	-.01			&	.04				&	.03	&	&	&	&	&	&	&	&	&	&	\\
9	&	SC	&	.14				&	.52{$^{***}$}	&	.35{$^{***}$}	&	.47{$^{***}$}	&	.23{$^{**}$}	&	.17{$^{*}$}		&	.32{$^{***}$}	&	.20{$^{**}$}	&	&	&&&	&&	&&\\
10	&	ST	&	.38{$^{***}$}	&	.19	{$^{*}$}	&	.24{$^{**}$}	&	.31{$^{***}$}	&	.32{$^{***}$}	&	.24{$^{**}$}	&	.39{$^{***}$}	&	-.01			&	.22{$^{**}$}	&		&&&&&&\\
11	&	CH	&	.36{$^{***}$}	&	.51{$^{***}$}	&	.52{$^{***}$}	&	.31{$^{***}$}	&	.52{$^{***}$}	&	.13				&	.33{$^{***}$}	&	.07				&	.31{$^{***}$}&	.17{$^{*}$}	&	&&&	&&&	\\
12	&	TG	&	.32{$^{***}$}	&	.33{$^{***}$}	&	.27{$^{***}$}	&	.43{$^{***}$}	&	.43{$^{***}$}	&	.16{$^{*}$}		&	.43{$^{***}$}	&	.11				&	.28{$^{***}$}&	.38{$^{***}$}	&	.22{$^{**}$}	&&	&	&&&	\\
13	&	RZ	&	.30{$^{***}$}	&	.53{$^{***}$}	&	.41{$^{***}$}	&	.55{$^{***}$}	&	.43{$^{***}$}	&	.27{$^{***}$}	&	.42{$^{***}$}	&	.08				&	.44{$^{***}$}&	.34{$^{***}$}	&	.38{$^{***}$}&	.33{$^{***}$}&&&&&	\\
14	&	WM	&	.40{$^{***}$}	&	.12				&	.29{$^{***}$}	&	.17{$^{*}$}		&	.26{$^{**}$}	&	.27{$^{***}$}	&	.39{$^{***}$}	&	.08				&	.10			&	.40{$^{***}$}	&	.18{$^{*}$}	&	.18{$^{*}$}		&	.13				&&&&	\\
15	&	KW	&	.26{$^{***}$}	&	.23{$^{**}$}	&	.28{$^{***}$}	&	.35{$^{***}$}	&	.40{$^{***}$}	&	.23{$^{**}$}	&	.56{$^{***}$}	&	.14				&	.29{$^{***}$}&	.52{$^{***}$}	&	.21{$^{**}$}&	.54{$^{***}$}	&	.36{$^{***}$}	&	.32{$^{***}$}	&&&	\\
16	&	ZZ	&	.29{$^{***}$}	&	.05				&	.04				&	.21{$^{**}$}	&	.01				&	.37{$^{***}$}	&	.10				&	.09				&	.05			&	.30{$^{***}$}&	.07				&	.08				&	.09				&	.39{$^{***}$}	&	.13				&					&	\\
17	&	OE	&	.09				&	.04				&	.03				&	.08				&	.01				&	.03				&	.13				&	.34{$^{***}$}	&	.16{$^{*}$}&	.03			&	-.06			&	.15{$^{*}$}		&	.15{$^{*}$}		&	.02				&	.16{$^{*}$}		& -.03				&	\\
18	&	WE	&	.39{$^{***}$}	&	.31{$^{***}$}	&	.27{$^{***}$}	&	.22{$^{**}$}		&	.28{$^{***}$}	&	.28{$^{***}$}	&	.09			&	-.02			&	.16{$^{*}$}	&	.23{$^{**}$}&	.27{$^{***}$}	&	.20{$^{**}$}	&	.34{$^{***}$}	&	.16{$^{*}$}		&	.22{$^{**}$}	&.19{$^{*}$}		&	-.10\\
			\hline			
			\end{tabular}
		
			\begin{tablenotes}[flushleft]
				\footnotesize				% font size
				\setlength\labelsep{0pt}	% no indent on second line
				\item \textit{Anmerkung.} Siehe \autoref{tab:bis_subtest_description} für eine Beschreibung der Subtests.
				\item {$^{*}$}$p<.05$. {$^{**}$}$p<.01$. {$^{***}$}$p<.001$ (zweiseitig).
			\end{tablenotes}
		\end{threeparttable}
	\end{adjustbox}
\end{sidewaystable}

\clearpage
\subsection{Zusammenhänge zwischen den Aufgaben \label{subsec:Zusammenhänge}}

Bevor ausgewählte Zusammenhänge zwischen den Aufgaben in den folgenden Abschnitten mit den Fragestellungen abgearbeitet werden, ist der Vollständigkeit halber in \autoref{tab:product_moment_correlations_manifest} eine Korrelationsmatrix abgebildet. 

Abgesehen von den bereits erwähnten Zusammenhängen zwischen den Bedingungen der \gls{ssauf}, der \gls{ha} respektive den Subtests des \gls{bist} ist an dieser Stelle gesondert auf Folgendes hinzuweisen:
Der \gls{si} wies eine negative Korrelation mit der $1.8^{\circ}$-Be\-ding\-ung auf ($r=-.28$, $p<.001$) und korrelierte positiv mit der $7.2^{\circ}$-Bedingung ($r=.66$, $p<.001$). 
Diese Zusammenhänge können als Hinweis dafür gesehen werden, dass der \gls{si} als Differenz zwischen der $82\,\%$-$\log_{10}$-Er\-ken\-nungs\-schwel\-le für die Mustergrösse $7.2^{\circ}$ und der $82\,\%$-$\log_{10}$-Er\-ken\-nungs\-schwel\-le für die Mustergrösse $1.8^{\circ}$ korrekt gebildet wurde.

Weiter korrelierte einzig die 0-bit-Bedingung der \gls{ha} signifikant mit der $1.8^{\circ}$-, der $3.6^{\circ}$- und der $5.4^{\circ}$-Bedingung der \gls{ssauf}, während alle restlichen Zusammenhänge zwischen den Bedingungen der beiden Aufgaben so gering ausfielen, dass sie bei der gewählten Irrtumswahrscheinlichkeit von weniger als $5\,\%$ nicht von 0 unterschieden werden konnten.

Die Bedingungen der \gls{ha} korrelierten erwartungsgemäss signifikant negativ mit psychometrischer Intelligenz \citep[$r=-.19$ bis $-.28$, alle $p\textnormal{s}<.05$; vgl. ][]{Sheppard2008}.

Die Bedingungen der \gls{ssauf} korrelierten mit Ausnahme des Zusammenhangs zwischen der $7.2^{\circ}$-Bedingung und dem \gls{zwert} des \gls{bist}s ($r=-.12$, $p=.10$) alle signifikant negativ mit psychometrischer Intelligenz ($r=-.16$ bis $-.19$, alle $p\textnormal{s}<.05$).

\begin{sidewaystable}
	\centering
	\captionsetup{labelsep = none}
	\caption[Produkt-Moment-Korrelationen zwischen der \gls{ssauf}, dem \gls{si}, der \gls{ha}, dem \textit{z}-Wert und dem \gls{gfaktor} des \gls{bist}s]{\newline  \textit{Produkt-Moment-Korrelationen zwischen den Bedingungen der \gls{ssauf}, dem \gls{si}, den Bedingungen der \gls{ha}, dem \textit{z}-Wert und dem \gls{gfaktor} des \gls{bist}s} \vspace{.2cm}}
	\label{tab:product_moment_correlations_manifest}
	\sisetup{table-space-text-post = $^{***}$}
%	\begin{adjustbox}{width=.90\textwidth, keepaspectratio}
	\begin{threeparttable}
		\begin{tabular}{
				l
				l
				S[table-format = 0.2, add-integer-zero=false]
				S[table-format = 0.2, add-integer-zero=false]
				S[table-format = 0.2, add-integer-zero=false]
				S[table-format = 0.2, add-integer-zero=false]
				S[table-format = 0.2, add-integer-zero=false]
				p{.001cm}
				S[table-format = 0.2, add-integer-zero=false]
				S[table-format = 0.2, add-integer-zero=false]
				S[table-format = 0.2, add-integer-zero=false]
				S[table-format = 0.2, add-integer-zero=false]
				p{.001cm}
				S[table-format = 0.2, add-integer-zero=false]
				c
				>{\centering\arraybackslash}p{1.2cm}
			}
			\hline
			
			
			&	& 	\multicolumn{5}{c}{\gls{ssauf}}	&	&	\multicolumn{4}{c}{\gls{ha}}	&	&	\multicolumn{2}{c}{\gls{bist}}	\\
			
			\cline{3-7}
			\cline{9-12}
			\cline{14-15}
			
	&	\multicolumn{1}{c}{Mass}			&	{1}				&	{2}				&	{3}				&	{4}			&	{5}			&	& {6}	& {7}	& {8}	&{9}&&{10}&{11} \\
\hline
1	&	$1.8^{\circ}$	&					&					&					&					&				&	& & & & &\\
2	&	$3.6^{\circ}$	&	.85{$^{***}$}	&					&					&					&				&	& & & & &\\
3	&	$5.4^{\circ}$	&	.73{$^{***}$}	&	.87{$^{***}$}	&					&					&				&	& & & & &\\
4	&	$7.2^{\circ}$	&	.54{$^{***}$}	&	.72{$^{***}$}	&	.87{$^{***}$}	&					&				&	& & & & &\\
5	&	SI 				&	-.28{$^{***}$}	&	.05				&	.34{$^{***}$}	&	.66{$^{***}$}	&				&	& & & & &\\
\rule{0pt}{4ex}%  EXTRA vertical height
6	&	0-bit			&	.17{$^{*}$}		&	.24{$^{**}$}	&	.25{$^{***}$}	&	.14				&	.01			&	& & & & &\\
7	&	1-bit			&	.09				&	.11				&	.13				&	.07				&	.00			&	&.76{$^{***}$}	&	&	&	&	&		\\
8	&	2-bit			&	.12				&	.08				&	.08				&	.04				&	-.06		&	&.58{$^{***}$}	&	.72{$^{***}$}	&	&	&	&		\\
9	&	2.58-bit		&	.14				&	.09				&	.12				&	.07				&	-.04		&	&.52{$^{***}$}	&	.66{$^{***}$}	&	.83{$^{***}$}	&	&	&		\\
\rule{0pt}{4ex}%  EXTRA vertical height
10	&	\textit{z}-Wert	&	-.16{$^{*}$}	&	-.17{$^{*}$}	&	-.16{$^{*}$}	&	-.12			&	.00			&	&-.19{$^{*}$}	&	-.27{$^{***}$}	&	-.28{$^{***}$}	&	-.28{$^{***}$}	&				&		\\
11	&	\gls{gfaktor}	&	-.18{$^{*}$}	&	-.19{$^{*}$}	&	-.19{$^{*}$}	&	-.16{$^{*}$}	&	-.01		&	&-.20{$^{**}$}	&	-.28{$^{***}$}	&	-.28{$^{***}$}	&	-.27{$^{***}$}	&				&	.98{$^{***}$}	& \hphantom{.10000}			\\	
			
			\hline
			
		\end{tabular}%
		%}
		\begin{tablenotes}[flushleft]
			\footnotesize				% font size
			\setlength\labelsep{0pt}	% no indent on second line
			\item \textit{Anmerkungen}. SI = \gls{si}; \gls{zwert} = Mittelwert aller 18 \textit{z}-standardisierten Subtests.
			\item {$^{*}$}$p<.05$. {$^{**}$}$p<.01$. {$^{***}$}$p<.001$ (zweiseitig).
		\end{tablenotes}
	\end{threeparttable}
	%\end{adjustbox}
\end{sidewaystable}



\clearpage
\section{1. Fragestellung}

Mit der ersten Fragestellung sollte geprüft werden, ob die von \citet{Melnick2013} berichteten Zusammenhänge zwischen der \gls{ssauf} und psychometrischer Intelligenz bestätigt werden können. 

Der von \citet{Melnick2013} berichtete Zusammenhang zwischen dem \gls{si} und IQ-Punkten (Studie 1 [$N=12$]: $r~=~.64$, $p~=~.02$ und Studie 2 [$N=53$]: $r~=~.71$, $p~<~.001$) konnte in der vorliegenden Arbeit nicht bestätigt werden: Der Zusammenhang zwischen dem \gls{si} und dem \gls{zwert} aus dem \gls{bist} betrug $r~=~.00$ ($p~=~.98$; siehe \autoref{fig:suppression_index_zscore_scatterplot}). 
Um zu prüfen, ob dieser Zusammenhang signifikant tiefer ausfiel als bei \citeauthor{Melnick2013}, wurden die Korrelationskoeffizienten in Fisher-\textit{Z}-Werte umgerechnet und  auf Unterschiedlichkeit geprüft \citep[][S. 54]{Cohen1983}. 
Dabei hat sich ergeben, dass sich der in der vorliegenden Arbeit ermittelte Korrelationskoeffizient $r~=~.00$ signifikant von den von \citet{Melnick2013} berichteten $r~=~.64$ und $r~=~.71$ unterschied ($z=2.22$, $p=.03$ respektive $z=5.53$, $p<.001$).

\begin{figure}[t]
	\centering
	\begin{adjustbox}{width=1\textwidth}
		% Created by tikzDevice version 0.10.1 on 2016-07-13 10:11:52
% !TEX encoding = UTF-8 Unicode
\begin{tikzpicture}[x=1pt,y=1pt]
\definecolor{fillColor}{RGB}{255,255,255}
\path[use as bounding box,fill=fillColor,fill opacity=0.00] (0,0) rectangle (505.89,505.89);
\begin{scope}
\path[clip] ( 48.60, 49.20) rectangle (505.29,505.89);
\definecolor{drawColor}{RGB}{0,0,0}
\definecolor{fillColor}{RGB}{0,0,0}

\path[draw=drawColor,line width= 0.4pt,line join=round,line cap=round,fill=fillColor] (246.99,210.69) circle (  1.50);

\path[draw=drawColor,line width= 0.4pt,line join=round,line cap=round,fill=fillColor] (338.26,337.74) circle (  1.50);

\path[draw=drawColor,line width= 0.4pt,line join=round,line cap=round,fill=fillColor] (215.28,308.42) circle (  1.50);

\path[draw=drawColor,line width= 0.4pt,line join=round,line cap=round,fill=fillColor] (213.52,303.01) circle (  1.50);

\path[draw=drawColor,line width= 0.4pt,line join=round,line cap=round,fill=fillColor] (192.73,229.83) circle (  1.50);

\path[draw=drawColor,line width= 0.4pt,line join=round,line cap=round,fill=fillColor] (219.51,328.60) circle (  1.50);

\path[draw=drawColor,line width= 0.4pt,line join=round,line cap=round,fill=fillColor] (240.65,311.25) circle (  1.50);

\path[draw=drawColor,line width= 0.4pt,line join=round,line cap=round,fill=fillColor] (189.91,363.71) circle (  1.50);

\path[draw=drawColor,line width= 0.4pt,line join=round,line cap=round,fill=fillColor] (147.97,265.36) circle (  1.50);

\path[draw=drawColor,line width= 0.4pt,line join=round,line cap=round,fill=fillColor] (283.99,240.38) circle (  1.50);

\path[draw=drawColor,line width= 0.4pt,line join=round,line cap=round,fill=fillColor] (313.24,240.61) circle (  1.50);

\path[draw=drawColor,line width= 0.4pt,line join=round,line cap=round,fill=fillColor] (183.56,362.71) circle (  1.50);

\path[draw=drawColor,line width= 0.4pt,line join=round,line cap=round,fill=fillColor] (238.89,289.05) circle (  1.50);

\path[draw=drawColor,line width= 0.4pt,line join=round,line cap=round,fill=fillColor] (268.84,293.23) circle (  1.50);

\path[draw=drawColor,line width= 0.4pt,line join=round,line cap=round,fill=fillColor] (270.95,273.67) circle (  1.50);

\path[draw=drawColor,line width= 0.4pt,line join=round,line cap=round,fill=fillColor] (227.61,322.59) circle (  1.50);

\path[draw=drawColor,line width= 0.4pt,line join=round,line cap=round,fill=fillColor] (212.81,361.66) circle (  1.50);

\path[draw=drawColor,line width= 0.4pt,line join=round,line cap=round,fill=fillColor] (176.87,276.62) circle (  1.50);

\path[draw=drawColor,line width= 0.4pt,line join=round,line cap=round,fill=fillColor] (130.35,266.31) circle (  1.50);

\path[draw=drawColor,line width= 0.4pt,line join=round,line cap=round,fill=fillColor] (213.16,425.87) circle (  1.50);

\path[draw=drawColor,line width= 0.4pt,line join=round,line cap=round,fill=fillColor] (312.18,269.65) circle (  1.50);

\path[draw=drawColor,line width= 0.4pt,line join=round,line cap=round,fill=fillColor] (171.93,260.98) circle (  1.50);

\path[draw=drawColor,line width= 0.4pt,line join=round,line cap=round,fill=fillColor] (196.60,336.77) circle (  1.50);

\path[draw=drawColor,line width= 0.4pt,line join=round,line cap=round,fill=fillColor] (269.19,309.64) circle (  1.50);

\path[draw=drawColor,line width= 0.4pt,line join=round,line cap=round,fill=fillColor] (183.56,393.43) circle (  1.50);

\path[draw=drawColor,line width= 0.4pt,line join=round,line cap=round,fill=fillColor] (203.30,280.02) circle (  1.50);

\path[draw=drawColor,line width= 0.4pt,line join=round,line cap=round,fill=fillColor] (184.97,392.74) circle (  1.50);

\path[draw=drawColor,line width= 0.4pt,line join=round,line cap=round,fill=fillColor] (187.09,300.15) circle (  1.50);

\path[draw=drawColor,line width= 0.4pt,line join=round,line cap=round,fill=fillColor] (212.46,289.57) circle (  1.50);

\path[draw=drawColor,line width= 0.4pt,line join=round,line cap=round,fill=fillColor] (204.71,328.43) circle (  1.50);

\path[draw=drawColor,line width= 0.4pt,line join=round,line cap=round,fill=fillColor] (249.46,299.40) circle (  1.50);

\path[draw=drawColor,line width= 0.4pt,line join=round,line cap=round,fill=fillColor] ( 70.80,205.94) circle (  1.50);

\path[draw=drawColor,line width= 0.4pt,line join=round,line cap=round,fill=fillColor] (187.44,361.70) circle (  1.50);

\path[draw=drawColor,line width= 0.4pt,line join=round,line cap=round,fill=fillColor] (205.06,275.48) circle (  1.50);

\path[draw=drawColor,line width= 0.4pt,line join=round,line cap=round,fill=fillColor] (272.01,350.84) circle (  1.50);

\path[draw=drawColor,line width= 0.4pt,line join=round,line cap=round,fill=fillColor] (203.65,368.38) circle (  1.50);

\path[draw=drawColor,line width= 0.4pt,line join=round,line cap=round,fill=fillColor] (159.25,303.47) circle (  1.50);

\path[draw=drawColor,line width= 0.4pt,line join=round,line cap=round,fill=fillColor] (164.89,336.12) circle (  1.50);

\path[draw=drawColor,line width= 0.4pt,line join=round,line cap=round,fill=fillColor] (186.03,315.32) circle (  1.50);

\path[draw=drawColor,line width= 0.4pt,line join=round,line cap=round,fill=fillColor] (113.79,321.06) circle (  1.50);

\path[draw=drawColor,line width= 0.4pt,line join=round,line cap=round,fill=fillColor] (189.91,309.24) circle (  1.50);

\path[draw=drawColor,line width= 0.4pt,line join=round,line cap=round,fill=fillColor] (299.50,343.44) circle (  1.50);

\path[draw=drawColor,line width= 0.4pt,line join=round,line cap=round,fill=fillColor] (165.94,346.64) circle (  1.50);

\path[draw=drawColor,line width= 0.4pt,line join=round,line cap=round,fill=fillColor] ( 97.23,341.98) circle (  1.50);

\path[draw=drawColor,line width= 0.4pt,line join=round,line cap=round,fill=fillColor] (274.83,369.42) circle (  1.50);

\path[draw=drawColor,line width= 0.4pt,line join=round,line cap=round,fill=fillColor] (213.16,305.41) circle (  1.50);

\path[draw=drawColor,line width= 0.4pt,line join=round,line cap=round,fill=fillColor] (219.86,206.75) circle (  1.50);

\path[draw=drawColor,line width= 0.4pt,line join=round,line cap=round,fill=fillColor] (235.72,174.42) circle (  1.50);

\path[draw=drawColor,line width= 0.4pt,line join=round,line cap=round,fill=fillColor] (345.31,207.39) circle (  1.50);

\path[draw=drawColor,line width= 0.4pt,line join=round,line cap=round,fill=fillColor] (212.11,204.59) circle (  1.50);

\path[draw=drawColor,line width= 0.4pt,line join=round,line cap=round,fill=fillColor] (242.41,251.41) circle (  1.50);

\path[draw=drawColor,line width= 0.4pt,line join=round,line cap=round,fill=fillColor] (176.87,256.97) circle (  1.50);

\path[draw=drawColor,line width= 0.4pt,line join=round,line cap=round,fill=fillColor] (173.70,175.21) circle (  1.50);

\path[draw=drawColor,line width= 0.4pt,line join=round,line cap=round,fill=fillColor] (239.59,216.17) circle (  1.50);

\path[draw=drawColor,line width= 0.4pt,line join=round,line cap=round,fill=fillColor] (247.34,322.97) circle (  1.50);

\path[draw=drawColor,line width= 0.4pt,line join=round,line cap=round,fill=fillColor] (208.58,172.54) circle (  1.50);

\path[draw=drawColor,line width= 0.4pt,line join=round,line cap=round,fill=fillColor] (165.94,281.35) circle (  1.50);

\path[draw=drawColor,line width= 0.4pt,line join=round,line cap=round,fill=fillColor] (254.39,284.42) circle (  1.50);

\path[draw=drawColor,line width= 0.4pt,line join=round,line cap=round,fill=fillColor] (241.71,217.48) circle (  1.50);

\path[draw=drawColor,line width= 0.4pt,line join=round,line cap=round,fill=fillColor] (322.40,295.70) circle (  1.50);

\path[draw=drawColor,line width= 0.4pt,line join=round,line cap=round,fill=fillColor] (252.28,252.42) circle (  1.50);

\path[draw=drawColor,line width= 0.4pt,line join=round,line cap=round,fill=fillColor] (200.83,296.97) circle (  1.50);

\path[draw=drawColor,line width= 0.4pt,line join=round,line cap=round,fill=fillColor] (169.82,279.59) circle (  1.50);

\path[draw=drawColor,line width= 0.4pt,line join=round,line cap=round,fill=fillColor] (124.71,337.67) circle (  1.50);

\path[draw=drawColor,line width= 0.4pt,line join=round,line cap=round,fill=fillColor] (208.93,300.92) circle (  1.50);

\path[draw=drawColor,line width= 0.4pt,line join=round,line cap=round,fill=fillColor] (181.45,273.08) circle (  1.50);

\path[draw=drawColor,line width= 0.4pt,line join=round,line cap=round,fill=fillColor] (182.15,125.32) circle (  1.50);

\path[draw=drawColor,line width= 0.4pt,line join=round,line cap=round,fill=fillColor] (200.48,228.65) circle (  1.50);

\path[draw=drawColor,line width= 0.4pt,line join=round,line cap=round,fill=fillColor] (157.84,253.13) circle (  1.50);

\path[draw=drawColor,line width= 0.4pt,line join=round,line cap=round,fill=fillColor] (194.13,310.07) circle (  1.50);

\path[draw=drawColor,line width= 0.4pt,line join=round,line cap=round,fill=fillColor] (223.73,317.97) circle (  1.50);

\path[draw=drawColor,line width= 0.4pt,line join=round,line cap=round,fill=fillColor] (199.77,280.63) circle (  1.50);

\path[draw=drawColor,line width= 0.4pt,line join=round,line cap=round,fill=fillColor] (163.12,301.30) circle (  1.50);

\path[draw=drawColor,line width= 0.4pt,line join=round,line cap=round,fill=fillColor] (218.10,343.09) circle (  1.50);

\path[draw=drawColor,line width= 0.4pt,line join=round,line cap=round,fill=fillColor] (231.49,369.94) circle (  1.50);

\path[draw=drawColor,line width= 0.4pt,line join=round,line cap=round,fill=fillColor] (292.10,272.72) circle (  1.50);

\path[draw=drawColor,line width= 0.4pt,line join=round,line cap=round,fill=fillColor] (149.38,249.98) circle (  1.50);

\path[draw=drawColor,line width= 0.4pt,line join=round,line cap=round,fill=fillColor] (187.79,382.95) circle (  1.50);

\path[draw=drawColor,line width= 0.4pt,line join=round,line cap=round,fill=fillColor] (179.69,286.20) circle (  1.50);

\path[draw=drawColor,line width= 0.4pt,line join=round,line cap=round,fill=fillColor] (202.24,232.58) circle (  1.50);

\path[draw=drawColor,line width= 0.4pt,line join=round,line cap=round,fill=fillColor] (182.15,255.12) circle (  1.50);

\path[draw=drawColor,line width= 0.4pt,line join=round,line cap=round,fill=fillColor] (190.26,257.67) circle (  1.50);

\path[draw=drawColor,line width= 0.4pt,line join=round,line cap=round,fill=fillColor] (190.61,212.20) circle (  1.50);

\path[draw=drawColor,line width= 0.4pt,line join=round,line cap=round,fill=fillColor] (251.93,249.25) circle (  1.50);

\path[draw=drawColor,line width= 0.4pt,line join=round,line cap=round,fill=fillColor] (122.60,252.30) circle (  1.50);

\path[draw=drawColor,line width= 0.4pt,line join=round,line cap=round,fill=fillColor] (187.44,307.14) circle (  1.50);

\path[draw=drawColor,line width= 0.4pt,line join=round,line cap=round,fill=fillColor] (207.53,322.37) circle (  1.50);

\path[draw=drawColor,line width= 0.4pt,line join=round,line cap=round,fill=fillColor] (181.10,258.40) circle (  1.50);

\path[draw=drawColor,line width= 0.4pt,line join=round,line cap=round,fill=fillColor] (368.92,417.21) circle (  1.50);

\path[draw=drawColor,line width= 0.4pt,line join=round,line cap=round,fill=fillColor] (193.43,259.50) circle (  1.50);

\path[draw=drawColor,line width= 0.4pt,line join=round,line cap=round,fill=fillColor] (159.25,280.17) circle (  1.50);

\path[draw=drawColor,line width= 0.4pt,line join=round,line cap=round,fill=fillColor] (176.16,310.09) circle (  1.50);

\path[draw=drawColor,line width= 0.4pt,line join=round,line cap=round,fill=fillColor] (308.66,283.77) circle (  1.50);

\path[draw=drawColor,line width= 0.4pt,line join=round,line cap=round,fill=fillColor] (199.77,410.68) circle (  1.50);

\path[draw=drawColor,line width= 0.4pt,line join=round,line cap=round,fill=fillColor] (219.51,275.95) circle (  1.50);

\path[draw=drawColor,line width= 0.4pt,line join=round,line cap=round,fill=fillColor] (281.17,239.72) circle (  1.50);

\path[draw=drawColor,line width= 0.4pt,line join=round,line cap=round,fill=fillColor] (121.90,251.37) circle (  1.50);

\path[draw=drawColor,line width= 0.4pt,line join=round,line cap=round,fill=fillColor] (167.00,272.44) circle (  1.50);

\path[draw=drawColor,line width= 0.4pt,line join=round,line cap=round,fill=fillColor] (170.88,277.61) circle (  1.50);

\path[draw=drawColor,line width= 0.4pt,line join=round,line cap=round,fill=fillColor] (310.07,224.78) circle (  1.50);

\path[draw=drawColor,line width= 0.4pt,line join=round,line cap=round,fill=fillColor] (249.11,205.38) circle (  1.50);

\path[draw=drawColor,line width= 0.4pt,line join=round,line cap=round,fill=fillColor] (261.09,315.37) circle (  1.50);

\path[draw=drawColor,line width= 0.4pt,line join=round,line cap=round,fill=fillColor] (184.97,208.99) circle (  1.50);

\path[draw=drawColor,line width= 0.4pt,line join=round,line cap=round,fill=fillColor] (219.51,364.76) circle (  1.50);

\path[draw=drawColor,line width= 0.4pt,line join=round,line cap=round,fill=fillColor] (269.90,282.87) circle (  1.50);

\path[draw=drawColor,line width= 0.4pt,line join=round,line cap=round,fill=fillColor] (231.14,298.49) circle (  1.50);

\path[draw=drawColor,line width= 0.4pt,line join=round,line cap=round,fill=fillColor] (249.46,258.38) circle (  1.50);

\path[draw=drawColor,line width= 0.4pt,line join=round,line cap=round,fill=fillColor] (169.47,282.30) circle (  1.50);

\path[draw=drawColor,line width= 0.4pt,line join=round,line cap=round,fill=fillColor] (132.47,336.35) circle (  1.50);

\path[draw=drawColor,line width= 0.4pt,line join=round,line cap=round,fill=fillColor] (176.87,253.03) circle (  1.50);

\path[draw=drawColor,line width= 0.4pt,line join=round,line cap=round,fill=fillColor] (204.00,260.44) circle (  1.50);

\path[draw=drawColor,line width= 0.4pt,line join=round,line cap=round,fill=fillColor] (157.49,181.83) circle (  1.50);

\path[draw=drawColor,line width= 0.4pt,line join=round,line cap=round,fill=fillColor] (200.83,191.79) circle (  1.50);

\path[draw=drawColor,line width= 0.4pt,line join=round,line cap=round,fill=fillColor] (276.59,252.55) circle (  1.50);

\path[draw=drawColor,line width= 0.4pt,line join=round,line cap=round,fill=fillColor] (202.94,226.44) circle (  1.50);

\path[draw=drawColor,line width= 0.4pt,line join=round,line cap=round,fill=fillColor] (186.03,281.93) circle (  1.50);

\path[draw=drawColor,line width= 0.4pt,line join=round,line cap=round,fill=fillColor] (275.54,318.25) circle (  1.50);

\path[draw=drawColor,line width= 0.4pt,line join=round,line cap=round,fill=fillColor] (178.28,265.29) circle (  1.50);

\path[draw=drawColor,line width= 0.4pt,line join=round,line cap=round,fill=fillColor] (140.92,305.91) circle (  1.50);

\path[draw=drawColor,line width= 0.4pt,line join=round,line cap=round,fill=fillColor] (207.17,327.49) circle (  1.50);

\path[draw=drawColor,line width= 0.4pt,line join=round,line cap=round,fill=fillColor] (213.87,258.99) circle (  1.50);

\path[draw=drawColor,line width= 0.4pt,line join=round,line cap=round,fill=fillColor] (234.31,175.60) circle (  1.50);

\path[draw=drawColor,line width= 0.4pt,line join=round,line cap=round,fill=fillColor] (204.00,268.25) circle (  1.50);

\path[draw=drawColor,line width= 0.4pt,line join=round,line cap=round,fill=fillColor] (266.02,159.45) circle (  1.50);

\path[draw=drawColor,line width= 0.4pt,line join=round,line cap=round,fill=fillColor] (187.09,203.99) circle (  1.50);

\path[draw=drawColor,line width= 0.4pt,line join=round,line cap=round,fill=fillColor] (137.40,247.53) circle (  1.50);

\path[draw=drawColor,line width= 0.4pt,line join=round,line cap=round,fill=fillColor] (207.17,296.50) circle (  1.50);

\path[draw=drawColor,line width= 0.4pt,line join=round,line cap=round,fill=fillColor] (148.32,204.39) circle (  1.50);

\path[draw=drawColor,line width= 0.4pt,line join=round,line cap=round,fill=fillColor] (236.07,178.75) circle (  1.50);

\path[draw=drawColor,line width= 0.4pt,line join=round,line cap=round,fill=fillColor] (155.37,251.61) circle (  1.50);

\path[draw=drawColor,line width= 0.4pt,line join=round,line cap=round,fill=fillColor] (143.74,215.86) circle (  1.50);

\path[draw=drawColor,line width= 0.4pt,line join=round,line cap=round,fill=fillColor] (160.66,238.71) circle (  1.50);

\path[draw=drawColor,line width= 0.4pt,line join=round,line cap=round,fill=fillColor] (191.67,254.26) circle (  1.50);

\path[draw=drawColor,line width= 0.4pt,line join=round,line cap=round,fill=fillColor] (220.21,237.83) circle (  1.50);

\path[draw=drawColor,line width= 0.4pt,line join=round,line cap=round,fill=fillColor] (201.53,299.68) circle (  1.50);

\path[draw=drawColor,line width= 0.4pt,line join=round,line cap=round,fill=fillColor] (153.61,240.90) circle (  1.50);

\path[draw=drawColor,line width= 0.4pt,line join=round,line cap=round,fill=fillColor] (314.65,273.11) circle (  1.50);

\path[draw=drawColor,line width= 0.4pt,line join=round,line cap=round,fill=fillColor] (216.33,290.29) circle (  1.50);

\path[draw=drawColor,line width= 0.4pt,line join=round,line cap=round,fill=fillColor] (181.45,254.52) circle (  1.50);

\path[draw=drawColor,line width= 0.4pt,line join=round,line cap=round,fill=fillColor] (202.59,337.65) circle (  1.50);

\path[draw=drawColor,line width= 0.4pt,line join=round,line cap=round,fill=fillColor] (213.16,315.78) circle (  1.50);

\path[draw=drawColor,line width= 0.4pt,line join=round,line cap=round,fill=fillColor] (183.56,336.50) circle (  1.50);

\path[draw=drawColor,line width= 0.4pt,line join=round,line cap=round,fill=fillColor] (230.43,108.58) circle (  1.50);

\path[draw=drawColor,line width= 0.4pt,line join=round,line cap=round,fill=fillColor] (151.50,279.46) circle (  1.50);

\path[draw=drawColor,line width= 0.4pt,line join=round,line cap=round,fill=fillColor] (262.85,286.67) circle (  1.50);

\path[draw=drawColor,line width= 0.4pt,line join=round,line cap=round,fill=fillColor] (204.00,297.10) circle (  1.50);

\path[draw=drawColor,line width= 0.4pt,line join=round,line cap=round,fill=fillColor] (184.27,304.32) circle (  1.50);

\path[draw=drawColor,line width= 0.4pt,line join=round,line cap=round,fill=fillColor] (244.88,340.38) circle (  1.50);

\path[draw=drawColor,line width= 0.4pt,line join=round,line cap=round,fill=fillColor] (305.49,259.38) circle (  1.50);

\path[draw=drawColor,line width= 0.4pt,line join=round,line cap=round,fill=fillColor] (237.83,278.14) circle (  1.50);

\path[draw=drawColor,line width= 0.4pt,line join=round,line cap=round,fill=fillColor] (145.86,251.38) circle (  1.50);

\path[draw=drawColor,line width= 0.4pt,line join=round,line cap=round,fill=fillColor] (171.93,216.35) circle (  1.50);

\path[draw=drawColor,line width= 0.4pt,line join=round,line cap=round,fill=fillColor] (278.71,354.16) circle (  1.50);

\path[draw=drawColor,line width= 0.4pt,line join=round,line cap=round,fill=fillColor] (284.35,240.02) circle (  1.50);

\path[draw=drawColor,line width= 0.4pt,line join=round,line cap=round,fill=fillColor] (224.09,287.50) circle (  1.50);

\path[draw=drawColor,line width= 0.4pt,line join=round,line cap=round,fill=fillColor] (197.66,319.95) circle (  1.50);

\path[draw=drawColor,line width= 0.4pt,line join=round,line cap=round,fill=fillColor] (291.39,260.17) circle (  1.50);

\path[draw=drawColor,line width= 0.4pt,line join=round,line cap=round,fill=fillColor] (171.58,247.04) circle (  1.50);

\path[draw=drawColor,line width= 0.4pt,line join=round,line cap=round,fill=fillColor] (237.83,310.48) circle (  1.50);

\path[draw=drawColor,line width= 0.4pt,line join=round,line cap=round,fill=fillColor] (182.86,214.00) circle (  1.50);

\path[draw=drawColor,line width= 0.4pt,line join=round,line cap=round,fill=fillColor] (225.14,274.33) circle (  1.50);

\path[draw=drawColor,line width= 0.4pt,line join=round,line cap=round,fill=fillColor] (281.17,310.00) circle (  1.50);

\path[draw=drawColor,line width= 0.4pt,line join=round,line cap=round,fill=fillColor] (184.27,253.48) circle (  1.50);

\path[draw=drawColor,line width= 0.4pt,line join=round,line cap=round,fill=fillColor] (417.90,201.12) circle (  1.50);

\path[draw=drawColor,line width= 0.4pt,line join=round,line cap=round,fill=fillColor] (172.29,243.88) circle (  1.50);

\path[draw=drawColor,line width= 0.4pt,line join=round,line cap=round,fill=fillColor] (179.69,240.24) circle (  1.50);

\path[draw=drawColor,line width= 0.4pt,line join=round,line cap=round,fill=fillColor] (305.49,230.01) circle (  1.50);

\path[draw=drawColor,line width= 0.4pt,line join=round,line cap=round,fill=fillColor] (189.91,211.88) circle (  1.50);

\path[draw=drawColor,line width= 0.4pt,line join=round,line cap=round,fill=fillColor] (219.51,266.17) circle (  1.50);

\path[draw=drawColor,line width= 0.4pt,line join=round,line cap=round,fill=fillColor] (244.53,378.35) circle (  1.50);

\path[draw=drawColor,line width= 0.4pt,line join=round,line cap=round,fill=fillColor] (216.33,385.82) circle (  1.50);

\path[draw=drawColor,line width= 0.4pt,line join=round,line cap=round,fill=fillColor] (205.06,282.10) circle (  1.50);

\path[draw=drawColor,line width= 0.4pt,line join=round,line cap=round,fill=fillColor] (310.42,274.93) circle (  1.50);

\path[draw=drawColor,line width= 0.4pt,line join=round,line cap=round,fill=fillColor] (448.20,248.67) circle (  1.50);

\path[draw=drawColor,line width= 0.4pt,line join=round,line cap=round,fill=fillColor] (202.24,270.70) circle (  1.50);

\path[draw=drawColor,line width= 0.4pt,line join=round,line cap=round,fill=fillColor] (130.00,371.58) circle (  1.50);

\path[draw=drawColor,line width= 0.4pt,line join=round,line cap=round,fill=fillColor] (327.34,319.76) circle (  1.50);
\end{scope}
\begin{scope}
\path[clip] (  0.00,  0.00) rectangle (505.89,505.89);
\definecolor{drawColor}{RGB}{0,0,0}

\node[text=drawColor,anchor=base,inner sep=0pt, outer sep=0pt, scale=  1.20] at (276.94,  3.60) {Suppression-Index};

\node[text=drawColor,rotate= 90.00,anchor=base,inner sep=0pt, outer sep=0pt, scale=  1.20] at ( 10.20,277.55) {BIS-Test \textit{z}-Wert};
\end{scope}
\begin{scope}
\path[clip] ( 48.60, 49.20) rectangle (505.29,505.89);
\definecolor{drawColor}{RGB}{0,0,0}

\path[draw=drawColor,line width= 0.4pt,line join=round,line cap=round] ( 48.60,279.19) -- (505.29,280.22);
\end{scope}
\begin{scope}
\path[clip] (  0.00,  0.00) rectangle (505.89,505.89);
\definecolor{drawColor}{RGB}{0,0,0}

\path[draw=drawColor,line width= 0.4pt,line join=round,line cap=round] ( 65.51, 49.20) -- (488.38, 49.20);

\path[draw=drawColor,line width= 0.4pt,line join=round,line cap=round] ( 65.51, 49.20) -- ( 65.51, 43.20);

\path[draw=drawColor,line width= 0.4pt,line join=round,line cap=round] (135.99, 49.20) -- (135.99, 43.20);

\path[draw=drawColor,line width= 0.4pt,line join=round,line cap=round] (206.47, 49.20) -- (206.47, 43.20);

\path[draw=drawColor,line width= 0.4pt,line join=round,line cap=round] (276.94, 49.20) -- (276.94, 43.20);

\path[draw=drawColor,line width= 0.4pt,line join=round,line cap=round] (347.42, 49.20) -- (347.42, 43.20);

\path[draw=drawColor,line width= 0.4pt,line join=round,line cap=round] (417.90, 49.20) -- (417.90, 43.20);

\path[draw=drawColor,line width= 0.4pt,line join=round,line cap=round] (488.38, 49.20) -- (488.38, 43.20);

\node[text=drawColor,anchor=base,inner sep=0pt, outer sep=0pt, scale=  1.20] at ( 65.51, 27.60) {-0.2};

\node[text=drawColor,anchor=base,inner sep=0pt, outer sep=0pt, scale=  1.20] at (135.99, 27.60) {0.0};

\node[text=drawColor,anchor=base,inner sep=0pt, outer sep=0pt, scale=  1.20] at (206.47, 27.60) {0.2};

\node[text=drawColor,anchor=base,inner sep=0pt, outer sep=0pt, scale=  1.20] at (276.94, 27.60) {0.4};

\node[text=drawColor,anchor=base,inner sep=0pt, outer sep=0pt, scale=  1.20] at (347.42, 27.60) {0.6};

\node[text=drawColor,anchor=base,inner sep=0pt, outer sep=0pt, scale=  1.20] at (417.90, 27.60) {0.8};

\node[text=drawColor,anchor=base,inner sep=0pt, outer sep=0pt, scale=  1.20] at (488.38, 27.60) {1.0};

\path[draw=drawColor,line width= 0.4pt,line join=round,line cap=round] ( 48.60, 66.11) -- ( 48.60,488.98);

\path[draw=drawColor,line width= 0.4pt,line join=round,line cap=round] ( 48.60, 66.11) -- ( 42.60, 66.11);

\path[draw=drawColor,line width= 0.4pt,line join=round,line cap=round] ( 48.60,171.83) -- ( 42.60,171.83);

\path[draw=drawColor,line width= 0.4pt,line join=round,line cap=round] ( 48.60,277.55) -- ( 42.60,277.55);

\path[draw=drawColor,line width= 0.4pt,line join=round,line cap=round] ( 48.60,383.26) -- ( 42.60,383.26);

\path[draw=drawColor,line width= 0.4pt,line join=round,line cap=round] ( 48.60,488.98) -- ( 42.60,488.98);

\node[text=drawColor,anchor=base east,inner sep=0pt, outer sep=0pt, scale=  1.20] at ( 36.60, 61.98) {-2};

\node[text=drawColor,anchor=base east,inner sep=0pt, outer sep=0pt, scale=  1.20] at ( 36.60,167.70) {-1};

\node[text=drawColor,anchor=base east,inner sep=0pt, outer sep=0pt, scale=  1.20] at ( 36.60,273.41) {0};

\node[text=drawColor,anchor=base east,inner sep=0pt, outer sep=0pt, scale=  1.20] at ( 36.60,379.13) {1};

\node[text=drawColor,anchor=base east,inner sep=0pt, outer sep=0pt, scale=  1.20] at ( 36.60,484.84) {2};
\end{scope}
\end{tikzpicture}

	\end{adjustbox}
	\caption[Zusammenhang zwischen dem \gls{si} und \gls{zwert} des \gls{bist}s]{Streudiagramm des Zusammenhangs zwischen dem \gls{si} und dem \gls{zwert} aus dem \gls{bist} ($r=.00$, $p=.98$).}
	\label{fig:suppression_index_zscore_scatterplot}
\end{figure}

Auch der von \citet{Melnick2013} in Studie 2 berichtete Zusammenhang zwischen der kleinsten Mustergrösse ($1.8^{\circ}$-Bedingung) und IQ-Punkten ($r=-.46$, $p<.001$) konnte nicht bestätigt werden: Die Korrelation zwischen der $1.8^{\circ}$-Bedingung und dem \gls{zwert} aus dem \gls{bist} betrug in der vorliegenden Arbeit $r=-.16$ ($p=.03$) und fiel damit signifikant tiefer aus, als bei \citeauthor{Melnick2013} ($z=2.09$, $p=.04$).

Gleichermassen nicht bestätigt werden konnten die von \citet{Melnick2013} berichteten Semipartialkorrelationen zwischen der kleinsten Mustergrösse ($1.8^{\circ}$-Bedingung), der grössten Mustergrösse ($7.2^{\circ}$-Bedingung) und psychometrischer Intelligenz: In Studie 2 von \citeauthor{Melnick2013} betrug die Semipartialkorrelation zwischen der kleinsten Mustergrösse und IQ-Punkten bei Kontrolle für die grösste Mustergrösse $r=-.71$ ($p<.001$) und zwischen der grössten Mustergrösse und IQ-Punkten bei Kontrolle für die kleinste Mustergrösse $r=.55$ ($p<.001$). Hoher IQ war bei \citeauthor{Melnick2013} im Vergleich zu tiefem IQ also mit tieferen $82\,\%$-Erkennungsschwellen bei kleiner Mustergrösse und mit höheren $82\,\%$-Erkennungsschwellen bei grosser Mustergrösse verbunden. 
In der vorliegenden Arbeit betrug der Semipartialkorrelationskoeffizient bei Kontrolle für die grösste Mustergrösse ($z$) zwischen der kleinsten Mustergrösse ($x$) und dem \gls{zwert} ($y$) aus dem \gls{bist} $r_{y(x.z)}= -.11$ ($p = .15$) und bei einer Kontrolle für die kleinste Mustergrösse ($z$) zwischen der grössten Mustergrösse ($x$) und dem \gls{zwert} ($y$) aus dem \gls{bist} $r_{y(x.z)}=-.04$ ($p = .57$). 
Ein Vergleich dieser unabhängigen Semipartialkorrelationskoeffizienten hat ergeben, dass die in der vorliegenden Arbeit erhaltenen Zusammenhänge signifikant geringer ausfielen als bei \citeauthor{Melnick2013} ($z=4.84$, $p<.001$ respektive $z=4.10$, $p<.001$).

Abschliessend zur ersten Fragestellung kann festgehalten werden, dass sowohl die von \citet{Melnick2013} berichteten Zusammenhänge zwischen dem \gls{si} und psychometrischer Intelligenz als auch die Zusammenhänge der einzelnen Bedingungen der \gls{ssauf} mit psychometrischer Intelligenz nicht bestätigt werden konnten.













\section{2. Fragestellung \label{sec:2Fragestellung}}

Mit der zweiten Fragestellung sollte geprüft werden, ob die aus der \gls{ssauf} mit einer exponentiellen Regression abgeleiteten Aufgabenparameter benutzt werden können, um psychometrische Intelligenz vorherzusagen.

Für jede \gls{vp} wurden die vier $82\,\%$-Erkennungsschwellen mit einer exponentiellen Regression der von \citet{Melnick2013} vorgeschlagenen Form $y=a \times e^{bx}$ vorhergesagt (siehe \autoref{fig:spatial_suppression_exponential_model}).
\begin{table}[b]
	%\flushleft
	\centering
	\captionsetup{labelsep = none}
	\caption[Deskriptive Angaben zur exponentiellen Regression für die Vorhersage der $82\,\%$-Er\-ken\-nungs\-schwel\-len durch die Mustergrössen der \gls{ssauf}]{\newline  \textit{Deskriptive Angaben zur exponentiellen Regression ($y=a \times e^{bx}$) für die Vorhersage der $82\,\%$-Er\-ken\-nungs\-schwel\-len durch die Mustergrössen der \gls{ssauf} und Kennwerte zur Verteilungsform der Daten} \vspace{.2cm}}
	\label{tab:spatial_suppression_exponential_model}
	\begin{adjustbox}{width=1\textwidth}
		\begin{threeparttable}
			\begin{tabular}{
					l
					S[table-format = 2.3]
					S[table-format = 2.3]
					S[table-format = 2.3]
					S[table-format = 3.3]
					S[table-format = 1.2]
					S[table-format = 2.2]
					S[table-format = <0.3, add-integer-zero=false]
				}
				\hline
				\multicolumn{1}{c}{Parameter}	& 	{\textit{M}}	&{\textit{SD}}	&	{Min}	&	{Max} 	&	{\textnormal{Schiefe}}	&	{\textnormal{Kurtosis}} & {S-W \textit{p}-Wert}\\
				\hline
				$a$			&		70			&	28			&	5		&	195		&	0.97					&	1.87					& 		<.001			\\
				$b$			&		.103		&	.081		&	-.079	&	.650	&	2.17					&	10.80					& 		<.001			\\
				\hline
			\end{tabular}

			\begin{tablenotes}[flushleft]
				\footnotesize				% font size
				\setlength\labelsep{0pt}	% no indent on second line
				\item \textit{Anmerkungen.} \textit{a}~=~Asymptote (in ms); \textit{b}~=~Steigung; Min~=~Minimum; Max~=Maximum; S-W~= Shapiro-Wilk-Test.
			\end{tablenotes}
		\end{threeparttable}
	\end{adjustbox}
\end{table}
\begin{figure}[t]
	\centering
	\begin{adjustbox}{width=1\textwidth}
		% Created by tikzDevice version 0.10.1 on 2016-07-26 08:19:51
% !TEX encoding = UTF-8 Unicode
\begin{tikzpicture}[x=1pt,y=1pt]
\definecolor{fillColor}{RGB}{255,255,255}
\path[use as bounding box,fill=fillColor,fill opacity=0.00] (0,0) rectangle (505.89,505.89);
\begin{scope}
\path[clip] ( 54.00, 78.00) rectangle (451.89,475.89);
\definecolor{drawColor}{RGB}{0,0,0}

\path[draw=drawColor,line width= 0.4pt,line join=round,line cap=round] (  0.00,235.32) --
	(  2.26,235.54) --
	(  4.66,235.77) --
	(  7.03,236.01) --
	(  9.37,236.24) --
	( 11.67,236.48) --
	( 13.94,236.72) --
	( 16.18,236.95) --
	( 18.38,237.19) --
	( 20.56,237.42) --
	( 22.71,237.66) --
	( 24.83,237.89) --
	( 26.92,238.13) --
	( 28.98,238.36) --
	( 31.02,238.60) --
	( 33.04,238.84) --
	( 35.02,239.07) --
	( 36.99,239.31) --
	( 38.93,239.54) --
	( 40.84,239.78) --
	( 42.74,240.01) --
	( 44.61,240.25) --
	( 46.46,240.49) --
	( 48.28,240.72) --
	( 50.09,240.96) --
	( 51.88,241.19) --
	( 53.65,241.43) --
	( 55.40,241.66) --
	( 57.13,241.90) --
	( 58.84,242.13) --
	( 60.53,242.37) --
	( 62.21,242.61) --
	( 63.86,242.84) --
	( 65.51,243.08) --
	( 67.13,243.31) --
	( 68.74,243.55) --
	( 70.33,243.78) --
	( 71.91,244.02) --
	( 73.47,244.25) --
	( 75.01,244.49) --
	( 76.54,244.73) --
	( 78.06,244.96) --
	( 79.56,245.20) --
	( 81.05,245.43) --
	( 82.53,245.67) --
	( 83.99,245.90) --
	( 85.44,246.14) --
	( 86.87,246.38) --
	( 88.29,246.61) --
	( 89.70,246.85) --
	( 91.10,247.08) --
	( 92.49,247.32) --
	( 93.86,247.55) --
	( 95.22,247.79) --
	( 96.57,248.02) --
	( 97.91,248.26) --
	( 99.24,248.50) --
	(100.55,248.73) --
	(101.86,248.97) --
	(103.16,249.20) --
	(104.44,249.44) --
	(105.72,249.67) --
	(106.98,249.91) --
	(108.24,250.15) --
	(109.48,250.38) --
	(110.72,250.62) --
	(111.94,250.85) --
	(113.16,251.09) --
	(114.37,251.32) --
	(115.57,251.56) --
	(116.75,251.79) --
	(117.94,252.03) --
	(119.11,252.27) --
	(120.27,252.50) --
	(121.43,252.74) --
	(122.57,252.97) --
	(123.71,253.21) --
	(124.84,253.44) --
	(125.97,253.68) --
	(127.08,253.91) --
	(128.19,254.15) --
	(129.29,254.39) --
	(130.38,254.62) --
	(131.47,254.86) --
	(132.54,255.09) --
	(133.61,255.33) --
	(134.68,255.56) --
	(135.73,255.80) --
	(136.78,256.04) --
	(137.82,256.27) --
	(138.86,256.51) --
	(139.89,256.74) --
	(140.91,256.98) --
	(141.93,257.21) --
	(142.94,257.45) --
	(143.94,257.68) --
	(144.94,257.92) --
	(145.93,258.16) --
	(146.91,258.39) --
	(147.89,258.63) --
	(148.86,258.86) --
	(149.83,259.10) --
	(150.79,259.33) --
	(151.75,259.57) --
	(152.69,259.81) --
	(153.64,260.04) --
	(154.58,260.28) --
	(155.51,260.51) --
	(156.44,260.75) --
	(157.36,260.98) --
	(158.28,261.22) --
	(159.19,261.45) --
	(160.10,261.69) --
	(161.00,261.93) --
	(161.89,262.16) --
	(162.78,262.40) --
	(163.67,262.63) --
	(164.55,262.87) --
	(165.43,263.10) --
	(166.30,263.34) --
	(167.17,263.57) --
	(168.03,263.81) --
	(168.89,264.05) --
	(169.74,264.28) --
	(170.59,264.52) --
	(171.43,264.75) --
	(172.27,264.99) --
	(173.11,265.22) --
	(173.94,265.46) --
	(174.77,265.70) --
	(175.59,265.93) --
	(176.41,266.17) --
	(177.22,266.40) --
	(178.03,266.64) --
	(178.84,266.87) --
	(179.64,267.11) --
	(180.44,267.34) --
	(181.23,267.58) --
	(182.02,267.82) --
	(182.81,268.05) --
	(183.59,268.29) --
	(184.37,268.52) --
	(185.15,268.76) --
	(185.92,268.99) --
	(186.68,269.23) --
	(187.45,269.47) --
	(188.21,269.70) --
	(188.96,269.94) --
	(189.72,270.17) --
	(190.47,270.41) --
	(191.21,270.64) --
	(191.96,270.88) --
	(192.69,271.11) --
	(193.43,271.35) --
	(194.16,271.59) --
	(194.89,271.82) --
	(195.62,272.06) --
	(196.34,272.29) --
	(197.06,272.53) --
	(197.77,272.76) --
	(198.49,273.00) --
	(199.20,273.23) --
	(199.90,273.47) --
	(200.61,273.71) --
	(201.31,273.94) --
	(202.00,274.18) --
	(202.70,274.41) --
	(203.39,274.65) --
	(204.08,274.88) --
	(204.76,275.12) --
	(205.44,275.36) --
	(206.12,275.59) --
	(206.80,275.83) --
	(207.47,276.06) --
	(208.15,276.30) --
	(208.81,276.53) --
	(209.48,276.77) --
	(210.14,277.00) --
	(210.80,277.24) --
	(211.46,277.48) --
	(212.11,277.71) --
	(212.76,277.95) --
	(213.41,278.18) --
	(214.06,278.42) --
	(214.70,278.65) --
	(215.35,278.89) --
	(215.98,279.13) --
	(216.62,279.36) --
	(217.25,279.60) --
	(217.88,279.83) --
	(218.51,280.07) --
	(219.14,280.30) --
	(219.76,280.54) --
	(220.38,280.77) --
	(221.00,281.01) --
	(221.62,281.25) --
	(222.23,281.48) --
	(222.85,281.72) --
	(223.46,281.95) --
	(224.06,282.19) --
	(224.67,282.42) --
	(225.27,282.66) --
	(225.87,282.89) --
	(226.47,283.13) --
	(227.07,283.37) --
	(227.66,283.60) --
	(228.25,283.84) --
	(228.84,284.07) --
	(229.43,284.31) --
	(230.01,284.54) --
	(230.59,284.78) --
	(231.18,285.02) --
	(231.75,285.25) --
	(232.33,285.49) --
	(232.91,285.72) --
	(233.48,285.96) --
	(234.05,286.19) --
	(234.62,286.43) --
	(235.18,286.66) --
	(235.75,286.90) --
	(236.31,287.14) --
	(236.87,287.37) --
	(237.43,287.61) --
	(237.99,287.84) --
	(238.54,288.08) --
	(239.09,288.31) --
	(239.64,288.55) --
	(240.19,288.79) --
	(240.74,289.02) --
	(241.28,289.26) --
	(241.83,289.49) --
	(242.37,289.73) --
	(242.91,289.96) --
	(243.45,290.20) --
	(243.98,290.43) --
	(244.52,290.67) --
	(245.05,290.91) --
	(245.58,291.14) --
	(246.11,291.38) --
	(246.64,291.61) --
	(247.16,291.85) --
	(247.69,292.08) --
	(248.21,292.32) --
	(248.73,292.55) --
	(249.25,292.79) --
	(249.76,293.03) --
	(250.28,293.26) --
	(250.79,293.50) --
	(251.30,293.73) --
	(251.81,293.97) --
	(252.32,294.20) --
	(252.83,294.44) --
	(253.34,294.68) --
	(253.84,294.91) --
	(254.34,295.15) --
	(254.84,295.38) --
	(255.34,295.62) --
	(255.84,295.85) --
	(256.34,296.09) --
	(256.83,296.32) --
	(257.32,296.56) --
	(257.82,296.80) --
	(258.31,297.03) --
	(258.79,297.27) --
	(259.28,297.50) --
	(259.77,297.74) --
	(260.25,297.97) --
	(260.73,298.21) --
	(261.21,298.45) --
	(261.69,298.68) --
	(262.17,298.92) --
	(262.65,299.15) --
	(263.12,299.39) --
	(263.60,299.62) --
	(264.07,299.86) --
	(264.54,300.09) --
	(265.01,300.33) --
	(265.48,300.57) --
	(265.95,300.80) --
	(266.41,301.04) --
	(266.88,301.27) --
	(267.34,301.51) --
	(267.80,301.74) --
	(268.26,301.98) --
	(268.72,302.21) --
	(269.18,302.45) --
	(269.64,302.69) --
	(270.09,302.92) --
	(270.55,303.16) --
	(271.00,303.39) --
	(271.45,303.63) --
	(271.90,303.86) --
	(272.35,304.10) --
	(272.80,304.34) --
	(273.24,304.57) --
	(273.69,304.81) --
	(274.13,305.04) --
	(274.57,305.28) --
	(275.02,305.51) --
	(275.46,305.75) --
	(275.90,305.98) --
	(276.33,306.22) --
	(276.77,306.46) --
	(277.20,306.69) --
	(277.64,306.93) --
	(278.07,307.16) --
	(278.50,307.40) --
	(278.93,307.63) --
	(279.36,307.87) --
	(279.79,308.11) --
	(280.22,308.34) --
	(280.65,308.58) --
	(281.07,308.81) --
	(281.49,309.05) --
	(281.92,309.28) --
	(282.34,309.52) --
	(282.76,309.75) --
	(283.18,309.99) --
	(283.60,310.23) --
	(284.01,310.46) --
	(284.43,310.70) --
	(284.85,310.93) --
	(285.26,311.17) --
	(285.67,311.40) --
	(286.08,311.64) --
	(286.50,311.87) --
	(286.90,312.11) --
	(287.31,312.35) --
	(287.72,312.58) --
	(288.13,312.82) --
	(288.53,313.05) --
	(288.94,313.29) --
	(289.34,313.52) --
	(289.74,313.76) --
	(290.15,314.00) --
	(290.55,314.23) --
	(290.95,314.47) --
	(291.34,314.70) --
	(291.74,314.94) --
	(292.14,315.17) --
	(292.53,315.41) --
	(292.93,315.64) --
	(293.32,315.88) --
	(293.71,316.12) --
	(294.11,316.35) --
	(294.50,316.59) --
	(294.89,316.82) --
	(295.28,317.06) --
	(295.66,317.29) --
	(296.05,317.53) --
	(296.44,317.77) --
	(296.82,318.00) --
	(297.21,318.24) --
	(297.59,318.47) --
	(297.97,318.71) --
	(298.35,318.94) --
	(298.73,319.18) --
	(299.11,319.41) --
	(299.49,319.65) --
	(299.87,319.89) --
	(300.25,320.12) --
	(300.62,320.36) --
	(301.00,320.59) --
	(301.37,320.83) --
	(301.74,321.06) --
	(302.12,321.30) --
	(302.49,321.53) --
	(302.86,321.77) --
	(303.23,322.01) --
	(303.60,322.24) --
	(303.97,322.48) --
	(304.33,322.71) --
	(304.70,322.95) --
	(305.07,323.18) --
	(305.43,323.42) --
	(305.80,323.66) --
	(306.16,323.89) --
	(306.52,324.13) --
	(306.88,324.36) --
	(307.24,324.60) --
	(307.60,324.83) --
	(307.96,325.07) --
	(308.32,325.30) --
	(308.68,325.54) --
	(309.04,325.78) --
	(309.39,326.01) --
	(309.75,326.25) --
	(310.10,326.48) --
	(310.45,326.72) --
	(310.81,326.95) --
	(311.16,327.19) --
	(311.51,327.43) --
	(311.86,327.66) --
	(312.21,327.90) --
	(312.56,328.13) --
	(312.91,328.37) --
	(313.26,328.60) --
	(313.60,328.84) --
	(313.95,329.07) --
	(314.29,329.31) --
	(314.64,329.55) --
	(314.98,329.78) --
	(315.32,330.02) --
	(315.67,330.25) --
	(316.01,330.49) --
	(316.35,330.72) --
	(316.69,330.96) --
	(317.03,331.19) --
	(317.37,331.43) --
	(317.71,331.67) --
	(318.04,331.90) --
	(318.38,332.14) --
	(318.71,332.37) --
	(319.05,332.61) --
	(319.38,332.84) --
	(319.72,333.08) --
	(320.05,333.32) --
	(320.38,333.55) --
	(320.71,333.79) --
	(321.05,334.02) --
	(321.38,334.26) --
	(321.71,334.49) --
	(322.03,334.73) --
	(322.36,334.96) --
	(322.69,335.20) --
	(323.02,335.44) --
	(323.34,335.67) --
	(323.67,335.91) --
	(323.99,336.14) --
	(324.32,336.38) --
	(324.64,336.61) --
	(324.96,336.85) --
	(325.29,337.09) --
	(325.61,337.32) --
	(325.93,337.56) --
	(326.25,337.79) --
	(326.57,338.03) --
	(326.89,338.26) --
	(327.21,338.50) --
	(327.52,338.73) --
	(327.84,338.97) --
	(328.16,339.21) --
	(328.47,339.44) --
	(328.79,339.68) --
	(329.10,339.91) --
	(329.42,340.15) --
	(329.73,340.38) --
	(330.04,340.62) --
	(330.36,340.85) --
	(330.67,341.09) --
	(330.98,341.33) --
	(331.29,341.56) --
	(331.60,341.80) --
	(331.91,342.03) --
	(332.22,342.27) --
	(332.52,342.50) --
	(332.83,342.74) --
	(333.14,342.98) --
	(333.44,343.21) --
	(333.75,343.45) --
	(334.06,343.68) --
	(334.36,343.92) --
	(334.66,344.15) --
	(334.97,344.39) --
	(335.27,344.62) --
	(335.57,344.86) --
	(335.87,345.10) --
	(336.17,345.33) --
	(336.48,345.57) --
	(336.78,345.80) --
	(337.07,346.04) --
	(337.37,346.27) --
	(337.67,346.51) --
	(337.97,346.75) --
	(338.27,346.98) --
	(338.56,347.22) --
	(338.86,347.45) --
	(339.15,347.69) --
	(339.45,347.92) --
	(339.74,348.16) --
	(340.04,348.39) --
	(340.33,348.63) --
	(340.62,348.87) --
	(340.92,349.10) --
	(341.21,349.34) --
	(341.50,349.57) --
	(341.79,349.81) --
	(342.08,350.04) --
	(342.37,350.28) --
	(342.66,350.51) --
	(342.95,350.75) --
	(343.23,350.99) --
	(343.52,351.22) --
	(343.81,351.46) --
	(344.10,351.69) --
	(344.38,351.93) --
	(344.67,352.16) --
	(344.95,352.40) --
	(345.24,352.64) --
	(345.52,352.87) --
	(345.80,353.11) --
	(346.09,353.34) --
	(346.37,353.58) --
	(346.65,353.81) --
	(346.93,354.05) --
	(347.21,354.28) --
	(347.49,354.52) --
	(347.77,354.76) --
	(348.05,354.99) --
	(348.33,355.23) --
	(348.61,355.46) --
	(348.89,355.70) --
	(349.17,355.93) --
	(349.44,356.17) --
	(349.72,356.40) --
	(350.00,356.64) --
	(350.27,356.88) --
	(350.55,357.11) --
	(350.82,357.35) --
	(351.10,357.58) --
	(351.37,357.82) --
	(351.64,358.05) --
	(351.92,358.29) --
	(352.19,358.53) --
	(352.46,358.76) --
	(352.73,359.00) --
	(353.00,359.23) --
	(353.27,359.47) --
	(353.54,359.70) --
	(353.81,359.94) --
	(354.08,360.17) --
	(354.35,360.41) --
	(354.62,360.65) --
	(354.89,360.88) --
	(355.15,361.12) --
	(355.42,361.35) --
	(355.69,361.59) --
	(355.95,361.82) --
	(356.22,362.06) --
	(356.48,362.30) --
	(356.75,362.53) --
	(357.01,362.77) --
	(357.28,363.00) --
	(357.54,363.24) --
	(357.80,363.47) --
	(358.07,363.71) --
	(358.33,363.94) --
	(358.59,364.18) --
	(358.85,364.42) --
	(359.11,364.65) --
	(359.37,364.89) --
	(359.63,365.12) --
	(359.89,365.36) --
	(360.15,365.59) --
	(360.41,365.83) --
	(360.67,366.06) --
	(360.93,366.30) --
	(361.18,366.54) --
	(361.44,366.77) --
	(361.70,367.01) --
	(361.95,367.24) --
	(362.21,367.48) --
	(362.46,367.71) --
	(362.72,367.95) --
	(362.97,368.19) --
	(363.23,368.42) --
	(363.48,368.66) --
	(363.73,368.89) --
	(363.99,369.13) --
	(364.24,369.36) --
	(364.49,369.60) --
	(364.74,369.83) --
	(365.00,370.07) --
	(365.25,370.31) --
	(365.50,370.54) --
	(365.75,370.78) --
	(366.00,371.01) --
	(366.25,371.25) --
	(366.50,371.48) --
	(366.74,371.72) --
	(366.99,371.96) --
	(367.24,372.19) --
	(367.49,372.43) --
	(367.73,372.66) --
	(367.98,372.90) --
	(368.23,373.13) --
	(368.47,373.37) --
	(368.72,373.60) --
	(368.96,373.84) --
	(369.21,374.08) --
	(369.45,374.31) --
	(369.70,374.55) --
	(369.94,374.78) --
	(370.18,375.02) --
	(370.43,375.25) --
	(370.67,375.49) --
	(370.91,375.72) --
	(371.15,375.96) --
	(371.40,376.20) --
	(371.64,376.43) --
	(371.88,376.67) --
	(372.12,376.90) --
	(372.36,377.14) --
	(372.60,377.37) --
	(372.84,377.61) --
	(373.08,377.85) --
	(373.32,378.08) --
	(373.55,378.32) --
	(373.79,378.55) --
	(374.03,378.79) --
	(374.27,379.02) --
	(374.50,379.26) --
	(374.74,379.49) --
	(374.98,379.73) --
	(375.21,379.97) --
	(375.45,380.20) --
	(375.68,380.44) --
	(375.92,380.67) --
	(376.15,380.91) --
	(376.39,381.14) --
	(376.62,381.38) --
	(376.85,381.62) --
	(377.09,381.85) --
	(377.32,382.09) --
	(377.55,382.32) --
	(377.78,382.56) --
	(378.01,382.79) --
	(378.25,383.03) --
	(378.48,383.26) --
	(378.71,383.50) --
	(378.94,383.74) --
	(379.17,383.97) --
	(379.40,384.21) --
	(379.63,384.44) --
	(379.86,384.68) --
	(380.09,384.91) --
	(380.31,385.15) --
	(380.54,385.38) --
	(380.77,385.62) --
	(381.00,385.86) --
	(381.22,386.09) --
	(381.45,386.33) --
	(381.68,386.56) --
	(381.90,386.80) --
	(382.13,387.03) --
	(382.35,387.27) --
	(382.58,387.51) --
	(382.80,387.74) --
	(383.03,387.98) --
	(383.25,388.21) --
	(383.48,388.45) --
	(383.70,388.68) --
	(383.92,388.92) --
	(384.15,389.15) --
	(384.37,389.39) --
	(384.59,389.63) --
	(384.81,389.86) --
	(385.04,390.10) --
	(385.26,390.33) --
	(385.48,390.57) --
	(385.70,390.80) --
	(385.92,391.04) --
	(386.14,391.28) --
	(386.36,391.51) --
	(386.58,391.75) --
	(386.80,391.98) --
	(387.02,392.22) --
	(387.24,392.45) --
	(387.46,392.69) --
	(387.67,392.92) --
	(387.89,393.16) --
	(388.11,393.40) --
	(388.33,393.63) --
	(388.54,393.87) --
	(388.76,394.10) --
	(388.98,394.34) --
	(389.19,394.57) --
	(389.41,394.81) --
	(389.62,395.04) --
	(389.84,395.28) --
	(390.05,395.52) --
	(390.27,395.75) --
	(390.48,395.99) --
	(390.70,396.22) --
	(390.91,396.46) --
	(391.12,396.69) --
	(391.34,396.93) --
	(391.55,397.17) --
	(391.76,397.40) --
	(391.98,397.64) --
	(392.19,397.87) --
	(392.40,398.11) --
	(392.61,398.34) --
	(392.82,398.58) --
	(393.03,398.81) --
	(393.24,399.05) --
	(393.45,399.29) --
	(393.66,399.52) --
	(393.87,399.76) --
	(394.08,399.99) --
	(394.29,400.23) --
	(394.50,400.46) --
	(394.71,400.70) --
	(394.92,400.94) --
	(395.13,401.17) --
	(395.33,401.41) --
	(395.54,401.64) --
	(395.75,401.88) --
	(395.96,402.11) --
	(396.16,402.35) --
	(396.37,402.58) --
	(396.58,402.82) --
	(396.78,403.06) --
	(396.99,403.29) --
	(397.19,403.53) --
	(397.40,403.76) --
	(397.60,404.00) --
	(397.81,404.23) --
	(398.01,404.47) --
	(398.22,404.70) --
	(398.42,404.94) --
	(398.63,405.18) --
	(398.83,405.41) --
	(399.03,405.65) --
	(399.23,405.88) --
	(399.44,406.12) --
	(399.64,406.35) --
	(399.84,406.59) --
	(400.04,406.83) --
	(400.25,407.06) --
	(400.45,407.30) --
	(400.65,407.53) --
	(400.85,407.77) --
	(401.05,408.00) --
	(401.25,408.24) --
	(401.45,408.47) --
	(401.65,408.71) --
	(401.85,408.95) --
	(402.05,409.18) --
	(402.25,409.42) --
	(402.45,409.65) --
	(402.65,409.89) --
	(402.84,410.12) --
	(403.04,410.36) --
	(403.24,410.60) --
	(403.44,410.83) --
	(403.64,411.07) --
	(403.83,411.30) --
	(404.03,411.54) --
	(404.23,411.77) --
	(404.42,412.01) --
	(404.62,412.24) --
	(404.81,412.48) --
	(405.01,412.72) --
	(405.21,412.95) --
	(405.40,413.19) --
	(405.60,413.42) --
	(405.79,413.66) --
	(405.99,413.89) --
	(406.18,414.13) --
	(406.37,414.36) --
	(406.57,414.60) --
	(406.76,414.84) --
	(406.95,415.07) --
	(407.15,415.31) --
	(407.34,415.54) --
	(407.53,415.78) --
	(407.73,416.01) --
	(407.92,416.25) --
	(408.11,416.49) --
	(408.30,416.72) --
	(408.49,416.96) --
	(408.68,417.19) --
	(408.88,417.43) --
	(409.07,417.66) --
	(409.26,417.90) --
	(409.45,418.13) --
	(409.64,418.37) --
	(409.83,418.61) --
	(410.02,418.84) --
	(410.21,419.08) --
	(410.40,419.31) --
	(410.58,419.55) --
	(410.77,419.78) --
	(410.96,420.02) --
	(411.15,420.26) --
	(411.34,420.49) --
	(411.53,420.73) --
	(411.71,420.96) --
	(411.90,421.20) --
	(412.09,421.43) --
	(412.28,421.67) --
	(412.46,421.90) --
	(412.65,422.14) --
	(412.84,422.38) --
	(413.02,422.61) --
	(413.21,422.85) --
	(413.39,423.08) --
	(413.58,423.32) --
	(413.76,423.55) --
	(413.95,423.79) --
	(414.13,424.02) --
	(414.32,424.26) --
	(414.50,424.50) --
	(414.69,424.73) --
	(414.87,424.97) --
	(415.06,425.20) --
	(415.24,425.44) --
	(415.42,425.67) --
	(415.61,425.91) --
	(415.79,426.15) --
	(415.97,426.38) --
	(416.15,426.62) --
	(416.34,426.85) --
	(416.52,427.09) --
	(416.70,427.32) --
	(416.88,427.56) --
	(417.06,427.79) --
	(417.24,428.03) --
	(417.43,428.27) --
	(417.61,428.50) --
	(417.79,428.74) --
	(417.97,428.97) --
	(418.15,429.21) --
	(418.33,429.44) --
	(418.51,429.68) --
	(418.69,429.92) --
	(418.87,430.15) --
	(419.05,430.39) --
	(419.23,430.62) --
	(419.40,430.86) --
	(419.58,431.09) --
	(419.76,431.33) --
	(419.94,431.56) --
	(420.12,431.80) --
	(420.30,432.04) --
	(420.47,432.27) --
	(420.65,432.51) --
	(420.83,432.74) --
	(421.01,432.98) --
	(421.18,433.21) --
	(421.36,433.45) --
	(421.54,433.68) --
	(421.71,433.92) --
	(421.89,434.16) --
	(422.06,434.39) --
	(422.24,434.63) --
	(422.41,434.86) --
	(422.59,435.10) --
	(422.77,435.33) --
	(422.94,435.57) --
	(423.12,435.81) --
	(423.29,436.04) --
	(423.46,436.28) --
	(423.64,436.51) --
	(423.81,436.75) --
	(423.99,436.98) --
	(424.16,437.22) --
	(424.33,437.45) --
	(424.51,437.69) --
	(424.68,437.93) --
	(424.85,438.16) --
	(425.03,438.40) --
	(425.20,438.63) --
	(425.37,438.87) --
	(425.54,439.10) --
	(425.71,439.34) --
	(425.89,439.58) --
	(426.06,439.81) --
	(426.23,440.05) --
	(426.40,440.28) --
	(426.57,440.52) --
	(426.74,440.75) --
	(426.91,440.99) --
	(427.08,441.22) --
	(427.25,441.46) --
	(427.42,441.70) --
	(427.59,441.93) --
	(427.76,442.17) --
	(427.93,442.40) --
	(428.10,442.64) --
	(428.27,442.87) --
	(428.44,443.11) --
	(428.61,443.34) --
	(428.78,443.58) --
	(428.95,443.82) --
	(429.11,444.05) --
	(429.28,444.29) --
	(429.45,444.52) --
	(429.62,444.76) --
	(429.79,444.99) --
	(429.95,445.23) --
	(430.12,445.47) --
	(430.29,445.70) --
	(430.46,445.94) --
	(430.62,446.17) --
	(430.79,446.41) --
	(430.95,446.64) --
	(431.12,446.88) --
	(431.29,447.11) --
	(431.45,447.35) --
	(431.62,447.59) --
	(431.78,447.82) --
	(431.95,448.06) --
	(432.11,448.29) --
	(432.28,448.53) --
	(432.44,448.76) --
	(432.61,449.00) --
	(432.77,449.24) --
	(432.94,449.47) --
	(433.10,449.71) --
	(433.27,449.94) --
	(433.43,450.18) --
	(433.59,450.41) --
	(433.76,450.65) --
	(433.92,450.88) --
	(434.08,451.12) --
	(434.25,451.36) --
	(434.41,451.59) --
	(434.57,451.83) --
	(434.74,452.06) --
	(434.90,452.30) --
	(435.06,452.53) --
	(435.22,452.77) --
	(435.38,453.00) --
	(435.55,453.24) --
	(435.71,453.48) --
	(435.87,453.71) --
	(436.03,453.95) --
	(436.19,454.18) --
	(436.35,454.42) --
	(436.51,454.65) --
	(436.67,454.89) --
	(436.83,455.13) --
	(436.99,455.36);
\end{scope}
\begin{scope}
\path[clip] (  0.00,  0.00) rectangle (505.89,505.89);
\definecolor{drawColor}{RGB}{0,0,0}

\node[text=drawColor,anchor=base,inner sep=0pt, outer sep=0pt, scale=  1.20] at (252.94, 32.40) {Mustergr{"o}sse ($^\circ$)};

\node[text=drawColor,rotate= 90.00,anchor=base,inner sep=0pt, outer sep=0pt, scale=  1.20] at ( 15.60,276.95) {Schwellensch{"a}tzungen 82 \% korrekt (ms)};
\end{scope}
\begin{scope}
\path[clip] (  0.00,  0.00) rectangle (505.89,505.89);
\definecolor{drawColor}{RGB}{0,0,0}

\path[draw=drawColor,line width= 0.4pt,line join=round,line cap=round] (162.78, 78.00) -- (384.59, 78.00);

\path[draw=drawColor,line width= 0.4pt,line join=round,line cap=round] (162.78, 78.00) -- (162.78, 72.00);

\path[draw=drawColor,line width= 0.4pt,line join=round,line cap=round] (273.69, 78.00) -- (273.69, 72.00);

\path[draw=drawColor,line width= 0.4pt,line join=round,line cap=round] (338.56, 78.00) -- (338.56, 72.00);

\path[draw=drawColor,line width= 0.4pt,line join=round,line cap=round] (384.59, 78.00) -- (384.59, 72.00);

\node[text=drawColor,anchor=base,inner sep=0pt, outer sep=0pt, scale=  1.20] at (162.78, 60.00) {1.8};

\node[text=drawColor,anchor=base,inner sep=0pt, outer sep=0pt, scale=  1.20] at (273.69, 60.00) {3.6};

\node[text=drawColor,anchor=base,inner sep=0pt, outer sep=0pt, scale=  1.20] at (338.56, 60.00) {5.4};

\node[text=drawColor,anchor=base,inner sep=0pt, outer sep=0pt, scale=  1.20] at (384.59, 60.00) {7.2};
\end{scope}
\begin{scope}
\path[clip] (  0.00,  0.00) rectangle (505.89,505.89);
\definecolor{drawColor}{RGB}{0,0,0}

\path[draw=drawColor,line width= 0.4pt,line join=round,line cap=round] ( 68.74, 78.00) --
	(437.15, 78.00);
\end{scope}
\begin{scope}
\path[clip] (  0.00,  0.00) rectangle (505.89,505.89);
\definecolor{drawColor}{RGB}{0,0,0}

\path[draw=drawColor,line width= 0.4pt,line join=round,line cap=round] ( 54.00, 92.74) -- ( 54.00,461.15);

\path[draw=drawColor,line width= 0.4pt,line join=round,line cap=round] ( 54.00, 92.74) -- ( 48.00, 92.74);

\path[draw=drawColor,line width= 0.4pt,line join=round,line cap=round] ( 54.00,143.82) -- ( 48.00,143.82);

\path[draw=drawColor,line width= 0.4pt,line join=round,line cap=round] ( 54.00,302.48) -- ( 48.00,302.48);

\path[draw=drawColor,line width= 0.4pt,line join=round,line cap=round] ( 54.00,395.30) -- ( 48.00,395.30);

\path[draw=drawColor,line width= 0.4pt,line join=round,line cap=round] ( 54.00,461.15) -- ( 48.00,461.15);

\node[text=drawColor,anchor=base east,inner sep=0pt, outer sep=0pt, scale=  1.20] at ( 45.60, 88.60) {0};

\node[text=drawColor,anchor=base east,inner sep=0pt, outer sep=0pt, scale=  1.20] at ( 45.60,139.68) {50};

\node[text=drawColor,anchor=base east,inner sep=0pt, outer sep=0pt, scale=  1.20] at ( 45.60,298.35) {100};

\node[text=drawColor,anchor=base east,inner sep=0pt, outer sep=0pt, scale=  1.20] at ( 45.60,391.17) {150};

\node[text=drawColor,anchor=base east,inner sep=0pt, outer sep=0pt, scale=  1.20] at ( 45.60,457.02) {200};
\end{scope}
\begin{scope}
\path[clip] (  0.00,  0.00) rectangle (505.89,505.89);
\definecolor{drawColor}{RGB}{255,255,255}
\definecolor{fillColor}{RGB}{255,255,255}

\path[draw=drawColor,line width= 0.4pt,line join=round,line cap=round,fill=fillColor] ( 51.02,116.71) rectangle ( 56.98,122.68);
\definecolor{drawColor}{RGB}{0,0,0}

\path[draw=drawColor,line width= 0.4pt,line join=round,line cap=round] ( 51.02,113.73) -- ( 56.98,119.70);

\path[draw=drawColor,line width= 0.4pt,line join=round,line cap=round] ( 51.02,119.70) -- ( 56.98,125.67);
\end{scope}
\begin{scope}
\path[clip] ( 54.00, 78.00) rectangle (451.89,475.89);
\definecolor{drawColor}{RGB}{0,0,0}
\definecolor{fillColor}{RGB}{0,0,0}

\path[draw=drawColor,line width= 0.4pt,line join=round,line cap=round,fill=fillColor] (162.78,268.26) circle (  2.25);

\path[draw=drawColor,line width= 0.4pt,line join=round,line cap=round,fill=fillColor] (273.69,289.73) circle (  2.25);

\path[draw=drawColor,line width= 0.4pt,line join=round,line cap=round,fill=fillColor] (338.56,338.41) circle (  2.25);

\path[draw=drawColor,line width= 0.4pt,line join=round,line cap=round,fill=fillColor] (384.59,397.21) circle (  2.25);

\path[draw=drawColor,line width= 0.4pt,line join=round,line cap=round] (162.78,262.53) -- (162.78,273.85);

\path[draw=drawColor,line width= 0.4pt,line join=round,line cap=round] (159.17,262.53) --
	(162.78,262.53) --
	(166.40,262.53);

\path[draw=drawColor,line width= 0.4pt,line join=round,line cap=round] (166.40,273.85) --
	(162.78,273.85) --
	(159.17,273.85);

\path[draw=drawColor,line width= 0.4pt,line join=round,line cap=round] (273.69,283.54) -- (273.69,295.77);

\path[draw=drawColor,line width= 0.4pt,line join=round,line cap=round] (270.07,283.54) --
	(273.69,283.54) --
	(277.30,283.54);

\path[draw=drawColor,line width= 0.4pt,line join=round,line cap=round] (277.30,295.77) --
	(273.69,295.77) --
	(270.07,295.77);

\path[draw=drawColor,line width= 0.4pt,line join=round,line cap=round] (338.56,330.49) -- (338.56,346.07);

\path[draw=drawColor,line width= 0.4pt,line join=round,line cap=round] (334.95,330.49) --
	(338.56,330.49) --
	(342.18,330.49);

\path[draw=drawColor,line width= 0.4pt,line join=round,line cap=round] (342.18,346.07) --
	(338.56,346.07) --
	(334.95,346.07);

\path[draw=drawColor,line width= 0.4pt,line join=round,line cap=round] (384.59,386.95) -- (384.59,407.03);

\path[draw=drawColor,line width= 0.4pt,line join=round,line cap=round] (380.98,386.95) --
	(384.59,386.95) --
	(388.21,386.95);

\path[draw=drawColor,line width= 0.4pt,line join=round,line cap=round] (388.21,407.03) --
	(384.59,407.03) --
	(380.98,407.03);
\end{scope}
\end{tikzpicture}

	\end{adjustbox}
	\caption[Exponentielles Modell zur Vorhersage der $82\,\%$-Er\-ken\-nungs\-schwel\-le durch die Mustergrösse der \gls{ssauf}]{Exponentieller Einfluss der Mustergrösse auf die $82\,\%$-Er\-ken\-nungs\-schwel\-le für horizontale Bewegung in der \gls{ssauf}. Eingezeichnet sind die Mittelwerte $\pm$ Standardfehler der Mittelwerte. Die x- und die y-Achse sind beide logarithmiert. $y=70 \times e^{0.103x}$.}
	\label{fig:spatial_suppression_exponential_model}
\end{figure}
Deskriptive Angaben zu den daraus resultierenden Parametern, der Asymptote $a$ und der Steigung $b$, sind in \autoref{tab:spatial_suppression_exponential_model} zu finden.
Weil der Determinationskoeffizient $R^2$ bei nicht-linearen Modellen kein adäquates Mass für die Anpassungsgüte des Modells an die Daten darstellt \citep{Spiess2010}, wurde für jede Person der \gls{rmse} berechnet. Der \gls{rmse} ist die Quadratwurzel aus dem Mittelwert der quadrierten Fehler und damit ein Mass für die durchschnittliche Abweichung der vorhergesagten Werte von den empirischen Werten. 
Obwohl der \gls{rmse} für einige \glspl{vp} sehr gross ausfiel, eignete sich ein exponentielles Modell zur Beschreibung der Daten für einen grossen Teil der \glspl{vp} sehr gut (siehe \autoref{fig:spatial_suppression_rmse_density}). Der Median betrug $6$ ms und das dritte Quartil lag bei $9$ ms (Minimum $=0.47$~ms, Maximum $=65.47$~ms).

\begin{figure}[t]
	\centering
	%	\captionsetup{font = small}
	\begin{adjustbox}{width=1\textwidth}
		% Created by tikzDevice version 0.10.1 on 2016-08-17 08:31:57
% !TEX encoding = UTF-8 Unicode
\begin{tikzpicture}[x=1pt,y=1pt]
\definecolor{fillColor}{RGB}{255,255,255}
\path[use as bounding box,fill=fillColor,fill opacity=0.00] (0,0) rectangle (361.35,144.54);
\begin{scope}
\path[clip] ( 42.00, 48.00) rectangle (361.35,138.54);
\definecolor{drawColor}{RGB}{0,0,0}

\path[draw=drawColor,line width= 0.4pt,line join=round,line cap=round] ( 46.94, 51.40) --
	( 47.51, 51.43) --
	( 48.09, 51.49) --
	( 48.66, 51.58) --
	( 49.23, 51.72) --
	( 49.80, 51.93) --
	( 50.37, 52.23) --
	( 50.95, 52.64) --
	( 51.52, 53.18) --
	( 52.09, 53.88) --
	( 52.66, 54.73) --
	( 53.23, 55.76) --
	( 53.81, 56.96) --
	( 54.38, 58.32) --
	( 54.95, 59.84) --
	( 55.52, 61.49) --
	( 56.10, 63.28) --
	( 56.67, 65.21) --
	( 57.24, 67.29) --
	( 57.81, 69.56) --
	( 58.38, 72.03) --
	( 58.96, 74.76) --
	( 59.53, 77.75) --
	( 60.10, 81.02) --
	( 60.67, 84.54) --
	( 61.24, 88.24) --
	( 61.82, 92.04) --
	( 62.39, 95.80) --
	( 62.96, 99.38) --
	( 63.53,102.62) --
	( 64.10,105.36) --
	( 64.68,107.54) --
	( 65.25,109.15) --
	( 65.82,110.21) --
	( 66.39,110.82) --
	( 66.96,111.13) --
	( 67.54,111.30) --
	( 68.11,111.50) --
	( 68.68,111.89) --
	( 69.25,112.56) --
	( 69.82,113.56) --
	( 70.40,114.85) --
	( 70.97,116.35) --
	( 71.54,117.87) --
	( 72.11,119.22) --
	( 72.68,120.22) --
	( 73.26,120.71) --
	( 73.83,120.63) --
	( 74.40,119.94) --
	( 74.97,118.70) --
	( 75.54,117.02) --
	( 76.12,115.06) --
	( 76.69,113.01) --
	( 77.26,111.02) --
	( 77.83,109.25) --
	( 78.40,107.80) --
	( 78.98,106.76) --
	( 79.55,106.11) --
	( 80.12,105.79) --
	( 80.69,105.74) --
	( 81.27,105.85) --
	( 81.84,106.03) --
	( 82.41,106.18) --
	( 82.98,106.19) --
	( 83.55,105.96) --
	( 84.13,105.40) --
	( 84.70,104.44) --
	( 85.27,103.03) --
	( 85.84,101.14) --
	( 86.41, 98.77) --
	( 86.99, 95.97) --
	( 87.56, 92.83) --
	( 88.13, 89.47) --
	( 88.70, 85.99) --
	( 89.27, 82.51) --
	( 89.85, 79.15) --
	( 90.42, 76.02) --
	( 90.99, 73.20) --
	( 91.56, 70.76) --
	( 92.13, 68.76) --
	( 92.71, 67.22) --
	( 93.28, 66.14) --
	( 93.85, 65.49) --
	( 94.42, 65.25) --
	( 94.99, 65.32) --
	( 95.57, 65.64) --
	( 96.14, 66.15) --
	( 96.71, 66.75) --
	( 97.28, 67.40) --
	( 97.85, 68.02) --
	( 98.43, 68.53) --
	( 99.00, 68.89) --
	( 99.57, 69.04) --
	(100.14, 68.95) --
	(100.71, 68.61) --
	(101.29, 68.03) --
	(101.86, 67.26) --
	(102.43, 66.36) --
	(103.00, 65.43) --
	(103.57, 64.52) --
	(104.15, 63.71) --
	(104.72, 63.04) --
	(105.29, 62.53) --
	(105.86, 62.18) --
	(106.44, 61.97) --
	(107.01, 61.87) --
	(107.58, 61.85) --
	(108.15, 61.87) --
	(108.72, 61.91) --
	(109.30, 61.92) --
	(109.87, 61.91) --
	(110.44, 61.87) --
	(111.01, 61.79) --
	(111.58, 61.69) --
	(112.16, 61.57) --
	(112.73, 61.43) --
	(113.30, 61.25) --
	(113.87, 61.06) --
	(114.44, 60.83) --
	(115.02, 60.57) --
	(115.59, 60.30) --
	(116.16, 60.02) --
	(116.73, 59.73) --
	(117.30, 59.47) --
	(117.88, 59.24) --
	(118.45, 59.05) --
	(119.02, 58.91) --
	(119.59, 58.82) --
	(120.16, 58.78) --
	(120.74, 58.79) --
	(121.31, 58.84) --
	(121.88, 58.90) --
	(122.45, 58.98) --
	(123.02, 59.06) --
	(123.60, 59.14) --
	(124.17, 59.20) --
	(124.74, 59.26) --
	(125.31, 59.32) --
	(125.88, 59.36) --
	(126.46, 59.40) --
	(127.03, 59.41) --
	(127.60, 59.37) --
	(128.17, 59.26) --
	(128.74, 59.06) --
	(129.32, 58.76) --
	(129.89, 58.36) --
	(130.46, 57.86) --
	(131.03, 57.27) --
	(131.61, 56.62) --
	(132.18, 55.94) --
	(132.75, 55.26) --
	(133.32, 54.61) --
	(133.89, 54.00) --
	(134.47, 53.46) --
	(135.04, 52.98) --
	(135.61, 52.59) --
	(136.18, 52.26) --
	(136.75, 52.00) --
	(137.33, 51.81) --
	(137.90, 51.67) --
	(138.47, 51.57) --
	(139.04, 51.51) --
	(139.61, 51.49) --
	(140.19, 51.50) --
	(140.76, 51.54) --
	(141.33, 51.61) --
	(141.90, 51.71) --
	(142.47, 51.84) --
	(143.05, 52.00) --
	(143.62, 52.19) --
	(144.19, 52.39) --
	(144.76, 52.59) --
	(145.33, 52.79) --
	(145.91, 52.95) --
	(146.48, 53.07) --
	(147.05, 53.14) --
	(147.62, 53.13) --
	(148.19, 53.06) --
	(148.77, 52.94) --
	(149.34, 52.77) --
	(149.91, 52.57) --
	(150.48, 52.37) --
	(151.05, 52.17) --
	(151.63, 51.98) --
	(152.20, 51.82) --
	(152.77, 51.69) --
	(153.34, 51.59) --
	(153.91, 51.51) --
	(154.49, 51.46) --
	(155.06, 51.44) --
	(155.63, 51.43) --
	(156.20, 51.44) --
	(156.77, 51.47) --
	(157.35, 51.53) --
	(157.92, 51.61) --
	(158.49, 51.72) --
	(159.06, 51.85) --
	(159.64, 52.02) --
	(160.21, 52.21) --
	(160.78, 52.41) --
	(161.35, 52.62) --
	(161.92, 52.81) --
	(162.50, 52.98) --
	(163.07, 53.09) --
	(163.64, 53.14) --
	(164.21, 53.13) --
	(164.78, 53.05) --
	(165.36, 52.92) --
	(165.93, 52.75) --
	(166.50, 52.55) --
	(167.07, 52.34) --
	(167.64, 52.14) --
	(168.22, 51.96) --
	(168.79, 51.80) --
	(169.36, 51.67) --
	(169.93, 51.57) --
	(170.50, 51.50) --
	(171.08, 51.45) --
	(171.65, 51.41) --
	(172.22, 51.39) --
	(172.79, 51.37) --
	(173.36, 51.37) --
	(173.94, 51.36) --
	(174.51, 51.36) --
	(175.08, 51.37) --
	(175.65, 51.38) --
	(176.22, 51.40) --
	(176.80, 51.43) --
	(177.37, 51.47) --
	(177.94, 51.54) --
	(178.51, 51.63) --
	(179.08, 51.76) --
	(179.66, 51.92) --
	(180.23, 52.11) --
	(180.80, 52.34) --
	(181.37, 52.60) --
	(181.94, 52.88) --
	(182.52, 53.17) --
	(183.09, 53.45) --
	(183.66, 53.71) --
	(184.23, 53.93) --
	(184.81, 54.11) --
	(185.38, 54.24) --
	(185.95, 54.29) --
	(186.52, 54.28) --
	(187.09, 54.21) --
	(187.67, 54.07) --
	(188.24, 53.88) --
	(188.81, 53.65) --
	(189.38, 53.39) --
	(189.95, 53.10) --
	(190.53, 52.82) --
	(191.10, 52.54) --
	(191.67, 52.29) --
	(192.24, 52.07) --
	(192.81, 51.88) --
	(193.39, 51.74) --
	(193.96, 51.64) --
	(194.53, 51.57) --
	(195.10, 51.54) --
	(195.67, 51.55) --
	(196.25, 51.59) --
	(196.82, 51.68) --
	(197.39, 51.81) --
	(197.96, 51.98) --
	(198.53, 52.21) --
	(199.11, 52.49) --
	(199.68, 52.82) --
	(200.25, 53.20) --
	(200.82, 53.60) --
	(201.39, 54.01) --
	(201.97, 54.41) --
	(202.54, 54.78) --
	(203.11, 55.12) --
	(203.68, 55.40) --
	(204.25, 55.62) --
	(204.83, 55.79) --
	(205.40, 55.89) --
	(205.97, 55.95) --
	(206.54, 55.96) --
	(207.11, 55.92) --
	(207.69, 55.84) --
	(208.26, 55.72) --
	(208.83, 55.55) --
	(209.40, 55.33) --
	(209.98, 55.07) --
	(210.55, 54.76) --
	(211.12, 54.41) --
	(211.69, 54.04) --
	(212.26, 53.66) --
	(212.84, 53.28) --
	(213.41, 52.91) --
	(213.98, 52.58) --
	(214.55, 52.29) --
	(215.12, 52.05) --
	(215.70, 51.85) --
	(216.27, 51.70) --
	(216.84, 51.58) --
	(217.41, 51.50) --
	(217.98, 51.45) --
	(218.56, 51.41) --
	(219.13, 51.39) --
	(219.70, 51.37) --
	(220.27, 51.36) --
	(220.84, 51.36) --
	(221.42, 51.36) --
	(221.99, 51.35) --
	(222.56, 51.35) --
	(223.13, 51.35) --
	(223.70, 51.35) --
	(224.28, 51.35) --
	(224.85, 51.35) --
	(225.42, 51.35) --
	(225.99, 51.35) --
	(226.56, 51.35) --
	(227.14, 51.35) --
	(227.71, 51.35) --
	(228.28, 51.35) --
	(228.85, 51.35) --
	(229.42, 51.35) --
	(230.00, 51.35) --
	(230.57, 51.35) --
	(231.14, 51.35) --
	(231.71, 51.35) --
	(232.28, 51.35) --
	(232.86, 51.35) --
	(233.43, 51.35) --
	(234.00, 51.35) --
	(234.57, 51.35) --
	(235.15, 51.35) --
	(235.72, 51.35) --
	(236.29, 51.35) --
	(236.86, 51.35) --
	(237.43, 51.35) --
	(238.01, 51.35) --
	(238.58, 51.35) --
	(239.15, 51.35) --
	(239.72, 51.35) --
	(240.29, 51.35) --
	(240.87, 51.35) --
	(241.44, 51.35) --
	(242.01, 51.35) --
	(242.58, 51.35) --
	(243.15, 51.35) --
	(243.73, 51.35) --
	(244.30, 51.35) --
	(244.87, 51.35) --
	(245.44, 51.35) --
	(246.01, 51.35) --
	(246.59, 51.35) --
	(247.16, 51.35) --
	(247.73, 51.35) --
	(248.30, 51.35) --
	(248.87, 51.35) --
	(249.45, 51.35) --
	(250.02, 51.35) --
	(250.59, 51.35) --
	(251.16, 51.35) --
	(251.73, 51.35) --
	(252.31, 51.35) --
	(252.88, 51.35) --
	(253.45, 51.36) --
	(254.02, 51.36) --
	(254.59, 51.36) --
	(255.17, 51.37) --
	(255.74, 51.38) --
	(256.31, 51.40) --
	(256.88, 51.44) --
	(257.45, 51.49) --
	(258.03, 51.55) --
	(258.60, 51.65) --
	(259.17, 51.77) --
	(259.74, 51.92) --
	(260.32, 52.09) --
	(260.89, 52.29) --
	(261.46, 52.50) --
	(262.03, 52.70) --
	(262.60, 52.88) --
	(263.18, 53.02) --
	(263.75, 53.11) --
	(264.32, 53.14) --
	(264.89, 53.10) --
	(265.46, 53.00) --
	(266.04, 52.85) --
	(266.61, 52.67) --
	(267.18, 52.47) --
	(267.75, 52.26) --
	(268.32, 52.07) --
	(268.90, 51.89) --
	(269.47, 51.74) --
	(270.04, 51.63) --
	(270.61, 51.54) --
	(271.18, 51.48) --
	(271.76, 51.43) --
	(272.33, 51.40) --
	(272.90, 51.38) --
	(273.47, 51.37) --
	(274.04, 51.36) --
	(274.62, 51.36) --
	(275.19, 51.36) --
	(275.76, 51.35) --
	(276.33, 51.35) --
	(276.90, 51.35) --
	(277.48, 51.35) --
	(278.05, 51.35) --
	(278.62, 51.35) --
	(279.19, 51.35) --
	(279.76, 51.35) --
	(280.34, 51.35) --
	(280.91, 51.35) --
	(281.48, 51.35) --
	(282.05, 51.35) --
	(282.62, 51.35) --
	(283.20, 51.35) --
	(283.77, 51.35) --
	(284.34, 51.35) --
	(284.91, 51.35) --
	(285.49, 51.35) --
	(286.06, 51.35) --
	(286.63, 51.35) --
	(287.20, 51.35) --
	(287.77, 51.35) --
	(288.35, 51.35) --
	(288.92, 51.35) --
	(289.49, 51.35) --
	(290.06, 51.35) --
	(290.63, 51.35) --
	(291.21, 51.35) --
	(291.78, 51.35) --
	(292.35, 51.35) --
	(292.92, 51.35) --
	(293.49, 51.35) --
	(294.07, 51.35) --
	(294.64, 51.35) --
	(295.21, 51.35) --
	(295.78, 51.35) --
	(296.35, 51.35) --
	(296.93, 51.35) --
	(297.50, 51.35) --
	(298.07, 51.35) --
	(298.64, 51.35) --
	(299.21, 51.35) --
	(299.79, 51.35) --
	(300.36, 51.35) --
	(300.93, 51.35) --
	(301.50, 51.35) --
	(302.07, 51.35) --
	(302.65, 51.35) --
	(303.22, 51.35) --
	(303.79, 51.35) --
	(304.36, 51.35) --
	(304.93, 51.35) --
	(305.51, 51.35) --
	(306.08, 51.35) --
	(306.65, 51.35) --
	(307.22, 51.35) --
	(307.79, 51.35) --
	(308.37, 51.35) --
	(308.94, 51.35) --
	(309.51, 51.35) --
	(310.08, 51.35) --
	(310.66, 51.35) --
	(311.23, 51.35) --
	(311.80, 51.35) --
	(312.37, 51.35) --
	(312.94, 51.35) --
	(313.52, 51.35) --
	(314.09, 51.35) --
	(314.66, 51.35) --
	(315.23, 51.35) --
	(315.80, 51.35) --
	(316.38, 51.35) --
	(316.95, 51.35) --
	(317.52, 51.35) --
	(318.09, 51.35) --
	(318.66, 51.35) --
	(319.24, 51.36) --
	(319.81, 51.36) --
	(320.38, 51.36) --
	(320.95, 51.37) --
	(321.52, 51.38) --
	(322.10, 51.39) --
	(322.67, 51.42) --
	(323.24, 51.45) --
	(323.81, 51.51) --
	(324.38, 51.59) --
	(324.96, 51.70) --
	(325.53, 51.83) --
	(326.10, 51.99) --
	(326.67, 52.18) --
	(327.24, 52.38) --
	(327.82, 52.58) --
	(328.39, 52.78) --
	(328.96, 52.94) --
	(329.53, 53.07) --
	(330.10, 53.13) --
	(330.68, 53.13) --
	(331.25, 53.07) --
	(331.82, 52.94) --
	(332.39, 52.78) --
	(332.96, 52.58) --
	(333.54, 52.38) --
	(334.11, 52.17) --
	(334.68, 51.99) --
	(335.25, 51.83) --
	(335.83, 51.69) --
	(336.40, 51.59) --
	(336.97, 51.51) --
	(337.54, 51.45) --
	(338.11, 51.41) --
	(338.69, 51.39) --
	(339.26, 51.37);
\end{scope}
\begin{scope}
\path[clip] (  0.00,  0.00) rectangle (361.35,144.54);
\definecolor{drawColor}{RGB}{0,0,0}

\node[text=drawColor,anchor=base,inner sep=0pt, outer sep=0pt, scale=  1.00] at (201.68,  8.40) {\textit{RMSE} (ms)};

\node[text=drawColor,rotate= 90.00,anchor=base,inner sep=0pt, outer sep=0pt, scale=  1.00] at (  9.60, 93.27) {Dichte};
\end{scope}
\begin{scope}
\path[clip] (  0.00,  0.00) rectangle (361.35,144.54);
\definecolor{drawColor}{RGB}{0,0,0}

\path[draw=drawColor,line width= 0.4pt,line join=round,line cap=round] ( 53.83, 48.00) -- (349.52, 48.00);

\path[draw=drawColor,line width= 0.4pt,line join=round,line cap=round] ( 53.83, 48.00) -- ( 53.83, 42.00);

\path[draw=drawColor,line width= 0.4pt,line join=round,line cap=round] ( 96.07, 48.00) -- ( 96.07, 42.00);

\path[draw=drawColor,line width= 0.4pt,line join=round,line cap=round] (138.31, 48.00) -- (138.31, 42.00);

\path[draw=drawColor,line width= 0.4pt,line join=round,line cap=round] (180.55, 48.00) -- (180.55, 42.00);

\path[draw=drawColor,line width= 0.4pt,line join=round,line cap=round] (222.80, 48.00) -- (222.80, 42.00);

\path[draw=drawColor,line width= 0.4pt,line join=round,line cap=round] (265.04, 48.00) -- (265.04, 42.00);

\path[draw=drawColor,line width= 0.4pt,line join=round,line cap=round] (307.28, 48.00) -- (307.28, 42.00);

\path[draw=drawColor,line width= 0.4pt,line join=round,line cap=round] (349.52, 48.00) -- (349.52, 42.00);

\node[text=drawColor,anchor=base,inner sep=0pt, outer sep=0pt, scale=  1.00] at ( 53.83, 30.00) {0};

\node[text=drawColor,anchor=base,inner sep=0pt, outer sep=0pt, scale=  1.00] at ( 96.07, 30.00) {10};

\node[text=drawColor,anchor=base,inner sep=0pt, outer sep=0pt, scale=  1.00] at (138.31, 30.00) {20};

\node[text=drawColor,anchor=base,inner sep=0pt, outer sep=0pt, scale=  1.00] at (180.55, 30.00) {30};

\node[text=drawColor,anchor=base,inner sep=0pt, outer sep=0pt, scale=  1.00] at (222.80, 30.00) {40};

\node[text=drawColor,anchor=base,inner sep=0pt, outer sep=0pt, scale=  1.00] at (265.04, 30.00) {50};

\node[text=drawColor,anchor=base,inner sep=0pt, outer sep=0pt, scale=  1.00] at (307.28, 30.00) {60};

\node[text=drawColor,anchor=base,inner sep=0pt, outer sep=0pt, scale=  1.00] at (349.52, 30.00) {70};

\path[draw=drawColor,line width= 0.4pt,line join=round,line cap=round] ( 42.00, 51.35) -- ( 42.00,135.19);

\path[draw=drawColor,line width= 0.4pt,line join=round,line cap=round] ( 42.00, 51.35) -- ( 36.00, 51.35);

\path[draw=drawColor,line width= 0.4pt,line join=round,line cap=round] ( 42.00, 79.30) -- ( 36.00, 79.30);

\path[draw=drawColor,line width= 0.4pt,line join=round,line cap=round] ( 42.00,107.24) -- ( 36.00,107.24);

\path[draw=drawColor,line width= 0.4pt,line join=round,line cap=round] ( 42.00,135.19) -- ( 36.00,135.19);

\node[text=drawColor,anchor=base east,inner sep=0pt, outer sep=0pt, scale=  1.00] at ( 33.60, 47.91) {0.00};

\node[text=drawColor,anchor=base east,inner sep=0pt, outer sep=0pt, scale=  1.00] at ( 33.60, 75.85) {0.05};

\node[text=drawColor,anchor=base east,inner sep=0pt, outer sep=0pt, scale=  1.00] at ( 33.60,103.80) {0.10};

\node[text=drawColor,anchor=base east,inner sep=0pt, outer sep=0pt, scale=  1.00] at ( 33.60,131.74) {0.15};

\path[draw=drawColor,line width= 0.2pt,line join=round,line cap=round] ( 55.81, 48.00) -- ( 55.81, 57.05);

\path[draw=drawColor,line width= 0.2pt,line join=round,line cap=round] ( 56.15, 48.00) -- ( 56.15, 57.05);

\path[draw=drawColor,line width= 0.2pt,line join=round,line cap=round] ( 57.16, 48.00) -- ( 57.16, 57.05);

\path[draw=drawColor,line width= 0.2pt,line join=round,line cap=round] ( 58.01, 48.00) -- ( 58.01, 57.05);

\path[draw=drawColor,line width= 0.2pt,line join=round,line cap=round] ( 59.19, 48.00) -- ( 59.19, 57.05);

\path[draw=drawColor,line width= 0.2pt,line join=round,line cap=round] ( 59.28, 48.00) -- ( 59.28, 57.05);

\path[draw=drawColor,line width= 0.2pt,line join=round,line cap=round] ( 60.08, 48.00) -- ( 60.08, 57.05);

\path[draw=drawColor,line width= 0.2pt,line join=round,line cap=round] ( 60.71, 48.00) -- ( 60.71, 57.05);

\path[draw=drawColor,line width= 0.2pt,line join=round,line cap=round] ( 61.39, 48.00) -- ( 61.39, 57.05);

\path[draw=drawColor,line width= 0.2pt,line join=round,line cap=round] ( 61.81, 48.00) -- ( 61.81, 57.05);

\path[draw=drawColor,line width= 0.2pt,line join=round,line cap=round] ( 62.11, 48.00) -- ( 62.11, 57.05);

\path[draw=drawColor,line width= 0.2pt,line join=round,line cap=round] ( 62.53, 48.00) -- ( 62.53, 57.05);

\path[draw=drawColor,line width= 0.2pt,line join=round,line cap=round] ( 62.95, 48.00) -- ( 62.95, 57.05);

\path[draw=drawColor,line width= 0.2pt,line join=round,line cap=round] ( 63.04, 48.00) -- ( 63.04, 57.05);

\path[draw=drawColor,line width= 0.2pt,line join=round,line cap=round] ( 63.37, 48.00) -- ( 63.37, 57.05);

\path[draw=drawColor,line width= 0.2pt,line join=round,line cap=round] ( 63.50, 48.00) -- ( 63.50, 57.05);

\path[draw=drawColor,line width= 0.2pt,line join=round,line cap=round] ( 64.05, 48.00) -- ( 64.05, 57.05);

\path[draw=drawColor,line width= 0.2pt,line join=round,line cap=round] ( 64.09, 48.00) -- ( 64.09, 57.05);

\path[draw=drawColor,line width= 0.2pt,line join=round,line cap=round] ( 64.13, 48.00) -- ( 64.13, 57.05);

\path[draw=drawColor,line width= 0.2pt,line join=round,line cap=round] ( 64.18, 48.00) -- ( 64.18, 57.05);

\path[draw=drawColor,line width= 0.2pt,line join=round,line cap=round] ( 64.26, 48.00) -- ( 64.26, 57.05);

\path[draw=drawColor,line width= 0.2pt,line join=round,line cap=round] ( 64.30, 48.00) -- ( 64.30, 57.05);

\path[draw=drawColor,line width= 0.2pt,line join=round,line cap=round] ( 64.43, 48.00) -- ( 64.43, 57.05);

\path[draw=drawColor,line width= 0.2pt,line join=round,line cap=round] ( 64.64, 48.00) -- ( 64.64, 57.05);

\path[draw=drawColor,line width= 0.2pt,line join=round,line cap=round] ( 64.77, 48.00) -- ( 64.77, 57.05);

\path[draw=drawColor,line width= 0.2pt,line join=round,line cap=round] ( 65.53, 48.00) -- ( 65.53, 57.05);

\path[draw=drawColor,line width= 0.2pt,line join=round,line cap=round] ( 65.57, 48.00) -- ( 65.57, 57.05);

\path[draw=drawColor,line width= 0.2pt,line join=round,line cap=round] ( 65.91, 48.00) -- ( 65.91, 57.05);

\path[draw=drawColor,line width= 0.2pt,line join=round,line cap=round] ( 65.95, 48.00) -- ( 65.95, 57.05);

\path[draw=drawColor,line width= 0.2pt,line join=round,line cap=round] ( 66.04, 48.00) -- ( 66.04, 57.05);

\path[draw=drawColor,line width= 0.2pt,line join=round,line cap=round] ( 66.20, 48.00) -- ( 66.20, 57.05);

\path[draw=drawColor,line width= 0.2pt,line join=round,line cap=round] ( 66.29, 48.00) -- ( 66.29, 57.05);

\path[draw=drawColor,line width= 0.2pt,line join=round,line cap=round] ( 66.54, 48.00) -- ( 66.54, 57.05);

\path[draw=drawColor,line width= 0.2pt,line join=round,line cap=round] ( 66.63, 48.00) -- ( 66.63, 57.05);

\path[draw=drawColor,line width= 0.2pt,line join=round,line cap=round] ( 66.80, 48.00) -- ( 66.80, 57.05);

\path[draw=drawColor,line width= 0.2pt,line join=round,line cap=round] ( 66.92, 48.00) -- ( 66.92, 57.05);

\path[draw=drawColor,line width= 0.2pt,line join=round,line cap=round] ( 67.43, 48.00) -- ( 67.43, 57.05);

\path[draw=drawColor,line width= 0.2pt,line join=round,line cap=round] ( 67.47, 48.00) -- ( 67.47, 57.05);

\path[draw=drawColor,line width= 0.2pt,line join=round,line cap=round] ( 67.73, 48.00) -- ( 67.73, 57.05);

\path[draw=drawColor,line width= 0.2pt,line join=round,line cap=round] ( 68.32, 48.00) -- ( 68.32, 57.05);

\path[draw=drawColor,line width= 0.2pt,line join=round,line cap=round] ( 68.44, 48.00) -- ( 68.44, 57.05);

\path[draw=drawColor,line width= 0.2pt,line join=round,line cap=round] ( 68.61, 48.00) -- ( 68.61, 57.05);

\path[draw=drawColor,line width= 0.2pt,line join=round,line cap=round] ( 68.61, 48.00) -- ( 68.61, 57.05);

\path[draw=drawColor,line width= 0.2pt,line join=round,line cap=round] ( 68.95, 48.00) -- ( 68.95, 57.05);

\path[draw=drawColor,line width= 0.2pt,line join=round,line cap=round] ( 70.13, 48.00) -- ( 70.13, 57.05);

\path[draw=drawColor,line width= 0.2pt,line join=round,line cap=round] ( 70.26, 48.00) -- ( 70.26, 57.05);

\path[draw=drawColor,line width= 0.2pt,line join=round,line cap=round] ( 70.47, 48.00) -- ( 70.47, 57.05);

\path[draw=drawColor,line width= 0.2pt,line join=round,line cap=round] ( 70.51, 48.00) -- ( 70.51, 57.05);

\path[draw=drawColor,line width= 0.2pt,line join=round,line cap=round] ( 70.77, 48.00) -- ( 70.77, 57.05);

\path[draw=drawColor,line width= 0.2pt,line join=round,line cap=round] ( 70.94, 48.00) -- ( 70.94, 57.05);

\path[draw=drawColor,line width= 0.2pt,line join=round,line cap=round] ( 71.02, 48.00) -- ( 71.02, 57.05);

\path[draw=drawColor,line width= 0.2pt,line join=round,line cap=round] ( 71.44, 48.00) -- ( 71.44, 57.05);

\path[draw=drawColor,line width= 0.2pt,line join=round,line cap=round] ( 71.48, 48.00) -- ( 71.48, 57.05);

\path[draw=drawColor,line width= 0.2pt,line join=round,line cap=round] ( 71.78, 48.00) -- ( 71.78, 57.05);

\path[draw=drawColor,line width= 0.2pt,line join=round,line cap=round] ( 72.08, 48.00) -- ( 72.08, 57.05);

\path[draw=drawColor,line width= 0.2pt,line join=round,line cap=round] ( 72.33, 48.00) -- ( 72.33, 57.05);

\path[draw=drawColor,line width= 0.2pt,line join=round,line cap=round] ( 72.54, 48.00) -- ( 72.54, 57.05);

\path[draw=drawColor,line width= 0.2pt,line join=round,line cap=round] ( 72.63, 48.00) -- ( 72.63, 57.05);

\path[draw=drawColor,line width= 0.2pt,line join=round,line cap=round] ( 72.84, 48.00) -- ( 72.84, 57.05);

\path[draw=drawColor,line width= 0.2pt,line join=round,line cap=round] ( 72.92, 48.00) -- ( 72.92, 57.05);

\path[draw=drawColor,line width= 0.2pt,line join=round,line cap=round] ( 72.96, 48.00) -- ( 72.96, 57.05);

\path[draw=drawColor,line width= 0.2pt,line join=round,line cap=round] ( 73.09, 48.00) -- ( 73.09, 57.05);

\path[draw=drawColor,line width= 0.2pt,line join=round,line cap=round] ( 73.26, 48.00) -- ( 73.26, 57.05);

\path[draw=drawColor,line width= 0.2pt,line join=round,line cap=round] ( 73.43, 48.00) -- ( 73.43, 57.05);

\path[draw=drawColor,line width= 0.2pt,line join=round,line cap=round] ( 73.47, 48.00) -- ( 73.47, 57.05);

\path[draw=drawColor,line width= 0.2pt,line join=round,line cap=round] ( 73.47, 48.00) -- ( 73.47, 57.05);

\path[draw=drawColor,line width= 0.2pt,line join=round,line cap=round] ( 73.55, 48.00) -- ( 73.55, 57.05);

\path[draw=drawColor,line width= 0.2pt,line join=round,line cap=round] ( 73.68, 48.00) -- ( 73.68, 57.05);

\path[draw=drawColor,line width= 0.2pt,line join=round,line cap=round] ( 74.27, 48.00) -- ( 74.27, 57.05);

\path[draw=drawColor,line width= 0.2pt,line join=round,line cap=round] ( 74.32, 48.00) -- ( 74.32, 57.05);

\path[draw=drawColor,line width= 0.2pt,line join=round,line cap=round] ( 74.32, 48.00) -- ( 74.32, 57.05);

\path[draw=drawColor,line width= 0.2pt,line join=round,line cap=round] ( 74.57, 48.00) -- ( 74.57, 57.05);

\path[draw=drawColor,line width= 0.2pt,line join=round,line cap=round] ( 74.61, 48.00) -- ( 74.61, 57.05);

\path[draw=drawColor,line width= 0.2pt,line join=round,line cap=round] ( 74.65, 48.00) -- ( 74.65, 57.05);

\path[draw=drawColor,line width= 0.2pt,line join=round,line cap=round] ( 74.99, 48.00) -- ( 74.99, 57.05);

\path[draw=drawColor,line width= 0.2pt,line join=round,line cap=round] ( 75.41, 48.00) -- ( 75.41, 57.05);

\path[draw=drawColor,line width= 0.2pt,line join=round,line cap=round] ( 75.41, 48.00) -- ( 75.41, 57.05);

\path[draw=drawColor,line width= 0.2pt,line join=round,line cap=round] ( 75.46, 48.00) -- ( 75.46, 57.05);

\path[draw=drawColor,line width= 0.2pt,line join=round,line cap=round] ( 75.62, 48.00) -- ( 75.62, 57.05);

\path[draw=drawColor,line width= 0.2pt,line join=round,line cap=round] ( 75.67, 48.00) -- ( 75.67, 57.05);

\path[draw=drawColor,line width= 0.2pt,line join=round,line cap=round] ( 75.75, 48.00) -- ( 75.75, 57.05);

\path[draw=drawColor,line width= 0.2pt,line join=round,line cap=round] ( 76.60, 48.00) -- ( 76.60, 57.05);

\path[draw=drawColor,line width= 0.2pt,line join=round,line cap=round] ( 76.68, 48.00) -- ( 76.68, 57.05);

\path[draw=drawColor,line width= 0.2pt,line join=round,line cap=round] ( 76.89, 48.00) -- ( 76.89, 57.05);

\path[draw=drawColor,line width= 0.2pt,line join=round,line cap=round] ( 76.93, 48.00) -- ( 76.93, 57.05);

\path[draw=drawColor,line width= 0.2pt,line join=round,line cap=round] ( 77.15, 48.00) -- ( 77.15, 57.05);

\path[draw=drawColor,line width= 0.2pt,line join=round,line cap=round] ( 77.61, 48.00) -- ( 77.61, 57.05);

\path[draw=drawColor,line width= 0.2pt,line join=round,line cap=round] ( 78.07, 48.00) -- ( 78.07, 57.05);

\path[draw=drawColor,line width= 0.2pt,line join=round,line cap=round] ( 78.45, 48.00) -- ( 78.45, 57.05);

\path[draw=drawColor,line width= 0.2pt,line join=round,line cap=round] ( 79.09, 48.00) -- ( 79.09, 57.05);

\path[draw=drawColor,line width= 0.2pt,line join=round,line cap=round] ( 79.68, 48.00) -- ( 79.68, 57.05);

\path[draw=drawColor,line width= 0.2pt,line join=round,line cap=round] ( 79.72, 48.00) -- ( 79.72, 57.05);

\path[draw=drawColor,line width= 0.2pt,line join=round,line cap=round] ( 79.76, 48.00) -- ( 79.76, 57.05);

\path[draw=drawColor,line width= 0.2pt,line join=round,line cap=round] ( 79.98, 48.00) -- ( 79.98, 57.05);

\path[draw=drawColor,line width= 0.2pt,line join=round,line cap=round] ( 80.14, 48.00) -- ( 80.14, 57.05);

\path[draw=drawColor,line width= 0.2pt,line join=round,line cap=round] ( 80.23, 48.00) -- ( 80.23, 57.05);

\path[draw=drawColor,line width= 0.2pt,line join=round,line cap=round] ( 80.31, 48.00) -- ( 80.31, 57.05);

\path[draw=drawColor,line width= 0.2pt,line join=round,line cap=round] ( 80.36, 48.00) -- ( 80.36, 57.05);

\path[draw=drawColor,line width= 0.2pt,line join=round,line cap=round] ( 80.40, 48.00) -- ( 80.40, 57.05);

\path[draw=drawColor,line width= 0.2pt,line join=round,line cap=round] ( 81.12, 48.00) -- ( 81.12, 57.05);

\path[draw=drawColor,line width= 0.2pt,line join=round,line cap=round] ( 81.71, 48.00) -- ( 81.71, 57.05);

\path[draw=drawColor,line width= 0.2pt,line join=round,line cap=round] ( 81.83, 48.00) -- ( 81.83, 57.05);

\path[draw=drawColor,line width= 0.2pt,line join=round,line cap=round] ( 82.00, 48.00) -- ( 82.00, 57.05);

\path[draw=drawColor,line width= 0.2pt,line join=round,line cap=round] ( 82.34, 48.00) -- ( 82.34, 57.05);

\path[draw=drawColor,line width= 0.2pt,line join=round,line cap=round] ( 82.47, 48.00) -- ( 82.47, 57.05);

\path[draw=drawColor,line width= 0.2pt,line join=round,line cap=round] ( 82.72, 48.00) -- ( 82.72, 57.05);

\path[draw=drawColor,line width= 0.2pt,line join=round,line cap=round] ( 83.02, 48.00) -- ( 83.02, 57.05);

\path[draw=drawColor,line width= 0.2pt,line join=round,line cap=round] ( 83.06, 48.00) -- ( 83.06, 57.05);

\path[draw=drawColor,line width= 0.2pt,line join=round,line cap=round] ( 83.65, 48.00) -- ( 83.65, 57.05);

\path[draw=drawColor,line width= 0.2pt,line join=round,line cap=round] ( 83.65, 48.00) -- ( 83.65, 57.05);

\path[draw=drawColor,line width= 0.2pt,line join=round,line cap=round] ( 84.28, 48.00) -- ( 84.28, 57.05);

\path[draw=drawColor,line width= 0.2pt,line join=round,line cap=round] ( 84.37, 48.00) -- ( 84.37, 57.05);

\path[draw=drawColor,line width= 0.2pt,line join=round,line cap=round] ( 84.50, 48.00) -- ( 84.50, 57.05);

\path[draw=drawColor,line width= 0.2pt,line join=round,line cap=round] ( 84.71, 48.00) -- ( 84.71, 57.05);

\path[draw=drawColor,line width= 0.2pt,line join=round,line cap=round] ( 84.83, 48.00) -- ( 84.83, 57.05);

\path[draw=drawColor,line width= 0.2pt,line join=round,line cap=round] ( 84.96, 48.00) -- ( 84.96, 57.05);

\path[draw=drawColor,line width= 0.2pt,line join=round,line cap=round] ( 84.96, 48.00) -- ( 84.96, 57.05);

\path[draw=drawColor,line width= 0.2pt,line join=round,line cap=round] ( 85.68, 48.00) -- ( 85.68, 57.05);

\path[draw=drawColor,line width= 0.2pt,line join=round,line cap=round] ( 85.76, 48.00) -- ( 85.76, 57.05);

\path[draw=drawColor,line width= 0.2pt,line join=round,line cap=round] ( 85.85, 48.00) -- ( 85.85, 57.05);

\path[draw=drawColor,line width= 0.2pt,line join=round,line cap=round] ( 85.89, 48.00) -- ( 85.89, 57.05);

\path[draw=drawColor,line width= 0.2pt,line join=round,line cap=round] ( 86.06, 48.00) -- ( 86.06, 57.05);

\path[draw=drawColor,line width= 0.2pt,line join=round,line cap=round] ( 86.10, 48.00) -- ( 86.10, 57.05);

\path[draw=drawColor,line width= 0.2pt,line join=round,line cap=round] ( 86.40, 48.00) -- ( 86.40, 57.05);

\path[draw=drawColor,line width= 0.2pt,line join=round,line cap=round] ( 86.44, 48.00) -- ( 86.44, 57.05);

\path[draw=drawColor,line width= 0.2pt,line join=round,line cap=round] ( 87.33, 48.00) -- ( 87.33, 57.05);

\path[draw=drawColor,line width= 0.2pt,line join=round,line cap=round] ( 87.58, 48.00) -- ( 87.58, 57.05);

\path[draw=drawColor,line width= 0.2pt,line join=round,line cap=round] ( 87.87, 48.00) -- ( 87.87, 57.05);

\path[draw=drawColor,line width= 0.2pt,line join=round,line cap=round] ( 88.13, 48.00) -- ( 88.13, 57.05);

\path[draw=drawColor,line width= 0.2pt,line join=round,line cap=round] ( 88.97, 48.00) -- ( 88.97, 57.05);

\path[draw=drawColor,line width= 0.2pt,line join=round,line cap=round] ( 89.31, 48.00) -- ( 89.31, 57.05);

\path[draw=drawColor,line width= 0.2pt,line join=round,line cap=round] ( 89.52, 48.00) -- ( 89.52, 57.05);

\path[draw=drawColor,line width= 0.2pt,line join=round,line cap=round] ( 89.78, 48.00) -- ( 89.78, 57.05);

\path[draw=drawColor,line width= 0.2pt,line join=round,line cap=round] ( 89.82, 48.00) -- ( 89.82, 57.05);

\path[draw=drawColor,line width= 0.2pt,line join=round,line cap=round] ( 91.34, 48.00) -- ( 91.34, 57.05);

\path[draw=drawColor,line width= 0.2pt,line join=round,line cap=round] ( 94.13, 48.00) -- ( 94.13, 57.05);

\path[draw=drawColor,line width= 0.2pt,line join=round,line cap=round] ( 94.76, 48.00) -- ( 94.76, 57.05);

\path[draw=drawColor,line width= 0.2pt,line join=round,line cap=round] ( 95.69, 48.00) -- ( 95.69, 57.05);

\path[draw=drawColor,line width= 0.2pt,line join=round,line cap=round] ( 97.00, 48.00) -- ( 97.00, 57.05);

\path[draw=drawColor,line width= 0.2pt,line join=round,line cap=round] ( 97.08, 48.00) -- ( 97.08, 57.05);

\path[draw=drawColor,line width= 0.2pt,line join=round,line cap=round] ( 99.41, 48.00) -- ( 99.41, 57.05);

\path[draw=drawColor,line width= 0.2pt,line join=round,line cap=round] ( 99.45, 48.00) -- ( 99.45, 57.05);

\path[draw=drawColor,line width= 0.2pt,line join=round,line cap=round] ( 99.70, 48.00) -- ( 99.70, 57.05);

\path[draw=drawColor,line width= 0.2pt,line join=round,line cap=round] (100.00, 48.00) -- (100.00, 57.05);

\path[draw=drawColor,line width= 0.2pt,line join=round,line cap=round] (100.89, 48.00) -- (100.89, 57.05);

\path[draw=drawColor,line width= 0.2pt,line join=round,line cap=round] (101.01, 48.00) -- (101.01, 57.05);

\path[draw=drawColor,line width= 0.2pt,line join=round,line cap=round] (101.77, 48.00) -- (101.77, 57.05);

\path[draw=drawColor,line width= 0.2pt,line join=round,line cap=round] (102.07, 48.00) -- (102.07, 57.05);

\path[draw=drawColor,line width= 0.2pt,line join=round,line cap=round] (105.36, 48.00) -- (105.36, 57.05);

\path[draw=drawColor,line width= 0.2pt,line join=round,line cap=round] (105.49, 48.00) -- (105.49, 57.05);

\path[draw=drawColor,line width= 0.2pt,line join=round,line cap=round] (108.11, 48.00) -- (108.11, 57.05);

\path[draw=drawColor,line width= 0.2pt,line join=round,line cap=round] (108.62, 48.00) -- (108.62, 57.05);

\path[draw=drawColor,line width= 0.2pt,line join=round,line cap=round] (110.01, 48.00) -- (110.01, 57.05);

\path[draw=drawColor,line width= 0.2pt,line join=round,line cap=round] (110.14, 48.00) -- (110.14, 57.05);

\path[draw=drawColor,line width= 0.2pt,line join=round,line cap=round] (113.43, 48.00) -- (113.43, 57.05);

\path[draw=drawColor,line width= 0.2pt,line join=round,line cap=round] (113.56, 48.00) -- (113.56, 57.05);

\path[draw=drawColor,line width= 0.2pt,line join=round,line cap=round] (114.23, 48.00) -- (114.23, 57.05);

\path[draw=drawColor,line width= 0.2pt,line join=round,line cap=round] (116.52, 48.00) -- (116.52, 57.05);

\path[draw=drawColor,line width= 0.2pt,line join=round,line cap=round] (118.25, 48.00) -- (118.25, 57.05);

\path[draw=drawColor,line width= 0.2pt,line join=round,line cap=round] (119.89, 48.00) -- (119.89, 57.05);

\path[draw=drawColor,line width= 0.2pt,line join=round,line cap=round] (122.94, 48.00) -- (122.94, 57.05);

\path[draw=drawColor,line width= 0.2pt,line join=round,line cap=round] (123.32, 48.00) -- (123.32, 57.05);

\path[draw=drawColor,line width= 0.2pt,line join=round,line cap=round] (124.84, 48.00) -- (124.84, 57.05);

\path[draw=drawColor,line width= 0.2pt,line join=round,line cap=round] (128.43, 48.00) -- (128.43, 57.05);

\path[draw=drawColor,line width= 0.2pt,line join=round,line cap=round] (128.72, 48.00) -- (128.72, 57.05);

\path[draw=drawColor,line width= 0.2pt,line join=round,line cap=round] (128.85, 48.00) -- (128.85, 57.05);

\path[draw=drawColor,line width= 0.2pt,line join=round,line cap=round] (132.02, 48.00) -- (132.02, 57.05);

\path[draw=drawColor,line width= 0.2pt,line join=round,line cap=round] (147.31, 48.00) -- (147.31, 57.05);

\path[draw=drawColor,line width= 0.2pt,line join=round,line cap=round] (163.83, 48.00) -- (163.83, 57.05);

\path[draw=drawColor,line width= 0.2pt,line join=round,line cap=round] (184.31, 48.00) -- (184.31, 57.05);

\path[draw=drawColor,line width= 0.2pt,line join=round,line cap=round] (188.03, 48.00) -- (188.03, 57.05);

\path[draw=drawColor,line width= 0.2pt,line join=round,line cap=round] (202.69, 48.00) -- (202.69, 57.05);

\path[draw=drawColor,line width= 0.2pt,line join=round,line cap=round] (205.01, 48.00) -- (205.01, 57.05);

\path[draw=drawColor,line width= 0.2pt,line join=round,line cap=round] (208.01, 48.00) -- (208.01, 57.05);

\path[draw=drawColor,line width= 0.2pt,line join=round,line cap=round] (210.67, 48.00) -- (210.67, 57.05);

\path[draw=drawColor,line width= 0.2pt,line join=round,line cap=round] (264.28, 48.00) -- (264.28, 57.05);

\path[draw=drawColor,line width= 0.2pt,line join=round,line cap=round] (330.39, 48.00) -- (330.39, 57.05);
\end{scope}
\end{tikzpicture}

	\end{adjustbox}
	\caption[Dichtefunktion des aus der \gls{ssauf} mit einer exponentiellen Regression abgeleiteten \gls{rmse}]{Dichtefunktion des aus der \gls{ssauf} mit einer exponentiellen Regression abgeleiteten Root Mean Square Error (\gls{rmse}; in Millisekunden).  Alle Datenpunkte sind auf der x-Achse mit vertikalen Strichen markiert.}
	\label{fig:spatial_suppression_rmse_density}
\end{figure}

Um erkennen zu können, ob der Zusammenhang zwischen den beiden Aufgabenparametern (Asymptote und Steigung), der Zusammenhang zwischen der Steigung und dem \gls{si} oder der Zusammenhang zwischen der Asymptote respektive der Steigung mit dem \gls{zwert} des \gls{bist}s durch diejenigen \glspl{vp} verzerrt wurde, bei welchen eine Beschreibung der Daten mit einem exponentiellen Modell nicht angebracht war, wurden diese Zusammenhänge in Abhängigkeit des \gls{rmse} bestimmt. 
Dafür wurde für jeden \gls{rmse} zwischen $1$~ms und $70$~ms eine Teilstichprobe mit \glspl{vp} gebildet, welche den gewählten \gls{rmse} (\gls{rmse}-Grenzwert) nicht überschritten haben. Die erste Teilstichprobe ($n=4$) bestand also aus \glspl{vp}, welche einen \gls{rmse} von nicht grösser $1$~ms aufwiesen. Die zweite Teilstichprobe ($n=11$) setzte sich aus \glspl{vp} zusammen, welche einen \gls{rmse} von nicht grösser $2$~ms aufwiesen. Die dritte Teilstichprobe ($n=32$) beinhaltete \glspl{vp}, welche einen \gls{rmse} von nicht grösser $3$~ms aufwiesen (usw.). Dieses Vorgehen wurde solange weitergeführt, bis die Teilstichprobe bei einem \gls{rmse}-Grenzwert von $65.47$~ms alle \glspl{vp} ($N=177$) beinhaltete. 
Damit liessen sich die Zusammenhänge über den ganzen \gls{rmse}-Grenz\-wert\-be\-reich bestimmen. Eine Teilstichprobe bei einem tiefen \gls{rmse}-Grenz\-wert umfasste also \glspl{vp}, welche Modell-konforme Daten aufwiesen, während eine Teilstichprobe bei einem hohen \gls{rmse}-Grenzwert auch \glspl{vp} beinhaltete, welche stärker vom exponentiellen Modell abwichen. 

Die Analyse hat ergeben, dass die Asymptote und die Steigung ab einem \gls{rmse}-Grenzwert von $1.4$ ms stark negativ miteinander zusammenhingen ($r=-.57$ bis $-.98$, alle $p\textnormal{s}<.05$). Eine visuelle Inspektion des Verlaufs deutete darauf hin, dass der \gls{rmse}-Grenzwert einen negativen Einfluss auf die Höhe des Zusammenhangs ausübte (siehe \autoref{fig:spatial_suppression_rmse_cutoff_asymtote_slope_suppressionindex}a).

\begin{figure}[htbp]
	\centering
	%	\captionsetup{font = small}
	\begin{adjustbox}{width=1\textwidth}
		\subfloat[Test][Zusammenhang ($r$) zwischen der Asymptote und der Steigung.]{% Created by tikzDevice version 0.10.1 on 2016-09-01 13:37:43
% !TEX encoding = UTF-8 Unicode
\begin{tikzpicture}[x=1pt,y=1pt]
\definecolor{fillColor}{RGB}{255,255,255}
\path[use as bounding box,fill=fillColor,fill opacity=0.00] (0,0) rectangle (361.35,216.81);
\begin{scope}
\path[clip] (  0.00,  0.00) rectangle (361.35,216.81);
\definecolor{drawColor}{RGB}{0,0,0}

\node[text=drawColor,anchor=base,inner sep=0pt, outer sep=0pt, scale=  1.00] at (201.68,  8.40) {\textit{RMSE}-Grenzwert (ms)};

\node[text=drawColor,rotate= 90.00,anchor=base,inner sep=0pt, outer sep=0pt, scale=  1.00] at (  9.60,129.40) {\textit{r}};
\end{scope}
\begin{scope}
\path[clip] (  0.00,  0.00) rectangle (361.35,216.81);
\definecolor{drawColor}{RGB}{0,0,0}

\path[draw=drawColor,line width= 0.4pt,line join=round,line cap=round] ( 58.05, 48.00) -- (349.52, 48.00);

\path[draw=drawColor,line width= 0.4pt,line join=round,line cap=round] ( 58.05, 48.00) -- ( 58.05, 42.00);

\path[draw=drawColor,line width= 0.4pt,line join=round,line cap=round] ( 96.07, 48.00) -- ( 96.07, 42.00);

\path[draw=drawColor,line width= 0.4pt,line join=round,line cap=round] (138.31, 48.00) -- (138.31, 42.00);

\path[draw=drawColor,line width= 0.4pt,line join=round,line cap=round] (180.55, 48.00) -- (180.55, 42.00);

\path[draw=drawColor,line width= 0.4pt,line join=round,line cap=round] (222.80, 48.00) -- (222.80, 42.00);

\path[draw=drawColor,line width= 0.4pt,line join=round,line cap=round] (265.04, 48.00) -- (265.04, 42.00);

\path[draw=drawColor,line width= 0.4pt,line join=round,line cap=round] (307.28, 48.00) -- (307.28, 42.00);

\path[draw=drawColor,line width= 0.4pt,line join=round,line cap=round] (349.52, 48.00) -- (349.52, 42.00);

\node[text=drawColor,anchor=base,inner sep=0pt, outer sep=0pt, scale=  1.00] at ( 58.05, 30.00) {1};

\node[text=drawColor,anchor=base,inner sep=0pt, outer sep=0pt, scale=  1.00] at ( 96.07, 30.00) {10};

\node[text=drawColor,anchor=base,inner sep=0pt, outer sep=0pt, scale=  1.00] at (138.31, 30.00) {20};

\node[text=drawColor,anchor=base,inner sep=0pt, outer sep=0pt, scale=  1.00] at (180.55, 30.00) {30};

\node[text=drawColor,anchor=base,inner sep=0pt, outer sep=0pt, scale=  1.00] at (222.80, 30.00) {40};

\node[text=drawColor,anchor=base,inner sep=0pt, outer sep=0pt, scale=  1.00] at (265.04, 30.00) {50};

\node[text=drawColor,anchor=base,inner sep=0pt, outer sep=0pt, scale=  1.00] at (307.28, 30.00) {60};

\node[text=drawColor,anchor=base,inner sep=0pt, outer sep=0pt, scale=  1.00] at (349.52, 30.00) {70};

\path[draw=drawColor,line width= 0.4pt,line join=round,line cap=round] ( 42.00, 54.03) -- ( 42.00,204.78);

\path[draw=drawColor,line width= 0.4pt,line join=round,line cap=round] ( 42.00, 54.03) -- ( 36.00, 54.03);

\path[draw=drawColor,line width= 0.4pt,line join=round,line cap=round] ( 42.00, 72.87) -- ( 36.00, 72.87);

\path[draw=drawColor,line width= 0.4pt,line join=round,line cap=round] ( 42.00, 91.72) -- ( 36.00, 91.72);

\path[draw=drawColor,line width= 0.4pt,line join=round,line cap=round] ( 42.00,110.56) -- ( 36.00,110.56);

\path[draw=drawColor,line width= 0.4pt,line join=round,line cap=round] ( 42.00,129.40) -- ( 36.00,129.40);

\path[draw=drawColor,line width= 0.4pt,line join=round,line cap=round] ( 42.00,148.25) -- ( 36.00,148.25);

\path[draw=drawColor,line width= 0.4pt,line join=round,line cap=round] ( 42.00,167.09) -- ( 36.00,167.09);

\path[draw=drawColor,line width= 0.4pt,line join=round,line cap=round] ( 42.00,185.94) -- ( 36.00,185.94);

\path[draw=drawColor,line width= 0.4pt,line join=round,line cap=round] ( 42.00,204.78) -- ( 36.00,204.78);

\node[text=drawColor,anchor=base east,inner sep=0pt, outer sep=0pt, scale=  1.00] at ( 33.60, 50.59) {--1.00};

\node[text=drawColor,anchor=base east,inner sep=0pt, outer sep=0pt, scale=  1.00] at ( 33.60, 69.43) {--.75};

\node[text=drawColor,anchor=base east,inner sep=0pt, outer sep=0pt, scale=  1.00] at ( 33.60, 88.27) {--.50};

\node[text=drawColor,anchor=base east,inner sep=0pt, outer sep=0pt, scale=  1.00] at ( 33.60,107.12) {--.25};

\node[text=drawColor,anchor=base east,inner sep=0pt, outer sep=0pt, scale=  1.00] at ( 33.60,125.96) {.00};

\node[text=drawColor,anchor=base east,inner sep=0pt, outer sep=0pt, scale=  1.00] at ( 33.60,144.81) {.25};

\node[text=drawColor,anchor=base east,inner sep=0pt, outer sep=0pt, scale=  1.00] at ( 33.60,163.65) {.50};

\node[text=drawColor,anchor=base east,inner sep=0pt, outer sep=0pt, scale=  1.00] at ( 33.60,182.49) {.75};

\node[text=drawColor,anchor=base east,inner sep=0pt, outer sep=0pt, scale=  1.00] at ( 33.60,201.34) {1.00};
\end{scope}
\begin{scope}
\path[clip] ( 42.00, 48.00) rectangle (361.35,210.81);
\definecolor{fillColor}{RGB}{190,190,190}

\path[fill=fillColor] ( 58.05, 54.12) --
	( 58.47, 54.12) --
	( 58.90, 54.12) --
	( 59.32, 54.55) --
	( 59.74, 54.55) --
	( 60.16, 55.07) --
	( 60.59, 55.07) --
	( 61.01, 55.41) --
	( 61.43, 55.01) --
	( 61.85, 54.50) --
	( 62.28, 54.55) --
	( 62.70, 54.55) --
	( 63.12, 54.96) --
	( 63.54, 57.10) --
	( 63.97, 57.10) --
	( 64.39, 57.59) --
	( 64.81, 60.84) --
	( 65.23, 60.84) --
	( 65.66, 61.25) --
	( 66.08, 61.73) --
	( 66.50, 63.33) --
	( 66.92, 64.16) --
	( 67.35, 64.16) --
	( 67.77, 65.40) --
	( 68.19, 65.40) --
	( 68.61, 66.02) --
	( 69.03, 66.25) --
	( 69.46, 66.25) --
	( 69.88, 66.25) --
	( 70.30, 65.42) --
	( 70.72, 65.89) --
	( 71.15, 65.64) --
	( 71.57, 65.68) --
	( 71.99, 64.74) --
	( 72.41, 64.98) --
	( 72.84, 65.58) --
	( 73.26, 65.54) --
	( 73.68, 64.40) --
	( 74.10, 64.40) --
	( 74.53, 65.11) --
	( 74.95, 65.96) --
	( 75.37, 66.16) --
	( 75.79, 66.72) --
	( 76.22, 66.72) --
	( 76.64, 66.73) --
	( 77.06, 67.01) --
	( 77.48, 67.03) --
	( 77.91, 66.98) --
	( 78.33, 67.15) --
	( 78.75, 67.25) --
	( 79.17, 67.28) --
	( 79.60, 67.28) --
	( 80.02, 67.03) --
	( 80.44, 67.52) --
	( 80.86, 67.52) --
	( 81.29, 67.51) --
	( 81.71, 67.52) --
	( 82.13, 67.29) --
	( 82.55, 66.78) --
	( 82.97, 66.87) --
	( 83.40, 67.34) --
	( 83.82, 67.82) --
	( 84.24, 67.82) --
	( 84.66, 67.95) --
	( 85.09, 68.06) --
	( 85.51, 68.06) --
	( 85.93, 67.99) --
	( 86.35, 67.98) --
	( 86.78, 67.80) --
	( 87.20, 67.80) --
	( 87.62, 67.87) --
	( 88.04, 68.19) --
	( 88.47, 68.38) --
	( 88.89, 68.38) --
	( 89.31, 68.49) --
	( 89.73, 68.36) --
	( 90.16, 68.45) --
	( 90.58, 68.45) --
	( 91.00, 68.45) --
	( 91.42, 68.74) --
	( 91.85, 68.74) --
	( 92.27, 68.74) --
	( 92.69, 68.74) --
	( 93.11, 68.74) --
	( 93.54, 68.74) --
	( 93.96, 68.74) --
	( 94.38, 68.79) --
	( 94.80, 69.05) --
	( 95.22, 69.05) --
	( 95.65, 69.05) --
	( 96.07, 69.49) --
	( 96.49, 69.49) --
	( 96.91, 69.49) --
	( 97.34, 69.38) --
	( 97.76, 69.38) --
	( 98.18, 69.38) --
	( 98.60, 69.38) --
	( 99.03, 69.38) --
	( 99.45, 69.00) --
	( 99.87, 69.12) --
	(100.29, 69.21) --
	(100.72, 69.21) --
	(101.14, 69.06) --
	(101.56, 69.06) --
	(101.98, 69.09) --
	(102.41, 69.13) --
	(102.83, 69.13) --
	(103.25, 69.13) --
	(103.67, 69.13) --
	(104.10, 69.13) --
	(104.52, 69.13) --
	(104.94, 69.13) --
	(105.36, 69.46) --
	(105.79, 69.44) --
	(106.21, 69.44) --
	(106.63, 69.44) --
	(107.05, 69.44) --
	(107.48, 69.44) --
	(107.90, 69.44) --
	(108.32, 69.44) --
	(108.74, 69.29) --
	(109.16, 69.29) --
	(109.59, 69.29) --
	(110.01, 69.68) --
	(110.43, 69.79) --
	(110.85, 69.79) --
	(111.28, 69.79) --
	(111.70, 69.79) --
	(112.12, 69.79) --
	(112.54, 69.79) --
	(112.97, 69.79) --
	(113.39, 69.79) --
	(113.81, 70.34) --
	(114.23, 70.33) --
	(114.66, 70.33) --
	(115.08, 70.33) --
	(115.50, 70.33) --
	(115.92, 70.33) --
	(116.35, 70.33) --
	(116.77, 70.29) --
	(117.19, 70.29) --
	(117.61, 70.29) --
	(118.04, 70.29) --
	(118.46, 70.39) --
	(118.88, 70.39) --
	(119.30, 70.39) --
	(119.73, 70.39) --
	(120.15, 70.95) --
	(120.57, 70.95) --
	(120.99, 70.95) --
	(121.42, 70.95) --
	(121.84, 70.95) --
	(122.26, 70.95) --
	(122.68, 70.95) --
	(123.10, 70.92) --
	(123.53, 71.02) --
	(123.95, 71.02) --
	(124.37, 71.02) --
	(124.79, 71.02) --
	(125.22, 71.16) --
	(125.64, 71.16) --
	(126.06, 71.16) --
	(126.48, 71.16) --
	(126.91, 71.16) --
	(127.33, 71.16) --
	(127.75, 71.16) --
	(128.17, 71.16) --
	(128.60, 71.17) --
	(129.02, 71.35) --
	(129.44, 71.35) --
	(129.86, 71.35) --
	(130.29, 71.35) --
	(130.71, 71.35) --
	(131.13, 71.35) --
	(131.55, 71.35) --
	(131.98, 71.35) --
	(132.40, 72.21) --
	(132.82, 72.21) --
	(133.24, 72.21) --
	(133.67, 72.21) --
	(134.09, 72.21) --
	(134.51, 72.21) --
	(134.93, 72.21) --
	(135.35, 72.21) --
	(135.78, 72.21) --
	(136.20, 72.21) --
	(136.62, 72.21) --
	(137.04, 72.21) --
	(137.47, 72.21) --
	(137.89, 72.21) --
	(138.31, 72.21) --
	(138.73, 72.21) --
	(139.16, 72.21) --
	(139.58, 72.21) --
	(140.00, 72.21) --
	(140.42, 72.21) --
	(140.85, 72.21) --
	(141.27, 72.21) --
	(141.69, 72.21) --
	(142.11, 72.21) --
	(142.54, 72.21) --
	(142.96, 72.21) --
	(143.38, 72.21) --
	(143.80, 72.21) --
	(144.23, 72.21) --
	(144.65, 72.21) --
	(145.07, 72.21) --
	(145.49, 72.21) --
	(145.92, 72.21) --
	(146.34, 72.21) --
	(146.76, 72.21) --
	(147.18, 72.21) --
	(147.61, 72.43) --
	(148.03, 72.43) --
	(148.45, 72.43) --
	(148.87, 72.43) --
	(149.29, 72.43) --
	(149.72, 72.43) --
	(150.14, 72.43) --
	(150.56, 72.43) --
	(150.98, 72.43) --
	(151.41, 72.43) --
	(151.83, 72.43) --
	(152.25, 72.43) --
	(152.67, 72.43) --
	(153.10, 72.43) --
	(153.52, 72.43) --
	(153.94, 72.43) --
	(154.36, 72.43) --
	(154.79, 72.43) --
	(155.21, 72.43) --
	(155.63, 72.43) --
	(156.05, 72.43) --
	(156.48, 72.43) --
	(156.90, 72.43) --
	(157.32, 72.43) --
	(157.74, 72.43) --
	(158.17, 72.43) --
	(158.59, 72.43) --
	(159.01, 72.43) --
	(159.43, 72.43) --
	(159.86, 72.43) --
	(160.28, 72.43) --
	(160.70, 72.43) --
	(161.12, 72.43) --
	(161.55, 72.43) --
	(161.97, 72.43) --
	(162.39, 72.43) --
	(162.81, 72.43) --
	(163.23, 72.43) --
	(163.66, 72.43) --
	(164.08, 73.01) --
	(164.50, 73.01) --
	(164.92, 73.01) --
	(165.35, 73.01) --
	(165.77, 73.01) --
	(166.19, 73.01) --
	(166.61, 73.01) --
	(167.04, 73.01) --
	(167.46, 73.01) --
	(167.88, 73.01) --
	(168.30, 73.01) --
	(168.73, 73.01) --
	(169.15, 73.01) --
	(169.57, 73.01) --
	(169.99, 73.01) --
	(170.42, 73.01) --
	(170.84, 73.01) --
	(171.26, 73.01) --
	(171.68, 73.01) --
	(172.11, 73.01) --
	(172.53, 73.01) --
	(172.95, 73.01) --
	(173.37, 73.01) --
	(173.80, 73.01) --
	(174.22, 73.01) --
	(174.64, 73.01) --
	(175.06, 73.01) --
	(175.48, 73.01) --
	(175.91, 73.01) --
	(176.33, 73.01) --
	(176.75, 73.01) --
	(177.17, 73.01) --
	(177.60, 73.01) --
	(178.02, 73.01) --
	(178.44, 73.01) --
	(178.86, 73.01) --
	(179.29, 73.01) --
	(179.71, 73.01) --
	(180.13, 73.01) --
	(180.55, 73.01) --
	(180.98, 73.01) --
	(181.40, 73.01) --
	(181.82, 73.01) --
	(182.24, 73.01) --
	(182.67, 73.01) --
	(183.09, 73.01) --
	(183.51, 73.01) --
	(183.93, 73.01) --
	(184.36, 74.19) --
	(184.78, 74.19) --
	(185.20, 74.19) --
	(185.62, 74.19) --
	(186.05, 74.19) --
	(186.47, 74.19) --
	(186.89, 74.19) --
	(187.31, 74.19) --
	(187.74, 74.19) --
	(188.16, 74.60) --
	(188.58, 74.60) --
	(189.00, 74.60) --
	(189.42, 74.60) --
	(189.85, 74.60) --
	(190.27, 74.60) --
	(190.69, 74.60) --
	(191.11, 74.60) --
	(191.54, 74.60) --
	(191.96, 74.60) --
	(192.38, 74.60) --
	(192.80, 74.60) --
	(193.23, 74.60) --
	(193.65, 74.60) --
	(194.07, 74.60) --
	(194.49, 74.60) --
	(194.92, 74.60) --
	(195.34, 74.60) --
	(195.76, 74.60) --
	(196.18, 74.60) --
	(196.61, 74.60) --
	(197.03, 74.60) --
	(197.45, 74.60) --
	(197.87, 74.60) --
	(198.30, 74.60) --
	(198.72, 74.60) --
	(199.14, 74.60) --
	(199.56, 74.60) --
	(199.99, 74.60) --
	(200.41, 74.60) --
	(200.83, 74.60) --
	(201.25, 74.60) --
	(201.68, 74.60) --
	(202.10, 74.60) --
	(202.52, 74.60) --
	(202.94, 75.63) --
	(203.36, 75.63) --
	(203.79, 75.63) --
	(204.21, 75.63) --
	(204.63, 75.63) --
	(205.05, 75.89) --
	(205.48, 75.89) --
	(205.90, 75.89) --
	(206.32, 75.89) --
	(206.74, 75.89) --
	(207.17, 75.89) --
	(207.59, 75.89) --
	(208.01, 76.39) --
	(208.43, 76.39) --
	(208.86, 76.39) --
	(209.28, 76.39) --
	(209.70, 76.39) --
	(210.12, 76.39) --
	(210.55, 76.39) --
	(210.97, 77.01) --
	(211.39, 77.01) --
	(211.81, 77.01) --
	(212.24, 77.01) --
	(212.66, 77.01) --
	(213.08, 77.01) --
	(213.50, 77.01) --
	(213.93, 77.01) --
	(214.35, 77.01) --
	(214.77, 77.01) --
	(215.19, 77.01) --
	(215.61, 77.01) --
	(216.04, 77.01) --
	(216.46, 77.01) --
	(216.88, 77.01) --
	(217.30, 77.01) --
	(217.73, 77.01) --
	(218.15, 77.01) --
	(218.57, 77.01) --
	(218.99, 77.01) --
	(219.42, 77.01) --
	(219.84, 77.01) --
	(220.26, 77.01) --
	(220.68, 77.01) --
	(221.11, 77.01) --
	(221.53, 77.01) --
	(221.95, 77.01) --
	(222.37, 77.01) --
	(222.80, 77.01) --
	(223.22, 77.01) --
	(223.64, 77.01) --
	(224.06, 77.01) --
	(224.49, 77.01) --
	(224.91, 77.01) --
	(225.33, 77.01) --
	(225.75, 77.01) --
	(226.18, 77.01) --
	(226.60, 77.01) --
	(227.02, 77.01) --
	(227.44, 77.01) --
	(227.87, 77.01) --
	(228.29, 77.01) --
	(228.71, 77.01) --
	(229.13, 77.01) --
	(229.55, 77.01) --
	(229.98, 77.01) --
	(230.40, 77.01) --
	(230.82, 77.01) --
	(231.24, 77.01) --
	(231.67, 77.01) --
	(232.09, 77.01) --
	(232.51, 77.01) --
	(232.93, 77.01) --
	(233.36, 77.01) --
	(233.78, 77.01) --
	(234.20, 77.01) --
	(234.62, 77.01) --
	(235.05, 77.01) --
	(235.47, 77.01) --
	(235.89, 77.01) --
	(236.31, 77.01) --
	(236.74, 77.01) --
	(237.16, 77.01) --
	(237.58, 77.01) --
	(238.00, 77.01) --
	(238.43, 77.01) --
	(238.85, 77.01) --
	(239.27, 77.01) --
	(239.69, 77.01) --
	(240.12, 77.01) --
	(240.54, 77.01) --
	(240.96, 77.01) --
	(241.38, 77.01) --
	(241.80, 77.01) --
	(242.23, 77.01) --
	(242.65, 77.01) --
	(243.07, 77.01) --
	(243.49, 77.01) --
	(243.92, 77.01) --
	(244.34, 77.01) --
	(244.76, 77.01) --
	(245.18, 77.01) --
	(245.61, 77.01) --
	(246.03, 77.01) --
	(246.45, 77.01) --
	(246.87, 77.01) --
	(247.30, 77.01) --
	(247.72, 77.01) --
	(248.14, 77.01) --
	(248.56, 77.01) --
	(248.99, 77.01) --
	(249.41, 77.01) --
	(249.83, 77.01) --
	(250.25, 77.01) --
	(250.68, 77.01) --
	(251.10, 77.01) --
	(251.52, 77.01) --
	(251.94, 77.01) --
	(252.37, 77.01) --
	(252.79, 77.01) --
	(253.21, 77.01) --
	(253.63, 77.01) --
	(254.06, 77.01) --
	(254.48, 77.01) --
	(254.90, 77.01) --
	(255.32, 77.01) --
	(255.74, 77.01) --
	(256.17, 77.01) --
	(256.59, 77.01) --
	(257.01, 77.01) --
	(257.43, 77.01) --
	(257.86, 77.01) --
	(258.28, 77.01) --
	(258.70, 77.01) --
	(259.12, 77.01) --
	(259.55, 77.01) --
	(259.97, 77.01) --
	(260.39, 77.01) --
	(260.81, 77.01) --
	(261.24, 77.01) --
	(261.66, 77.01) --
	(262.08, 77.01) --
	(262.50, 77.01) --
	(262.93, 77.01) --
	(263.35, 77.01) --
	(263.77, 77.01) --
	(264.19, 77.01) --
	(264.62, 78.04) --
	(265.04, 78.04) --
	(265.46, 78.04) --
	(265.88, 78.04) --
	(266.31, 78.04) --
	(266.73, 78.04) --
	(267.15, 78.04) --
	(267.57, 78.04) --
	(268.00, 78.04) --
	(268.42, 78.04) --
	(268.84, 78.04) --
	(269.26, 78.04) --
	(269.68, 78.04) --
	(270.11, 78.04) --
	(270.53, 78.04) --
	(270.95, 78.04) --
	(271.37, 78.04) --
	(271.80, 78.04) --
	(272.22, 78.04) --
	(272.64, 78.04) --
	(273.06, 78.04) --
	(273.49, 78.04) --
	(273.91, 78.04) --
	(274.33, 78.04) --
	(274.75, 78.04) --
	(275.18, 78.04) --
	(275.60, 78.04) --
	(276.02, 78.04) --
	(276.44, 78.04) --
	(276.87, 78.04) --
	(277.29, 78.04) --
	(277.71, 78.04) --
	(278.13, 78.04) --
	(278.56, 78.04) --
	(278.98, 78.04) --
	(279.40, 78.04) --
	(279.82, 78.04) --
	(280.25, 78.04) --
	(280.67, 78.04) --
	(281.09, 78.04) --
	(281.51, 78.04) --
	(281.93, 78.04) --
	(282.36, 78.04) --
	(282.78, 78.04) --
	(283.20, 78.04) --
	(283.62, 78.04) --
	(284.05, 78.04) --
	(284.47, 78.04) --
	(284.89, 78.04) --
	(285.31, 78.04) --
	(285.74, 78.04) --
	(286.16, 78.04) --
	(286.58, 78.04) --
	(287.00, 78.04) --
	(287.43, 78.04) --
	(287.85, 78.04) --
	(288.27, 78.04) --
	(288.69, 78.04) --
	(289.12, 78.04) --
	(289.54, 78.04) --
	(289.96, 78.04) --
	(290.38, 78.04) --
	(290.81, 78.04) --
	(291.23, 78.04) --
	(291.65, 78.04) --
	(292.07, 78.04) --
	(292.50, 78.04) --
	(292.92, 78.04) --
	(293.34, 78.04) --
	(293.76, 78.04) --
	(294.19, 78.04) --
	(294.61, 78.04) --
	(295.03, 78.04) --
	(295.45, 78.04) --
	(295.87, 78.04) --
	(296.30, 78.04) --
	(296.72, 78.04) --
	(297.14, 78.04) --
	(297.56, 78.04) --
	(297.99, 78.04) --
	(298.41, 78.04) --
	(298.83, 78.04) --
	(299.25, 78.04) --
	(299.68, 78.04) --
	(300.10, 78.04) --
	(300.52, 78.04) --
	(300.94, 78.04) --
	(301.37, 78.04) --
	(301.79, 78.04) --
	(302.21, 78.04) --
	(302.63, 78.04) --
	(303.06, 78.04) --
	(303.48, 78.04) --
	(303.90, 78.04) --
	(304.32, 78.04) --
	(304.75, 78.04) --
	(305.17, 78.04) --
	(305.59, 78.04) --
	(306.01, 78.04) --
	(306.44, 78.04) --
	(306.86, 78.04) --
	(307.28, 78.04) --
	(307.70, 78.04) --
	(308.12, 78.04) --
	(308.55, 78.04) --
	(308.97, 78.04) --
	(309.39, 78.04) --
	(309.81, 78.04) --
	(310.24, 78.04) --
	(310.66, 78.04) --
	(311.08, 78.04) --
	(311.50, 78.04) --
	(311.93, 78.04) --
	(312.35, 78.04) --
	(312.77, 78.04) --
	(313.19, 78.04) --
	(313.62, 78.04) --
	(314.04, 78.04) --
	(314.46, 78.04) --
	(314.88, 78.04) --
	(315.31, 78.04) --
	(315.73, 78.04) --
	(316.15, 78.04) --
	(316.57, 78.04) --
	(317.00, 78.04) --
	(317.42, 78.04) --
	(317.84, 78.04) --
	(318.26, 78.04) --
	(318.69, 78.04) --
	(319.11, 78.04) --
	(319.53, 78.04) --
	(319.95, 78.04) --
	(320.38, 78.04) --
	(320.80, 78.04) --
	(321.22, 78.04) --
	(321.64, 78.04) --
	(322.06, 78.04) --
	(322.49, 78.04) --
	(322.91, 78.04) --
	(323.33, 78.04) --
	(323.75, 78.04) --
	(324.18, 78.04) --
	(324.60, 78.04) --
	(325.02, 78.04) --
	(325.44, 78.04) --
	(325.87, 78.04) --
	(326.29, 78.04) --
	(326.71, 78.04) --
	(327.13, 78.04) --
	(327.56, 78.04) --
	(327.98, 78.04) --
	(328.40, 78.04) --
	(328.82, 78.04) --
	(329.25, 78.04) --
	(329.67, 78.04) --
	(330.09, 78.04) --
	(330.51, 79.43) --
	(330.94, 79.43) --
	(331.36, 79.43) --
	(331.78, 79.43) --
	(332.20, 79.43) --
	(332.63, 79.43) --
	(333.05, 79.43) --
	(333.47, 79.43) --
	(333.89, 79.43) --
	(334.32, 79.43) --
	(334.74, 79.43) --
	(335.16, 79.43) --
	(335.58, 79.43) --
	(336.00, 79.43) --
	(336.43, 79.43) --
	(336.85, 79.43) --
	(337.27, 79.43) --
	(337.69, 79.43) --
	(338.12, 79.43) --
	(338.54, 79.43) --
	(338.96, 79.43) --
	(339.38, 79.43) --
	(339.81, 79.43) --
	(340.23, 79.43) --
	(340.65, 79.43) --
	(341.07, 79.43) --
	(341.50, 79.43) --
	(341.92, 79.43) --
	(342.34, 79.43) --
	(342.76, 79.43) --
	(343.19, 79.43) --
	(343.61, 79.43) --
	(344.03, 79.43) --
	(344.45, 79.43) --
	(344.88, 79.43) --
	(345.30, 79.43) --
	(345.72, 79.43) --
	(346.14, 79.43) --
	(346.57, 79.43) --
	(346.99, 79.43) --
	(347.41, 79.43) --
	(347.83, 79.43) --
	(348.25, 79.43) --
	(348.68, 79.43) --
	(349.10, 79.43) --
	(349.52, 79.43) --
	(349.52, 94.51) --
	(349.10, 94.51) --
	(348.68, 94.51) --
	(348.25, 94.51) --
	(347.83, 94.51) --
	(347.41, 94.51) --
	(346.99, 94.51) --
	(346.57, 94.51) --
	(346.14, 94.51) --
	(345.72, 94.51) --
	(345.30, 94.51) --
	(344.88, 94.51) --
	(344.45, 94.51) --
	(344.03, 94.51) --
	(343.61, 94.51) --
	(343.19, 94.51) --
	(342.76, 94.51) --
	(342.34, 94.51) --
	(341.92, 94.51) --
	(341.50, 94.51) --
	(341.07, 94.51) --
	(340.65, 94.51) --
	(340.23, 94.51) --
	(339.81, 94.51) --
	(339.38, 94.51) --
	(338.96, 94.51) --
	(338.54, 94.51) --
	(338.12, 94.51) --
	(337.69, 94.51) --
	(337.27, 94.51) --
	(336.85, 94.51) --
	(336.43, 94.51) --
	(336.00, 94.51) --
	(335.58, 94.51) --
	(335.16, 94.51) --
	(334.74, 94.51) --
	(334.32, 94.51) --
	(333.89, 94.51) --
	(333.47, 94.51) --
	(333.05, 94.51) --
	(332.63, 94.51) --
	(332.20, 94.51) --
	(331.78, 94.51) --
	(331.36, 94.51) --
	(330.94, 94.51) --
	(330.51, 94.51) --
	(330.09, 92.59) --
	(329.67, 92.59) --
	(329.25, 92.59) --
	(328.82, 92.59) --
	(328.40, 92.59) --
	(327.98, 92.59) --
	(327.56, 92.59) --
	(327.13, 92.59) --
	(326.71, 92.59) --
	(326.29, 92.59) --
	(325.87, 92.59) --
	(325.44, 92.59) --
	(325.02, 92.59) --
	(324.60, 92.59) --
	(324.18, 92.59) --
	(323.75, 92.59) --
	(323.33, 92.59) --
	(322.91, 92.59) --
	(322.49, 92.59) --
	(322.06, 92.59) --
	(321.64, 92.59) --
	(321.22, 92.59) --
	(320.80, 92.59) --
	(320.38, 92.59) --
	(319.95, 92.59) --
	(319.53, 92.59) --
	(319.11, 92.59) --
	(318.69, 92.59) --
	(318.26, 92.59) --
	(317.84, 92.59) --
	(317.42, 92.59) --
	(317.00, 92.59) --
	(316.57, 92.59) --
	(316.15, 92.59) --
	(315.73, 92.59) --
	(315.31, 92.59) --
	(314.88, 92.59) --
	(314.46, 92.59) --
	(314.04, 92.59) --
	(313.62, 92.59) --
	(313.19, 92.59) --
	(312.77, 92.59) --
	(312.35, 92.59) --
	(311.93, 92.59) --
	(311.50, 92.59) --
	(311.08, 92.59) --
	(310.66, 92.59) --
	(310.24, 92.59) --
	(309.81, 92.59) --
	(309.39, 92.59) --
	(308.97, 92.59) --
	(308.55, 92.59) --
	(308.12, 92.59) --
	(307.70, 92.59) --
	(307.28, 92.59) --
	(306.86, 92.59) --
	(306.44, 92.59) --
	(306.01, 92.59) --
	(305.59, 92.59) --
	(305.17, 92.59) --
	(304.75, 92.59) --
	(304.32, 92.59) --
	(303.90, 92.59) --
	(303.48, 92.59) --
	(303.06, 92.59) --
	(302.63, 92.59) --
	(302.21, 92.59) --
	(301.79, 92.59) --
	(301.37, 92.59) --
	(300.94, 92.59) --
	(300.52, 92.59) --
	(300.10, 92.59) --
	(299.68, 92.59) --
	(299.25, 92.59) --
	(298.83, 92.59) --
	(298.41, 92.59) --
	(297.99, 92.59) --
	(297.56, 92.59) --
	(297.14, 92.59) --
	(296.72, 92.59) --
	(296.30, 92.59) --
	(295.87, 92.59) --
	(295.45, 92.59) --
	(295.03, 92.59) --
	(294.61, 92.59) --
	(294.19, 92.59) --
	(293.76, 92.59) --
	(293.34, 92.59) --
	(292.92, 92.59) --
	(292.50, 92.59) --
	(292.07, 92.59) --
	(291.65, 92.59) --
	(291.23, 92.59) --
	(290.81, 92.59) --
	(290.38, 92.59) --
	(289.96, 92.59) --
	(289.54, 92.59) --
	(289.12, 92.59) --
	(288.69, 92.59) --
	(288.27, 92.59) --
	(287.85, 92.59) --
	(287.43, 92.59) --
	(287.00, 92.59) --
	(286.58, 92.59) --
	(286.16, 92.59) --
	(285.74, 92.59) --
	(285.31, 92.59) --
	(284.89, 92.59) --
	(284.47, 92.59) --
	(284.05, 92.59) --
	(283.62, 92.59) --
	(283.20, 92.59) --
	(282.78, 92.59) --
	(282.36, 92.59) --
	(281.93, 92.59) --
	(281.51, 92.59) --
	(281.09, 92.59) --
	(280.67, 92.59) --
	(280.25, 92.59) --
	(279.82, 92.59) --
	(279.40, 92.59) --
	(278.98, 92.59) --
	(278.56, 92.59) --
	(278.13, 92.59) --
	(277.71, 92.59) --
	(277.29, 92.59) --
	(276.87, 92.59) --
	(276.44, 92.59) --
	(276.02, 92.59) --
	(275.60, 92.59) --
	(275.18, 92.59) --
	(274.75, 92.59) --
	(274.33, 92.59) --
	(273.91, 92.59) --
	(273.49, 92.59) --
	(273.06, 92.59) --
	(272.64, 92.59) --
	(272.22, 92.59) --
	(271.80, 92.59) --
	(271.37, 92.59) --
	(270.95, 92.59) --
	(270.53, 92.59) --
	(270.11, 92.59) --
	(269.68, 92.59) --
	(269.26, 92.59) --
	(268.84, 92.59) --
	(268.42, 92.59) --
	(268.00, 92.59) --
	(267.57, 92.59) --
	(267.15, 92.59) --
	(266.73, 92.59) --
	(266.31, 92.59) --
	(265.88, 92.59) --
	(265.46, 92.59) --
	(265.04, 92.59) --
	(264.62, 92.59) --
	(264.19, 91.18) --
	(263.77, 91.18) --
	(263.35, 91.18) --
	(262.93, 91.18) --
	(262.50, 91.18) --
	(262.08, 91.18) --
	(261.66, 91.18) --
	(261.24, 91.18) --
	(260.81, 91.18) --
	(260.39, 91.18) --
	(259.97, 91.18) --
	(259.55, 91.18) --
	(259.12, 91.18) --
	(258.70, 91.18) --
	(258.28, 91.18) --
	(257.86, 91.18) --
	(257.43, 91.18) --
	(257.01, 91.18) --
	(256.59, 91.18) --
	(256.17, 91.18) --
	(255.74, 91.18) --
	(255.32, 91.18) --
	(254.90, 91.18) --
	(254.48, 91.18) --
	(254.06, 91.18) --
	(253.63, 91.18) --
	(253.21, 91.18) --
	(252.79, 91.18) --
	(252.37, 91.18) --
	(251.94, 91.18) --
	(251.52, 91.18) --
	(251.10, 91.18) --
	(250.68, 91.18) --
	(250.25, 91.18) --
	(249.83, 91.18) --
	(249.41, 91.18) --
	(248.99, 91.18) --
	(248.56, 91.18) --
	(248.14, 91.18) --
	(247.72, 91.18) --
	(247.30, 91.18) --
	(246.87, 91.18) --
	(246.45, 91.18) --
	(246.03, 91.18) --
	(245.61, 91.18) --
	(245.18, 91.18) --
	(244.76, 91.18) --
	(244.34, 91.18) --
	(243.92, 91.18) --
	(243.49, 91.18) --
	(243.07, 91.18) --
	(242.65, 91.18) --
	(242.23, 91.18) --
	(241.80, 91.18) --
	(241.38, 91.18) --
	(240.96, 91.18) --
	(240.54, 91.18) --
	(240.12, 91.18) --
	(239.69, 91.18) --
	(239.27, 91.18) --
	(238.85, 91.18) --
	(238.43, 91.18) --
	(238.00, 91.18) --
	(237.58, 91.18) --
	(237.16, 91.18) --
	(236.74, 91.18) --
	(236.31, 91.18) --
	(235.89, 91.18) --
	(235.47, 91.18) --
	(235.05, 91.18) --
	(234.62, 91.18) --
	(234.20, 91.18) --
	(233.78, 91.18) --
	(233.36, 91.18) --
	(232.93, 91.18) --
	(232.51, 91.18) --
	(232.09, 91.18) --
	(231.67, 91.18) --
	(231.24, 91.18) --
	(230.82, 91.18) --
	(230.40, 91.18) --
	(229.98, 91.18) --
	(229.55, 91.18) --
	(229.13, 91.18) --
	(228.71, 91.18) --
	(228.29, 91.18) --
	(227.87, 91.18) --
	(227.44, 91.18) --
	(227.02, 91.18) --
	(226.60, 91.18) --
	(226.18, 91.18) --
	(225.75, 91.18) --
	(225.33, 91.18) --
	(224.91, 91.18) --
	(224.49, 91.18) --
	(224.06, 91.18) --
	(223.64, 91.18) --
	(223.22, 91.18) --
	(222.80, 91.18) --
	(222.37, 91.18) --
	(221.95, 91.18) --
	(221.53, 91.18) --
	(221.11, 91.18) --
	(220.68, 91.18) --
	(220.26, 91.18) --
	(219.84, 91.18) --
	(219.42, 91.18) --
	(218.99, 91.18) --
	(218.57, 91.18) --
	(218.15, 91.18) --
	(217.73, 91.18) --
	(217.30, 91.18) --
	(216.88, 91.18) --
	(216.46, 91.18) --
	(216.04, 91.18) --
	(215.61, 91.18) --
	(215.19, 91.18) --
	(214.77, 91.18) --
	(214.35, 91.18) --
	(213.93, 91.18) --
	(213.50, 91.18) --
	(213.08, 91.18) --
	(212.66, 91.18) --
	(212.24, 91.18) --
	(211.81, 91.18) --
	(211.39, 91.18) --
	(210.97, 91.18) --
	(210.55, 90.33) --
	(210.12, 90.33) --
	(209.70, 90.33) --
	(209.28, 90.33) --
	(208.86, 90.33) --
	(208.43, 90.33) --
	(208.01, 90.33) --
	(207.59, 89.65) --
	(207.17, 89.65) --
	(206.74, 89.65) --
	(206.32, 89.65) --
	(205.90, 89.65) --
	(205.48, 89.65) --
	(205.05, 89.65) --
	(204.63, 89.33) --
	(204.21, 89.33) --
	(203.79, 89.33) --
	(203.36, 89.33) --
	(202.94, 89.33) --
	(202.52, 87.85) --
	(202.10, 87.85) --
	(201.68, 87.85) --
	(201.25, 87.85) --
	(200.83, 87.85) --
	(200.41, 87.85) --
	(199.99, 87.85) --
	(199.56, 87.85) --
	(199.14, 87.85) --
	(198.72, 87.85) --
	(198.30, 87.85) --
	(197.87, 87.85) --
	(197.45, 87.85) --
	(197.03, 87.85) --
	(196.61, 87.85) --
	(196.18, 87.85) --
	(195.76, 87.85) --
	(195.34, 87.85) --
	(194.92, 87.85) --
	(194.49, 87.85) --
	(194.07, 87.85) --
	(193.65, 87.85) --
	(193.23, 87.85) --
	(192.80, 87.85) --
	(192.38, 87.85) --
	(191.96, 87.85) --
	(191.54, 87.85) --
	(191.11, 87.85) --
	(190.69, 87.85) --
	(190.27, 87.85) --
	(189.85, 87.85) --
	(189.42, 87.85) --
	(189.00, 87.85) --
	(188.58, 87.85) --
	(188.16, 87.85) --
	(187.74, 87.30) --
	(187.31, 87.30) --
	(186.89, 87.30) --
	(186.47, 87.30) --
	(186.05, 87.30) --
	(185.62, 87.30) --
	(185.20, 87.30) --
	(184.78, 87.30) --
	(184.36, 87.30) --
	(183.93, 85.58) --
	(183.51, 85.58) --
	(183.09, 85.58) --
	(182.67, 85.58) --
	(182.24, 85.58) --
	(181.82, 85.58) --
	(181.40, 85.58) --
	(180.98, 85.58) --
	(180.55, 85.58) --
	(180.13, 85.58) --
	(179.71, 85.58) --
	(179.29, 85.58) --
	(178.86, 85.58) --
	(178.44, 85.58) --
	(178.02, 85.58) --
	(177.60, 85.58) --
	(177.17, 85.58) --
	(176.75, 85.58) --
	(176.33, 85.58) --
	(175.91, 85.58) --
	(175.48, 85.58) --
	(175.06, 85.58) --
	(174.64, 85.58) --
	(174.22, 85.58) --
	(173.80, 85.58) --
	(173.37, 85.58) --
	(172.95, 85.58) --
	(172.53, 85.58) --
	(172.11, 85.58) --
	(171.68, 85.58) --
	(171.26, 85.58) --
	(170.84, 85.58) --
	(170.42, 85.58) --
	(169.99, 85.58) --
	(169.57, 85.58) --
	(169.15, 85.58) --
	(168.73, 85.58) --
	(168.30, 85.58) --
	(167.88, 85.58) --
	(167.46, 85.58) --
	(167.04, 85.58) --
	(166.61, 85.58) --
	(166.19, 85.58) --
	(165.77, 85.58) --
	(165.35, 85.58) --
	(164.92, 85.58) --
	(164.50, 85.58) --
	(164.08, 85.58) --
	(163.66, 84.75) --
	(163.23, 84.75) --
	(162.81, 84.75) --
	(162.39, 84.75) --
	(161.97, 84.75) --
	(161.55, 84.75) --
	(161.12, 84.75) --
	(160.70, 84.75) --
	(160.28, 84.75) --
	(159.86, 84.75) --
	(159.43, 84.75) --
	(159.01, 84.75) --
	(158.59, 84.75) --
	(158.17, 84.75) --
	(157.74, 84.75) --
	(157.32, 84.75) --
	(156.90, 84.75) --
	(156.48, 84.75) --
	(156.05, 84.75) --
	(155.63, 84.75) --
	(155.21, 84.75) --
	(154.79, 84.75) --
	(154.36, 84.75) --
	(153.94, 84.75) --
	(153.52, 84.75) --
	(153.10, 84.75) --
	(152.67, 84.75) --
	(152.25, 84.75) --
	(151.83, 84.75) --
	(151.41, 84.75) --
	(150.98, 84.75) --
	(150.56, 84.75) --
	(150.14, 84.75) --
	(149.72, 84.75) --
	(149.29, 84.75) --
	(148.87, 84.75) --
	(148.45, 84.75) --
	(148.03, 84.75) --
	(147.61, 84.75) --
	(147.18, 84.47) --
	(146.76, 84.47) --
	(146.34, 84.47) --
	(145.92, 84.47) --
	(145.49, 84.47) --
	(145.07, 84.47) --
	(144.65, 84.47) --
	(144.23, 84.47) --
	(143.80, 84.47) --
	(143.38, 84.47) --
	(142.96, 84.47) --
	(142.54, 84.47) --
	(142.11, 84.47) --
	(141.69, 84.47) --
	(141.27, 84.47) --
	(140.85, 84.47) --
	(140.42, 84.47) --
	(140.00, 84.47) --
	(139.58, 84.47) --
	(139.16, 84.47) --
	(138.73, 84.47) --
	(138.31, 84.47) --
	(137.89, 84.47) --
	(137.47, 84.47) --
	(137.04, 84.47) --
	(136.62, 84.47) --
	(136.20, 84.47) --
	(135.78, 84.47) --
	(135.35, 84.47) --
	(134.93, 84.47) --
	(134.51, 84.47) --
	(134.09, 84.47) --
	(133.67, 84.47) --
	(133.24, 84.47) --
	(132.82, 84.47) --
	(132.40, 84.47) --
	(131.98, 83.20) --
	(131.55, 83.20) --
	(131.13, 83.20) --
	(130.71, 83.20) --
	(130.29, 83.20) --
	(129.86, 83.20) --
	(129.44, 83.20) --
	(129.02, 83.20) --
	(128.60, 83.01) --
	(128.17, 83.04) --
	(127.75, 83.04) --
	(127.33, 83.04) --
	(126.91, 83.04) --
	(126.48, 83.04) --
	(126.06, 83.04) --
	(125.64, 83.04) --
	(125.22, 83.04) --
	(124.79, 82.87) --
	(124.37, 82.87) --
	(123.95, 82.87) --
	(123.53, 82.87) --
	(123.10, 82.75) --
	(122.68, 82.85) --
	(122.26, 82.85) --
	(121.84, 82.85) --
	(121.42, 82.85) --
	(120.99, 82.85) --
	(120.57, 82.85) --
	(120.15, 82.85) --
	(119.73, 82.03) --
	(119.30, 82.03) --
	(118.88, 82.03) --
	(118.46, 82.03) --
	(118.04, 81.91) --
	(117.61, 81.91) --
	(117.19, 81.91) --
	(116.77, 81.91) --
	(116.35, 82.03) --
	(115.92, 82.03) --
	(115.50, 82.03) --
	(115.08, 82.03) --
	(114.66, 82.03) --
	(114.23, 82.03) --
	(113.81, 82.09) --
	(113.39, 81.32) --
	(112.97, 81.32) --
	(112.54, 81.32) --
	(112.12, 81.32) --
	(111.70, 81.32) --
	(111.28, 81.32) --
	(110.85, 81.32) --
	(110.43, 81.32) --
	(110.01, 81.18) --
	(109.59, 80.61) --
	(109.16, 80.61) --
	(108.74, 80.61) --
	(108.32, 80.89) --
	(107.90, 80.95) --
	(107.48, 80.95) --
	(107.05, 80.95) --
	(106.63, 80.95) --
	(106.21, 80.95) --
	(105.79, 80.95) --
	(105.36, 81.02) --
	(104.94, 80.54) --
	(104.52, 80.54) --
	(104.10, 80.54) --
	(103.67, 80.54) --
	(103.25, 80.54) --
	(102.83, 80.54) --
	(102.41, 80.54) --
	(101.98, 80.54) --
	(101.56, 80.54) --
	(101.14, 80.54) --
	(100.72, 80.87) --
	(100.29, 80.87) --
	( 99.87, 80.78) --
	( 99.45, 80.64) --
	( 99.03, 81.36) --
	( 98.60, 81.36) --
	( 98.18, 81.36) --
	( 97.76, 81.36) --
	( 97.34, 81.36) --
	( 96.91, 81.66) --
	( 96.49, 81.66) --
	( 96.07, 81.66) --
	( 95.65, 80.99) --
	( 95.22, 80.99) --
	( 94.80, 80.99) --
	( 94.38, 80.62) --
	( 93.96, 80.60) --
	( 93.54, 80.60) --
	( 93.11, 80.60) --
	( 92.69, 80.60) --
	( 92.27, 80.60) --
	( 91.85, 80.60) --
	( 91.42, 80.60) --
	( 91.00, 80.18) --
	( 90.58, 80.18) --
	( 90.16, 80.18) --
	( 89.73, 80.14) --
	( 89.31, 80.41) --
	( 88.89, 80.35) --
	( 88.47, 80.35) --
	( 88.04, 80.10) --
	( 87.62, 79.61) --
	( 87.20, 79.62) --
	( 86.78, 79.62) --
	( 86.35, 80.05) --
	( 85.93, 80.19) --
	( 85.51, 80.59) --
	( 85.09, 80.59) --
	( 84.66, 80.69) --
	( 84.24, 80.70) --
	( 83.82, 80.70) --
	( 83.40, 80.00) --
	( 82.97, 79.32) --
	( 82.55, 79.24) --
	( 82.13, 80.32) --
	( 81.71, 80.90) --
	( 81.29, 80.98) --
	( 80.86, 81.09) --
	( 80.44, 81.09) --
	( 80.02, 80.67) --
	( 79.60, 81.55) --
	( 79.17, 81.55) --
	( 78.75, 81.60) --
	( 78.33, 81.53) --
	( 77.91, 81.32) --
	( 77.48, 81.53) --
	( 77.06, 81.61) --
	( 76.64, 81.45) --
	( 76.22, 81.55) --
	( 75.79, 81.55) --
	( 75.37, 81.25) --
	( 74.95, 81.00) --
	( 74.53, 79.71) --
	( 74.10, 78.66) --
	( 73.68, 78.66) --
	( 73.26, 81.97) --
	( 72.84, 82.87) --
	( 72.41, 82.22) --
	( 71.99, 82.15) --
	( 71.57, 84.55) --
	( 71.15, 85.01) --
	( 70.72, 86.52) --
	( 70.30, 86.09) --
	( 69.88, 88.83) --
	( 69.46, 88.83) --
	( 69.03, 88.83) --
	( 68.61, 88.67) --
	( 68.19, 88.94) --
	( 67.77, 88.94) --
	( 67.35, 87.18) --
	( 66.92, 87.18) --
	( 66.50, 87.18) --
	( 66.08, 83.54) --
	( 65.66, 84.11) --
	( 65.23, 84.33) --
	( 64.81, 84.33) --
	( 64.39, 73.27) --
	( 63.97, 77.35) --
	( 63.54, 77.35) --
	( 63.12, 63.32) --
	( 62.70, 60.82) --
	( 62.28, 61.94) --
	( 61.85, 62.67) --
	( 61.43, 74.87) --
	( 61.01, 89.46) --
	( 60.59, 93.15) --
	( 60.16, 93.15) --
	( 59.74, 90.37) --
	( 59.32, 90.37) --
	( 58.90,146.14) --
	( 58.47,146.14) --
	( 58.05,146.14) --
	cycle;
\definecolor{drawColor}{RGB}{0,0,0}

\path[draw=drawColor,line width= 0.4pt,dash pattern=on 7pt off 3pt ,line join=round,line cap=round] ( 42.00,129.40) -- (361.35,129.40);

\path[draw=drawColor,line width= 0.4pt,line join=round,line cap=round] ( 58.05, 58.59) --
	( 58.47, 58.59) --
	( 58.90, 58.59) --
	( 59.32, 58.85) --
	( 59.74, 58.85) --
	( 60.16, 61.12) --
	( 60.59, 61.12) --
	( 61.01, 61.65) --
	( 61.43, 58.76) --
	( 61.85, 56.08) --
	( 62.28, 56.09) --
	( 62.70, 55.93) --
	( 63.12, 57.01) --
	( 63.54, 62.79) --
	( 63.97, 62.79) --
	( 64.39, 62.50) --
	( 64.81, 68.85) --
	( 65.23, 68.85) --
	( 65.66, 69.21) --
	( 66.08, 69.51) --
	( 66.50, 72.09) --
	( 66.92, 72.83) --
	( 67.35, 72.83) --
	( 67.77, 74.46) --
	( 68.19, 74.46) --
	( 68.61, 74.88) --
	( 69.03, 75.12) --
	( 69.46, 75.12) --
	( 69.88, 75.12) --
	( 70.30, 73.53) --
	( 70.72, 74.05) --
	( 71.15, 73.34) --
	( 71.57, 73.21) --
	( 71.99, 71.66) --
	( 72.41, 71.87) --
	( 72.84, 72.55) --
	( 73.26, 72.21) --
	( 73.68, 70.19) --
	( 74.10, 70.19) --
	( 74.53, 71.09) --
	( 74.95, 72.17) --
	( 75.37, 72.41) --
	( 75.79, 72.92) --
	( 76.22, 72.92) --
	( 76.64, 72.89) --
	( 77.06, 73.15) --
	( 77.48, 73.13) --
	( 77.91, 73.02) --
	( 78.33, 73.22) --
	( 78.75, 73.31) --
	( 79.17, 73.31) --
	( 79.60, 73.31) --
	( 80.02, 72.82) --
	( 80.44, 73.31) --
	( 80.86, 73.31) --
	( 81.29, 73.26) --
	( 81.71, 73.24) --
	( 82.13, 72.86) --
	( 82.55, 72.10) --
	( 82.97, 72.20) --
	( 83.40, 72.77) --
	( 83.82, 73.36) --
	( 84.24, 73.36) --
	( 84.66, 73.45) --
	( 85.09, 73.48) --
	( 85.51, 73.48) --
	( 85.93, 73.28) --
	( 86.35, 73.22) --
	( 86.78, 72.93) --
	( 87.20, 72.93) --
	( 87.62, 72.97) --
	( 88.04, 73.38) --
	( 88.47, 73.60) --
	( 88.89, 73.60) --
	( 89.31, 73.69) --
	( 89.73, 73.50) --
	( 90.16, 73.58) --
	( 90.58, 73.58) --
	( 91.00, 73.58) --
	( 91.42, 73.94) --
	( 91.85, 73.94) --
	( 92.27, 73.94) --
	( 92.69, 73.94) --
	( 93.11, 73.94) --
	( 93.54, 73.94) --
	( 93.96, 73.94) --
	( 94.38, 73.97) --
	( 94.80, 74.29) --
	( 95.22, 74.29) --
	( 95.65, 74.29) --
	( 96.07, 74.84) --
	( 96.49, 74.84) --
	( 96.91, 74.84) --
	( 97.34, 74.65) --
	( 97.76, 74.65) --
	( 98.18, 74.65) --
	( 98.60, 74.65) --
	( 99.03, 74.65) --
	( 99.45, 74.12) --
	( 99.87, 74.25) --
	(100.29, 74.34) --
	(100.72, 74.34) --
	(101.14, 74.11) --
	(101.56, 74.11) --
	(101.98, 74.14) --
	(102.41, 74.16) --
	(102.83, 74.16) --
	(103.25, 74.16) --
	(103.67, 74.16) --
	(104.10, 74.16) --
	(104.52, 74.16) --
	(104.94, 74.16) --
	(105.36, 74.56) --
	(105.79, 74.52) --
	(106.21, 74.52) --
	(106.63, 74.52) --
	(107.05, 74.52) --
	(107.48, 74.52) --
	(107.90, 74.52) --
	(108.32, 74.50) --
	(108.74, 74.29) --
	(109.16, 74.29) --
	(109.59, 74.29) --
	(110.01, 74.77) --
	(110.43, 74.90) --
	(110.85, 74.90) --
	(111.28, 74.90) --
	(111.70, 74.90) --
	(112.12, 74.90) --
	(112.54, 74.90) --
	(112.97, 74.90) --
	(113.39, 74.90) --
	(113.81, 75.55) --
	(114.23, 75.53) --
	(114.66, 75.53) --
	(115.08, 75.53) --
	(115.50, 75.53) --
	(115.92, 75.53) --
	(116.35, 75.53) --
	(116.77, 75.45) --
	(117.19, 75.45) --
	(117.61, 75.45) --
	(118.04, 75.45) --
	(118.46, 75.56) --
	(118.88, 75.56) --
	(119.30, 75.56) --
	(119.73, 75.56) --
	(120.15, 76.25) --
	(120.57, 76.25) --
	(120.99, 76.25) --
	(121.42, 76.25) --
	(121.84, 76.25) --
	(122.26, 76.25) --
	(122.68, 76.25) --
	(123.10, 76.19) --
	(123.53, 76.30) --
	(123.95, 76.30) --
	(124.37, 76.30) --
	(124.79, 76.30) --
	(125.22, 76.46) --
	(125.64, 76.46) --
	(126.06, 76.46) --
	(126.48, 76.46) --
	(126.91, 76.46) --
	(127.33, 76.46) --
	(127.75, 76.46) --
	(128.17, 76.46) --
	(128.60, 76.45) --
	(129.02, 76.65) --
	(129.44, 76.65) --
	(129.86, 76.65) --
	(130.29, 76.65) --
	(130.71, 76.65) --
	(131.13, 76.65) --
	(131.55, 76.65) --
	(131.98, 76.65) --
	(132.40, 77.70) --
	(132.82, 77.70) --
	(133.24, 77.70) --
	(133.67, 77.70) --
	(134.09, 77.70) --
	(134.51, 77.70) --
	(134.93, 77.70) --
	(135.35, 77.70) --
	(135.78, 77.70) --
	(136.20, 77.70) --
	(136.62, 77.70) --
	(137.04, 77.70) --
	(137.47, 77.70) --
	(137.89, 77.70) --
	(138.31, 77.70) --
	(138.73, 77.70) --
	(139.16, 77.70) --
	(139.58, 77.70) --
	(140.00, 77.70) --
	(140.42, 77.70) --
	(140.85, 77.70) --
	(141.27, 77.70) --
	(141.69, 77.70) --
	(142.11, 77.70) --
	(142.54, 77.70) --
	(142.96, 77.70) --
	(143.38, 77.70) --
	(143.80, 77.70) --
	(144.23, 77.70) --
	(144.65, 77.70) --
	(145.07, 77.70) --
	(145.49, 77.70) --
	(145.92, 77.70) --
	(146.34, 77.70) --
	(146.76, 77.70) --
	(147.18, 77.70) --
	(147.61, 77.95) --
	(148.03, 77.95) --
	(148.45, 77.95) --
	(148.87, 77.95) --
	(149.29, 77.95) --
	(149.72, 77.95) --
	(150.14, 77.95) --
	(150.56, 77.95) --
	(150.98, 77.95) --
	(151.41, 77.95) --
	(151.83, 77.95) --
	(152.25, 77.95) --
	(152.67, 77.95) --
	(153.10, 77.95) --
	(153.52, 77.95) --
	(153.94, 77.95) --
	(154.36, 77.95) --
	(154.79, 77.95) --
	(155.21, 77.95) --
	(155.63, 77.95) --
	(156.05, 77.95) --
	(156.48, 77.95) --
	(156.90, 77.95) --
	(157.32, 77.95) --
	(157.74, 77.95) --
	(158.17, 77.95) --
	(158.59, 77.95) --
	(159.01, 77.95) --
	(159.43, 77.95) --
	(159.86, 77.95) --
	(160.28, 77.95) --
	(160.70, 77.95) --
	(161.12, 77.95) --
	(161.55, 77.95) --
	(161.97, 77.95) --
	(162.39, 77.95) --
	(162.81, 77.95) --
	(163.23, 77.95) --
	(163.66, 77.95) --
	(164.08, 78.65) --
	(164.50, 78.65) --
	(164.92, 78.65) --
	(165.35, 78.65) --
	(165.77, 78.65) --
	(166.19, 78.65) --
	(166.61, 78.65) --
	(167.04, 78.65) --
	(167.46, 78.65) --
	(167.88, 78.65) --
	(168.30, 78.65) --
	(168.73, 78.65) --
	(169.15, 78.65) --
	(169.57, 78.65) --
	(169.99, 78.65) --
	(170.42, 78.65) --
	(170.84, 78.65) --
	(171.26, 78.65) --
	(171.68, 78.65) --
	(172.11, 78.65) --
	(172.53, 78.65) --
	(172.95, 78.65) --
	(173.37, 78.65) --
	(173.80, 78.65) --
	(174.22, 78.65) --
	(174.64, 78.65) --
	(175.06, 78.65) --
	(175.48, 78.65) --
	(175.91, 78.65) --
	(176.33, 78.65) --
	(176.75, 78.65) --
	(177.17, 78.65) --
	(177.60, 78.65) --
	(178.02, 78.65) --
	(178.44, 78.65) --
	(178.86, 78.65) --
	(179.29, 78.65) --
	(179.71, 78.65) --
	(180.13, 78.65) --
	(180.55, 78.65) --
	(180.98, 78.65) --
	(181.40, 78.65) --
	(181.82, 78.65) --
	(182.24, 78.65) --
	(182.67, 78.65) --
	(183.09, 78.65) --
	(183.51, 78.65) --
	(183.93, 78.65) --
	(184.36, 80.10) --
	(184.78, 80.10) --
	(185.20, 80.10) --
	(185.62, 80.10) --
	(186.05, 80.10) --
	(186.47, 80.10) --
	(186.89, 80.10) --
	(187.31, 80.10) --
	(187.74, 80.10) --
	(188.16, 80.58) --
	(188.58, 80.58) --
	(189.00, 80.58) --
	(189.42, 80.58) --
	(189.85, 80.58) --
	(190.27, 80.58) --
	(190.69, 80.58) --
	(191.11, 80.58) --
	(191.54, 80.58) --
	(191.96, 80.58) --
	(192.38, 80.58) --
	(192.80, 80.58) --
	(193.23, 80.58) --
	(193.65, 80.58) --
	(194.07, 80.58) --
	(194.49, 80.58) --
	(194.92, 80.58) --
	(195.34, 80.58) --
	(195.76, 80.58) --
	(196.18, 80.58) --
	(196.61, 80.58) --
	(197.03, 80.58) --
	(197.45, 80.58) --
	(197.87, 80.58) --
	(198.30, 80.58) --
	(198.72, 80.58) --
	(199.14, 80.58) --
	(199.56, 80.58) --
	(199.99, 80.58) --
	(200.41, 80.58) --
	(200.83, 80.58) --
	(201.25, 80.58) --
	(201.68, 80.58) --
	(202.10, 80.58) --
	(202.52, 80.58) --
	(202.94, 81.84) --
	(203.36, 81.84) --
	(203.79, 81.84) --
	(204.21, 81.84) --
	(204.63, 81.84) --
	(205.05, 82.12) --
	(205.48, 82.12) --
	(205.90, 82.12) --
	(206.32, 82.12) --
	(206.74, 82.12) --
	(207.17, 82.12) --
	(207.59, 82.12) --
	(208.01, 82.71) --
	(208.43, 82.71) --
	(208.86, 82.71) --
	(209.28, 82.71) --
	(209.70, 82.71) --
	(210.12, 82.71) --
	(210.55, 82.71) --
	(210.97, 83.45) --
	(211.39, 83.45) --
	(211.81, 83.45) --
	(212.24, 83.45) --
	(212.66, 83.45) --
	(213.08, 83.45) --
	(213.50, 83.45) --
	(213.93, 83.45) --
	(214.35, 83.45) --
	(214.77, 83.45) --
	(215.19, 83.45) --
	(215.61, 83.45) --
	(216.04, 83.45) --
	(216.46, 83.45) --
	(216.88, 83.45) --
	(217.30, 83.45) --
	(217.73, 83.45) --
	(218.15, 83.45) --
	(218.57, 83.45) --
	(218.99, 83.45) --
	(219.42, 83.45) --
	(219.84, 83.45) --
	(220.26, 83.45) --
	(220.68, 83.45) --
	(221.11, 83.45) --
	(221.53, 83.45) --
	(221.95, 83.45) --
	(222.37, 83.45) --
	(222.80, 83.45) --
	(223.22, 83.45) --
	(223.64, 83.45) --
	(224.06, 83.45) --
	(224.49, 83.45) --
	(224.91, 83.45) --
	(225.33, 83.45) --
	(225.75, 83.45) --
	(226.18, 83.45) --
	(226.60, 83.45) --
	(227.02, 83.45) --
	(227.44, 83.45) --
	(227.87, 83.45) --
	(228.29, 83.45) --
	(228.71, 83.45) --
	(229.13, 83.45) --
	(229.55, 83.45) --
	(229.98, 83.45) --
	(230.40, 83.45) --
	(230.82, 83.45) --
	(231.24, 83.45) --
	(231.67, 83.45) --
	(232.09, 83.45) --
	(232.51, 83.45) --
	(232.93, 83.45) --
	(233.36, 83.45) --
	(233.78, 83.45) --
	(234.20, 83.45) --
	(234.62, 83.45) --
	(235.05, 83.45) --
	(235.47, 83.45) --
	(235.89, 83.45) --
	(236.31, 83.45) --
	(236.74, 83.45) --
	(237.16, 83.45) --
	(237.58, 83.45) --
	(238.00, 83.45) --
	(238.43, 83.45) --
	(238.85, 83.45) --
	(239.27, 83.45) --
	(239.69, 83.45) --
	(240.12, 83.45) --
	(240.54, 83.45) --
	(240.96, 83.45) --
	(241.38, 83.45) --
	(241.80, 83.45) --
	(242.23, 83.45) --
	(242.65, 83.45) --
	(243.07, 83.45) --
	(243.49, 83.45) --
	(243.92, 83.45) --
	(244.34, 83.45) --
	(244.76, 83.45) --
	(245.18, 83.45) --
	(245.61, 83.45) --
	(246.03, 83.45) --
	(246.45, 83.45) --
	(246.87, 83.45) --
	(247.30, 83.45) --
	(247.72, 83.45) --
	(248.14, 83.45) --
	(248.56, 83.45) --
	(248.99, 83.45) --
	(249.41, 83.45) --
	(249.83, 83.45) --
	(250.25, 83.45) --
	(250.68, 83.45) --
	(251.10, 83.45) --
	(251.52, 83.45) --
	(251.94, 83.45) --
	(252.37, 83.45) --
	(252.79, 83.45) --
	(253.21, 83.45) --
	(253.63, 83.45) --
	(254.06, 83.45) --
	(254.48, 83.45) --
	(254.90, 83.45) --
	(255.32, 83.45) --
	(255.74, 83.45) --
	(256.17, 83.45) --
	(256.59, 83.45) --
	(257.01, 83.45) --
	(257.43, 83.45) --
	(257.86, 83.45) --
	(258.28, 83.45) --
	(258.70, 83.45) --
	(259.12, 83.45) --
	(259.55, 83.45) --
	(259.97, 83.45) --
	(260.39, 83.45) --
	(260.81, 83.45) --
	(261.24, 83.45) --
	(261.66, 83.45) --
	(262.08, 83.45) --
	(262.50, 83.45) --
	(262.93, 83.45) --
	(263.35, 83.45) --
	(263.77, 83.45) --
	(264.19, 83.45) --
	(264.62, 84.68) --
	(265.04, 84.68) --
	(265.46, 84.68) --
	(265.88, 84.68) --
	(266.31, 84.68) --
	(266.73, 84.68) --
	(267.15, 84.68) --
	(267.57, 84.68) --
	(268.00, 84.68) --
	(268.42, 84.68) --
	(268.84, 84.68) --
	(269.26, 84.68) --
	(269.68, 84.68) --
	(270.11, 84.68) --
	(270.53, 84.68) --
	(270.95, 84.68) --
	(271.37, 84.68) --
	(271.80, 84.68) --
	(272.22, 84.68) --
	(272.64, 84.68) --
	(273.06, 84.68) --
	(273.49, 84.68) --
	(273.91, 84.68) --
	(274.33, 84.68) --
	(274.75, 84.68) --
	(275.18, 84.68) --
	(275.60, 84.68) --
	(276.02, 84.68) --
	(276.44, 84.68) --
	(276.87, 84.68) --
	(277.29, 84.68) --
	(277.71, 84.68) --
	(278.13, 84.68) --
	(278.56, 84.68) --
	(278.98, 84.68) --
	(279.40, 84.68) --
	(279.82, 84.68) --
	(280.25, 84.68) --
	(280.67, 84.68) --
	(281.09, 84.68) --
	(281.51, 84.68) --
	(281.93, 84.68) --
	(282.36, 84.68) --
	(282.78, 84.68) --
	(283.20, 84.68) --
	(283.62, 84.68) --
	(284.05, 84.68) --
	(284.47, 84.68) --
	(284.89, 84.68) --
	(285.31, 84.68) --
	(285.74, 84.68) --
	(286.16, 84.68) --
	(286.58, 84.68) --
	(287.00, 84.68) --
	(287.43, 84.68) --
	(287.85, 84.68) --
	(288.27, 84.68) --
	(288.69, 84.68) --
	(289.12, 84.68) --
	(289.54, 84.68) --
	(289.96, 84.68) --
	(290.38, 84.68) --
	(290.81, 84.68) --
	(291.23, 84.68) --
	(291.65, 84.68) --
	(292.07, 84.68) --
	(292.50, 84.68) --
	(292.92, 84.68) --
	(293.34, 84.68) --
	(293.76, 84.68) --
	(294.19, 84.68) --
	(294.61, 84.68) --
	(295.03, 84.68) --
	(295.45, 84.68) --
	(295.87, 84.68) --
	(296.30, 84.68) --
	(296.72, 84.68) --
	(297.14, 84.68) --
	(297.56, 84.68) --
	(297.99, 84.68) --
	(298.41, 84.68) --
	(298.83, 84.68) --
	(299.25, 84.68) --
	(299.68, 84.68) --
	(300.10, 84.68) --
	(300.52, 84.68) --
	(300.94, 84.68) --
	(301.37, 84.68) --
	(301.79, 84.68) --
	(302.21, 84.68) --
	(302.63, 84.68) --
	(303.06, 84.68) --
	(303.48, 84.68) --
	(303.90, 84.68) --
	(304.32, 84.68) --
	(304.75, 84.68) --
	(305.17, 84.68) --
	(305.59, 84.68) --
	(306.01, 84.68) --
	(306.44, 84.68) --
	(306.86, 84.68) --
	(307.28, 84.68) --
	(307.70, 84.68) --
	(308.12, 84.68) --
	(308.55, 84.68) --
	(308.97, 84.68) --
	(309.39, 84.68) --
	(309.81, 84.68) --
	(310.24, 84.68) --
	(310.66, 84.68) --
	(311.08, 84.68) --
	(311.50, 84.68) --
	(311.93, 84.68) --
	(312.35, 84.68) --
	(312.77, 84.68) --
	(313.19, 84.68) --
	(313.62, 84.68) --
	(314.04, 84.68) --
	(314.46, 84.68) --
	(314.88, 84.68) --
	(315.31, 84.68) --
	(315.73, 84.68) --
	(316.15, 84.68) --
	(316.57, 84.68) --
	(317.00, 84.68) --
	(317.42, 84.68) --
	(317.84, 84.68) --
	(318.26, 84.68) --
	(318.69, 84.68) --
	(319.11, 84.68) --
	(319.53, 84.68) --
	(319.95, 84.68) --
	(320.38, 84.68) --
	(320.80, 84.68) --
	(321.22, 84.68) --
	(321.64, 84.68) --
	(322.06, 84.68) --
	(322.49, 84.68) --
	(322.91, 84.68) --
	(323.33, 84.68) --
	(323.75, 84.68) --
	(324.18, 84.68) --
	(324.60, 84.68) --
	(325.02, 84.68) --
	(325.44, 84.68) --
	(325.87, 84.68) --
	(326.29, 84.68) --
	(326.71, 84.68) --
	(327.13, 84.68) --
	(327.56, 84.68) --
	(327.98, 84.68) --
	(328.40, 84.68) --
	(328.82, 84.68) --
	(329.25, 84.68) --
	(329.67, 84.68) --
	(330.09, 84.68) --
	(330.51, 86.34) --
	(330.94, 86.34) --
	(331.36, 86.34) --
	(331.78, 86.34) --
	(332.20, 86.34) --
	(332.63, 86.34) --
	(333.05, 86.34) --
	(333.47, 86.34) --
	(333.89, 86.34) --
	(334.32, 86.34) --
	(334.74, 86.34) --
	(335.16, 86.34) --
	(335.58, 86.34) --
	(336.00, 86.34) --
	(336.43, 86.34) --
	(336.85, 86.34) --
	(337.27, 86.34) --
	(337.69, 86.34) --
	(338.12, 86.34) --
	(338.54, 86.34) --
	(338.96, 86.34) --
	(339.38, 86.34) --
	(339.81, 86.34) --
	(340.23, 86.34) --
	(340.65, 86.34) --
	(341.07, 86.34) --
	(341.50, 86.34) --
	(341.92, 86.34) --
	(342.34, 86.34) --
	(342.76, 86.34) --
	(343.19, 86.34) --
	(343.61, 86.34) --
	(344.03, 86.34) --
	(344.45, 86.34) --
	(344.88, 86.34) --
	(345.30, 86.34) --
	(345.72, 86.34) --
	(346.14, 86.34) --
	(346.57, 86.34) --
	(346.99, 86.34) --
	(347.41, 86.34) --
	(347.83, 86.34) --
	(348.25, 86.34) --
	(348.68, 86.34) --
	(349.10, 86.34) --
	(349.52, 86.34);
\end{scope}
\end{tikzpicture}
}
	\end{adjustbox}
	\newline
	\begin{adjustbox}{width=1\textwidth}
		\subfloat[Test][Zusammenhang ($r$) zwischen der Steigung und dem \gls{si}.]{% Created by tikzDevice version 0.10.1 on 2016-08-22 08:52:12
% !TEX encoding = UTF-8 Unicode
\begin{tikzpicture}[x=1pt,y=1pt]
\definecolor{fillColor}{RGB}{255,255,255}
\path[use as bounding box,fill=fillColor,fill opacity=0.00] (0,0) rectangle (361.35,216.81);
\begin{scope}
\path[clip] (  0.00,  0.00) rectangle (361.35,216.81);
\definecolor{drawColor}{RGB}{0,0,0}

\node[text=drawColor,anchor=base,inner sep=0pt, outer sep=0pt, scale=  1.00] at (201.68,  8.40) {\textit{RMSE}-Grenzwert (ms)};

\node[text=drawColor,rotate= 90.00,anchor=base,inner sep=0pt, outer sep=0pt, scale=  1.00] at (  9.60,129.40) {\textit{r}};
\end{scope}
\begin{scope}
\path[clip] (  0.00,  0.00) rectangle (361.35,216.81);
\definecolor{drawColor}{RGB}{0,0,0}

\path[draw=drawColor,line width= 0.4pt,line join=round,line cap=round] ( 58.05, 48.00) -- (349.52, 48.00);

\path[draw=drawColor,line width= 0.4pt,line join=round,line cap=round] ( 58.05, 48.00) -- ( 58.05, 42.00);

\path[draw=drawColor,line width= 0.4pt,line join=round,line cap=round] ( 96.07, 48.00) -- ( 96.07, 42.00);

\path[draw=drawColor,line width= 0.4pt,line join=round,line cap=round] (138.31, 48.00) -- (138.31, 42.00);

\path[draw=drawColor,line width= 0.4pt,line join=round,line cap=round] (180.55, 48.00) -- (180.55, 42.00);

\path[draw=drawColor,line width= 0.4pt,line join=round,line cap=round] (222.80, 48.00) -- (222.80, 42.00);

\path[draw=drawColor,line width= 0.4pt,line join=round,line cap=round] (265.04, 48.00) -- (265.04, 42.00);

\path[draw=drawColor,line width= 0.4pt,line join=round,line cap=round] (307.28, 48.00) -- (307.28, 42.00);

\path[draw=drawColor,line width= 0.4pt,line join=round,line cap=round] (349.52, 48.00) -- (349.52, 42.00);

\node[text=drawColor,anchor=base,inner sep=0pt, outer sep=0pt, scale=  1.00] at ( 58.05, 30.00) {1};

\node[text=drawColor,anchor=base,inner sep=0pt, outer sep=0pt, scale=  1.00] at ( 96.07, 30.00) {10};

\node[text=drawColor,anchor=base,inner sep=0pt, outer sep=0pt, scale=  1.00] at (138.31, 30.00) {20};

\node[text=drawColor,anchor=base,inner sep=0pt, outer sep=0pt, scale=  1.00] at (180.55, 30.00) {30};

\node[text=drawColor,anchor=base,inner sep=0pt, outer sep=0pt, scale=  1.00] at (222.80, 30.00) {40};

\node[text=drawColor,anchor=base,inner sep=0pt, outer sep=0pt, scale=  1.00] at (265.04, 30.00) {50};

\node[text=drawColor,anchor=base,inner sep=0pt, outer sep=0pt, scale=  1.00] at (307.28, 30.00) {60};

\node[text=drawColor,anchor=base,inner sep=0pt, outer sep=0pt, scale=  1.00] at (349.52, 30.00) {70};

\path[draw=drawColor,line width= 0.4pt,line join=round,line cap=round] ( 42.00, 54.03) -- ( 42.00,204.78);

\path[draw=drawColor,line width= 0.4pt,line join=round,line cap=round] ( 42.00, 54.03) -- ( 36.00, 54.03);

\path[draw=drawColor,line width= 0.4pt,line join=round,line cap=round] ( 42.00, 72.87) -- ( 36.00, 72.87);

\path[draw=drawColor,line width= 0.4pt,line join=round,line cap=round] ( 42.00, 91.72) -- ( 36.00, 91.72);

\path[draw=drawColor,line width= 0.4pt,line join=round,line cap=round] ( 42.00,110.56) -- ( 36.00,110.56);

\path[draw=drawColor,line width= 0.4pt,line join=round,line cap=round] ( 42.00,129.40) -- ( 36.00,129.40);

\path[draw=drawColor,line width= 0.4pt,line join=round,line cap=round] ( 42.00,148.25) -- ( 36.00,148.25);

\path[draw=drawColor,line width= 0.4pt,line join=round,line cap=round] ( 42.00,167.09) -- ( 36.00,167.09);

\path[draw=drawColor,line width= 0.4pt,line join=round,line cap=round] ( 42.00,185.94) -- ( 36.00,185.94);

\path[draw=drawColor,line width= 0.4pt,line join=round,line cap=round] ( 42.00,204.78) -- ( 36.00,204.78);

\node[text=drawColor,anchor=base east,inner sep=0pt, outer sep=0pt, scale=  1.00] at ( 33.60, 50.59) {--1.00};

\node[text=drawColor,anchor=base east,inner sep=0pt, outer sep=0pt, scale=  1.00] at ( 33.60, 69.43) {--.75};

\node[text=drawColor,anchor=base east,inner sep=0pt, outer sep=0pt, scale=  1.00] at ( 33.60, 88.27) {--.50};

\node[text=drawColor,anchor=base east,inner sep=0pt, outer sep=0pt, scale=  1.00] at ( 33.60,107.12) {--.25};

\node[text=drawColor,anchor=base east,inner sep=0pt, outer sep=0pt, scale=  1.00] at ( 33.60,125.96) {.00};

\node[text=drawColor,anchor=base east,inner sep=0pt, outer sep=0pt, scale=  1.00] at ( 33.60,144.81) {.25};

\node[text=drawColor,anchor=base east,inner sep=0pt, outer sep=0pt, scale=  1.00] at ( 33.60,163.65) {.50};

\node[text=drawColor,anchor=base east,inner sep=0pt, outer sep=0pt, scale=  1.00] at ( 33.60,182.49) {.75};

\node[text=drawColor,anchor=base east,inner sep=0pt, outer sep=0pt, scale=  1.00] at ( 33.60,201.34) {1.00};
\end{scope}
\begin{scope}
\path[clip] ( 42.00, 48.00) rectangle (361.35,210.81);
\definecolor{fillColor}{RGB}{190,190,190}

\path[fill=fillColor] ( 58.05,202.75) --
	( 58.47,202.75) --
	( 58.90,202.75) --
	( 59.32,204.27) --
	( 59.74,204.27) --
	( 60.16,204.45) --
	( 60.59,204.45) --
	( 61.01,203.84) --
	( 61.43,202.03) --
	( 61.85,203.71) --
	( 62.28,203.70) --
	( 62.70,203.90) --
	( 63.12,203.83) --
	( 63.54,203.79) --
	( 63.97,203.79) --
	( 64.39,203.24) --
	( 64.81,203.55) --
	( 65.23,203.55) --
	( 65.66,203.73) --
	( 66.08,203.68) --
	( 66.50,203.74) --
	( 66.92,203.99) --
	( 67.35,203.99) --
	( 67.77,203.89) --
	( 68.19,203.89) --
	( 68.61,203.85) --
	( 69.03,203.86) --
	( 69.46,203.86) --
	( 69.88,203.86) --
	( 70.30,203.82) --
	( 70.72,203.80) --
	( 71.15,203.87) --
	( 71.57,203.84) --
	( 71.99,203.89) --
	( 72.41,203.89) --
	( 72.84,203.88) --
	( 73.26,203.79) --
	( 73.68,203.82) --
	( 74.10,203.82) --
	( 74.53,203.65) --
	( 74.95,203.58) --
	( 75.37,203.63) --
	( 75.79,203.65) --
	( 76.22,203.65) --
	( 76.64,203.65) --
	( 77.06,203.60) --
	( 77.48,203.57) --
	( 77.91,203.55) --
	( 78.33,203.54) --
	( 78.75,203.54) --
	( 79.17,203.54) --
	( 79.60,203.54) --
	( 80.02,203.71) --
	( 80.44,203.58) --
	( 80.86,203.58) --
	( 81.29,203.51) --
	( 81.71,203.52) --
	( 82.13,203.58) --
	( 82.55,203.57) --
	( 82.97,203.56) --
	( 83.40,203.58) --
	( 83.82,203.50) --
	( 84.24,203.50) --
	( 84.66,203.49) --
	( 85.09,203.40) --
	( 85.51,203.40) --
	( 85.93,203.30) --
	( 86.35,203.30) --
	( 86.78,203.30) --
	( 87.20,203.30) --
	( 87.62,203.30) --
	( 88.04,203.31) --
	( 88.47,203.33) --
	( 88.89,203.33) --
	( 89.31,203.40) --
	( 89.73,203.35) --
	( 90.16,203.38) --
	( 90.58,203.38) --
	( 91.00,203.38) --
	( 91.42,203.39) --
	( 91.85,203.39) --
	( 92.27,203.39) --
	( 92.69,203.39) --
	( 93.11,203.39) --
	( 93.54,203.39) --
	( 93.96,203.39) --
	( 94.38,203.43) --
	( 94.80,203.43) --
	( 95.22,203.43) --
	( 95.65,203.43) --
	( 96.07,203.43) --
	( 96.49,203.43) --
	( 96.91,203.43) --
	( 97.34,203.45) --
	( 97.76,203.45) --
	( 98.18,203.45) --
	( 98.60,203.45) --
	( 99.03,203.45) --
	( 99.45,203.48) --
	( 99.87,203.46) --
	(100.29,203.44) --
	(100.72,203.44) --
	(101.14,203.46) --
	(101.56,203.46) --
	(101.98,203.27) --
	(102.41,203.27) --
	(102.83,203.27) --
	(103.25,203.27) --
	(103.67,203.27) --
	(104.10,203.27) --
	(104.52,203.27) --
	(104.94,203.27) --
	(105.36,203.27) --
	(105.79,203.26) --
	(106.21,203.26) --
	(106.63,203.26) --
	(107.05,203.26) --
	(107.48,203.26) --
	(107.90,203.26) --
	(108.32,203.22) --
	(108.74,203.24) --
	(109.16,203.24) --
	(109.59,203.24) --
	(110.01,203.24) --
	(110.43,203.24) --
	(110.85,203.24) --
	(111.28,203.24) --
	(111.70,203.24) --
	(112.12,203.24) --
	(112.54,203.24) --
	(112.97,203.24) --
	(113.39,203.24) --
	(113.81,202.86) --
	(114.23,202.84) --
	(114.66,202.84) --
	(115.08,202.84) --
	(115.50,202.84) --
	(115.92,202.84) --
	(116.35,202.84) --
	(116.77,202.82) --
	(117.19,202.82) --
	(117.61,202.82) --
	(118.04,202.82) --
	(118.46,202.81) --
	(118.88,202.81) --
	(119.30,202.81) --
	(119.73,202.81) --
	(120.15,202.88) --
	(120.57,202.88) --
	(120.99,202.88) --
	(121.42,202.88) --
	(121.84,202.88) --
	(122.26,202.88) --
	(122.68,202.88) --
	(123.10,202.89) --
	(123.53,202.90) --
	(123.95,202.90) --
	(124.37,202.90) --
	(124.79,202.90) --
	(125.22,202.92) --
	(125.64,202.92) --
	(126.06,202.92) --
	(126.48,202.92) --
	(126.91,202.92) --
	(127.33,202.92) --
	(127.75,202.92) --
	(128.17,202.92) --
	(128.60,202.92) --
	(129.02,202.85) --
	(129.44,202.85) --
	(129.86,202.85) --
	(130.29,202.85) --
	(130.71,202.85) --
	(131.13,202.85) --
	(131.55,202.85) --
	(131.98,202.85) --
	(132.40,202.90) --
	(132.82,202.90) --
	(133.24,202.90) --
	(133.67,202.90) --
	(134.09,202.90) --
	(134.51,202.90) --
	(134.93,202.90) --
	(135.35,202.90) --
	(135.78,202.90) --
	(136.20,202.90) --
	(136.62,202.90) --
	(137.04,202.90) --
	(137.47,202.90) --
	(137.89,202.90) --
	(138.31,202.90) --
	(138.73,202.90) --
	(139.16,202.90) --
	(139.58,202.90) --
	(140.00,202.90) --
	(140.42,202.90) --
	(140.85,202.90) --
	(141.27,202.90) --
	(141.69,202.90) --
	(142.11,202.90) --
	(142.54,202.90) --
	(142.96,202.90) --
	(143.38,202.90) --
	(143.80,202.90) --
	(144.23,202.90) --
	(144.65,202.90) --
	(145.07,202.90) --
	(145.49,202.90) --
	(145.92,202.90) --
	(146.34,202.90) --
	(146.76,202.90) --
	(147.18,202.90) --
	(147.61,202.92) --
	(148.03,202.92) --
	(148.45,202.92) --
	(148.87,202.92) --
	(149.29,202.92) --
	(149.72,202.92) --
	(150.14,202.92) --
	(150.56,202.92) --
	(150.98,202.92) --
	(151.41,202.92) --
	(151.83,202.92) --
	(152.25,202.92) --
	(152.67,202.92) --
	(153.10,202.92) --
	(153.52,202.92) --
	(153.94,202.92) --
	(154.36,202.92) --
	(154.79,202.92) --
	(155.21,202.92) --
	(155.63,202.92) --
	(156.05,202.92) --
	(156.48,202.92) --
	(156.90,202.92) --
	(157.32,202.92) --
	(157.74,202.92) --
	(158.17,202.92) --
	(158.59,202.92) --
	(159.01,202.92) --
	(159.43,202.92) --
	(159.86,202.92) --
	(160.28,202.92) --
	(160.70,202.92) --
	(161.12,202.92) --
	(161.55,202.92) --
	(161.97,202.92) --
	(162.39,202.92) --
	(162.81,202.92) --
	(163.23,202.92) --
	(163.66,202.92) --
	(164.08,202.86) --
	(164.50,202.86) --
	(164.92,202.86) --
	(165.35,202.86) --
	(165.77,202.86) --
	(166.19,202.86) --
	(166.61,202.86) --
	(167.04,202.86) --
	(167.46,202.86) --
	(167.88,202.86) --
	(168.30,202.86) --
	(168.73,202.86) --
	(169.15,202.86) --
	(169.57,202.86) --
	(169.99,202.86) --
	(170.42,202.86) --
	(170.84,202.86) --
	(171.26,202.86) --
	(171.68,202.86) --
	(172.11,202.86) --
	(172.53,202.86) --
	(172.95,202.86) --
	(173.37,202.86) --
	(173.80,202.86) --
	(174.22,202.86) --
	(174.64,202.86) --
	(175.06,202.86) --
	(175.48,202.86) --
	(175.91,202.86) --
	(176.33,202.86) --
	(176.75,202.86) --
	(177.17,202.86) --
	(177.60,202.86) --
	(178.02,202.86) --
	(178.44,202.86) --
	(178.86,202.86) --
	(179.29,202.86) --
	(179.71,202.86) --
	(180.13,202.86) --
	(180.55,202.86) --
	(180.98,202.86) --
	(181.40,202.86) --
	(181.82,202.86) --
	(182.24,202.86) --
	(182.67,202.86) --
	(183.09,202.86) --
	(183.51,202.86) --
	(183.93,202.86) --
	(184.36,203.02) --
	(184.78,203.02) --
	(185.20,203.02) --
	(185.62,203.02) --
	(186.05,203.02) --
	(186.47,203.02) --
	(186.89,203.02) --
	(187.31,203.02) --
	(187.74,203.02) --
	(188.16,202.91) --
	(188.58,202.91) --
	(189.00,202.91) --
	(189.42,202.91) --
	(189.85,202.91) --
	(190.27,202.91) --
	(190.69,202.91) --
	(191.11,202.91) --
	(191.54,202.91) --
	(191.96,202.91) --
	(192.38,202.91) --
	(192.80,202.91) --
	(193.23,202.91) --
	(193.65,202.91) --
	(194.07,202.91) --
	(194.49,202.91) --
	(194.92,202.91) --
	(195.34,202.91) --
	(195.76,202.91) --
	(196.18,202.91) --
	(196.61,202.91) --
	(197.03,202.91) --
	(197.45,202.91) --
	(197.87,202.91) --
	(198.30,202.91) --
	(198.72,202.91) --
	(199.14,202.91) --
	(199.56,202.91) --
	(199.99,202.91) --
	(200.41,202.91) --
	(200.83,202.91) --
	(201.25,202.91) --
	(201.68,202.91) --
	(202.10,202.91) --
	(202.52,202.91) --
	(202.94,202.94) --
	(203.36,202.94) --
	(203.79,202.94) --
	(204.21,202.94) --
	(204.63,202.94) --
	(205.05,200.93) --
	(205.48,200.93) --
	(205.90,200.93) --
	(206.32,200.93) --
	(206.74,200.93) --
	(207.17,200.93) --
	(207.59,200.93) --
	(208.01,200.94) --
	(208.43,200.94) --
	(208.86,200.94) --
	(209.28,200.94) --
	(209.70,200.94) --
	(210.12,200.94) --
	(210.55,200.94) --
	(210.97,200.92) --
	(211.39,200.92) --
	(211.81,200.92) --
	(212.24,200.92) --
	(212.66,200.92) --
	(213.08,200.92) --
	(213.50,200.92) --
	(213.93,200.92) --
	(214.35,200.92) --
	(214.77,200.92) --
	(215.19,200.92) --
	(215.61,200.92) --
	(216.04,200.92) --
	(216.46,200.92) --
	(216.88,200.92) --
	(217.30,200.92) --
	(217.73,200.92) --
	(218.15,200.92) --
	(218.57,200.92) --
	(218.99,200.92) --
	(219.42,200.92) --
	(219.84,200.92) --
	(220.26,200.92) --
	(220.68,200.92) --
	(221.11,200.92) --
	(221.53,200.92) --
	(221.95,200.92) --
	(222.37,200.92) --
	(222.80,200.92) --
	(223.22,200.92) --
	(223.64,200.92) --
	(224.06,200.92) --
	(224.49,200.92) --
	(224.91,200.92) --
	(225.33,200.92) --
	(225.75,200.92) --
	(226.18,200.92) --
	(226.60,200.92) --
	(227.02,200.92) --
	(227.44,200.92) --
	(227.87,200.92) --
	(228.29,200.92) --
	(228.71,200.92) --
	(229.13,200.92) --
	(229.55,200.92) --
	(229.98,200.92) --
	(230.40,200.92) --
	(230.82,200.92) --
	(231.24,200.92) --
	(231.67,200.92) --
	(232.09,200.92) --
	(232.51,200.92) --
	(232.93,200.92) --
	(233.36,200.92) --
	(233.78,200.92) --
	(234.20,200.92) --
	(234.62,200.92) --
	(235.05,200.92) --
	(235.47,200.92) --
	(235.89,200.92) --
	(236.31,200.92) --
	(236.74,200.92) --
	(237.16,200.92) --
	(237.58,200.92) --
	(238.00,200.92) --
	(238.43,200.92) --
	(238.85,200.92) --
	(239.27,200.92) --
	(239.69,200.92) --
	(240.12,200.92) --
	(240.54,200.92) --
	(240.96,200.92) --
	(241.38,200.92) --
	(241.80,200.92) --
	(242.23,200.92) --
	(242.65,200.92) --
	(243.07,200.92) --
	(243.49,200.92) --
	(243.92,200.92) --
	(244.34,200.92) --
	(244.76,200.92) --
	(245.18,200.92) --
	(245.61,200.92) --
	(246.03,200.92) --
	(246.45,200.92) --
	(246.87,200.92) --
	(247.30,200.92) --
	(247.72,200.92) --
	(248.14,200.92) --
	(248.56,200.92) --
	(248.99,200.92) --
	(249.41,200.92) --
	(249.83,200.92) --
	(250.25,200.92) --
	(250.68,200.92) --
	(251.10,200.92) --
	(251.52,200.92) --
	(251.94,200.92) --
	(252.37,200.92) --
	(252.79,200.92) --
	(253.21,200.92) --
	(253.63,200.92) --
	(254.06,200.92) --
	(254.48,200.92) --
	(254.90,200.92) --
	(255.32,200.92) --
	(255.74,200.92) --
	(256.17,200.92) --
	(256.59,200.92) --
	(257.01,200.92) --
	(257.43,200.92) --
	(257.86,200.92) --
	(258.28,200.92) --
	(258.70,200.92) --
	(259.12,200.92) --
	(259.55,200.92) --
	(259.97,200.92) --
	(260.39,200.92) --
	(260.81,200.92) --
	(261.24,200.92) --
	(261.66,200.92) --
	(262.08,200.92) --
	(262.50,200.92) --
	(262.93,200.92) --
	(263.35,200.92) --
	(263.77,200.92) --
	(264.19,200.92) --
	(264.62,200.91) --
	(265.04,200.91) --
	(265.46,200.91) --
	(265.88,200.91) --
	(266.31,200.91) --
	(266.73,200.91) --
	(267.15,200.91) --
	(267.57,200.91) --
	(268.00,200.91) --
	(268.42,200.91) --
	(268.84,200.91) --
	(269.26,200.91) --
	(269.68,200.91) --
	(270.11,200.91) --
	(270.53,200.91) --
	(270.95,200.91) --
	(271.37,200.91) --
	(271.80,200.91) --
	(272.22,200.91) --
	(272.64,200.91) --
	(273.06,200.91) --
	(273.49,200.91) --
	(273.91,200.91) --
	(274.33,200.91) --
	(274.75,200.91) --
	(275.18,200.91) --
	(275.60,200.91) --
	(276.02,200.91) --
	(276.44,200.91) --
	(276.87,200.91) --
	(277.29,200.91) --
	(277.71,200.91) --
	(278.13,200.91) --
	(278.56,200.91) --
	(278.98,200.91) --
	(279.40,200.91) --
	(279.82,200.91) --
	(280.25,200.91) --
	(280.67,200.91) --
	(281.09,200.91) --
	(281.51,200.91) --
	(281.93,200.91) --
	(282.36,200.91) --
	(282.78,200.91) --
	(283.20,200.91) --
	(283.62,200.91) --
	(284.05,200.91) --
	(284.47,200.91) --
	(284.89,200.91) --
	(285.31,200.91) --
	(285.74,200.91) --
	(286.16,200.91) --
	(286.58,200.91) --
	(287.00,200.91) --
	(287.43,200.91) --
	(287.85,200.91) --
	(288.27,200.91) --
	(288.69,200.91) --
	(289.12,200.91) --
	(289.54,200.91) --
	(289.96,200.91) --
	(290.38,200.91) --
	(290.81,200.91) --
	(291.23,200.91) --
	(291.65,200.91) --
	(292.07,200.91) --
	(292.50,200.91) --
	(292.92,200.91) --
	(293.34,200.91) --
	(293.76,200.91) --
	(294.19,200.91) --
	(294.61,200.91) --
	(295.03,200.91) --
	(295.45,200.91) --
	(295.87,200.91) --
	(296.30,200.91) --
	(296.72,200.91) --
	(297.14,200.91) --
	(297.56,200.91) --
	(297.99,200.91) --
	(298.41,200.91) --
	(298.83,200.91) --
	(299.25,200.91) --
	(299.68,200.91) --
	(300.10,200.91) --
	(300.52,200.91) --
	(300.94,200.91) --
	(301.37,200.91) --
	(301.79,200.91) --
	(302.21,200.91) --
	(302.63,200.91) --
	(303.06,200.91) --
	(303.48,200.91) --
	(303.90,200.91) --
	(304.32,200.91) --
	(304.75,200.91) --
	(305.17,200.91) --
	(305.59,200.91) --
	(306.01,200.91) --
	(306.44,200.91) --
	(306.86,200.91) --
	(307.28,200.91) --
	(307.70,200.91) --
	(308.12,200.91) --
	(308.55,200.91) --
	(308.97,200.91) --
	(309.39,200.91) --
	(309.81,200.91) --
	(310.24,200.91) --
	(310.66,200.91) --
	(311.08,200.91) --
	(311.50,200.91) --
	(311.93,200.91) --
	(312.35,200.91) --
	(312.77,200.91) --
	(313.19,200.91) --
	(313.62,200.91) --
	(314.04,200.91) --
	(314.46,200.91) --
	(314.88,200.91) --
	(315.31,200.91) --
	(315.73,200.91) --
	(316.15,200.91) --
	(316.57,200.91) --
	(317.00,200.91) --
	(317.42,200.91) --
	(317.84,200.91) --
	(318.26,200.91) --
	(318.69,200.91) --
	(319.11,200.91) --
	(319.53,200.91) --
	(319.95,200.91) --
	(320.38,200.91) --
	(320.80,200.91) --
	(321.22,200.91) --
	(321.64,200.91) --
	(322.06,200.91) --
	(322.49,200.91) --
	(322.91,200.91) --
	(323.33,200.91) --
	(323.75,200.91) --
	(324.18,200.91) --
	(324.60,200.91) --
	(325.02,200.91) --
	(325.44,200.91) --
	(325.87,200.91) --
	(326.29,200.91) --
	(326.71,200.91) --
	(327.13,200.91) --
	(327.56,200.91) --
	(327.98,200.91) --
	(328.40,200.91) --
	(328.82,200.91) --
	(329.25,200.91) --
	(329.67,200.91) --
	(330.09,200.91) --
	(330.51,200.88) --
	(330.94,200.88) --
	(331.36,200.88) --
	(331.78,200.88) --
	(332.20,200.88) --
	(332.63,200.88) --
	(333.05,200.88) --
	(333.47,200.88) --
	(333.89,200.88) --
	(334.32,200.88) --
	(334.74,200.88) --
	(335.16,200.88) --
	(335.58,200.88) --
	(336.00,200.88) --
	(336.43,200.88) --
	(336.85,200.88) --
	(337.27,200.88) --
	(337.69,200.88) --
	(338.12,200.88) --
	(338.54,200.88) --
	(338.96,200.88) --
	(339.38,200.88) --
	(339.81,200.88) --
	(340.23,200.88) --
	(340.65,200.88) --
	(341.07,200.88) --
	(341.50,200.88) --
	(341.92,200.88) --
	(342.34,200.88) --
	(342.76,200.88) --
	(343.19,200.88) --
	(343.61,200.88) --
	(344.03,200.88) --
	(344.45,200.88) --
	(344.88,200.88) --
	(345.30,200.88) --
	(345.72,200.88) --
	(346.14,200.88) --
	(346.57,200.88) --
	(346.99,200.88) --
	(347.41,200.88) --
	(347.83,200.88) --
	(348.25,200.88) --
	(348.68,200.88) --
	(349.10,200.88) --
	(349.52,200.88) --
	(349.52,202.60) --
	(349.10,202.60) --
	(348.68,202.60) --
	(348.25,202.60) --
	(347.83,202.60) --
	(347.41,202.60) --
	(346.99,202.60) --
	(346.57,202.60) --
	(346.14,202.60) --
	(345.72,202.60) --
	(345.30,202.60) --
	(344.88,202.60) --
	(344.45,202.60) --
	(344.03,202.60) --
	(343.61,202.60) --
	(343.19,202.60) --
	(342.76,202.60) --
	(342.34,202.60) --
	(341.92,202.60) --
	(341.50,202.60) --
	(341.07,202.60) --
	(340.65,202.60) --
	(340.23,202.60) --
	(339.81,202.60) --
	(339.38,202.60) --
	(338.96,202.60) --
	(338.54,202.60) --
	(338.12,202.60) --
	(337.69,202.60) --
	(337.27,202.60) --
	(336.85,202.60) --
	(336.43,202.60) --
	(336.00,202.60) --
	(335.58,202.60) --
	(335.16,202.60) --
	(334.74,202.60) --
	(334.32,202.60) --
	(333.89,202.60) --
	(333.47,202.60) --
	(333.05,202.60) --
	(332.63,202.60) --
	(332.20,202.60) --
	(331.78,202.60) --
	(331.36,202.60) --
	(330.94,202.60) --
	(330.51,202.60) --
	(330.09,202.62) --
	(329.67,202.62) --
	(329.25,202.62) --
	(328.82,202.62) --
	(328.40,202.62) --
	(327.98,202.62) --
	(327.56,202.62) --
	(327.13,202.62) --
	(326.71,202.62) --
	(326.29,202.62) --
	(325.87,202.62) --
	(325.44,202.62) --
	(325.02,202.62) --
	(324.60,202.62) --
	(324.18,202.62) --
	(323.75,202.62) --
	(323.33,202.62) --
	(322.91,202.62) --
	(322.49,202.62) --
	(322.06,202.62) --
	(321.64,202.62) --
	(321.22,202.62) --
	(320.80,202.62) --
	(320.38,202.62) --
	(319.95,202.62) --
	(319.53,202.62) --
	(319.11,202.62) --
	(318.69,202.62) --
	(318.26,202.62) --
	(317.84,202.62) --
	(317.42,202.62) --
	(317.00,202.62) --
	(316.57,202.62) --
	(316.15,202.62) --
	(315.73,202.62) --
	(315.31,202.62) --
	(314.88,202.62) --
	(314.46,202.62) --
	(314.04,202.62) --
	(313.62,202.62) --
	(313.19,202.62) --
	(312.77,202.62) --
	(312.35,202.62) --
	(311.93,202.62) --
	(311.50,202.62) --
	(311.08,202.62) --
	(310.66,202.62) --
	(310.24,202.62) --
	(309.81,202.62) --
	(309.39,202.62) --
	(308.97,202.62) --
	(308.55,202.62) --
	(308.12,202.62) --
	(307.70,202.62) --
	(307.28,202.62) --
	(306.86,202.62) --
	(306.44,202.62) --
	(306.01,202.62) --
	(305.59,202.62) --
	(305.17,202.62) --
	(304.75,202.62) --
	(304.32,202.62) --
	(303.90,202.62) --
	(303.48,202.62) --
	(303.06,202.62) --
	(302.63,202.62) --
	(302.21,202.62) --
	(301.79,202.62) --
	(301.37,202.62) --
	(300.94,202.62) --
	(300.52,202.62) --
	(300.10,202.62) --
	(299.68,202.62) --
	(299.25,202.62) --
	(298.83,202.62) --
	(298.41,202.62) --
	(297.99,202.62) --
	(297.56,202.62) --
	(297.14,202.62) --
	(296.72,202.62) --
	(296.30,202.62) --
	(295.87,202.62) --
	(295.45,202.62) --
	(295.03,202.62) --
	(294.61,202.62) --
	(294.19,202.62) --
	(293.76,202.62) --
	(293.34,202.62) --
	(292.92,202.62) --
	(292.50,202.62) --
	(292.07,202.62) --
	(291.65,202.62) --
	(291.23,202.62) --
	(290.81,202.62) --
	(290.38,202.62) --
	(289.96,202.62) --
	(289.54,202.62) --
	(289.12,202.62) --
	(288.69,202.62) --
	(288.27,202.62) --
	(287.85,202.62) --
	(287.43,202.62) --
	(287.00,202.62) --
	(286.58,202.62) --
	(286.16,202.62) --
	(285.74,202.62) --
	(285.31,202.62) --
	(284.89,202.62) --
	(284.47,202.62) --
	(284.05,202.62) --
	(283.62,202.62) --
	(283.20,202.62) --
	(282.78,202.62) --
	(282.36,202.62) --
	(281.93,202.62) --
	(281.51,202.62) --
	(281.09,202.62) --
	(280.67,202.62) --
	(280.25,202.62) --
	(279.82,202.62) --
	(279.40,202.62) --
	(278.98,202.62) --
	(278.56,202.62) --
	(278.13,202.62) --
	(277.71,202.62) --
	(277.29,202.62) --
	(276.87,202.62) --
	(276.44,202.62) --
	(276.02,202.62) --
	(275.60,202.62) --
	(275.18,202.62) --
	(274.75,202.62) --
	(274.33,202.62) --
	(273.91,202.62) --
	(273.49,202.62) --
	(273.06,202.62) --
	(272.64,202.62) --
	(272.22,202.62) --
	(271.80,202.62) --
	(271.37,202.62) --
	(270.95,202.62) --
	(270.53,202.62) --
	(270.11,202.62) --
	(269.68,202.62) --
	(269.26,202.62) --
	(268.84,202.62) --
	(268.42,202.62) --
	(268.00,202.62) --
	(267.57,202.62) --
	(267.15,202.62) --
	(266.73,202.62) --
	(266.31,202.62) --
	(265.88,202.62) --
	(265.46,202.62) --
	(265.04,202.62) --
	(264.62,202.62) --
	(264.19,202.63) --
	(263.77,202.63) --
	(263.35,202.63) --
	(262.93,202.63) --
	(262.50,202.63) --
	(262.08,202.63) --
	(261.66,202.63) --
	(261.24,202.63) --
	(260.81,202.63) --
	(260.39,202.63) --
	(259.97,202.63) --
	(259.55,202.63) --
	(259.12,202.63) --
	(258.70,202.63) --
	(258.28,202.63) --
	(257.86,202.63) --
	(257.43,202.63) --
	(257.01,202.63) --
	(256.59,202.63) --
	(256.17,202.63) --
	(255.74,202.63) --
	(255.32,202.63) --
	(254.90,202.63) --
	(254.48,202.63) --
	(254.06,202.63) --
	(253.63,202.63) --
	(253.21,202.63) --
	(252.79,202.63) --
	(252.37,202.63) --
	(251.94,202.63) --
	(251.52,202.63) --
	(251.10,202.63) --
	(250.68,202.63) --
	(250.25,202.63) --
	(249.83,202.63) --
	(249.41,202.63) --
	(248.99,202.63) --
	(248.56,202.63) --
	(248.14,202.63) --
	(247.72,202.63) --
	(247.30,202.63) --
	(246.87,202.63) --
	(246.45,202.63) --
	(246.03,202.63) --
	(245.61,202.63) --
	(245.18,202.63) --
	(244.76,202.63) --
	(244.34,202.63) --
	(243.92,202.63) --
	(243.49,202.63) --
	(243.07,202.63) --
	(242.65,202.63) --
	(242.23,202.63) --
	(241.80,202.63) --
	(241.38,202.63) --
	(240.96,202.63) --
	(240.54,202.63) --
	(240.12,202.63) --
	(239.69,202.63) --
	(239.27,202.63) --
	(238.85,202.63) --
	(238.43,202.63) --
	(238.00,202.63) --
	(237.58,202.63) --
	(237.16,202.63) --
	(236.74,202.63) --
	(236.31,202.63) --
	(235.89,202.63) --
	(235.47,202.63) --
	(235.05,202.63) --
	(234.62,202.63) --
	(234.20,202.63) --
	(233.78,202.63) --
	(233.36,202.63) --
	(232.93,202.63) --
	(232.51,202.63) --
	(232.09,202.63) --
	(231.67,202.63) --
	(231.24,202.63) --
	(230.82,202.63) --
	(230.40,202.63) --
	(229.98,202.63) --
	(229.55,202.63) --
	(229.13,202.63) --
	(228.71,202.63) --
	(228.29,202.63) --
	(227.87,202.63) --
	(227.44,202.63) --
	(227.02,202.63) --
	(226.60,202.63) --
	(226.18,202.63) --
	(225.75,202.63) --
	(225.33,202.63) --
	(224.91,202.63) --
	(224.49,202.63) --
	(224.06,202.63) --
	(223.64,202.63) --
	(223.22,202.63) --
	(222.80,202.63) --
	(222.37,202.63) --
	(221.95,202.63) --
	(221.53,202.63) --
	(221.11,202.63) --
	(220.68,202.63) --
	(220.26,202.63) --
	(219.84,202.63) --
	(219.42,202.63) --
	(218.99,202.63) --
	(218.57,202.63) --
	(218.15,202.63) --
	(217.73,202.63) --
	(217.30,202.63) --
	(216.88,202.63) --
	(216.46,202.63) --
	(216.04,202.63) --
	(215.61,202.63) --
	(215.19,202.63) --
	(214.77,202.63) --
	(214.35,202.63) --
	(213.93,202.63) --
	(213.50,202.63) --
	(213.08,202.63) --
	(212.66,202.63) --
	(212.24,202.63) --
	(211.81,202.63) --
	(211.39,202.63) --
	(210.97,202.63) --
	(210.55,202.64) --
	(210.12,202.64) --
	(209.70,202.64) --
	(209.28,202.64) --
	(208.86,202.64) --
	(208.43,202.64) --
	(208.01,202.64) --
	(207.59,202.65) --
	(207.17,202.65) --
	(206.74,202.65) --
	(206.32,202.65) --
	(205.90,202.65) --
	(205.48,202.65) --
	(205.05,202.65) --
	(204.63,203.77) --
	(204.21,203.77) --
	(203.79,203.77) --
	(203.36,203.77) --
	(202.94,203.77) --
	(202.52,203.75) --
	(202.10,203.75) --
	(201.68,203.75) --
	(201.25,203.75) --
	(200.83,203.75) --
	(200.41,203.75) --
	(199.99,203.75) --
	(199.56,203.75) --
	(199.14,203.75) --
	(198.72,203.75) --
	(198.30,203.75) --
	(197.87,203.75) --
	(197.45,203.75) --
	(197.03,203.75) --
	(196.61,203.75) --
	(196.18,203.75) --
	(195.76,203.75) --
	(195.34,203.75) --
	(194.92,203.75) --
	(194.49,203.75) --
	(194.07,203.75) --
	(193.65,203.75) --
	(193.23,203.75) --
	(192.80,203.75) --
	(192.38,203.75) --
	(191.96,203.75) --
	(191.54,203.75) --
	(191.11,203.75) --
	(190.69,203.75) --
	(190.27,203.75) --
	(189.85,203.75) --
	(189.42,203.75) --
	(189.00,203.75) --
	(188.58,203.75) --
	(188.16,203.75) --
	(187.74,203.81) --
	(187.31,203.81) --
	(186.89,203.81) --
	(186.47,203.81) --
	(186.05,203.81) --
	(185.62,203.81) --
	(185.20,203.81) --
	(184.78,203.81) --
	(184.36,203.81) --
	(183.93,203.73) --
	(183.51,203.73) --
	(183.09,203.73) --
	(182.67,203.73) --
	(182.24,203.73) --
	(181.82,203.73) --
	(181.40,203.73) --
	(180.98,203.73) --
	(180.55,203.73) --
	(180.13,203.73) --
	(179.71,203.73) --
	(179.29,203.73) --
	(178.86,203.73) --
	(178.44,203.73) --
	(178.02,203.73) --
	(177.60,203.73) --
	(177.17,203.73) --
	(176.75,203.73) --
	(176.33,203.73) --
	(175.91,203.73) --
	(175.48,203.73) --
	(175.06,203.73) --
	(174.64,203.73) --
	(174.22,203.73) --
	(173.80,203.73) --
	(173.37,203.73) --
	(172.95,203.73) --
	(172.53,203.73) --
	(172.11,203.73) --
	(171.68,203.73) --
	(171.26,203.73) --
	(170.84,203.73) --
	(170.42,203.73) --
	(169.99,203.73) --
	(169.57,203.73) --
	(169.15,203.73) --
	(168.73,203.73) --
	(168.30,203.73) --
	(167.88,203.73) --
	(167.46,203.73) --
	(167.04,203.73) --
	(166.61,203.73) --
	(166.19,203.73) --
	(165.77,203.73) --
	(165.35,203.73) --
	(164.92,203.73) --
	(164.50,203.73) --
	(164.08,203.73) --
	(163.66,203.76) --
	(163.23,203.76) --
	(162.81,203.76) --
	(162.39,203.76) --
	(161.97,203.76) --
	(161.55,203.76) --
	(161.12,203.76) --
	(160.70,203.76) --
	(160.28,203.76) --
	(159.86,203.76) --
	(159.43,203.76) --
	(159.01,203.76) --
	(158.59,203.76) --
	(158.17,203.76) --
	(157.74,203.76) --
	(157.32,203.76) --
	(156.90,203.76) --
	(156.48,203.76) --
	(156.05,203.76) --
	(155.63,203.76) --
	(155.21,203.76) --
	(154.79,203.76) --
	(154.36,203.76) --
	(153.94,203.76) --
	(153.52,203.76) --
	(153.10,203.76) --
	(152.67,203.76) --
	(152.25,203.76) --
	(151.83,203.76) --
	(151.41,203.76) --
	(150.98,203.76) --
	(150.56,203.76) --
	(150.14,203.76) --
	(149.72,203.76) --
	(149.29,203.76) --
	(148.87,203.76) --
	(148.45,203.76) --
	(148.03,203.76) --
	(147.61,203.76) --
	(147.18,203.76) --
	(146.76,203.76) --
	(146.34,203.76) --
	(145.92,203.76) --
	(145.49,203.76) --
	(145.07,203.76) --
	(144.65,203.76) --
	(144.23,203.76) --
	(143.80,203.76) --
	(143.38,203.76) --
	(142.96,203.76) --
	(142.54,203.76) --
	(142.11,203.76) --
	(141.69,203.76) --
	(141.27,203.76) --
	(140.85,203.76) --
	(140.42,203.76) --
	(140.00,203.76) --
	(139.58,203.76) --
	(139.16,203.76) --
	(138.73,203.76) --
	(138.31,203.76) --
	(137.89,203.76) --
	(137.47,203.76) --
	(137.04,203.76) --
	(136.62,203.76) --
	(136.20,203.76) --
	(135.78,203.76) --
	(135.35,203.76) --
	(134.93,203.76) --
	(134.51,203.76) --
	(134.09,203.76) --
	(133.67,203.76) --
	(133.24,203.76) --
	(132.82,203.76) --
	(132.40,203.76) --
	(131.98,203.73) --
	(131.55,203.73) --
	(131.13,203.73) --
	(130.71,203.73) --
	(130.29,203.73) --
	(129.86,203.73) --
	(129.44,203.73) --
	(129.02,203.73) --
	(128.60,203.77) --
	(128.17,203.77) --
	(127.75,203.77) --
	(127.33,203.77) --
	(126.91,203.77) --
	(126.48,203.77) --
	(126.06,203.77) --
	(125.64,203.77) --
	(125.22,203.77) --
	(124.79,203.76) --
	(124.37,203.76) --
	(123.95,203.76) --
	(123.53,203.76) --
	(123.10,203.76) --
	(122.68,203.76) --
	(122.26,203.76) --
	(121.84,203.76) --
	(121.42,203.76) --
	(120.99,203.76) --
	(120.57,203.76) --
	(120.15,203.76) --
	(119.73,203.72) --
	(119.30,203.72) --
	(118.88,203.72) --
	(118.46,203.72) --
	(118.04,203.73) --
	(117.61,203.73) --
	(117.19,203.73) --
	(116.77,203.73) --
	(116.35,203.74) --
	(115.92,203.74) --
	(115.50,203.74) --
	(115.08,203.74) --
	(114.66,203.74) --
	(114.23,203.74) --
	(113.81,203.75) --
	(113.39,203.96) --
	(112.97,203.96) --
	(112.54,203.96) --
	(112.12,203.96) --
	(111.70,203.96) --
	(111.28,203.96) --
	(110.85,203.96) --
	(110.43,203.96) --
	(110.01,203.96) --
	(109.59,203.96) --
	(109.16,203.96) --
	(108.74,203.96) --
	(108.32,203.96) --
	(107.90,203.98) --
	(107.48,203.98) --
	(107.05,203.98) --
	(106.63,203.98) --
	(106.21,203.98) --
	(105.79,203.98) --
	(105.36,203.98) --
	(104.94,203.99) --
	(104.52,203.99) --
	(104.10,203.99) --
	(103.67,203.99) --
	(103.25,203.99) --
	(102.83,203.99) --
	(102.41,203.99) --
	(101.98,203.99) --
	(101.56,204.09) --
	(101.14,204.09) --
	(100.72,204.09) --
	(100.29,204.09) --
	( 99.87,204.10) --
	( 99.45,204.11) --
	( 99.03,204.10) --
	( 98.60,204.10) --
	( 98.18,204.10) --
	( 97.76,204.10) --
	( 97.34,204.10) --
	( 96.91,204.09) --
	( 96.49,204.09) --
	( 96.07,204.09) --
	( 95.65,204.09) --
	( 95.22,204.09) --
	( 94.80,204.09) --
	( 94.38,204.09) --
	( 93.96,204.07) --
	( 93.54,204.07) --
	( 93.11,204.07) --
	( 92.69,204.07) --
	( 92.27,204.07) --
	( 91.85,204.07) --
	( 91.42,204.07) --
	( 91.00,204.07) --
	( 90.58,204.07) --
	( 90.16,204.07) --
	( 89.73,204.06) --
	( 89.31,204.09) --
	( 88.89,204.06) --
	( 88.47,204.06) --
	( 88.04,204.05) --
	( 87.62,204.04) --
	( 87.20,204.05) --
	( 86.78,204.05) --
	( 86.35,204.05) --
	( 85.93,204.06) --
	( 85.51,204.11) --
	( 85.09,204.11) --
	( 84.66,204.17) --
	( 84.24,204.18) --
	( 83.82,204.18) --
	( 83.40,204.22) --
	( 82.97,204.21) --
	( 82.55,204.22) --
	( 82.13,204.23) --
	( 81.71,204.21) --
	( 81.29,204.21) --
	( 80.86,204.24) --
	( 80.44,204.24) --
	( 80.02,204.31) --
	( 79.60,204.24) --
	( 79.17,204.24) --
	( 78.75,204.25) --
	( 78.33,204.25) --
	( 77.91,204.26) --
	( 77.48,204.27) --
	( 77.06,204.28) --
	( 76.64,204.31) --
	( 76.22,204.31) --
	( 75.79,204.31) --
	( 75.37,204.32) --
	( 74.95,204.30) --
	( 74.53,204.34) --
	( 74.10,204.42) --
	( 73.68,204.42) --
	( 73.26,204.42) --
	( 72.84,204.46) --
	( 72.41,204.48) --
	( 71.99,204.48) --
	( 71.57,204.47) --
	( 71.15,204.49) --
	( 70.72,204.48) --
	( 70.30,204.49) --
	( 69.88,204.51) --
	( 69.46,204.51) --
	( 69.03,204.51) --
	( 68.61,204.51) --
	( 68.19,204.54) --
	( 67.77,204.54) --
	( 67.35,204.58) --
	( 66.92,204.58) --
	( 66.50,204.54) --
	( 66.08,204.53) --
	( 65.66,204.57) --
	( 65.23,204.55) --
	( 64.81,204.55) --
	( 64.39,204.52) --
	( 63.97,204.67) --
	( 63.54,204.67) --
	( 63.12,204.69) --
	( 62.70,204.71) --
	( 62.28,204.71) --
	( 61.85,204.72) --
	( 61.43,204.67) --
	( 61.01,204.75) --
	( 60.59,204.77) --
	( 60.16,204.77) --
	( 59.74,204.77) --
	( 59.32,204.77) --
	( 58.90,204.78) --
	( 58.47,204.78) --
	( 58.05,204.78) --
	cycle;
\definecolor{drawColor}{RGB}{0,0,0}

\path[draw=drawColor,line width= 0.4pt,dash pattern=on 7pt off 3pt ,line join=round,line cap=round] ( 42.00,129.40) -- (361.35,129.40);

\path[draw=drawColor,line width= 0.4pt,line join=round,line cap=round] ( 58.05,204.74) --
	( 58.47,204.74) --
	( 58.90,204.74) --
	( 59.32,204.73) --
	( 59.74,204.73) --
	( 60.16,204.73) --
	( 60.59,204.73) --
	( 61.01,204.62) --
	( 61.43,204.22) --
	( 61.85,204.54) --
	( 62.28,204.51) --
	( 62.70,204.54) --
	( 63.12,204.49) --
	( 63.54,204.44) --
	( 63.97,204.44) --
	( 64.39,204.15) --
	( 64.81,204.24) --
	( 65.23,204.24) --
	( 65.66,204.31) --
	( 66.08,204.26) --
	( 66.50,204.27) --
	( 66.92,204.38) --
	( 67.35,204.38) --
	( 67.77,204.31) --
	( 68.19,204.31) --
	( 68.61,204.28) --
	( 69.03,204.28) --
	( 69.46,204.28) --
	( 69.88,204.28) --
	( 70.30,204.25) --
	( 70.72,204.23) --
	( 71.15,204.26) --
	( 71.57,204.24) --
	( 71.99,204.27) --
	( 72.41,204.26) --
	( 72.84,204.25) --
	( 73.26,204.18) --
	( 73.68,204.19) --
	( 74.10,204.19) --
	( 74.53,204.08) --
	( 74.95,204.02) --
	( 75.37,204.05) --
	( 75.79,204.05) --
	( 76.22,204.05) --
	( 76.64,204.05) --
	( 77.06,204.01) --
	( 77.48,203.99) --
	( 77.91,203.98) --
	( 78.33,203.97) --
	( 78.75,203.97) --
	( 79.17,203.97) --
	( 79.60,203.97) --
	( 80.02,204.07) --
	( 80.44,203.97) --
	( 80.86,203.97) --
	( 81.29,203.93) --
	( 81.71,203.93) --
	( 82.13,203.97) --
	( 82.55,203.96) --
	( 82.97,203.95) --
	( 83.40,203.96) --
	( 83.82,203.90) --
	( 84.24,203.90) --
	( 84.66,203.89) --
	( 85.09,203.82) --
	( 85.51,203.82) --
	( 85.93,203.74) --
	( 86.35,203.74) --
	( 86.78,203.74) --
	( 87.20,203.74) --
	( 87.62,203.73) --
	( 88.04,203.74) --
	( 88.47,203.76) --
	( 88.89,203.76) --
	( 89.31,203.80) --
	( 89.73,203.77) --
	( 90.16,203.78) --
	( 90.58,203.78) --
	( 91.00,203.78) --
	( 91.42,203.79) --
	( 91.85,203.79) --
	( 92.27,203.79) --
	( 92.69,203.79) --
	( 93.11,203.79) --
	( 93.54,203.79) --
	( 93.96,203.79) --
	( 94.38,203.81) --
	( 94.80,203.81) --
	( 95.22,203.81) --
	( 95.65,203.81) --
	( 96.07,203.82) --
	( 96.49,203.82) --
	( 96.91,203.82) --
	( 97.34,203.83) --
	( 97.76,203.83) --
	( 98.18,203.83) --
	( 98.60,203.83) --
	( 99.03,203.83) --
	( 99.45,203.85) --
	( 99.87,203.83) --
	(100.29,203.82) --
	(100.72,203.82) --
	(101.14,203.82) --
	(101.56,203.82) --
	(101.98,203.69) --
	(102.41,203.69) --
	(102.83,203.69) --
	(103.25,203.69) --
	(103.67,203.69) --
	(104.10,203.69) --
	(104.52,203.69) --
	(104.94,203.69) --
	(105.36,203.68) --
	(105.79,203.68) --
	(106.21,203.68) --
	(106.63,203.68) --
	(107.05,203.68) --
	(107.48,203.68) --
	(107.90,203.68) --
	(108.32,203.65) --
	(108.74,203.66) --
	(109.16,203.66) --
	(109.59,203.66) --
	(110.01,203.66) --
	(110.43,203.66) --
	(110.85,203.66) --
	(111.28,203.66) --
	(111.70,203.66) --
	(112.12,203.66) --
	(112.54,203.66) --
	(112.97,203.66) --
	(113.39,203.66) --
	(113.81,203.37) --
	(114.23,203.36) --
	(114.66,203.36) --
	(115.08,203.36) --
	(115.50,203.36) --
	(115.92,203.36) --
	(116.35,203.36) --
	(116.77,203.34) --
	(117.19,203.34) --
	(117.61,203.34) --
	(118.04,203.34) --
	(118.46,203.34) --
	(118.88,203.34) --
	(119.30,203.34) --
	(119.73,203.34) --
	(120.15,203.38) --
	(120.57,203.38) --
	(120.99,203.38) --
	(121.42,203.38) --
	(121.84,203.38) --
	(122.26,203.38) --
	(122.68,203.38) --
	(123.10,203.39) --
	(123.53,203.40) --
	(123.95,203.40) --
	(124.37,203.40) --
	(124.79,203.40) --
	(125.22,203.41) --
	(125.64,203.41) --
	(126.06,203.41) --
	(126.48,203.41) --
	(126.91,203.41) --
	(127.33,203.41) --
	(127.75,203.41) --
	(128.17,203.41) --
	(128.60,203.41) --
	(129.02,203.36) --
	(129.44,203.36) --
	(129.86,203.36) --
	(130.29,203.36) --
	(130.71,203.36) --
	(131.13,203.36) --
	(131.55,203.36) --
	(131.98,203.36) --
	(132.40,203.39) --
	(132.82,203.39) --
	(133.24,203.39) --
	(133.67,203.39) --
	(134.09,203.39) --
	(134.51,203.39) --
	(134.93,203.39) --
	(135.35,203.39) --
	(135.78,203.39) --
	(136.20,203.39) --
	(136.62,203.39) --
	(137.04,203.39) --
	(137.47,203.39) --
	(137.89,203.39) --
	(138.31,203.39) --
	(138.73,203.39) --
	(139.16,203.39) --
	(139.58,203.39) --
	(140.00,203.39) --
	(140.42,203.39) --
	(140.85,203.39) --
	(141.27,203.39) --
	(141.69,203.39) --
	(142.11,203.39) --
	(142.54,203.39) --
	(142.96,203.39) --
	(143.38,203.39) --
	(143.80,203.39) --
	(144.23,203.39) --
	(144.65,203.39) --
	(145.07,203.39) --
	(145.49,203.39) --
	(145.92,203.39) --
	(146.34,203.39) --
	(146.76,203.39) --
	(147.18,203.39) --
	(147.61,203.40) --
	(148.03,203.40) --
	(148.45,203.40) --
	(148.87,203.40) --
	(149.29,203.40) --
	(149.72,203.40) --
	(150.14,203.40) --
	(150.56,203.40) --
	(150.98,203.40) --
	(151.41,203.40) --
	(151.83,203.40) --
	(152.25,203.40) --
	(152.67,203.40) --
	(153.10,203.40) --
	(153.52,203.40) --
	(153.94,203.40) --
	(154.36,203.40) --
	(154.79,203.40) --
	(155.21,203.40) --
	(155.63,203.40) --
	(156.05,203.40) --
	(156.48,203.40) --
	(156.90,203.40) --
	(157.32,203.40) --
	(157.74,203.40) --
	(158.17,203.40) --
	(158.59,203.40) --
	(159.01,203.40) --
	(159.43,203.40) --
	(159.86,203.40) --
	(160.28,203.40) --
	(160.70,203.40) --
	(161.12,203.40) --
	(161.55,203.40) --
	(161.97,203.40) --
	(162.39,203.40) --
	(162.81,203.40) --
	(163.23,203.40) --
	(163.66,203.40) --
	(164.08,203.36) --
	(164.50,203.36) --
	(164.92,203.36) --
	(165.35,203.36) --
	(165.77,203.36) --
	(166.19,203.36) --
	(166.61,203.36) --
	(167.04,203.36) --
	(167.46,203.36) --
	(167.88,203.36) --
	(168.30,203.36) --
	(168.73,203.36) --
	(169.15,203.36) --
	(169.57,203.36) --
	(169.99,203.36) --
	(170.42,203.36) --
	(170.84,203.36) --
	(171.26,203.36) --
	(171.68,203.36) --
	(172.11,203.36) --
	(172.53,203.36) --
	(172.95,203.36) --
	(173.37,203.36) --
	(173.80,203.36) --
	(174.22,203.36) --
	(174.64,203.36) --
	(175.06,203.36) --
	(175.48,203.36) --
	(175.91,203.36) --
	(176.33,203.36) --
	(176.75,203.36) --
	(177.17,203.36) --
	(177.60,203.36) --
	(178.02,203.36) --
	(178.44,203.36) --
	(178.86,203.36) --
	(179.29,203.36) --
	(179.71,203.36) --
	(180.13,203.36) --
	(180.55,203.36) --
	(180.98,203.36) --
	(181.40,203.36) --
	(181.82,203.36) --
	(182.24,203.36) --
	(182.67,203.36) --
	(183.09,203.36) --
	(183.51,203.36) --
	(183.93,203.36) --
	(184.36,203.48) --
	(184.78,203.48) --
	(185.20,203.48) --
	(185.62,203.48) --
	(186.05,203.48) --
	(186.47,203.48) --
	(186.89,203.48) --
	(187.31,203.48) --
	(187.74,203.48) --
	(188.16,203.39) --
	(188.58,203.39) --
	(189.00,203.39) --
	(189.42,203.39) --
	(189.85,203.39) --
	(190.27,203.39) --
	(190.69,203.39) --
	(191.11,203.39) --
	(191.54,203.39) --
	(191.96,203.39) --
	(192.38,203.39) --
	(192.80,203.39) --
	(193.23,203.39) --
	(193.65,203.39) --
	(194.07,203.39) --
	(194.49,203.39) --
	(194.92,203.39) --
	(195.34,203.39) --
	(195.76,203.39) --
	(196.18,203.39) --
	(196.61,203.39) --
	(197.03,203.39) --
	(197.45,203.39) --
	(197.87,203.39) --
	(198.30,203.39) --
	(198.72,203.39) --
	(199.14,203.39) --
	(199.56,203.39) --
	(199.99,203.39) --
	(200.41,203.39) --
	(200.83,203.39) --
	(201.25,203.39) --
	(201.68,203.39) --
	(202.10,203.39) --
	(202.52,203.39) --
	(202.94,203.41) --
	(203.36,203.41) --
	(203.79,203.41) --
	(204.21,203.41) --
	(204.63,203.41) --
	(205.05,201.91) --
	(205.48,201.91) --
	(205.90,201.91) --
	(206.32,201.91) --
	(206.74,201.91) --
	(207.17,201.91) --
	(207.59,201.91) --
	(208.01,201.91) --
	(208.43,201.91) --
	(208.86,201.91) --
	(209.28,201.91) --
	(209.70,201.91) --
	(210.12,201.91) --
	(210.55,201.91) --
	(210.97,201.90) --
	(211.39,201.90) --
	(211.81,201.90) --
	(212.24,201.90) --
	(212.66,201.90) --
	(213.08,201.90) --
	(213.50,201.90) --
	(213.93,201.90) --
	(214.35,201.90) --
	(214.77,201.90) --
	(215.19,201.90) --
	(215.61,201.90) --
	(216.04,201.90) --
	(216.46,201.90) --
	(216.88,201.90) --
	(217.30,201.90) --
	(217.73,201.90) --
	(218.15,201.90) --
	(218.57,201.90) --
	(218.99,201.90) --
	(219.42,201.90) --
	(219.84,201.90) --
	(220.26,201.90) --
	(220.68,201.90) --
	(221.11,201.90) --
	(221.53,201.90) --
	(221.95,201.90) --
	(222.37,201.90) --
	(222.80,201.90) --
	(223.22,201.90) --
	(223.64,201.90) --
	(224.06,201.90) --
	(224.49,201.90) --
	(224.91,201.90) --
	(225.33,201.90) --
	(225.75,201.90) --
	(226.18,201.90) --
	(226.60,201.90) --
	(227.02,201.90) --
	(227.44,201.90) --
	(227.87,201.90) --
	(228.29,201.90) --
	(228.71,201.90) --
	(229.13,201.90) --
	(229.55,201.90) --
	(229.98,201.90) --
	(230.40,201.90) --
	(230.82,201.90) --
	(231.24,201.90) --
	(231.67,201.90) --
	(232.09,201.90) --
	(232.51,201.90) --
	(232.93,201.90) --
	(233.36,201.90) --
	(233.78,201.90) --
	(234.20,201.90) --
	(234.62,201.90) --
	(235.05,201.90) --
	(235.47,201.90) --
	(235.89,201.90) --
	(236.31,201.90) --
	(236.74,201.90) --
	(237.16,201.90) --
	(237.58,201.90) --
	(238.00,201.90) --
	(238.43,201.90) --
	(238.85,201.90) --
	(239.27,201.90) --
	(239.69,201.90) --
	(240.12,201.90) --
	(240.54,201.90) --
	(240.96,201.90) --
	(241.38,201.90) --
	(241.80,201.90) --
	(242.23,201.90) --
	(242.65,201.90) --
	(243.07,201.90) --
	(243.49,201.90) --
	(243.92,201.90) --
	(244.34,201.90) --
	(244.76,201.90) --
	(245.18,201.90) --
	(245.61,201.90) --
	(246.03,201.90) --
	(246.45,201.90) --
	(246.87,201.90) --
	(247.30,201.90) --
	(247.72,201.90) --
	(248.14,201.90) --
	(248.56,201.90) --
	(248.99,201.90) --
	(249.41,201.90) --
	(249.83,201.90) --
	(250.25,201.90) --
	(250.68,201.90) --
	(251.10,201.90) --
	(251.52,201.90) --
	(251.94,201.90) --
	(252.37,201.90) --
	(252.79,201.90) --
	(253.21,201.90) --
	(253.63,201.90) --
	(254.06,201.90) --
	(254.48,201.90) --
	(254.90,201.90) --
	(255.32,201.90) --
	(255.74,201.90) --
	(256.17,201.90) --
	(256.59,201.90) --
	(257.01,201.90) --
	(257.43,201.90) --
	(257.86,201.90) --
	(258.28,201.90) --
	(258.70,201.90) --
	(259.12,201.90) --
	(259.55,201.90) --
	(259.97,201.90) --
	(260.39,201.90) --
	(260.81,201.90) --
	(261.24,201.90) --
	(261.66,201.90) --
	(262.08,201.90) --
	(262.50,201.90) --
	(262.93,201.90) --
	(263.35,201.90) --
	(263.77,201.90) --
	(264.19,201.90) --
	(264.62,201.89) --
	(265.04,201.89) --
	(265.46,201.89) --
	(265.88,201.89) --
	(266.31,201.89) --
	(266.73,201.89) --
	(267.15,201.89) --
	(267.57,201.89) --
	(268.00,201.89) --
	(268.42,201.89) --
	(268.84,201.89) --
	(269.26,201.89) --
	(269.68,201.89) --
	(270.11,201.89) --
	(270.53,201.89) --
	(270.95,201.89) --
	(271.37,201.89) --
	(271.80,201.89) --
	(272.22,201.89) --
	(272.64,201.89) --
	(273.06,201.89) --
	(273.49,201.89) --
	(273.91,201.89) --
	(274.33,201.89) --
	(274.75,201.89) --
	(275.18,201.89) --
	(275.60,201.89) --
	(276.02,201.89) --
	(276.44,201.89) --
	(276.87,201.89) --
	(277.29,201.89) --
	(277.71,201.89) --
	(278.13,201.89) --
	(278.56,201.89) --
	(278.98,201.89) --
	(279.40,201.89) --
	(279.82,201.89) --
	(280.25,201.89) --
	(280.67,201.89) --
	(281.09,201.89) --
	(281.51,201.89) --
	(281.93,201.89) --
	(282.36,201.89) --
	(282.78,201.89) --
	(283.20,201.89) --
	(283.62,201.89) --
	(284.05,201.89) --
	(284.47,201.89) --
	(284.89,201.89) --
	(285.31,201.89) --
	(285.74,201.89) --
	(286.16,201.89) --
	(286.58,201.89) --
	(287.00,201.89) --
	(287.43,201.89) --
	(287.85,201.89) --
	(288.27,201.89) --
	(288.69,201.89) --
	(289.12,201.89) --
	(289.54,201.89) --
	(289.96,201.89) --
	(290.38,201.89) --
	(290.81,201.89) --
	(291.23,201.89) --
	(291.65,201.89) --
	(292.07,201.89) --
	(292.50,201.89) --
	(292.92,201.89) --
	(293.34,201.89) --
	(293.76,201.89) --
	(294.19,201.89) --
	(294.61,201.89) --
	(295.03,201.89) --
	(295.45,201.89) --
	(295.87,201.89) --
	(296.30,201.89) --
	(296.72,201.89) --
	(297.14,201.89) --
	(297.56,201.89) --
	(297.99,201.89) --
	(298.41,201.89) --
	(298.83,201.89) --
	(299.25,201.89) --
	(299.68,201.89) --
	(300.10,201.89) --
	(300.52,201.89) --
	(300.94,201.89) --
	(301.37,201.89) --
	(301.79,201.89) --
	(302.21,201.89) --
	(302.63,201.89) --
	(303.06,201.89) --
	(303.48,201.89) --
	(303.90,201.89) --
	(304.32,201.89) --
	(304.75,201.89) --
	(305.17,201.89) --
	(305.59,201.89) --
	(306.01,201.89) --
	(306.44,201.89) --
	(306.86,201.89) --
	(307.28,201.89) --
	(307.70,201.89) --
	(308.12,201.89) --
	(308.55,201.89) --
	(308.97,201.89) --
	(309.39,201.89) --
	(309.81,201.89) --
	(310.24,201.89) --
	(310.66,201.89) --
	(311.08,201.89) --
	(311.50,201.89) --
	(311.93,201.89) --
	(312.35,201.89) --
	(312.77,201.89) --
	(313.19,201.89) --
	(313.62,201.89) --
	(314.04,201.89) --
	(314.46,201.89) --
	(314.88,201.89) --
	(315.31,201.89) --
	(315.73,201.89) --
	(316.15,201.89) --
	(316.57,201.89) --
	(317.00,201.89) --
	(317.42,201.89) --
	(317.84,201.89) --
	(318.26,201.89) --
	(318.69,201.89) --
	(319.11,201.89) --
	(319.53,201.89) --
	(319.95,201.89) --
	(320.38,201.89) --
	(320.80,201.89) --
	(321.22,201.89) --
	(321.64,201.89) --
	(322.06,201.89) --
	(322.49,201.89) --
	(322.91,201.89) --
	(323.33,201.89) --
	(323.75,201.89) --
	(324.18,201.89) --
	(324.60,201.89) --
	(325.02,201.89) --
	(325.44,201.89) --
	(325.87,201.89) --
	(326.29,201.89) --
	(326.71,201.89) --
	(327.13,201.89) --
	(327.56,201.89) --
	(327.98,201.89) --
	(328.40,201.89) --
	(328.82,201.89) --
	(329.25,201.89) --
	(329.67,201.89) --
	(330.09,201.89) --
	(330.51,201.86) --
	(330.94,201.86) --
	(331.36,201.86) --
	(331.78,201.86) --
	(332.20,201.86) --
	(332.63,201.86) --
	(333.05,201.86) --
	(333.47,201.86) --
	(333.89,201.86) --
	(334.32,201.86) --
	(334.74,201.86) --
	(335.16,201.86) --
	(335.58,201.86) --
	(336.00,201.86) --
	(336.43,201.86) --
	(336.85,201.86) --
	(337.27,201.86) --
	(337.69,201.86) --
	(338.12,201.86) --
	(338.54,201.86) --
	(338.96,201.86) --
	(339.38,201.86) --
	(339.81,201.86) --
	(340.23,201.86) --
	(340.65,201.86) --
	(341.07,201.86) --
	(341.50,201.86) --
	(341.92,201.86) --
	(342.34,201.86) --
	(342.76,201.86) --
	(343.19,201.86) --
	(343.61,201.86) --
	(344.03,201.86) --
	(344.45,201.86) --
	(344.88,201.86) --
	(345.30,201.86) --
	(345.72,201.86) --
	(346.14,201.86) --
	(346.57,201.86) --
	(346.99,201.86) --
	(347.41,201.86) --
	(347.83,201.86) --
	(348.25,201.86) --
	(348.68,201.86) --
	(349.10,201.86) --
	(349.52,201.86);
\end{scope}
\end{tikzpicture}
}
	\end{adjustbox}
	\caption[Einfluss des \gls{rmse}-Grenzwerts der \gls{ssauf} auf den Zusammenhang zwischen der Asymptote, der Steigung und dem \gls{si}]{Einfluss des \gls{rmse}-Grenzwerts auf ($a$) den Zusammenhang zwischen den aus der \gls{ssauf} mit einer exponentiellen Regression abgeleiteten Aufgabenparametern der Asymptote und der Steigung respektive auf ($b$) den Zusammenhang zwischen der Steigung und den \gls{si}. Die durchgezogene Linie kennzeichnet den Verlauf des Zusammenhangs. Der graue Bereich beschreibt das $95\,\%$-Konfidenzintervall.}
	\label{fig:spatial_suppression_rmse_cutoff_asymtote_slope_suppressionindex}
\end{figure}

Die Steigung korrelierte über den ganzen \gls{rmse}-Grenzwertbereich stark positiv mit dem \gls{si} ($r=.96$ bis $.99$, alle $p\textnormal{s}<.001$; siehe \autoref{fig:spatial_suppression_rmse_cutoff_asymtote_slope_suppressionindex}b). Die tiefste Schätzung ($r=.96$, $p<.001$) unterschied sich dabei signifikant ($z=7.85$, $p<.001$) von dem von \citet{Melnick2013} berichteten Zusammenhang ($r>.996$).

Der Zusammenhang zwischen der Asymptote und dem \gls{zwert} des \gls{bist}s fiel in Abhängigkeit des \gls{rmse}-Grenzwerts weniger eindeutig aus (siehe \autoref{fig:spatial_suppression_asymtote_slope_zscore}a).
Während der Zusammenhang über einen grossen Teil des tieferen \gls{rmse}-Grenzwertbereichs nicht signifikant war, unterschritt der Zusammenhang zwischen $8.6$ und $9.5$ ms ($r = -.17$, $p = .049$), zwischen $26.1$ und $35.7$ ms ($r = -.15$ bis $ -.16$, alle $p\textnormal{s} < .048$) und ab $36.5$ ms ($r = -.16$ bis $-.18$, alle $p\textnormal{s} < .03$) die Signifikanzgrenze. In den erwähnten Bereichen war eine tiefe Asymptote also tendenziell mit einem hohen \gls{zwert} verbunden. Eine visuelle Inspektion des Verlaufs liess keine Aussage darüber zu, ob der  \gls{rmse}-Grenzwert einen positiven oder negativen Einfluss auf die Höhe des Zusammenhangs ausübte.

Die Steigung und der \gls{zwert} des \gls{bist}s korrelierten unabhängig vom \gls{rmse}-Grenzwert nicht signifikant miteinander ($r=-.16$ bis $.62$, alle $p\textnormal{s}>.08$; siehe \autoref{fig:spatial_suppression_asymtote_slope_zscore}b). 
Um für den Vergleich zwischen dem von \citet{Melnick2013} berichteten Zusammenhang zwischen ihrer Steigung (Studie 1: $b=0.116$ und Studie 2: $b=0.139$) und IQ-Punkten und dem in der vorliegenden Arbeit ermittelten Zusammenhang zwischen der Steigung ($b=0.103$) und dem \gls{zwert} die bestmögliche Teststärke zu erhalten, wurde die Gesamtstichprobe (\gls{rmse}-Grenzwert = $65.47$ ms) verwendet. Die Analyse hat ergeben, dass sich der in der vorliegenden Arbeit ermittelte Zusammenhang ($r=.00$, $p=.97$) signifikant von dem von \citeauthor{Melnick2013} berichteten Zusammenhang ($r=.68$) unterschied ($z=5.61$, $p<.001$).

Betrachtet man die mit der Gesamtstichprobe erhaltenen Ergebnisse, kann abschliessend zur zweiten Fragestellung Folgendes festgehalten werden:  
Die aus der exponentiellen Regression abgeleitete Asymptote korrelierte in der vorliegenden Arbeit leicht negativ mit dem \gls{zwert} des \gls{bist}s ($r=-.16$, $p=.03$). Die Steigung, der zweite mit der exponentiellen Regression abgeleitete Aufgabenparameter, hing nicht signifikant mit dem \gls{zwert} zusammen ($r=.00$, $p=.97$) und bestätigte damit den von \citet{Melnick2013} berichteten Zusammenhang nicht. 

\begin{figure}[htbp]
	\centering
	%	\captionsetup{font = small}
	\begin{adjustbox}{width=1\textwidth}
		\subfloat[Test][Zusammenhang ($r$) zwischen der Asymptote und dem \gls{zwert} des \gls{bist}s.]{% Created by tikzDevice version 0.10.1 on 2016-08-18 17:15:49
% !TEX encoding = UTF-8 Unicode
\begin{tikzpicture}[x=1pt,y=1pt]
\definecolor{fillColor}{RGB}{255,255,255}
\path[use as bounding box,fill=fillColor,fill opacity=0.00] (0,0) rectangle (361.35,216.81);
\begin{scope}
\path[clip] (  0.00,  0.00) rectangle (361.35,216.81);
\definecolor{drawColor}{RGB}{0,0,0}

\node[text=drawColor,anchor=base,inner sep=0pt, outer sep=0pt, scale=  1.00] at (201.68,  8.40) {\textit{RMSE} Grenzwert (ms)};

\node[text=drawColor,rotate= 90.00,anchor=base,inner sep=0pt, outer sep=0pt, scale=  1.00] at (  9.60,129.40) {\textit{r}};
\end{scope}
\begin{scope}
\path[clip] (  0.00,  0.00) rectangle (361.35,216.81);
\definecolor{drawColor}{RGB}{0,0,0}

\path[draw=drawColor,line width= 0.4pt,line join=round,line cap=round] ( 58.05, 48.00) -- (349.52, 48.00);

\path[draw=drawColor,line width= 0.4pt,line join=round,line cap=round] ( 58.05, 48.00) -- ( 58.05, 42.00);

\path[draw=drawColor,line width= 0.4pt,line join=round,line cap=round] ( 96.07, 48.00) -- ( 96.07, 42.00);

\path[draw=drawColor,line width= 0.4pt,line join=round,line cap=round] (138.31, 48.00) -- (138.31, 42.00);

\path[draw=drawColor,line width= 0.4pt,line join=round,line cap=round] (180.55, 48.00) -- (180.55, 42.00);

\path[draw=drawColor,line width= 0.4pt,line join=round,line cap=round] (222.80, 48.00) -- (222.80, 42.00);

\path[draw=drawColor,line width= 0.4pt,line join=round,line cap=round] (265.04, 48.00) -- (265.04, 42.00);

\path[draw=drawColor,line width= 0.4pt,line join=round,line cap=round] (307.28, 48.00) -- (307.28, 42.00);

\path[draw=drawColor,line width= 0.4pt,line join=round,line cap=round] (349.52, 48.00) -- (349.52, 42.00);

\node[text=drawColor,anchor=base,inner sep=0pt, outer sep=0pt, scale=  1.00] at ( 58.05, 30.00) {1};

\node[text=drawColor,anchor=base,inner sep=0pt, outer sep=0pt, scale=  1.00] at ( 96.07, 30.00) {10};

\node[text=drawColor,anchor=base,inner sep=0pt, outer sep=0pt, scale=  1.00] at (138.31, 30.00) {20};

\node[text=drawColor,anchor=base,inner sep=0pt, outer sep=0pt, scale=  1.00] at (180.55, 30.00) {30};

\node[text=drawColor,anchor=base,inner sep=0pt, outer sep=0pt, scale=  1.00] at (222.80, 30.00) {40};

\node[text=drawColor,anchor=base,inner sep=0pt, outer sep=0pt, scale=  1.00] at (265.04, 30.00) {50};

\node[text=drawColor,anchor=base,inner sep=0pt, outer sep=0pt, scale=  1.00] at (307.28, 30.00) {60};

\node[text=drawColor,anchor=base,inner sep=0pt, outer sep=0pt, scale=  1.00] at (349.52, 30.00) {70};

\path[draw=drawColor,line width= 0.4pt,line join=round,line cap=round] ( 42.00, 54.03) -- ( 42.00,204.78);

\path[draw=drawColor,line width= 0.4pt,line join=round,line cap=round] ( 42.00, 54.03) -- ( 36.00, 54.03);

\path[draw=drawColor,line width= 0.4pt,line join=round,line cap=round] ( 42.00, 72.87) -- ( 36.00, 72.87);

\path[draw=drawColor,line width= 0.4pt,line join=round,line cap=round] ( 42.00, 91.72) -- ( 36.00, 91.72);

\path[draw=drawColor,line width= 0.4pt,line join=round,line cap=round] ( 42.00,110.56) -- ( 36.00,110.56);

\path[draw=drawColor,line width= 0.4pt,line join=round,line cap=round] ( 42.00,129.40) -- ( 36.00,129.40);

\path[draw=drawColor,line width= 0.4pt,line join=round,line cap=round] ( 42.00,148.25) -- ( 36.00,148.25);

\path[draw=drawColor,line width= 0.4pt,line join=round,line cap=round] ( 42.00,167.09) -- ( 36.00,167.09);

\path[draw=drawColor,line width= 0.4pt,line join=round,line cap=round] ( 42.00,185.94) -- ( 36.00,185.94);

\path[draw=drawColor,line width= 0.4pt,line join=round,line cap=round] ( 42.00,204.78) -- ( 36.00,204.78);

\node[text=drawColor,anchor=base east,inner sep=0pt, outer sep=0pt, scale=  1.00] at ( 33.60, 50.59) {--1.00};

\node[text=drawColor,anchor=base east,inner sep=0pt, outer sep=0pt, scale=  1.00] at ( 33.60, 69.43) {--.75};

\node[text=drawColor,anchor=base east,inner sep=0pt, outer sep=0pt, scale=  1.00] at ( 33.60, 88.27) {--.50};

\node[text=drawColor,anchor=base east,inner sep=0pt, outer sep=0pt, scale=  1.00] at ( 33.60,107.12) {--.25};

\node[text=drawColor,anchor=base east,inner sep=0pt, outer sep=0pt, scale=  1.00] at ( 33.60,125.96) {.00};

\node[text=drawColor,anchor=base east,inner sep=0pt, outer sep=0pt, scale=  1.00] at ( 33.60,144.81) {.25};

\node[text=drawColor,anchor=base east,inner sep=0pt, outer sep=0pt, scale=  1.00] at ( 33.60,163.65) {.50};

\node[text=drawColor,anchor=base east,inner sep=0pt, outer sep=0pt, scale=  1.00] at ( 33.60,182.49) {.75};

\node[text=drawColor,anchor=base east,inner sep=0pt, outer sep=0pt, scale=  1.00] at ( 33.60,201.34) {1.00};
\end{scope}
\begin{scope}
\path[clip] ( 42.00, 48.00) rectangle (361.35,210.81);
\definecolor{fillColor}{RGB}{190,190,190}

\path[fill=fillColor] ( 58.05, 55.59) --
	( 58.47, 55.59) --
	( 58.90, 55.59) --
	( 59.32, 62.55) --
	( 59.74, 62.55) --
	( 60.16, 64.79) --
	( 60.59, 64.79) --
	( 61.01, 67.35) --
	( 61.43, 77.30) --
	( 61.85, 87.17) --
	( 62.28, 85.93) --
	( 62.70, 82.40) --
	( 63.12, 96.60) --
	( 63.54, 99.20) --
	( 63.97, 99.20) --
	( 64.39,102.99) --
	( 64.81,104.47) --
	( 65.23,104.47) --
	( 65.66,109.98) --
	( 66.08,113.18) --
	( 66.50,113.07) --
	( 66.92,118.02) --
	( 67.35,118.02) --
	( 67.77,114.40) --
	( 68.19,114.40) --
	( 68.61,111.89) --
	( 69.03,112.30) --
	( 69.46,112.30) --
	( 69.88,112.30) --
	( 70.30,110.47) --
	( 70.72,110.94) --
	( 71.15,111.41) --
	( 71.57,111.33) --
	( 71.99,110.04) --
	( 72.41,110.35) --
	( 72.84,111.72) --
	( 73.26,110.41) --
	( 73.68,115.58) --
	( 74.10,115.58) --
	( 74.53,115.15) --
	( 74.95,114.44) --
	( 75.37,113.62) --
	( 75.79,113.25) --
	( 76.22,113.25) --
	( 76.64,113.65) --
	( 77.06,113.77) --
	( 77.48,113.17) --
	( 77.91,112.99) --
	( 78.33,113.19) --
	( 78.75,113.23) --
	( 79.17,113.33) --
	( 79.60,113.33) --
	( 80.02,110.91) --
	( 80.44,105.88) --
	( 80.86,105.88) --
	( 81.29,106.42) --
	( 81.71,106.63) --
	( 82.13,106.03) --
	( 82.55,105.30) --
	( 82.97,105.36) --
	( 83.40,105.52) --
	( 83.82,105.00) --
	( 84.24,105.00) --
	( 84.66,104.84) --
	( 85.09,104.24) --
	( 85.51,104.24) --
	( 85.93,104.28) --
	( 86.35,104.61) --
	( 86.78,104.07) --
	( 87.20,104.07) --
	( 87.62,104.33) --
	( 88.04,104.52) --
	( 88.47,104.57) --
	( 88.89,104.57) --
	( 89.31,104.89) --
	( 89.73,106.39) --
	( 90.16,104.54) --
	( 90.58,104.54) --
	( 91.00,104.54) --
	( 91.42,104.54) --
	( 91.85,104.54) --
	( 92.27,104.54) --
	( 92.69,104.54) --
	( 93.11,104.54) --
	( 93.54,104.54) --
	( 93.96,104.54) --
	( 94.38,105.05) --
	( 94.80,105.24) --
	( 95.22,105.24) --
	( 95.65,105.24) --
	( 96.07,105.31) --
	( 96.49,105.31) --
	( 96.91,105.31) --
	( 97.34,106.30) --
	( 97.76,106.30) --
	( 98.18,106.30) --
	( 98.60,106.30) --
	( 99.03,106.30) --
	( 99.45,106.20) --
	( 99.87,106.31) --
	(100.29,106.11) --
	(100.72,106.11) --
	(101.14,106.08) --
	(101.56,106.08) --
	(101.98,106.26) --
	(102.41,106.19) --
	(102.83,106.19) --
	(103.25,106.19) --
	(103.67,106.19) --
	(104.10,106.19) --
	(104.52,106.19) --
	(104.94,106.19) --
	(105.36,106.46) --
	(105.79,107.20) --
	(106.21,107.20) --
	(106.63,107.20) --
	(107.05,107.20) --
	(107.48,107.20) --
	(107.90,107.20) --
	(108.32,107.23) --
	(108.74,107.84) --
	(109.16,107.84) --
	(109.59,107.84) --
	(110.01,107.11) --
	(110.43,107.20) --
	(110.85,107.20) --
	(111.28,107.20) --
	(111.70,107.20) --
	(112.12,107.20) --
	(112.54,107.20) --
	(112.97,107.20) --
	(113.39,107.20) --
	(113.81,107.79) --
	(114.23,107.95) --
	(114.66,107.95) --
	(115.08,107.95) --
	(115.50,107.95) --
	(115.92,107.95) --
	(116.35,107.95) --
	(116.77,107.41) --
	(117.19,107.41) --
	(117.61,107.41) --
	(118.04,107.41) --
	(118.46,107.73) --
	(118.88,107.73) --
	(119.30,107.73) --
	(119.73,107.73) --
	(120.15,108.30) --
	(120.57,108.30) --
	(120.99,108.30) --
	(121.42,108.30) --
	(121.84,108.30) --
	(122.26,108.30) --
	(122.68,108.30) --
	(123.10,108.56) --
	(123.53,108.05) --
	(123.95,108.05) --
	(124.37,108.05) --
	(124.79,108.05) --
	(125.22,108.24) --
	(125.64,108.24) --
	(126.06,108.24) --
	(126.48,108.24) --
	(126.91,108.24) --
	(127.33,108.24) --
	(127.75,108.24) --
	(128.17,108.24) --
	(128.60,108.37) --
	(129.02,108.40) --
	(129.44,108.40) --
	(129.86,108.40) --
	(130.29,108.40) --
	(130.71,108.40) --
	(131.13,108.40) --
	(131.55,108.40) --
	(131.98,108.40) --
	(132.40,108.32) --
	(132.82,108.32) --
	(133.24,108.32) --
	(133.67,108.32) --
	(134.09,108.32) --
	(134.51,108.32) --
	(134.93,108.32) --
	(135.35,108.32) --
	(135.78,108.32) --
	(136.20,108.32) --
	(136.62,108.32) --
	(137.04,108.32) --
	(137.47,108.32) --
	(137.89,108.32) --
	(138.31,108.32) --
	(138.73,108.32) --
	(139.16,108.32) --
	(139.58,108.32) --
	(140.00,108.32) --
	(140.42,108.32) --
	(140.85,108.32) --
	(141.27,108.32) --
	(141.69,108.32) --
	(142.11,108.32) --
	(142.54,108.32) --
	(142.96,108.32) --
	(143.38,108.32) --
	(143.80,108.32) --
	(144.23,108.32) --
	(144.65,108.32) --
	(145.07,108.32) --
	(145.49,108.32) --
	(145.92,108.32) --
	(146.34,108.32) --
	(146.76,108.32) --
	(147.18,108.32) --
	(147.61,108.42) --
	(148.03,108.42) --
	(148.45,108.42) --
	(148.87,108.42) --
	(149.29,108.42) --
	(149.72,108.42) --
	(150.14,108.42) --
	(150.56,108.42) --
	(150.98,108.42) --
	(151.41,108.42) --
	(151.83,108.42) --
	(152.25,108.42) --
	(152.67,108.42) --
	(153.10,108.42) --
	(153.52,108.42) --
	(153.94,108.42) --
	(154.36,108.42) --
	(154.79,108.42) --
	(155.21,108.42) --
	(155.63,108.42) --
	(156.05,108.42) --
	(156.48,108.42) --
	(156.90,108.42) --
	(157.32,108.42) --
	(157.74,108.42) --
	(158.17,108.42) --
	(158.59,108.42) --
	(159.01,108.42) --
	(159.43,108.42) --
	(159.86,108.42) --
	(160.28,108.42) --
	(160.70,108.42) --
	(161.12,108.42) --
	(161.55,108.42) --
	(161.97,108.42) --
	(162.39,108.42) --
	(162.81,108.42) --
	(163.23,108.42) --
	(163.66,108.42) --
	(164.08,106.71) --
	(164.50,106.71) --
	(164.92,106.71) --
	(165.35,106.71) --
	(165.77,106.71) --
	(166.19,106.71) --
	(166.61,106.71) --
	(167.04,106.71) --
	(167.46,106.71) --
	(167.88,106.71) --
	(168.30,106.71) --
	(168.73,106.71) --
	(169.15,106.71) --
	(169.57,106.71) --
	(169.99,106.71) --
	(170.42,106.71) --
	(170.84,106.71) --
	(171.26,106.71) --
	(171.68,106.71) --
	(172.11,106.71) --
	(172.53,106.71) --
	(172.95,106.71) --
	(173.37,106.71) --
	(173.80,106.71) --
	(174.22,106.71) --
	(174.64,106.71) --
	(175.06,106.71) --
	(175.48,106.71) --
	(175.91,106.71) --
	(176.33,106.71) --
	(176.75,106.71) --
	(177.17,106.71) --
	(177.60,106.71) --
	(178.02,106.71) --
	(178.44,106.71) --
	(178.86,106.71) --
	(179.29,106.71) --
	(179.71,106.71) --
	(180.13,106.71) --
	(180.55,106.71) --
	(180.98,106.71) --
	(181.40,106.71) --
	(181.82,106.71) --
	(182.24,106.71) --
	(182.67,106.71) --
	(183.09,106.71) --
	(183.51,106.71) --
	(183.93,106.71) --
	(184.36,107.09) --
	(184.78,107.09) --
	(185.20,107.09) --
	(185.62,107.09) --
	(186.05,107.09) --
	(186.47,107.09) --
	(186.89,107.09) --
	(187.31,107.09) --
	(187.74,107.09) --
	(188.16,106.85) --
	(188.58,106.85) --
	(189.00,106.85) --
	(189.42,106.85) --
	(189.85,106.85) --
	(190.27,106.85) --
	(190.69,106.85) --
	(191.11,106.85) --
	(191.54,106.85) --
	(191.96,106.85) --
	(192.38,106.85) --
	(192.80,106.85) --
	(193.23,106.85) --
	(193.65,106.85) --
	(194.07,106.85) --
	(194.49,106.85) --
	(194.92,106.85) --
	(195.34,106.85) --
	(195.76,106.85) --
	(196.18,106.85) --
	(196.61,106.85) --
	(197.03,106.85) --
	(197.45,106.85) --
	(197.87,106.85) --
	(198.30,106.85) --
	(198.72,106.85) --
	(199.14,106.85) --
	(199.56,106.85) --
	(199.99,106.85) --
	(200.41,106.85) --
	(200.83,106.85) --
	(201.25,106.85) --
	(201.68,106.85) --
	(202.10,106.85) --
	(202.52,106.85) --
	(202.94,106.84) --
	(203.36,106.84) --
	(203.79,106.84) --
	(204.21,106.84) --
	(204.63,106.84) --
	(205.05,107.66) --
	(205.48,107.66) --
	(205.90,107.66) --
	(206.32,107.66) --
	(206.74,107.66) --
	(207.17,107.66) --
	(207.59,107.66) --
	(208.01,105.82) --
	(208.43,105.82) --
	(208.86,105.82) --
	(209.28,105.82) --
	(209.70,105.82) --
	(210.12,105.82) --
	(210.55,105.82) --
	(210.97,105.10) --
	(211.39,105.10) --
	(211.81,105.10) --
	(212.24,105.10) --
	(212.66,105.10) --
	(213.08,105.10) --
	(213.50,105.10) --
	(213.93,105.10) --
	(214.35,105.10) --
	(214.77,105.10) --
	(215.19,105.10) --
	(215.61,105.10) --
	(216.04,105.10) --
	(216.46,105.10) --
	(216.88,105.10) --
	(217.30,105.10) --
	(217.73,105.10) --
	(218.15,105.10) --
	(218.57,105.10) --
	(218.99,105.10) --
	(219.42,105.10) --
	(219.84,105.10) --
	(220.26,105.10) --
	(220.68,105.10) --
	(221.11,105.10) --
	(221.53,105.10) --
	(221.95,105.10) --
	(222.37,105.10) --
	(222.80,105.10) --
	(223.22,105.10) --
	(223.64,105.10) --
	(224.06,105.10) --
	(224.49,105.10) --
	(224.91,105.10) --
	(225.33,105.10) --
	(225.75,105.10) --
	(226.18,105.10) --
	(226.60,105.10) --
	(227.02,105.10) --
	(227.44,105.10) --
	(227.87,105.10) --
	(228.29,105.10) --
	(228.71,105.10) --
	(229.13,105.10) --
	(229.55,105.10) --
	(229.98,105.10) --
	(230.40,105.10) --
	(230.82,105.10) --
	(231.24,105.10) --
	(231.67,105.10) --
	(232.09,105.10) --
	(232.51,105.10) --
	(232.93,105.10) --
	(233.36,105.10) --
	(233.78,105.10) --
	(234.20,105.10) --
	(234.62,105.10) --
	(235.05,105.10) --
	(235.47,105.10) --
	(235.89,105.10) --
	(236.31,105.10) --
	(236.74,105.10) --
	(237.16,105.10) --
	(237.58,105.10) --
	(238.00,105.10) --
	(238.43,105.10) --
	(238.85,105.10) --
	(239.27,105.10) --
	(239.69,105.10) --
	(240.12,105.10) --
	(240.54,105.10) --
	(240.96,105.10) --
	(241.38,105.10) --
	(241.80,105.10) --
	(242.23,105.10) --
	(242.65,105.10) --
	(243.07,105.10) --
	(243.49,105.10) --
	(243.92,105.10) --
	(244.34,105.10) --
	(244.76,105.10) --
	(245.18,105.10) --
	(245.61,105.10) --
	(246.03,105.10) --
	(246.45,105.10) --
	(246.87,105.10) --
	(247.30,105.10) --
	(247.72,105.10) --
	(248.14,105.10) --
	(248.56,105.10) --
	(248.99,105.10) --
	(249.41,105.10) --
	(249.83,105.10) --
	(250.25,105.10) --
	(250.68,105.10) --
	(251.10,105.10) --
	(251.52,105.10) --
	(251.94,105.10) --
	(252.37,105.10) --
	(252.79,105.10) --
	(253.21,105.10) --
	(253.63,105.10) --
	(254.06,105.10) --
	(254.48,105.10) --
	(254.90,105.10) --
	(255.32,105.10) --
	(255.74,105.10) --
	(256.17,105.10) --
	(256.59,105.10) --
	(257.01,105.10) --
	(257.43,105.10) --
	(257.86,105.10) --
	(258.28,105.10) --
	(258.70,105.10) --
	(259.12,105.10) --
	(259.55,105.10) --
	(259.97,105.10) --
	(260.39,105.10) --
	(260.81,105.10) --
	(261.24,105.10) --
	(261.66,105.10) --
	(262.08,105.10) --
	(262.50,105.10) --
	(262.93,105.10) --
	(263.35,105.10) --
	(263.77,105.10) --
	(264.19,105.10) --
	(264.62,105.28) --
	(265.04,105.28) --
	(265.46,105.28) --
	(265.88,105.28) --
	(266.31,105.28) --
	(266.73,105.28) --
	(267.15,105.28) --
	(267.57,105.28) --
	(268.00,105.28) --
	(268.42,105.28) --
	(268.84,105.28) --
	(269.26,105.28) --
	(269.68,105.28) --
	(270.11,105.28) --
	(270.53,105.28) --
	(270.95,105.28) --
	(271.37,105.28) --
	(271.80,105.28) --
	(272.22,105.28) --
	(272.64,105.28) --
	(273.06,105.28) --
	(273.49,105.28) --
	(273.91,105.28) --
	(274.33,105.28) --
	(274.75,105.28) --
	(275.18,105.28) --
	(275.60,105.28) --
	(276.02,105.28) --
	(276.44,105.28) --
	(276.87,105.28) --
	(277.29,105.28) --
	(277.71,105.28) --
	(278.13,105.28) --
	(278.56,105.28) --
	(278.98,105.28) --
	(279.40,105.28) --
	(279.82,105.28) --
	(280.25,105.28) --
	(280.67,105.28) --
	(281.09,105.28) --
	(281.51,105.28) --
	(281.93,105.28) --
	(282.36,105.28) --
	(282.78,105.28) --
	(283.20,105.28) --
	(283.62,105.28) --
	(284.05,105.28) --
	(284.47,105.28) --
	(284.89,105.28) --
	(285.31,105.28) --
	(285.74,105.28) --
	(286.16,105.28) --
	(286.58,105.28) --
	(287.00,105.28) --
	(287.43,105.28) --
	(287.85,105.28) --
	(288.27,105.28) --
	(288.69,105.28) --
	(289.12,105.28) --
	(289.54,105.28) --
	(289.96,105.28) --
	(290.38,105.28) --
	(290.81,105.28) --
	(291.23,105.28) --
	(291.65,105.28) --
	(292.07,105.28) --
	(292.50,105.28) --
	(292.92,105.28) --
	(293.34,105.28) --
	(293.76,105.28) --
	(294.19,105.28) --
	(294.61,105.28) --
	(295.03,105.28) --
	(295.45,105.28) --
	(295.87,105.28) --
	(296.30,105.28) --
	(296.72,105.28) --
	(297.14,105.28) --
	(297.56,105.28) --
	(297.99,105.28) --
	(298.41,105.28) --
	(298.83,105.28) --
	(299.25,105.28) --
	(299.68,105.28) --
	(300.10,105.28) --
	(300.52,105.28) --
	(300.94,105.28) --
	(301.37,105.28) --
	(301.79,105.28) --
	(302.21,105.28) --
	(302.63,105.28) --
	(303.06,105.28) --
	(303.48,105.28) --
	(303.90,105.28) --
	(304.32,105.28) --
	(304.75,105.28) --
	(305.17,105.28) --
	(305.59,105.28) --
	(306.01,105.28) --
	(306.44,105.28) --
	(306.86,105.28) --
	(307.28,105.28) --
	(307.70,105.28) --
	(308.12,105.28) --
	(308.55,105.28) --
	(308.97,105.28) --
	(309.39,105.28) --
	(309.81,105.28) --
	(310.24,105.28) --
	(310.66,105.28) --
	(311.08,105.28) --
	(311.50,105.28) --
	(311.93,105.28) --
	(312.35,105.28) --
	(312.77,105.28) --
	(313.19,105.28) --
	(313.62,105.28) --
	(314.04,105.28) --
	(314.46,105.28) --
	(314.88,105.28) --
	(315.31,105.28) --
	(315.73,105.28) --
	(316.15,105.28) --
	(316.57,105.28) --
	(317.00,105.28) --
	(317.42,105.28) --
	(317.84,105.28) --
	(318.26,105.28) --
	(318.69,105.28) --
	(319.11,105.28) --
	(319.53,105.28) --
	(319.95,105.28) --
	(320.38,105.28) --
	(320.80,105.28) --
	(321.22,105.28) --
	(321.64,105.28) --
	(322.06,105.28) --
	(322.49,105.28) --
	(322.91,105.28) --
	(323.33,105.28) --
	(323.75,105.28) --
	(324.18,105.28) --
	(324.60,105.28) --
	(325.02,105.28) --
	(325.44,105.28) --
	(325.87,105.28) --
	(326.29,105.28) --
	(326.71,105.28) --
	(327.13,105.28) --
	(327.56,105.28) --
	(327.98,105.28) --
	(328.40,105.28) --
	(328.82,105.28) --
	(329.25,105.28) --
	(329.67,105.28) --
	(330.09,105.28) --
	(330.51,106.59) --
	(330.94,106.59) --
	(331.36,106.59) --
	(331.78,106.59) --
	(332.20,106.59) --
	(332.63,106.59) --
	(333.05,106.59) --
	(333.47,106.59) --
	(333.89,106.59) --
	(334.32,106.59) --
	(334.74,106.59) --
	(335.16,106.59) --
	(335.58,106.59) --
	(336.00,106.59) --
	(336.43,106.59) --
	(336.85,106.59) --
	(337.27,106.59) --
	(337.69,106.59) --
	(338.12,106.59) --
	(338.54,106.59) --
	(338.96,106.59) --
	(339.38,106.59) --
	(339.81,106.59) --
	(340.23,106.59) --
	(340.65,106.59) --
	(341.07,106.59) --
	(341.50,106.59) --
	(341.92,106.59) --
	(342.34,106.59) --
	(342.76,106.59) --
	(343.19,106.59) --
	(343.61,106.59) --
	(344.03,106.59) --
	(344.45,106.59) --
	(344.88,106.59) --
	(345.30,106.59) --
	(345.72,106.59) --
	(346.14,106.59) --
	(346.57,106.59) --
	(346.99,106.59) --
	(347.41,106.59) --
	(347.83,106.59) --
	(348.25,106.59) --
	(348.68,106.59) --
	(349.10,106.59) --
	(349.52,106.59) --
	(349.52,128.25) --
	(349.10,128.25) --
	(348.68,128.25) --
	(348.25,128.25) --
	(347.83,128.25) --
	(347.41,128.25) --
	(346.99,128.25) --
	(346.57,128.25) --
	(346.14,128.25) --
	(345.72,128.25) --
	(345.30,128.25) --
	(344.88,128.25) --
	(344.45,128.25) --
	(344.03,128.25) --
	(343.61,128.25) --
	(343.19,128.25) --
	(342.76,128.25) --
	(342.34,128.25) --
	(341.92,128.25) --
	(341.50,128.25) --
	(341.07,128.25) --
	(340.65,128.25) --
	(340.23,128.25) --
	(339.81,128.25) --
	(339.38,128.25) --
	(338.96,128.25) --
	(338.54,128.25) --
	(338.12,128.25) --
	(337.69,128.25) --
	(337.27,128.25) --
	(336.85,128.25) --
	(336.43,128.25) --
	(336.00,128.25) --
	(335.58,128.25) --
	(335.16,128.25) --
	(334.74,128.25) --
	(334.32,128.25) --
	(333.89,128.25) --
	(333.47,128.25) --
	(333.05,128.25) --
	(332.63,128.25) --
	(332.20,128.25) --
	(331.78,128.25) --
	(331.36,128.25) --
	(330.94,128.25) --
	(330.51,128.25) --
	(330.09,126.87) --
	(329.67,126.87) --
	(329.25,126.87) --
	(328.82,126.87) --
	(328.40,126.87) --
	(327.98,126.87) --
	(327.56,126.87) --
	(327.13,126.87) --
	(326.71,126.87) --
	(326.29,126.87) --
	(325.87,126.87) --
	(325.44,126.87) --
	(325.02,126.87) --
	(324.60,126.87) --
	(324.18,126.87) --
	(323.75,126.87) --
	(323.33,126.87) --
	(322.91,126.87) --
	(322.49,126.87) --
	(322.06,126.87) --
	(321.64,126.87) --
	(321.22,126.87) --
	(320.80,126.87) --
	(320.38,126.87) --
	(319.95,126.87) --
	(319.53,126.87) --
	(319.11,126.87) --
	(318.69,126.87) --
	(318.26,126.87) --
	(317.84,126.87) --
	(317.42,126.87) --
	(317.00,126.87) --
	(316.57,126.87) --
	(316.15,126.87) --
	(315.73,126.87) --
	(315.31,126.87) --
	(314.88,126.87) --
	(314.46,126.87) --
	(314.04,126.87) --
	(313.62,126.87) --
	(313.19,126.87) --
	(312.77,126.87) --
	(312.35,126.87) --
	(311.93,126.87) --
	(311.50,126.87) --
	(311.08,126.87) --
	(310.66,126.87) --
	(310.24,126.87) --
	(309.81,126.87) --
	(309.39,126.87) --
	(308.97,126.87) --
	(308.55,126.87) --
	(308.12,126.87) --
	(307.70,126.87) --
	(307.28,126.87) --
	(306.86,126.87) --
	(306.44,126.87) --
	(306.01,126.87) --
	(305.59,126.87) --
	(305.17,126.87) --
	(304.75,126.87) --
	(304.32,126.87) --
	(303.90,126.87) --
	(303.48,126.87) --
	(303.06,126.87) --
	(302.63,126.87) --
	(302.21,126.87) --
	(301.79,126.87) --
	(301.37,126.87) --
	(300.94,126.87) --
	(300.52,126.87) --
	(300.10,126.87) --
	(299.68,126.87) --
	(299.25,126.87) --
	(298.83,126.87) --
	(298.41,126.87) --
	(297.99,126.87) --
	(297.56,126.87) --
	(297.14,126.87) --
	(296.72,126.87) --
	(296.30,126.87) --
	(295.87,126.87) --
	(295.45,126.87) --
	(295.03,126.87) --
	(294.61,126.87) --
	(294.19,126.87) --
	(293.76,126.87) --
	(293.34,126.87) --
	(292.92,126.87) --
	(292.50,126.87) --
	(292.07,126.87) --
	(291.65,126.87) --
	(291.23,126.87) --
	(290.81,126.87) --
	(290.38,126.87) --
	(289.96,126.87) --
	(289.54,126.87) --
	(289.12,126.87) --
	(288.69,126.87) --
	(288.27,126.87) --
	(287.85,126.87) --
	(287.43,126.87) --
	(287.00,126.87) --
	(286.58,126.87) --
	(286.16,126.87) --
	(285.74,126.87) --
	(285.31,126.87) --
	(284.89,126.87) --
	(284.47,126.87) --
	(284.05,126.87) --
	(283.62,126.87) --
	(283.20,126.87) --
	(282.78,126.87) --
	(282.36,126.87) --
	(281.93,126.87) --
	(281.51,126.87) --
	(281.09,126.87) --
	(280.67,126.87) --
	(280.25,126.87) --
	(279.82,126.87) --
	(279.40,126.87) --
	(278.98,126.87) --
	(278.56,126.87) --
	(278.13,126.87) --
	(277.71,126.87) --
	(277.29,126.87) --
	(276.87,126.87) --
	(276.44,126.87) --
	(276.02,126.87) --
	(275.60,126.87) --
	(275.18,126.87) --
	(274.75,126.87) --
	(274.33,126.87) --
	(273.91,126.87) --
	(273.49,126.87) --
	(273.06,126.87) --
	(272.64,126.87) --
	(272.22,126.87) --
	(271.80,126.87) --
	(271.37,126.87) --
	(270.95,126.87) --
	(270.53,126.87) --
	(270.11,126.87) --
	(269.68,126.87) --
	(269.26,126.87) --
	(268.84,126.87) --
	(268.42,126.87) --
	(268.00,126.87) --
	(267.57,126.87) --
	(267.15,126.87) --
	(266.73,126.87) --
	(266.31,126.87) --
	(265.88,126.87) --
	(265.46,126.87) --
	(265.04,126.87) --
	(264.62,126.87) --
	(264.19,126.73) --
	(263.77,126.73) --
	(263.35,126.73) --
	(262.93,126.73) --
	(262.50,126.73) --
	(262.08,126.73) --
	(261.66,126.73) --
	(261.24,126.73) --
	(260.81,126.73) --
	(260.39,126.73) --
	(259.97,126.73) --
	(259.55,126.73) --
	(259.12,126.73) --
	(258.70,126.73) --
	(258.28,126.73) --
	(257.86,126.73) --
	(257.43,126.73) --
	(257.01,126.73) --
	(256.59,126.73) --
	(256.17,126.73) --
	(255.74,126.73) --
	(255.32,126.73) --
	(254.90,126.73) --
	(254.48,126.73) --
	(254.06,126.73) --
	(253.63,126.73) --
	(253.21,126.73) --
	(252.79,126.73) --
	(252.37,126.73) --
	(251.94,126.73) --
	(251.52,126.73) --
	(251.10,126.73) --
	(250.68,126.73) --
	(250.25,126.73) --
	(249.83,126.73) --
	(249.41,126.73) --
	(248.99,126.73) --
	(248.56,126.73) --
	(248.14,126.73) --
	(247.72,126.73) --
	(247.30,126.73) --
	(246.87,126.73) --
	(246.45,126.73) --
	(246.03,126.73) --
	(245.61,126.73) --
	(245.18,126.73) --
	(244.76,126.73) --
	(244.34,126.73) --
	(243.92,126.73) --
	(243.49,126.73) --
	(243.07,126.73) --
	(242.65,126.73) --
	(242.23,126.73) --
	(241.80,126.73) --
	(241.38,126.73) --
	(240.96,126.73) --
	(240.54,126.73) --
	(240.12,126.73) --
	(239.69,126.73) --
	(239.27,126.73) --
	(238.85,126.73) --
	(238.43,126.73) --
	(238.00,126.73) --
	(237.58,126.73) --
	(237.16,126.73) --
	(236.74,126.73) --
	(236.31,126.73) --
	(235.89,126.73) --
	(235.47,126.73) --
	(235.05,126.73) --
	(234.62,126.73) --
	(234.20,126.73) --
	(233.78,126.73) --
	(233.36,126.73) --
	(232.93,126.73) --
	(232.51,126.73) --
	(232.09,126.73) --
	(231.67,126.73) --
	(231.24,126.73) --
	(230.82,126.73) --
	(230.40,126.73) --
	(229.98,126.73) --
	(229.55,126.73) --
	(229.13,126.73) --
	(228.71,126.73) --
	(228.29,126.73) --
	(227.87,126.73) --
	(227.44,126.73) --
	(227.02,126.73) --
	(226.60,126.73) --
	(226.18,126.73) --
	(225.75,126.73) --
	(225.33,126.73) --
	(224.91,126.73) --
	(224.49,126.73) --
	(224.06,126.73) --
	(223.64,126.73) --
	(223.22,126.73) --
	(222.80,126.73) --
	(222.37,126.73) --
	(221.95,126.73) --
	(221.53,126.73) --
	(221.11,126.73) --
	(220.68,126.73) --
	(220.26,126.73) --
	(219.84,126.73) --
	(219.42,126.73) --
	(218.99,126.73) --
	(218.57,126.73) --
	(218.15,126.73) --
	(217.73,126.73) --
	(217.30,126.73) --
	(216.88,126.73) --
	(216.46,126.73) --
	(216.04,126.73) --
	(215.61,126.73) --
	(215.19,126.73) --
	(214.77,126.73) --
	(214.35,126.73) --
	(213.93,126.73) --
	(213.50,126.73) --
	(213.08,126.73) --
	(212.66,126.73) --
	(212.24,126.73) --
	(211.81,126.73) --
	(211.39,126.73) --
	(210.97,126.73) --
	(210.55,127.60) --
	(210.12,127.60) --
	(209.70,127.60) --
	(209.28,127.60) --
	(208.86,127.60) --
	(208.43,127.60) --
	(208.01,127.60) --
	(207.59,129.69) --
	(207.17,129.69) --
	(206.74,129.69) --
	(206.32,129.69) --
	(205.90,129.69) --
	(205.48,129.69) --
	(205.05,129.69) --
	(204.63,128.86) --
	(204.21,128.86) --
	(203.79,128.86) --
	(203.36,128.86) --
	(202.94,128.86) --
	(202.52,128.94) --
	(202.10,128.94) --
	(201.68,128.94) --
	(201.25,128.94) --
	(200.83,128.94) --
	(200.41,128.94) --
	(199.99,128.94) --
	(199.56,128.94) --
	(199.14,128.94) --
	(198.72,128.94) --
	(198.30,128.94) --
	(197.87,128.94) --
	(197.45,128.94) --
	(197.03,128.94) --
	(196.61,128.94) --
	(196.18,128.94) --
	(195.76,128.94) --
	(195.34,128.94) --
	(194.92,128.94) --
	(194.49,128.94) --
	(194.07,128.94) --
	(193.65,128.94) --
	(193.23,128.94) --
	(192.80,128.94) --
	(192.38,128.94) --
	(191.96,128.94) --
	(191.54,128.94) --
	(191.11,128.94) --
	(190.69,128.94) --
	(190.27,128.94) --
	(189.85,128.94) --
	(189.42,128.94) --
	(189.00,128.94) --
	(188.58,128.94) --
	(188.16,128.94) --
	(187.74,129.27) --
	(187.31,129.27) --
	(186.89,129.27) --
	(186.47,129.27) --
	(186.05,129.27) --
	(185.62,129.27) --
	(185.20,129.27) --
	(184.78,129.27) --
	(184.36,129.27) --
	(183.93,128.92) --
	(183.51,128.92) --
	(183.09,128.92) --
	(182.67,128.92) --
	(182.24,128.92) --
	(181.82,128.92) --
	(181.40,128.92) --
	(180.98,128.92) --
	(180.55,128.92) --
	(180.13,128.92) --
	(179.71,128.92) --
	(179.29,128.92) --
	(178.86,128.92) --
	(178.44,128.92) --
	(178.02,128.92) --
	(177.60,128.92) --
	(177.17,128.92) --
	(176.75,128.92) --
	(176.33,128.92) --
	(175.91,128.92) --
	(175.48,128.92) --
	(175.06,128.92) --
	(174.64,128.92) --
	(174.22,128.92) --
	(173.80,128.92) --
	(173.37,128.92) --
	(172.95,128.92) --
	(172.53,128.92) --
	(172.11,128.92) --
	(171.68,128.92) --
	(171.26,128.92) --
	(170.84,128.92) --
	(170.42,128.92) --
	(169.99,128.92) --
	(169.57,128.92) --
	(169.15,128.92) --
	(168.73,128.92) --
	(168.30,128.92) --
	(167.88,128.92) --
	(167.46,128.92) --
	(167.04,128.92) --
	(166.61,128.92) --
	(166.19,128.92) --
	(165.77,128.92) --
	(165.35,128.92) --
	(164.92,128.92) --
	(164.50,128.92) --
	(164.08,128.92) --
	(163.66,130.85) --
	(163.23,130.85) --
	(162.81,130.85) --
	(162.39,130.85) --
	(161.97,130.85) --
	(161.55,130.85) --
	(161.12,130.85) --
	(160.70,130.85) --
	(160.28,130.85) --
	(159.86,130.85) --
	(159.43,130.85) --
	(159.01,130.85) --
	(158.59,130.85) --
	(158.17,130.85) --
	(157.74,130.85) --
	(157.32,130.85) --
	(156.90,130.85) --
	(156.48,130.85) --
	(156.05,130.85) --
	(155.63,130.85) --
	(155.21,130.85) --
	(154.79,130.85) --
	(154.36,130.85) --
	(153.94,130.85) --
	(153.52,130.85) --
	(153.10,130.85) --
	(152.67,130.85) --
	(152.25,130.85) --
	(151.83,130.85) --
	(151.41,130.85) --
	(150.98,130.85) --
	(150.56,130.85) --
	(150.14,130.85) --
	(149.72,130.85) --
	(149.29,130.85) --
	(148.87,130.85) --
	(148.45,130.85) --
	(148.03,130.85) --
	(147.61,130.85) --
	(147.18,130.82) --
	(146.76,130.82) --
	(146.34,130.82) --
	(145.92,130.82) --
	(145.49,130.82) --
	(145.07,130.82) --
	(144.65,130.82) --
	(144.23,130.82) --
	(143.80,130.82) --
	(143.38,130.82) --
	(142.96,130.82) --
	(142.54,130.82) --
	(142.11,130.82) --
	(141.69,130.82) --
	(141.27,130.82) --
	(140.85,130.82) --
	(140.42,130.82) --
	(140.00,130.82) --
	(139.58,130.82) --
	(139.16,130.82) --
	(138.73,130.82) --
	(138.31,130.82) --
	(137.89,130.82) --
	(137.47,130.82) --
	(137.04,130.82) --
	(136.62,130.82) --
	(136.20,130.82) --
	(135.78,130.82) --
	(135.35,130.82) --
	(134.93,130.82) --
	(134.51,130.82) --
	(134.09,130.82) --
	(133.67,130.82) --
	(133.24,130.82) --
	(132.82,130.82) --
	(132.40,130.82) --
	(131.98,130.97) --
	(131.55,130.97) --
	(131.13,130.97) --
	(130.71,130.97) --
	(130.29,130.97) --
	(129.86,130.97) --
	(129.44,130.97) --
	(129.02,130.97) --
	(128.60,131.08) --
	(128.17,131.01) --
	(127.75,131.01) --
	(127.33,131.01) --
	(126.91,131.01) --
	(126.48,131.01) --
	(126.06,131.01) --
	(125.64,131.01) --
	(125.22,131.01) --
	(124.79,130.88) --
	(124.37,130.88) --
	(123.95,130.88) --
	(123.53,130.88) --
	(123.10,131.51) --
	(122.68,131.30) --
	(122.26,131.30) --
	(121.84,131.30) --
	(121.42,131.30) --
	(120.99,131.30) --
	(120.57,131.30) --
	(120.15,131.30) --
	(119.73,130.76) --
	(119.30,130.76) --
	(118.88,130.76) --
	(118.46,130.76) --
	(118.04,130.49) --
	(117.61,130.49) --
	(117.19,130.49) --
	(116.77,130.49) --
	(116.35,131.15) --
	(115.92,131.15) --
	(115.50,131.15) --
	(115.08,131.15) --
	(114.66,131.15) --
	(114.23,131.15) --
	(113.81,131.05) --
	(113.39,130.57) --
	(112.97,130.57) --
	(112.54,130.57) --
	(112.12,130.57) --
	(111.70,130.57) --
	(111.28,130.57) --
	(110.85,130.57) --
	(110.43,130.57) --
	(110.01,130.55) --
	(109.59,131.43) --
	(109.16,131.43) --
	(108.74,131.43) --
	(108.32,130.84) --
	(107.90,130.89) --
	(107.48,130.89) --
	(107.05,130.89) --
	(106.63,130.89) --
	(106.21,130.89) --
	(105.79,130.89) --
	(105.36,130.17) --
	(104.94,129.95) --
	(104.52,129.95) --
	(104.10,129.95) --
	(103.67,129.95) --
	(103.25,129.95) --
	(102.83,129.95) --
	(102.41,129.95) --
	(101.98,130.11) --
	(101.56,130.00) --
	(101.14,130.00) --
	(100.72,130.21) --
	(100.29,130.21) --
	( 99.87,130.51) --
	( 99.45,130.48) --
	( 99.03,130.78) --
	( 98.60,130.78) --
	( 98.18,130.78) --
	( 97.76,130.78) --
	( 97.34,130.78) --
	( 96.91,129.87) --
	( 96.49,129.87) --
	( 96.07,129.87) --
	( 95.65,129.88) --
	( 95.22,129.88) --
	( 94.80,129.88) --
	( 94.38,129.76) --
	( 93.96,129.29) --
	( 93.54,129.29) --
	( 93.11,129.29) --
	( 92.69,129.29) --
	( 92.27,129.29) --
	( 91.85,129.29) --
	( 91.42,129.29) --
	( 91.00,129.39) --
	( 90.58,129.39) --
	( 90.16,129.39) --
	( 89.73,131.64) --
	( 89.31,130.09) --
	( 88.89,129.93) --
	( 88.47,129.93) --
	( 88.04,129.98) --
	( 87.62,129.87) --
	( 87.20,129.80) --
	( 86.78,129.80) --
	( 86.35,130.62) --
	( 85.93,130.48) --
	( 85.51,130.91) --
	( 85.09,130.91) --
	( 84.66,132.08) --
	( 84.24,132.65) --
	( 83.82,132.65) --
	( 83.40,133.50) --
	( 82.97,133.60) --
	( 82.55,133.67) --
	( 82.13,134.77) --
	( 81.71,135.73) --
	( 81.29,135.65) --
	( 80.86,135.21) --
	( 80.44,135.21) --
	( 80.02,141.39) --
	( 79.60,144.54) --
	( 79.17,144.54) --
	( 78.75,144.62) --
	( 78.33,144.77) --
	( 77.91,144.74) --
	( 77.48,145.11) --
	( 77.06,145.90) --
	( 76.64,146.36) --
	( 76.22,146.17) --
	( 75.79,146.17) --
	( 75.37,147.83) --
	( 74.95,148.86) --
	( 74.53,150.25) --
	( 74.10,151.41) --
	( 73.68,151.41) --
	( 73.26,147.76) --
	( 72.84,150.31) --
	( 72.41,149.97) --
	( 71.99,150.40) --
	( 71.57,152.03) --
	( 71.15,152.90) --
	( 70.72,153.70) --
	( 70.30,154.16) --
	( 69.88,156.82) --
	( 69.46,156.82) --
	( 69.03,156.82) --
	( 68.61,156.94) --
	( 68.19,161.30) --
	( 67.77,161.30) --
	( 67.35,166.00) --
	( 66.92,166.00) --
	( 66.50,164.69) --
	( 66.08,166.32) --
	( 65.66,166.36) --
	( 65.23,163.76) --
	( 64.81,163.76) --
	( 64.39,166.18) --
	( 63.97,173.13) --
	( 63.54,173.13) --
	( 63.12,175.70) --
	( 62.70,168.57) --
	( 62.28,176.29) --
	( 61.85,181.43) --
	( 61.43,177.27) --
	( 61.01,169.14) --
	( 60.59,173.85) --
	( 60.16,173.85) --
	( 59.74,181.73) --
	( 59.32,181.73) --
	( 58.90,199.31) --
	( 58.47,199.31) --
	( 58.05,199.31) --
	cycle;
\definecolor{drawColor}{RGB}{0,0,0}

\path[draw=drawColor,line width= 0.4pt,dash pattern=on 7pt off 3pt ,line join=round,line cap=round] ( 42.00,129.40) -- (361.35,129.40);

\path[draw=drawColor,line width= 0.4pt,line join=round,line cap=round] ( 58.05,106.06) --
	( 58.47,106.06) --
	( 58.90,106.06) --
	( 59.32,109.14) --
	( 59.74,109.14) --
	( 60.16,107.26) --
	( 60.59,107.26) --
	( 61.01,108.12) --
	( 61.43,125.61) --
	( 61.85,137.48) --
	( 62.28,132.07) --
	( 62.70,123.57) --
	( 63.12,138.75) --
	( 63.54,138.33) --
	( 63.97,138.33) --
	( 64.39,135.69) --
	( 64.81,134.98) --
	( 65.23,134.98) --
	( 65.66,139.62) --
	( 66.08,141.26) --
	( 66.50,140.16) --
	( 66.92,143.48) --
	( 67.35,143.48) --
	( 67.77,138.77) --
	( 68.19,138.77) --
	( 68.61,134.91) --
	( 69.03,135.05) --
	( 69.46,135.05) --
	( 69.88,135.05) --
	( 70.30,132.58) --
	( 70.72,132.58) --
	( 71.15,132.38) --
	( 71.57,131.86) --
	( 71.99,130.28) --
	( 72.41,130.22) --
	( 72.84,131.13) --
	( 73.26,129.07) --
	( 73.68,133.74) --
	( 74.10,133.74) --
	( 74.53,132.89) --
	( 74.95,131.77) --
	( 75.37,130.80) --
	( 75.79,129.73) --
	( 76.22,129.73) --
	( 76.64,130.04) --
	( 77.06,129.85) --
	( 77.48,129.13) --
	( 77.91,128.84) --
	( 78.33,128.96) --
	( 78.75,128.90) --
	( 79.17,128.91) --
	( 79.60,128.91) --
	( 80.02,126.01) --
	( 80.44,120.19) --
	( 80.86,120.19) --
	( 81.29,120.71) --
	( 81.71,120.86) --
	( 82.13,120.06) --
	( 82.55,119.11) --
	( 82.97,119.11) --
	( 83.40,119.15) --
	( 83.82,118.45) --
	( 84.24,118.45) --
	( 84.66,118.08) --
	( 85.09,117.18) --
	( 85.51,117.18) --
	( 85.93,116.99) --
	( 86.35,117.25) --
	( 86.78,116.55) --
	( 87.20,116.55) --
	( 87.62,116.73) --
	( 88.04,116.88) --
	( 88.47,116.88) --
	( 88.89,116.88) --
	( 89.31,117.14) --
	( 89.73,118.71) --
	( 90.16,116.61) --
	( 90.58,116.61) --
	( 91.00,116.61) --
	( 91.42,116.56) --
	( 91.85,116.56) --
	( 92.27,116.56) --
	( 92.69,116.56) --
	( 93.11,116.56) --
	( 93.54,116.56) --
	( 93.96,116.56) --
	( 94.38,117.06) --
	( 94.80,117.23) --
	( 95.22,117.23) --
	( 95.65,117.23) --
	( 96.07,117.26) --
	( 96.49,117.26) --
	( 96.91,117.26) --
	( 97.34,118.24) --
	( 97.76,118.24) --
	( 98.18,118.24) --
	( 98.60,118.24) --
	( 99.03,118.24) --
	( 99.45,118.04) --
	( 99.87,118.11) --
	(100.29,117.86) --
	(100.72,117.86) --
	(101.14,117.74) --
	(101.56,117.74) --
	(101.98,117.89) --
	(102.41,117.78) --
	(102.83,117.78) --
	(103.25,117.78) --
	(103.67,117.78) --
	(104.10,117.78) --
	(104.52,117.78) --
	(104.94,117.78) --
	(105.36,118.03) --
	(105.79,118.77) --
	(106.21,118.77) --
	(106.63,118.77) --
	(107.05,118.77) --
	(107.48,118.77) --
	(107.90,118.77) --
	(108.32,118.77) --
	(108.74,119.39) --
	(109.16,119.39) --
	(109.59,119.39) --
	(110.01,118.56) --
	(110.43,118.62) --
	(110.85,118.62) --
	(111.28,118.62) --
	(111.70,118.62) --
	(112.12,118.62) --
	(112.54,118.62) --
	(112.97,118.62) --
	(113.39,118.62) --
	(113.81,119.17) --
	(114.23,119.31) --
	(114.66,119.31) --
	(115.08,119.31) --
	(115.50,119.31) --
	(115.92,119.31) --
	(116.35,119.31) --
	(116.77,118.69) --
	(117.19,118.69) --
	(117.61,118.69) --
	(118.04,118.69) --
	(118.46,119.00) --
	(118.88,119.00) --
	(119.30,119.00) --
	(119.73,119.00) --
	(120.15,119.57) --
	(120.57,119.57) --
	(120.99,119.57) --
	(121.42,119.57) --
	(121.84,119.57) --
	(122.26,119.57) --
	(122.68,119.57) --
	(123.10,119.81) --
	(123.53,119.23) --
	(123.95,119.23) --
	(124.37,119.23) --
	(124.79,119.23) --
	(125.22,119.39) --
	(125.64,119.39) --
	(126.06,119.39) --
	(126.48,119.39) --
	(126.91,119.39) --
	(127.33,119.39) --
	(127.75,119.39) --
	(128.17,119.39) --
	(128.60,119.50) --
	(129.02,119.45) --
	(129.44,119.45) --
	(129.86,119.45) --
	(130.29,119.45) --
	(130.71,119.45) --
	(131.13,119.45) --
	(131.55,119.45) --
	(131.98,119.45) --
	(132.40,119.34) --
	(132.82,119.34) --
	(133.24,119.34) --
	(133.67,119.34) --
	(134.09,119.34) --
	(134.51,119.34) --
	(134.93,119.34) --
	(135.35,119.34) --
	(135.78,119.34) --
	(136.20,119.34) --
	(136.62,119.34) --
	(137.04,119.34) --
	(137.47,119.34) --
	(137.89,119.34) --
	(138.31,119.34) --
	(138.73,119.34) --
	(139.16,119.34) --
	(139.58,119.34) --
	(140.00,119.34) --
	(140.42,119.34) --
	(140.85,119.34) --
	(141.27,119.34) --
	(141.69,119.34) --
	(142.11,119.34) --
	(142.54,119.34) --
	(142.96,119.34) --
	(143.38,119.34) --
	(143.80,119.34) --
	(144.23,119.34) --
	(144.65,119.34) --
	(145.07,119.34) --
	(145.49,119.34) --
	(145.92,119.34) --
	(146.34,119.34) --
	(146.76,119.34) --
	(147.18,119.34) --
	(147.61,119.41) --
	(148.03,119.41) --
	(148.45,119.41) --
	(148.87,119.41) --
	(149.29,119.41) --
	(149.72,119.41) --
	(150.14,119.41) --
	(150.56,119.41) --
	(150.98,119.41) --
	(151.41,119.41) --
	(151.83,119.41) --
	(152.25,119.41) --
	(152.67,119.41) --
	(153.10,119.41) --
	(153.52,119.41) --
	(153.94,119.41) --
	(154.36,119.41) --
	(154.79,119.41) --
	(155.21,119.41) --
	(155.63,119.41) --
	(156.05,119.41) --
	(156.48,119.41) --
	(156.90,119.41) --
	(157.32,119.41) --
	(157.74,119.41) --
	(158.17,119.41) --
	(158.59,119.41) --
	(159.01,119.41) --
	(159.43,119.41) --
	(159.86,119.41) --
	(160.28,119.41) --
	(160.70,119.41) --
	(161.12,119.41) --
	(161.55,119.41) --
	(161.97,119.41) --
	(162.39,119.41) --
	(162.81,119.41) --
	(163.23,119.41) --
	(163.66,119.41) --
	(164.08,117.56) --
	(164.50,117.56) --
	(164.92,117.56) --
	(165.35,117.56) --
	(165.77,117.56) --
	(166.19,117.56) --
	(166.61,117.56) --
	(167.04,117.56) --
	(167.46,117.56) --
	(167.88,117.56) --
	(168.30,117.56) --
	(168.73,117.56) --
	(169.15,117.56) --
	(169.57,117.56) --
	(169.99,117.56) --
	(170.42,117.56) --
	(170.84,117.56) --
	(171.26,117.56) --
	(171.68,117.56) --
	(172.11,117.56) --
	(172.53,117.56) --
	(172.95,117.56) --
	(173.37,117.56) --
	(173.80,117.56) --
	(174.22,117.56) --
	(174.64,117.56) --
	(175.06,117.56) --
	(175.48,117.56) --
	(175.91,117.56) --
	(176.33,117.56) --
	(176.75,117.56) --
	(177.17,117.56) --
	(177.60,117.56) --
	(178.02,117.56) --
	(178.44,117.56) --
	(178.86,117.56) --
	(179.29,117.56) --
	(179.71,117.56) --
	(180.13,117.56) --
	(180.55,117.56) --
	(180.98,117.56) --
	(181.40,117.56) --
	(181.82,117.56) --
	(182.24,117.56) --
	(182.67,117.56) --
	(183.09,117.56) --
	(183.51,117.56) --
	(183.93,117.56) --
	(184.36,117.93) --
	(184.78,117.93) --
	(185.20,117.93) --
	(185.62,117.93) --
	(186.05,117.93) --
	(186.47,117.93) --
	(186.89,117.93) --
	(187.31,117.93) --
	(187.74,117.93) --
	(188.16,117.64) --
	(188.58,117.64) --
	(189.00,117.64) --
	(189.42,117.64) --
	(189.85,117.64) --
	(190.27,117.64) --
	(190.69,117.64) --
	(191.11,117.64) --
	(191.54,117.64) --
	(191.96,117.64) --
	(192.38,117.64) --
	(192.80,117.64) --
	(193.23,117.64) --
	(193.65,117.64) --
	(194.07,117.64) --
	(194.49,117.64) --
	(194.92,117.64) --
	(195.34,117.64) --
	(195.76,117.64) --
	(196.18,117.64) --
	(196.61,117.64) --
	(197.03,117.64) --
	(197.45,117.64) --
	(197.87,117.64) --
	(198.30,117.64) --
	(198.72,117.64) --
	(199.14,117.64) --
	(199.56,117.64) --
	(199.99,117.64) --
	(200.41,117.64) --
	(200.83,117.64) --
	(201.25,117.64) --
	(201.68,117.64) --
	(202.10,117.64) --
	(202.52,117.64) --
	(202.94,117.59) --
	(203.36,117.59) --
	(203.79,117.59) --
	(204.21,117.59) --
	(204.63,117.59) --
	(205.05,118.43) --
	(205.48,118.43) --
	(205.90,118.43) --
	(206.32,118.43) --
	(206.74,118.43) --
	(207.17,118.43) --
	(207.59,118.43) --
	(208.01,116.43) --
	(208.43,116.43) --
	(208.86,116.43) --
	(209.28,116.43) --
	(209.70,116.43) --
	(210.12,116.43) --
	(210.55,116.43) --
	(210.97,115.63) --
	(211.39,115.63) --
	(211.81,115.63) --
	(212.24,115.63) --
	(212.66,115.63) --
	(213.08,115.63) --
	(213.50,115.63) --
	(213.93,115.63) --
	(214.35,115.63) --
	(214.77,115.63) --
	(215.19,115.63) --
	(215.61,115.63) --
	(216.04,115.63) --
	(216.46,115.63) --
	(216.88,115.63) --
	(217.30,115.63) --
	(217.73,115.63) --
	(218.15,115.63) --
	(218.57,115.63) --
	(218.99,115.63) --
	(219.42,115.63) --
	(219.84,115.63) --
	(220.26,115.63) --
	(220.68,115.63) --
	(221.11,115.63) --
	(221.53,115.63) --
	(221.95,115.63) --
	(222.37,115.63) --
	(222.80,115.63) --
	(223.22,115.63) --
	(223.64,115.63) --
	(224.06,115.63) --
	(224.49,115.63) --
	(224.91,115.63) --
	(225.33,115.63) --
	(225.75,115.63) --
	(226.18,115.63) --
	(226.60,115.63) --
	(227.02,115.63) --
	(227.44,115.63) --
	(227.87,115.63) --
	(228.29,115.63) --
	(228.71,115.63) --
	(229.13,115.63) --
	(229.55,115.63) --
	(229.98,115.63) --
	(230.40,115.63) --
	(230.82,115.63) --
	(231.24,115.63) --
	(231.67,115.63) --
	(232.09,115.63) --
	(232.51,115.63) --
	(232.93,115.63) --
	(233.36,115.63) --
	(233.78,115.63) --
	(234.20,115.63) --
	(234.62,115.63) --
	(235.05,115.63) --
	(235.47,115.63) --
	(235.89,115.63) --
	(236.31,115.63) --
	(236.74,115.63) --
	(237.16,115.63) --
	(237.58,115.63) --
	(238.00,115.63) --
	(238.43,115.63) --
	(238.85,115.63) --
	(239.27,115.63) --
	(239.69,115.63) --
	(240.12,115.63) --
	(240.54,115.63) --
	(240.96,115.63) --
	(241.38,115.63) --
	(241.80,115.63) --
	(242.23,115.63) --
	(242.65,115.63) --
	(243.07,115.63) --
	(243.49,115.63) --
	(243.92,115.63) --
	(244.34,115.63) --
	(244.76,115.63) --
	(245.18,115.63) --
	(245.61,115.63) --
	(246.03,115.63) --
	(246.45,115.63) --
	(246.87,115.63) --
	(247.30,115.63) --
	(247.72,115.63) --
	(248.14,115.63) --
	(248.56,115.63) --
	(248.99,115.63) --
	(249.41,115.63) --
	(249.83,115.63) --
	(250.25,115.63) --
	(250.68,115.63) --
	(251.10,115.63) --
	(251.52,115.63) --
	(251.94,115.63) --
	(252.37,115.63) --
	(252.79,115.63) --
	(253.21,115.63) --
	(253.63,115.63) --
	(254.06,115.63) --
	(254.48,115.63) --
	(254.90,115.63) --
	(255.32,115.63) --
	(255.74,115.63) --
	(256.17,115.63) --
	(256.59,115.63) --
	(257.01,115.63) --
	(257.43,115.63) --
	(257.86,115.63) --
	(258.28,115.63) --
	(258.70,115.63) --
	(259.12,115.63) --
	(259.55,115.63) --
	(259.97,115.63) --
	(260.39,115.63) --
	(260.81,115.63) --
	(261.24,115.63) --
	(261.66,115.63) --
	(262.08,115.63) --
	(262.50,115.63) --
	(262.93,115.63) --
	(263.35,115.63) --
	(263.77,115.63) --
	(264.19,115.63) --
	(264.62,115.79) --
	(265.04,115.79) --
	(265.46,115.79) --
	(265.88,115.79) --
	(266.31,115.79) --
	(266.73,115.79) --
	(267.15,115.79) --
	(267.57,115.79) --
	(268.00,115.79) --
	(268.42,115.79) --
	(268.84,115.79) --
	(269.26,115.79) --
	(269.68,115.79) --
	(270.11,115.79) --
	(270.53,115.79) --
	(270.95,115.79) --
	(271.37,115.79) --
	(271.80,115.79) --
	(272.22,115.79) --
	(272.64,115.79) --
	(273.06,115.79) --
	(273.49,115.79) --
	(273.91,115.79) --
	(274.33,115.79) --
	(274.75,115.79) --
	(275.18,115.79) --
	(275.60,115.79) --
	(276.02,115.79) --
	(276.44,115.79) --
	(276.87,115.79) --
	(277.29,115.79) --
	(277.71,115.79) --
	(278.13,115.79) --
	(278.56,115.79) --
	(278.98,115.79) --
	(279.40,115.79) --
	(279.82,115.79) --
	(280.25,115.79) --
	(280.67,115.79) --
	(281.09,115.79) --
	(281.51,115.79) --
	(281.93,115.79) --
	(282.36,115.79) --
	(282.78,115.79) --
	(283.20,115.79) --
	(283.62,115.79) --
	(284.05,115.79) --
	(284.47,115.79) --
	(284.89,115.79) --
	(285.31,115.79) --
	(285.74,115.79) --
	(286.16,115.79) --
	(286.58,115.79) --
	(287.00,115.79) --
	(287.43,115.79) --
	(287.85,115.79) --
	(288.27,115.79) --
	(288.69,115.79) --
	(289.12,115.79) --
	(289.54,115.79) --
	(289.96,115.79) --
	(290.38,115.79) --
	(290.81,115.79) --
	(291.23,115.79) --
	(291.65,115.79) --
	(292.07,115.79) --
	(292.50,115.79) --
	(292.92,115.79) --
	(293.34,115.79) --
	(293.76,115.79) --
	(294.19,115.79) --
	(294.61,115.79) --
	(295.03,115.79) --
	(295.45,115.79) --
	(295.87,115.79) --
	(296.30,115.79) --
	(296.72,115.79) --
	(297.14,115.79) --
	(297.56,115.79) --
	(297.99,115.79) --
	(298.41,115.79) --
	(298.83,115.79) --
	(299.25,115.79) --
	(299.68,115.79) --
	(300.10,115.79) --
	(300.52,115.79) --
	(300.94,115.79) --
	(301.37,115.79) --
	(301.79,115.79) --
	(302.21,115.79) --
	(302.63,115.79) --
	(303.06,115.79) --
	(303.48,115.79) --
	(303.90,115.79) --
	(304.32,115.79) --
	(304.75,115.79) --
	(305.17,115.79) --
	(305.59,115.79) --
	(306.01,115.79) --
	(306.44,115.79) --
	(306.86,115.79) --
	(307.28,115.79) --
	(307.70,115.79) --
	(308.12,115.79) --
	(308.55,115.79) --
	(308.97,115.79) --
	(309.39,115.79) --
	(309.81,115.79) --
	(310.24,115.79) --
	(310.66,115.79) --
	(311.08,115.79) --
	(311.50,115.79) --
	(311.93,115.79) --
	(312.35,115.79) --
	(312.77,115.79) --
	(313.19,115.79) --
	(313.62,115.79) --
	(314.04,115.79) --
	(314.46,115.79) --
	(314.88,115.79) --
	(315.31,115.79) --
	(315.73,115.79) --
	(316.15,115.79) --
	(316.57,115.79) --
	(317.00,115.79) --
	(317.42,115.79) --
	(317.84,115.79) --
	(318.26,115.79) --
	(318.69,115.79) --
	(319.11,115.79) --
	(319.53,115.79) --
	(319.95,115.79) --
	(320.38,115.79) --
	(320.80,115.79) --
	(321.22,115.79) --
	(321.64,115.79) --
	(322.06,115.79) --
	(322.49,115.79) --
	(322.91,115.79) --
	(323.33,115.79) --
	(323.75,115.79) --
	(324.18,115.79) --
	(324.60,115.79) --
	(325.02,115.79) --
	(325.44,115.79) --
	(325.87,115.79) --
	(326.29,115.79) --
	(326.71,115.79) --
	(327.13,115.79) --
	(327.56,115.79) --
	(327.98,115.79) --
	(328.40,115.79) --
	(328.82,115.79) --
	(329.25,115.79) --
	(329.67,115.79) --
	(330.09,115.79) --
	(330.51,117.16) --
	(330.94,117.16) --
	(331.36,117.16) --
	(331.78,117.16) --
	(332.20,117.16) --
	(332.63,117.16) --
	(333.05,117.16) --
	(333.47,117.16) --
	(333.89,117.16) --
	(334.32,117.16) --
	(334.74,117.16) --
	(335.16,117.16) --
	(335.58,117.16) --
	(336.00,117.16) --
	(336.43,117.16) --
	(336.85,117.16) --
	(337.27,117.16) --
	(337.69,117.16) --
	(338.12,117.16) --
	(338.54,117.16) --
	(338.96,117.16) --
	(339.38,117.16) --
	(339.81,117.16) --
	(340.23,117.16) --
	(340.65,117.16) --
	(341.07,117.16) --
	(341.50,117.16) --
	(341.92,117.16) --
	(342.34,117.16) --
	(342.76,117.16) --
	(343.19,117.16) --
	(343.61,117.16) --
	(344.03,117.16) --
	(344.45,117.16) --
	(344.88,117.16) --
	(345.30,117.16) --
	(345.72,117.16) --
	(346.14,117.16) --
	(346.57,117.16) --
	(346.99,117.16) --
	(347.41,117.16) --
	(347.83,117.16) --
	(348.25,117.16) --
	(348.68,117.16) --
	(349.10,117.16) --
	(349.52,117.16);
\end{scope}
\end{tikzpicture}
}
	\end{adjustbox}
	\newline
	\begin{adjustbox}{width=1\textwidth}
		\subfloat[Test][Zusammenhang ($r$) zwischen der Steigung und dem \gls{zwert} des \gls{bist}s.]{% Created by tikzDevice version 0.10.1 on 2016-08-18 15:54:08
% !TEX encoding = UTF-8 Unicode
\begin{tikzpicture}[x=1pt,y=1pt]
\definecolor{fillColor}{RGB}{255,255,255}
\path[use as bounding box,fill=fillColor,fill opacity=0.00] (0,0) rectangle (361.35,216.81);
\begin{scope}
\path[clip] (  0.00,  0.00) rectangle (361.35,216.81);
\definecolor{drawColor}{RGB}{0,0,0}

\node[text=drawColor,anchor=base,inner sep=0pt, outer sep=0pt, scale=  1.00] at (201.68,  8.40) {\textit{RMSE} Grenzwert (ms)};

\node[text=drawColor,rotate= 90.00,anchor=base,inner sep=0pt, outer sep=0pt, scale=  1.00] at (  9.60,129.40) {\textit{r}};
\end{scope}
\begin{scope}
\path[clip] (  0.00,  0.00) rectangle (361.35,216.81);
\definecolor{drawColor}{RGB}{0,0,0}

\path[draw=drawColor,line width= 0.4pt,line join=round,line cap=round] ( 58.05, 48.00) -- (349.52, 48.00);

\path[draw=drawColor,line width= 0.4pt,line join=round,line cap=round] ( 58.05, 48.00) -- ( 58.05, 42.00);

\path[draw=drawColor,line width= 0.4pt,line join=round,line cap=round] ( 96.07, 48.00) -- ( 96.07, 42.00);

\path[draw=drawColor,line width= 0.4pt,line join=round,line cap=round] (138.31, 48.00) -- (138.31, 42.00);

\path[draw=drawColor,line width= 0.4pt,line join=round,line cap=round] (180.55, 48.00) -- (180.55, 42.00);

\path[draw=drawColor,line width= 0.4pt,line join=round,line cap=round] (222.80, 48.00) -- (222.80, 42.00);

\path[draw=drawColor,line width= 0.4pt,line join=round,line cap=round] (265.04, 48.00) -- (265.04, 42.00);

\path[draw=drawColor,line width= 0.4pt,line join=round,line cap=round] (307.28, 48.00) -- (307.28, 42.00);

\path[draw=drawColor,line width= 0.4pt,line join=round,line cap=round] (349.52, 48.00) -- (349.52, 42.00);

\node[text=drawColor,anchor=base,inner sep=0pt, outer sep=0pt, scale=  1.00] at ( 58.05, 30.00) {1};

\node[text=drawColor,anchor=base,inner sep=0pt, outer sep=0pt, scale=  1.00] at ( 96.07, 30.00) {10};

\node[text=drawColor,anchor=base,inner sep=0pt, outer sep=0pt, scale=  1.00] at (138.31, 30.00) {20};

\node[text=drawColor,anchor=base,inner sep=0pt, outer sep=0pt, scale=  1.00] at (180.55, 30.00) {30};

\node[text=drawColor,anchor=base,inner sep=0pt, outer sep=0pt, scale=  1.00] at (222.80, 30.00) {40};

\node[text=drawColor,anchor=base,inner sep=0pt, outer sep=0pt, scale=  1.00] at (265.04, 30.00) {50};

\node[text=drawColor,anchor=base,inner sep=0pt, outer sep=0pt, scale=  1.00] at (307.28, 30.00) {60};

\node[text=drawColor,anchor=base,inner sep=0pt, outer sep=0pt, scale=  1.00] at (349.52, 30.00) {70};

\path[draw=drawColor,line width= 0.4pt,line join=round,line cap=round] ( 42.00, 54.03) -- ( 42.00,204.78);

\path[draw=drawColor,line width= 0.4pt,line join=round,line cap=round] ( 42.00, 54.03) -- ( 36.00, 54.03);

\path[draw=drawColor,line width= 0.4pt,line join=round,line cap=round] ( 42.00, 72.87) -- ( 36.00, 72.87);

\path[draw=drawColor,line width= 0.4pt,line join=round,line cap=round] ( 42.00, 91.72) -- ( 36.00, 91.72);

\path[draw=drawColor,line width= 0.4pt,line join=round,line cap=round] ( 42.00,110.56) -- ( 36.00,110.56);

\path[draw=drawColor,line width= 0.4pt,line join=round,line cap=round] ( 42.00,129.40) -- ( 36.00,129.40);

\path[draw=drawColor,line width= 0.4pt,line join=round,line cap=round] ( 42.00,148.25) -- ( 36.00,148.25);

\path[draw=drawColor,line width= 0.4pt,line join=round,line cap=round] ( 42.00,167.09) -- ( 36.00,167.09);

\path[draw=drawColor,line width= 0.4pt,line join=round,line cap=round] ( 42.00,185.94) -- ( 36.00,185.94);

\path[draw=drawColor,line width= 0.4pt,line join=round,line cap=round] ( 42.00,204.78) -- ( 36.00,204.78);

\node[text=drawColor,anchor=base east,inner sep=0pt, outer sep=0pt, scale=  1.00] at ( 33.60, 50.59) {--1.00};

\node[text=drawColor,anchor=base east,inner sep=0pt, outer sep=0pt, scale=  1.00] at ( 33.60, 69.43) {--.75};

\node[text=drawColor,anchor=base east,inner sep=0pt, outer sep=0pt, scale=  1.00] at ( 33.60, 88.27) {--.50};

\node[text=drawColor,anchor=base east,inner sep=0pt, outer sep=0pt, scale=  1.00] at ( 33.60,107.12) {--.25};

\node[text=drawColor,anchor=base east,inner sep=0pt, outer sep=0pt, scale=  1.00] at ( 33.60,125.96) {.00};

\node[text=drawColor,anchor=base east,inner sep=0pt, outer sep=0pt, scale=  1.00] at ( 33.60,144.81) {.25};

\node[text=drawColor,anchor=base east,inner sep=0pt, outer sep=0pt, scale=  1.00] at ( 33.60,163.65) {.50};

\node[text=drawColor,anchor=base east,inner sep=0pt, outer sep=0pt, scale=  1.00] at ( 33.60,182.49) {.75};

\node[text=drawColor,anchor=base east,inner sep=0pt, outer sep=0pt, scale=  1.00] at ( 33.60,201.34) {1.00};
\end{scope}
\begin{scope}
\path[clip] ( 42.00, 48.00) rectangle (361.35,210.81);
\definecolor{fillColor}{RGB}{190,190,190}

\path[fill=fillColor] ( 58.05, 65.66) --
	( 58.47, 65.66) --
	( 58.90, 65.66) --
	( 59.32, 75.62) --
	( 59.74, 75.62) --
	( 60.16, 88.17) --
	( 60.59, 88.17) --
	( 61.01, 94.01) --
	( 61.43, 84.92) --
	( 61.85, 79.22) --
	( 62.28, 83.78) --
	( 62.70, 90.58) --
	( 63.12, 82.25) --
	( 63.54, 86.39) --
	( 63.97, 86.39) --
	( 64.39, 94.21) --
	( 64.81,100.55) --
	( 65.23,100.55) --
	( 65.66, 98.11) --
	( 66.08, 98.83) --
	( 66.50,100.73) --
	( 66.92, 94.37) --
	( 67.35, 94.37) --
	( 67.77, 98.91) --
	( 68.19, 98.91) --
	( 68.61,100.30) --
	( 69.03,100.91) --
	( 69.46,100.91) --
	( 69.88,100.91) --
	( 70.30,103.85) --
	( 70.72,103.90) --
	( 71.15,104.99) --
	( 71.57,106.00) --
	( 71.99,107.46) --
	( 72.41,109.04) --
	( 72.84,108.98) --
	( 73.26,112.68) --
	( 73.68,106.08) --
	( 74.10,106.08) --
	( 74.53,107.19) --
	( 74.95,107.96) --
	( 75.37,110.53) --
	( 75.79,111.13) --
	( 76.22,111.13) --
	( 76.64,110.94) --
	( 77.06,111.32) --
	( 77.48,111.83) --
	( 77.91,112.24) --
	( 78.33,112.56) --
	( 78.75,113.36) --
	( 79.17,113.44) --
	( 79.60,113.44) --
	( 80.02,118.75) --
	( 80.44,120.96) --
	( 80.86,120.96) --
	( 81.29,120.52) --
	( 81.71,120.48) --
	( 82.13,122.32) --
	( 82.55,123.54) --
	( 82.97,123.63) --
	( 83.40,124.22) --
	( 83.82,124.42) --
	( 84.24,124.42) --
	( 84.66,124.45) --
	( 85.09,123.46) --
	( 85.51,123.46) --
	( 85.93,124.24) --
	( 86.35,124.00) --
	( 86.78,124.83) --
	( 87.20,124.83) --
	( 87.62,125.11) --
	( 88.04,124.37) --
	( 88.47,124.39) --
	( 88.89,124.39) --
	( 89.31,124.59) --
	( 89.73,123.29) --
	( 90.16,127.92) --
	( 90.58,127.92) --
	( 91.00,127.92) --
	( 91.42,127.54) --
	( 91.85,127.54) --
	( 92.27,127.54) --
	( 92.69,127.54) --
	( 93.11,127.54) --
	( 93.54,127.54) --
	( 93.96,127.54) --
	( 94.38,126.35) --
	( 94.80,126.87) --
	( 95.22,126.87) --
	( 95.65,126.87) --
	( 96.07,126.53) --
	( 96.49,126.53) --
	( 96.91,126.53) --
	( 97.34,126.36) --
	( 97.76,126.36) --
	( 98.18,126.36) --
	( 98.60,126.36) --
	( 99.03,126.36) --
	( 99.45,126.75) --
	( 99.87,126.79) --
	(100.29,126.75) --
	(100.72,126.75) --
	(101.14,126.97) --
	(101.56,126.97) --
	(101.98,126.61) --
	(102.41,126.61) --
	(102.83,126.61) --
	(103.25,126.61) --
	(103.67,126.61) --
	(104.10,126.61) --
	(104.52,126.61) --
	(104.94,126.61) --
	(105.36,126.71) --
	(105.79,126.28) --
	(106.21,126.28) --
	(106.63,126.28) --
	(107.05,126.28) --
	(107.48,126.28) --
	(107.90,126.28) --
	(108.32,126.33) --
	(108.74,125.61) --
	(109.16,125.61) --
	(109.59,125.61) --
	(110.01,125.47) --
	(110.43,125.64) --
	(110.85,125.64) --
	(111.28,125.64) --
	(111.70,125.64) --
	(112.12,125.64) --
	(112.54,125.64) --
	(112.97,125.64) --
	(113.39,125.64) --
	(113.81,125.56) --
	(114.23,125.21) --
	(114.66,125.21) --
	(115.08,125.21) --
	(115.50,125.21) --
	(115.92,125.21) --
	(116.35,125.21) --
	(116.77,125.77) --
	(117.19,125.77) --
	(117.61,125.77) --
	(118.04,125.77) --
	(118.46,125.76) --
	(118.88,125.76) --
	(119.30,125.76) --
	(119.73,125.76) --
	(120.15,123.43) --
	(120.57,123.43) --
	(120.99,123.43) --
	(121.42,123.43) --
	(121.84,123.43) --
	(122.26,123.43) --
	(122.68,123.43) --
	(123.10,122.91) --
	(123.53,123.19) --
	(123.95,123.19) --
	(124.37,123.19) --
	(124.79,123.19) --
	(125.22,122.59) --
	(125.64,122.59) --
	(126.06,122.59) --
	(126.48,122.59) --
	(126.91,122.59) --
	(127.33,122.59) --
	(127.75,122.59) --
	(128.17,122.59) --
	(128.60,122.59) --
	(129.02,122.57) --
	(129.44,122.57) --
	(129.86,122.57) --
	(130.29,122.57) --
	(130.71,122.57) --
	(131.13,122.57) --
	(131.55,122.57) --
	(131.98,122.57) --
	(132.40,121.84) --
	(132.82,121.84) --
	(133.24,121.84) --
	(133.67,121.84) --
	(134.09,121.84) --
	(134.51,121.84) --
	(134.93,121.84) --
	(135.35,121.84) --
	(135.78,121.84) --
	(136.20,121.84) --
	(136.62,121.84) --
	(137.04,121.84) --
	(137.47,121.84) --
	(137.89,121.84) --
	(138.31,121.84) --
	(138.73,121.84) --
	(139.16,121.84) --
	(139.58,121.84) --
	(140.00,121.84) --
	(140.42,121.84) --
	(140.85,121.84) --
	(141.27,121.84) --
	(141.69,121.84) --
	(142.11,121.84) --
	(142.54,121.84) --
	(142.96,121.84) --
	(143.38,121.84) --
	(143.80,121.84) --
	(144.23,121.84) --
	(144.65,121.84) --
	(145.07,121.84) --
	(145.49,121.84) --
	(145.92,121.84) --
	(146.34,121.84) --
	(146.76,121.84) --
	(147.18,121.84) --
	(147.61,121.63) --
	(148.03,121.63) --
	(148.45,121.63) --
	(148.87,121.63) --
	(149.29,121.63) --
	(149.72,121.63) --
	(150.14,121.63) --
	(150.56,121.63) --
	(150.98,121.63) --
	(151.41,121.63) --
	(151.83,121.63) --
	(152.25,121.63) --
	(152.67,121.63) --
	(153.10,121.63) --
	(153.52,121.63) --
	(153.94,121.63) --
	(154.36,121.63) --
	(154.79,121.63) --
	(155.21,121.63) --
	(155.63,121.63) --
	(156.05,121.63) --
	(156.48,121.63) --
	(156.90,121.63) --
	(157.32,121.63) --
	(157.74,121.63) --
	(158.17,121.63) --
	(158.59,121.63) --
	(159.01,121.63) --
	(159.43,121.63) --
	(159.86,121.63) --
	(160.28,121.63) --
	(160.70,121.63) --
	(161.12,121.63) --
	(161.55,121.63) --
	(161.97,121.63) --
	(162.39,121.63) --
	(162.81,121.63) --
	(163.23,121.63) --
	(163.66,121.63) --
	(164.08,122.21) --
	(164.50,122.21) --
	(164.92,122.21) --
	(165.35,122.21) --
	(165.77,122.21) --
	(166.19,122.21) --
	(166.61,122.21) --
	(167.04,122.21) --
	(167.46,122.21) --
	(167.88,122.21) --
	(168.30,122.21) --
	(168.73,122.21) --
	(169.15,122.21) --
	(169.57,122.21) --
	(169.99,122.21) --
	(170.42,122.21) --
	(170.84,122.21) --
	(171.26,122.21) --
	(171.68,122.21) --
	(172.11,122.21) --
	(172.53,122.21) --
	(172.95,122.21) --
	(173.37,122.21) --
	(173.80,122.21) --
	(174.22,122.21) --
	(174.64,122.21) --
	(175.06,122.21) --
	(175.48,122.21) --
	(175.91,122.21) --
	(176.33,122.21) --
	(176.75,122.21) --
	(177.17,122.21) --
	(177.60,122.21) --
	(178.02,122.21) --
	(178.44,122.21) --
	(178.86,122.21) --
	(179.29,122.21) --
	(179.71,122.21) --
	(180.13,122.21) --
	(180.55,122.21) --
	(180.98,122.21) --
	(181.40,122.21) --
	(181.82,122.21) --
	(182.24,122.21) --
	(182.67,122.21) --
	(183.09,122.21) --
	(183.51,122.21) --
	(183.93,122.21) --
	(184.36,119.63) --
	(184.78,119.63) --
	(185.20,119.63) --
	(185.62,119.63) --
	(186.05,119.63) --
	(186.47,119.63) --
	(186.89,119.63) --
	(187.31,119.63) --
	(187.74,119.63) --
	(188.16,121.03) --
	(188.58,121.03) --
	(189.00,121.03) --
	(189.42,121.03) --
	(189.85,121.03) --
	(190.27,121.03) --
	(190.69,121.03) --
	(191.11,121.03) --
	(191.54,121.03) --
	(191.96,121.03) --
	(192.38,121.03) --
	(192.80,121.03) --
	(193.23,121.03) --
	(193.65,121.03) --
	(194.07,121.03) --
	(194.49,121.03) --
	(194.92,121.03) --
	(195.34,121.03) --
	(195.76,121.03) --
	(196.18,121.03) --
	(196.61,121.03) --
	(197.03,121.03) --
	(197.45,121.03) --
	(197.87,121.03) --
	(198.30,121.03) --
	(198.72,121.03) --
	(199.14,121.03) --
	(199.56,121.03) --
	(199.99,121.03) --
	(200.41,121.03) --
	(200.83,121.03) --
	(201.25,121.03) --
	(201.68,121.03) --
	(202.10,121.03) --
	(202.52,121.03) --
	(202.94,120.84) --
	(203.36,120.84) --
	(203.79,120.84) --
	(204.21,120.84) --
	(204.63,120.84) --
	(205.05,118.78) --
	(205.48,118.78) --
	(205.90,118.78) --
	(206.32,118.78) --
	(206.74,118.78) --
	(207.17,118.78) --
	(207.59,118.78) --
	(208.01,118.56) --
	(208.43,118.56) --
	(208.86,118.56) --
	(209.28,118.56) --
	(209.70,118.56) --
	(210.12,118.56) --
	(210.55,118.56) --
	(210.97,118.42) --
	(211.39,118.42) --
	(211.81,118.42) --
	(212.24,118.42) --
	(212.66,118.42) --
	(213.08,118.42) --
	(213.50,118.42) --
	(213.93,118.42) --
	(214.35,118.42) --
	(214.77,118.42) --
	(215.19,118.42) --
	(215.61,118.42) --
	(216.04,118.42) --
	(216.46,118.42) --
	(216.88,118.42) --
	(217.30,118.42) --
	(217.73,118.42) --
	(218.15,118.42) --
	(218.57,118.42) --
	(218.99,118.42) --
	(219.42,118.42) --
	(219.84,118.42) --
	(220.26,118.42) --
	(220.68,118.42) --
	(221.11,118.42) --
	(221.53,118.42) --
	(221.95,118.42) --
	(222.37,118.42) --
	(222.80,118.42) --
	(223.22,118.42) --
	(223.64,118.42) --
	(224.06,118.42) --
	(224.49,118.42) --
	(224.91,118.42) --
	(225.33,118.42) --
	(225.75,118.42) --
	(226.18,118.42) --
	(226.60,118.42) --
	(227.02,118.42) --
	(227.44,118.42) --
	(227.87,118.42) --
	(228.29,118.42) --
	(228.71,118.42) --
	(229.13,118.42) --
	(229.55,118.42) --
	(229.98,118.42) --
	(230.40,118.42) --
	(230.82,118.42) --
	(231.24,118.42) --
	(231.67,118.42) --
	(232.09,118.42) --
	(232.51,118.42) --
	(232.93,118.42) --
	(233.36,118.42) --
	(233.78,118.42) --
	(234.20,118.42) --
	(234.62,118.42) --
	(235.05,118.42) --
	(235.47,118.42) --
	(235.89,118.42) --
	(236.31,118.42) --
	(236.74,118.42) --
	(237.16,118.42) --
	(237.58,118.42) --
	(238.00,118.42) --
	(238.43,118.42) --
	(238.85,118.42) --
	(239.27,118.42) --
	(239.69,118.42) --
	(240.12,118.42) --
	(240.54,118.42) --
	(240.96,118.42) --
	(241.38,118.42) --
	(241.80,118.42) --
	(242.23,118.42) --
	(242.65,118.42) --
	(243.07,118.42) --
	(243.49,118.42) --
	(243.92,118.42) --
	(244.34,118.42) --
	(244.76,118.42) --
	(245.18,118.42) --
	(245.61,118.42) --
	(246.03,118.42) --
	(246.45,118.42) --
	(246.87,118.42) --
	(247.30,118.42) --
	(247.72,118.42) --
	(248.14,118.42) --
	(248.56,118.42) --
	(248.99,118.42) --
	(249.41,118.42) --
	(249.83,118.42) --
	(250.25,118.42) --
	(250.68,118.42) --
	(251.10,118.42) --
	(251.52,118.42) --
	(251.94,118.42) --
	(252.37,118.42) --
	(252.79,118.42) --
	(253.21,118.42) --
	(253.63,118.42) --
	(254.06,118.42) --
	(254.48,118.42) --
	(254.90,118.42) --
	(255.32,118.42) --
	(255.74,118.42) --
	(256.17,118.42) --
	(256.59,118.42) --
	(257.01,118.42) --
	(257.43,118.42) --
	(257.86,118.42) --
	(258.28,118.42) --
	(258.70,118.42) --
	(259.12,118.42) --
	(259.55,118.42) --
	(259.97,118.42) --
	(260.39,118.42) --
	(260.81,118.42) --
	(261.24,118.42) --
	(261.66,118.42) --
	(262.08,118.42) --
	(262.50,118.42) --
	(262.93,118.42) --
	(263.35,118.42) --
	(263.77,118.42) --
	(264.19,118.42) --
	(264.62,118.52) --
	(265.04,118.52) --
	(265.46,118.52) --
	(265.88,118.52) --
	(266.31,118.52) --
	(266.73,118.52) --
	(267.15,118.52) --
	(267.57,118.52) --
	(268.00,118.52) --
	(268.42,118.52) --
	(268.84,118.52) --
	(269.26,118.52) --
	(269.68,118.52) --
	(270.11,118.52) --
	(270.53,118.52) --
	(270.95,118.52) --
	(271.37,118.52) --
	(271.80,118.52) --
	(272.22,118.52) --
	(272.64,118.52) --
	(273.06,118.52) --
	(273.49,118.52) --
	(273.91,118.52) --
	(274.33,118.52) --
	(274.75,118.52) --
	(275.18,118.52) --
	(275.60,118.52) --
	(276.02,118.52) --
	(276.44,118.52) --
	(276.87,118.52) --
	(277.29,118.52) --
	(277.71,118.52) --
	(278.13,118.52) --
	(278.56,118.52) --
	(278.98,118.52) --
	(279.40,118.52) --
	(279.82,118.52) --
	(280.25,118.52) --
	(280.67,118.52) --
	(281.09,118.52) --
	(281.51,118.52) --
	(281.93,118.52) --
	(282.36,118.52) --
	(282.78,118.52) --
	(283.20,118.52) --
	(283.62,118.52) --
	(284.05,118.52) --
	(284.47,118.52) --
	(284.89,118.52) --
	(285.31,118.52) --
	(285.74,118.52) --
	(286.16,118.52) --
	(286.58,118.52) --
	(287.00,118.52) --
	(287.43,118.52) --
	(287.85,118.52) --
	(288.27,118.52) --
	(288.69,118.52) --
	(289.12,118.52) --
	(289.54,118.52) --
	(289.96,118.52) --
	(290.38,118.52) --
	(290.81,118.52) --
	(291.23,118.52) --
	(291.65,118.52) --
	(292.07,118.52) --
	(292.50,118.52) --
	(292.92,118.52) --
	(293.34,118.52) --
	(293.76,118.52) --
	(294.19,118.52) --
	(294.61,118.52) --
	(295.03,118.52) --
	(295.45,118.52) --
	(295.87,118.52) --
	(296.30,118.52) --
	(296.72,118.52) --
	(297.14,118.52) --
	(297.56,118.52) --
	(297.99,118.52) --
	(298.41,118.52) --
	(298.83,118.52) --
	(299.25,118.52) --
	(299.68,118.52) --
	(300.10,118.52) --
	(300.52,118.52) --
	(300.94,118.52) --
	(301.37,118.52) --
	(301.79,118.52) --
	(302.21,118.52) --
	(302.63,118.52) --
	(303.06,118.52) --
	(303.48,118.52) --
	(303.90,118.52) --
	(304.32,118.52) --
	(304.75,118.52) --
	(305.17,118.52) --
	(305.59,118.52) --
	(306.01,118.52) --
	(306.44,118.52) --
	(306.86,118.52) --
	(307.28,118.52) --
	(307.70,118.52) --
	(308.12,118.52) --
	(308.55,118.52) --
	(308.97,118.52) --
	(309.39,118.52) --
	(309.81,118.52) --
	(310.24,118.52) --
	(310.66,118.52) --
	(311.08,118.52) --
	(311.50,118.52) --
	(311.93,118.52) --
	(312.35,118.52) --
	(312.77,118.52) --
	(313.19,118.52) --
	(313.62,118.52) --
	(314.04,118.52) --
	(314.46,118.52) --
	(314.88,118.52) --
	(315.31,118.52) --
	(315.73,118.52) --
	(316.15,118.52) --
	(316.57,118.52) --
	(317.00,118.52) --
	(317.42,118.52) --
	(317.84,118.52) --
	(318.26,118.52) --
	(318.69,118.52) --
	(319.11,118.52) --
	(319.53,118.52) --
	(319.95,118.52) --
	(320.38,118.52) --
	(320.80,118.52) --
	(321.22,118.52) --
	(321.64,118.52) --
	(322.06,118.52) --
	(322.49,118.52) --
	(322.91,118.52) --
	(323.33,118.52) --
	(323.75,118.52) --
	(324.18,118.52) --
	(324.60,118.52) --
	(325.02,118.52) --
	(325.44,118.52) --
	(325.87,118.52) --
	(326.29,118.52) --
	(326.71,118.52) --
	(327.13,118.52) --
	(327.56,118.52) --
	(327.98,118.52) --
	(328.40,118.52) --
	(328.82,118.52) --
	(329.25,118.52) --
	(329.67,118.52) --
	(330.09,118.52) --
	(330.51,118.49) --
	(330.94,118.49) --
	(331.36,118.49) --
	(331.78,118.49) --
	(332.20,118.49) --
	(332.63,118.49) --
	(333.05,118.49) --
	(333.47,118.49) --
	(333.89,118.49) --
	(334.32,118.49) --
	(334.74,118.49) --
	(335.16,118.49) --
	(335.58,118.49) --
	(336.00,118.49) --
	(336.43,118.49) --
	(336.85,118.49) --
	(337.27,118.49) --
	(337.69,118.49) --
	(338.12,118.49) --
	(338.54,118.49) --
	(338.96,118.49) --
	(339.38,118.49) --
	(339.81,118.49) --
	(340.23,118.49) --
	(340.65,118.49) --
	(341.07,118.49) --
	(341.50,118.49) --
	(341.92,118.49) --
	(342.34,118.49) --
	(342.76,118.49) --
	(343.19,118.49) --
	(343.61,118.49) --
	(344.03,118.49) --
	(344.45,118.49) --
	(344.88,118.49) --
	(345.30,118.49) --
	(345.72,118.49) --
	(346.14,118.49) --
	(346.57,118.49) --
	(346.99,118.49) --
	(347.41,118.49) --
	(347.83,118.49) --
	(348.25,118.49) --
	(348.68,118.49) --
	(349.10,118.49) --
	(349.52,118.49) --
	(349.52,140.72) --
	(349.10,140.72) --
	(348.68,140.72) --
	(348.25,140.72) --
	(347.83,140.72) --
	(347.41,140.72) --
	(346.99,140.72) --
	(346.57,140.72) --
	(346.14,140.72) --
	(345.72,140.72) --
	(345.30,140.72) --
	(344.88,140.72) --
	(344.45,140.72) --
	(344.03,140.72) --
	(343.61,140.72) --
	(343.19,140.72) --
	(342.76,140.72) --
	(342.34,140.72) --
	(341.92,140.72) --
	(341.50,140.72) --
	(341.07,140.72) --
	(340.65,140.72) --
	(340.23,140.72) --
	(339.81,140.72) --
	(339.38,140.72) --
	(338.96,140.72) --
	(338.54,140.72) --
	(338.12,140.72) --
	(337.69,140.72) --
	(337.27,140.72) --
	(336.85,140.72) --
	(336.43,140.72) --
	(336.00,140.72) --
	(335.58,140.72) --
	(335.16,140.72) --
	(334.74,140.72) --
	(334.32,140.72) --
	(333.89,140.72) --
	(333.47,140.72) --
	(333.05,140.72) --
	(332.63,140.72) --
	(332.20,140.72) --
	(331.78,140.72) --
	(331.36,140.72) --
	(330.94,140.72) --
	(330.51,140.72) --
	(330.09,140.82) --
	(329.67,140.82) --
	(329.25,140.82) --
	(328.82,140.82) --
	(328.40,140.82) --
	(327.98,140.82) --
	(327.56,140.82) --
	(327.13,140.82) --
	(326.71,140.82) --
	(326.29,140.82) --
	(325.87,140.82) --
	(325.44,140.82) --
	(325.02,140.82) --
	(324.60,140.82) --
	(324.18,140.82) --
	(323.75,140.82) --
	(323.33,140.82) --
	(322.91,140.82) --
	(322.49,140.82) --
	(322.06,140.82) --
	(321.64,140.82) --
	(321.22,140.82) --
	(320.80,140.82) --
	(320.38,140.82) --
	(319.95,140.82) --
	(319.53,140.82) --
	(319.11,140.82) --
	(318.69,140.82) --
	(318.26,140.82) --
	(317.84,140.82) --
	(317.42,140.82) --
	(317.00,140.82) --
	(316.57,140.82) --
	(316.15,140.82) --
	(315.73,140.82) --
	(315.31,140.82) --
	(314.88,140.82) --
	(314.46,140.82) --
	(314.04,140.82) --
	(313.62,140.82) --
	(313.19,140.82) --
	(312.77,140.82) --
	(312.35,140.82) --
	(311.93,140.82) --
	(311.50,140.82) --
	(311.08,140.82) --
	(310.66,140.82) --
	(310.24,140.82) --
	(309.81,140.82) --
	(309.39,140.82) --
	(308.97,140.82) --
	(308.55,140.82) --
	(308.12,140.82) --
	(307.70,140.82) --
	(307.28,140.82) --
	(306.86,140.82) --
	(306.44,140.82) --
	(306.01,140.82) --
	(305.59,140.82) --
	(305.17,140.82) --
	(304.75,140.82) --
	(304.32,140.82) --
	(303.90,140.82) --
	(303.48,140.82) --
	(303.06,140.82) --
	(302.63,140.82) --
	(302.21,140.82) --
	(301.79,140.82) --
	(301.37,140.82) --
	(300.94,140.82) --
	(300.52,140.82) --
	(300.10,140.82) --
	(299.68,140.82) --
	(299.25,140.82) --
	(298.83,140.82) --
	(298.41,140.82) --
	(297.99,140.82) --
	(297.56,140.82) --
	(297.14,140.82) --
	(296.72,140.82) --
	(296.30,140.82) --
	(295.87,140.82) --
	(295.45,140.82) --
	(295.03,140.82) --
	(294.61,140.82) --
	(294.19,140.82) --
	(293.76,140.82) --
	(293.34,140.82) --
	(292.92,140.82) --
	(292.50,140.82) --
	(292.07,140.82) --
	(291.65,140.82) --
	(291.23,140.82) --
	(290.81,140.82) --
	(290.38,140.82) --
	(289.96,140.82) --
	(289.54,140.82) --
	(289.12,140.82) --
	(288.69,140.82) --
	(288.27,140.82) --
	(287.85,140.82) --
	(287.43,140.82) --
	(287.00,140.82) --
	(286.58,140.82) --
	(286.16,140.82) --
	(285.74,140.82) --
	(285.31,140.82) --
	(284.89,140.82) --
	(284.47,140.82) --
	(284.05,140.82) --
	(283.62,140.82) --
	(283.20,140.82) --
	(282.78,140.82) --
	(282.36,140.82) --
	(281.93,140.82) --
	(281.51,140.82) --
	(281.09,140.82) --
	(280.67,140.82) --
	(280.25,140.82) --
	(279.82,140.82) --
	(279.40,140.82) --
	(278.98,140.82) --
	(278.56,140.82) --
	(278.13,140.82) --
	(277.71,140.82) --
	(277.29,140.82) --
	(276.87,140.82) --
	(276.44,140.82) --
	(276.02,140.82) --
	(275.60,140.82) --
	(275.18,140.82) --
	(274.75,140.82) --
	(274.33,140.82) --
	(273.91,140.82) --
	(273.49,140.82) --
	(273.06,140.82) --
	(272.64,140.82) --
	(272.22,140.82) --
	(271.80,140.82) --
	(271.37,140.82) --
	(270.95,140.82) --
	(270.53,140.82) --
	(270.11,140.82) --
	(269.68,140.82) --
	(269.26,140.82) --
	(268.84,140.82) --
	(268.42,140.82) --
	(268.00,140.82) --
	(267.57,140.82) --
	(267.15,140.82) --
	(266.73,140.82) --
	(266.31,140.82) --
	(265.88,140.82) --
	(265.46,140.82) --
	(265.04,140.82) --
	(264.62,140.82) --
	(264.19,140.78) --
	(263.77,140.78) --
	(263.35,140.78) --
	(262.93,140.78) --
	(262.50,140.78) --
	(262.08,140.78) --
	(261.66,140.78) --
	(261.24,140.78) --
	(260.81,140.78) --
	(260.39,140.78) --
	(259.97,140.78) --
	(259.55,140.78) --
	(259.12,140.78) --
	(258.70,140.78) --
	(258.28,140.78) --
	(257.86,140.78) --
	(257.43,140.78) --
	(257.01,140.78) --
	(256.59,140.78) --
	(256.17,140.78) --
	(255.74,140.78) --
	(255.32,140.78) --
	(254.90,140.78) --
	(254.48,140.78) --
	(254.06,140.78) --
	(253.63,140.78) --
	(253.21,140.78) --
	(252.79,140.78) --
	(252.37,140.78) --
	(251.94,140.78) --
	(251.52,140.78) --
	(251.10,140.78) --
	(250.68,140.78) --
	(250.25,140.78) --
	(249.83,140.78) --
	(249.41,140.78) --
	(248.99,140.78) --
	(248.56,140.78) --
	(248.14,140.78) --
	(247.72,140.78) --
	(247.30,140.78) --
	(246.87,140.78) --
	(246.45,140.78) --
	(246.03,140.78) --
	(245.61,140.78) --
	(245.18,140.78) --
	(244.76,140.78) --
	(244.34,140.78) --
	(243.92,140.78) --
	(243.49,140.78) --
	(243.07,140.78) --
	(242.65,140.78) --
	(242.23,140.78) --
	(241.80,140.78) --
	(241.38,140.78) --
	(240.96,140.78) --
	(240.54,140.78) --
	(240.12,140.78) --
	(239.69,140.78) --
	(239.27,140.78) --
	(238.85,140.78) --
	(238.43,140.78) --
	(238.00,140.78) --
	(237.58,140.78) --
	(237.16,140.78) --
	(236.74,140.78) --
	(236.31,140.78) --
	(235.89,140.78) --
	(235.47,140.78) --
	(235.05,140.78) --
	(234.62,140.78) --
	(234.20,140.78) --
	(233.78,140.78) --
	(233.36,140.78) --
	(232.93,140.78) --
	(232.51,140.78) --
	(232.09,140.78) --
	(231.67,140.78) --
	(231.24,140.78) --
	(230.82,140.78) --
	(230.40,140.78) --
	(229.98,140.78) --
	(229.55,140.78) --
	(229.13,140.78) --
	(228.71,140.78) --
	(228.29,140.78) --
	(227.87,140.78) --
	(227.44,140.78) --
	(227.02,140.78) --
	(226.60,140.78) --
	(226.18,140.78) --
	(225.75,140.78) --
	(225.33,140.78) --
	(224.91,140.78) --
	(224.49,140.78) --
	(224.06,140.78) --
	(223.64,140.78) --
	(223.22,140.78) --
	(222.80,140.78) --
	(222.37,140.78) --
	(221.95,140.78) --
	(221.53,140.78) --
	(221.11,140.78) --
	(220.68,140.78) --
	(220.26,140.78) --
	(219.84,140.78) --
	(219.42,140.78) --
	(218.99,140.78) --
	(218.57,140.78) --
	(218.15,140.78) --
	(217.73,140.78) --
	(217.30,140.78) --
	(216.88,140.78) --
	(216.46,140.78) --
	(216.04,140.78) --
	(215.61,140.78) --
	(215.19,140.78) --
	(214.77,140.78) --
	(214.35,140.78) --
	(213.93,140.78) --
	(213.50,140.78) --
	(213.08,140.78) --
	(212.66,140.78) --
	(212.24,140.78) --
	(211.81,140.78) --
	(211.39,140.78) --
	(210.97,140.78) --
	(210.55,140.99) --
	(210.12,140.99) --
	(209.70,140.99) --
	(209.28,140.99) --
	(208.86,140.99) --
	(208.43,140.99) --
	(208.01,140.99) --
	(207.59,141.27) --
	(207.17,141.27) --
	(206.74,141.27) --
	(206.32,141.27) --
	(205.90,141.27) --
	(205.48,141.27) --
	(205.05,141.27) --
	(204.63,143.37) --
	(204.21,143.37) --
	(203.79,143.37) --
	(203.36,143.37) --
	(202.94,143.37) --
	(202.52,143.62) --
	(202.10,143.62) --
	(201.68,143.62) --
	(201.25,143.62) --
	(200.83,143.62) --
	(200.41,143.62) --
	(199.99,143.62) --
	(199.56,143.62) --
	(199.14,143.62) --
	(198.72,143.62) --
	(198.30,143.62) --
	(197.87,143.62) --
	(197.45,143.62) --
	(197.03,143.62) --
	(196.61,143.62) --
	(196.18,143.62) --
	(195.76,143.62) --
	(195.34,143.62) --
	(194.92,143.62) --
	(194.49,143.62) --
	(194.07,143.62) --
	(193.65,143.62) --
	(193.23,143.62) --
	(192.80,143.62) --
	(192.38,143.62) --
	(191.96,143.62) --
	(191.54,143.62) --
	(191.11,143.62) --
	(190.69,143.62) --
	(190.27,143.62) --
	(189.85,143.62) --
	(189.42,143.62) --
	(189.00,143.62) --
	(188.58,143.62) --
	(188.16,143.62) --
	(187.74,142.31) --
	(187.31,142.31) --
	(186.89,142.31) --
	(186.47,142.31) --
	(186.05,142.31) --
	(185.62,142.31) --
	(185.20,142.31) --
	(184.78,142.31) --
	(184.36,142.31) --
	(183.93,144.90) --
	(183.51,144.90) --
	(183.09,144.90) --
	(182.67,144.90) --
	(182.24,144.90) --
	(181.82,144.90) --
	(181.40,144.90) --
	(180.98,144.90) --
	(180.55,144.90) --
	(180.13,144.90) --
	(179.71,144.90) --
	(179.29,144.90) --
	(178.86,144.90) --
	(178.44,144.90) --
	(178.02,144.90) --
	(177.60,144.90) --
	(177.17,144.90) --
	(176.75,144.90) --
	(176.33,144.90) --
	(175.91,144.90) --
	(175.48,144.90) --
	(175.06,144.90) --
	(174.64,144.90) --
	(174.22,144.90) --
	(173.80,144.90) --
	(173.37,144.90) --
	(172.95,144.90) --
	(172.53,144.90) --
	(172.11,144.90) --
	(171.68,144.90) --
	(171.26,144.90) --
	(170.84,144.90) --
	(170.42,144.90) --
	(169.99,144.90) --
	(169.57,144.90) --
	(169.15,144.90) --
	(168.73,144.90) --
	(168.30,144.90) --
	(167.88,144.90) --
	(167.46,144.90) --
	(167.04,144.90) --
	(166.61,144.90) --
	(166.19,144.90) --
	(165.77,144.90) --
	(165.35,144.90) --
	(164.92,144.90) --
	(164.50,144.90) --
	(164.08,144.90) --
	(163.66,144.40) --
	(163.23,144.40) --
	(162.81,144.40) --
	(162.39,144.40) --
	(161.97,144.40) --
	(161.55,144.40) --
	(161.12,144.40) --
	(160.70,144.40) --
	(160.28,144.40) --
	(159.86,144.40) --
	(159.43,144.40) --
	(159.01,144.40) --
	(158.59,144.40) --
	(158.17,144.40) --
	(157.74,144.40) --
	(157.32,144.40) --
	(156.90,144.40) --
	(156.48,144.40) --
	(156.05,144.40) --
	(155.63,144.40) --
	(155.21,144.40) --
	(154.79,144.40) --
	(154.36,144.40) --
	(153.94,144.40) --
	(153.52,144.40) --
	(153.10,144.40) --
	(152.67,144.40) --
	(152.25,144.40) --
	(151.83,144.40) --
	(151.41,144.40) --
	(150.98,144.40) --
	(150.56,144.40) --
	(150.14,144.40) --
	(149.72,144.40) --
	(149.29,144.40) --
	(148.87,144.40) --
	(148.45,144.40) --
	(148.03,144.40) --
	(147.61,144.40) --
	(147.18,144.67) --
	(146.76,144.67) --
	(146.34,144.67) --
	(145.92,144.67) --
	(145.49,144.67) --
	(145.07,144.67) --
	(144.65,144.67) --
	(144.23,144.67) --
	(143.80,144.67) --
	(143.38,144.67) --
	(142.96,144.67) --
	(142.54,144.67) --
	(142.11,144.67) --
	(141.69,144.67) --
	(141.27,144.67) --
	(140.85,144.67) --
	(140.42,144.67) --
	(140.00,144.67) --
	(139.58,144.67) --
	(139.16,144.67) --
	(138.73,144.67) --
	(138.31,144.67) --
	(137.89,144.67) --
	(137.47,144.67) --
	(137.04,144.67) --
	(136.62,144.67) --
	(136.20,144.67) --
	(135.78,144.67) --
	(135.35,144.67) --
	(134.93,144.67) --
	(134.51,144.67) --
	(134.09,144.67) --
	(133.67,144.67) --
	(133.24,144.67) --
	(132.82,144.67) --
	(132.40,144.67) --
	(131.98,145.44) --
	(131.55,145.44) --
	(131.13,145.44) --
	(130.71,145.44) --
	(130.29,145.44) --
	(129.86,145.44) --
	(129.44,145.44) --
	(129.02,145.44) --
	(128.60,145.60) --
	(128.17,145.67) --
	(127.75,145.67) --
	(127.33,145.67) --
	(126.91,145.67) --
	(126.48,145.67) --
	(126.06,145.67) --
	(125.64,145.67) --
	(125.22,145.67) --
	(124.79,146.32) --
	(124.37,146.32) --
	(123.95,146.32) --
	(123.53,146.32) --
	(123.10,146.12) --
	(122.68,146.68) --
	(122.26,146.68) --
	(121.84,146.68) --
	(121.42,146.68) --
	(120.99,146.68) --
	(120.57,146.68) --
	(120.15,146.68) --
	(119.73,148.95) --
	(119.30,148.95) --
	(118.88,148.95) --
	(118.46,148.95) --
	(118.04,149.04) --
	(117.61,149.04) --
	(117.19,149.04) --
	(116.77,149.04) --
	(116.35,148.58) --
	(115.92,148.58) --
	(115.50,148.58) --
	(115.08,148.58) --
	(114.66,148.58) --
	(114.23,148.58) --
	(113.81,148.98) --
	(113.39,149.20) --
	(112.97,149.20) --
	(112.54,149.20) --
	(112.12,149.20) --
	(111.70,149.20) --
	(111.28,149.20) --
	(110.85,149.20) --
	(110.43,149.20) --
	(110.01,149.12) --
	(109.59,149.33) --
	(109.16,149.33) --
	(108.74,149.33) --
	(108.32,150.07) --
	(107.90,150.10) --
	(107.48,150.10) --
	(107.05,150.10) --
	(106.63,150.10) --
	(106.21,150.10) --
	(105.79,150.10) --
	(105.36,150.57) --
	(104.94,150.57) --
	(104.52,150.57) --
	(104.10,150.57) --
	(103.67,150.57) --
	(103.25,150.57) --
	(102.83,150.57) --
	(102.41,150.57) --
	(101.98,150.64) --
	(101.56,151.05) --
	(101.14,151.05) --
	(100.72,151.01) --
	(100.29,151.01) --
	( 99.87,151.13) --
	( 99.45,151.17) --
	( 99.03,150.98) --
	( 98.60,150.98) --
	( 98.18,150.98) --
	( 97.76,150.98) --
	( 97.34,150.98) --
	( 96.91,151.31) --
	( 96.49,151.31) --
	( 96.07,151.31) --
	( 95.65,151.71) --
	( 95.22,151.71) --
	( 94.80,151.71) --
	( 94.38,151.32) --
	( 93.96,152.49) --
	( 93.54,152.49) --
	( 93.11,152.49) --
	( 92.69,152.49) --
	( 92.27,152.49) --
	( 91.85,152.49) --
	( 91.42,152.49) --
	( 91.00,152.92) --
	( 90.58,152.92) --
	( 90.16,152.92) --
	( 89.73,148.84) --
	( 89.31,150.15) --
	( 88.89,150.15) --
	( 88.47,150.15) --
	( 88.04,150.24) --
	( 87.62,151.01) --
	( 87.20,150.96) --
	( 86.78,150.96) --
	( 86.35,150.39) --
	( 85.93,150.83) --
	( 85.51,150.54) --
	( 85.09,150.54) --
	( 84.66,151.91) --
	( 84.24,152.24) --
	( 83.82,152.24) --
	( 83.40,152.30) --
	( 82.97,152.01) --
	( 82.55,152.06) --
	( 82.13,151.21) --
	( 81.71,149.76) --
	( 81.29,149.94) --
	( 80.86,150.50) --
	( 80.44,150.50) --
	( 80.02,149.17) --
	( 79.60,144.66) --
	( 79.17,144.66) --
	( 78.75,144.75) --
	( 78.33,144.12) --
	( 77.91,143.98) --
	( 77.48,143.75) --
	( 77.06,143.42) --
	( 76.64,143.63) --
	( 76.22,144.02) --
	( 75.79,144.02) --
	( 75.37,144.72) --
	( 74.95,142.29) --
	( 74.53,142.22) --
	( 74.10,141.82) --
	( 73.68,141.82) --
	( 73.26,150.01) --
	( 72.84,147.57) --
	( 72.41,148.67) --
	( 71.99,147.80) --
	( 71.57,146.68) --
	( 71.15,146.43) --
	( 70.72,146.59) --
	( 70.30,147.50) --
	( 69.88,145.32) --
	( 69.46,145.32) --
	( 69.03,145.32) --
	( 68.61,145.19) --
	( 68.19,146.03) --
	( 67.77,146.03) --
	( 67.35,142.75) --
	( 66.92,142.75) --
	( 66.50,153.24) --
	( 66.08,153.01) --
	( 65.66,155.31) --
	( 65.23,160.10) --
	( 64.81,160.10) --
	( 64.39,157.61) --
	( 63.97,160.50) --
	( 63.54,160.50) --
	( 63.12,161.07) --
	( 62.70,176.69) --
	( 62.28,174.21) --
	( 61.85,173.91) --
	( 61.43,184.20) --
	( 61.01,193.21) --
	( 60.59,195.21) --
	( 60.16,195.21) --
	( 59.74,195.61) --
	( 59.32,195.61) --
	( 58.90,204.07) --
	( 58.47,204.07) --
	( 58.05,204.07) --
	cycle;
\definecolor{drawColor}{RGB}{0,0,0}

\path[draw=drawColor,line width= 0.4pt,dash pattern=on 7pt off 3pt ,line join=round,line cap=round] ( 42.00,129.40) -- (361.35,129.40);

\path[draw=drawColor,line width= 0.4pt,line join=round,line cap=round] ( 58.05,175.85) --
	( 58.47,175.85) --
	( 58.90,175.85) --
	( 59.32,146.95) --
	( 59.74,146.95) --
	( 60.16,155.81) --
	( 60.59,155.81) --
	( 61.01,155.89) --
	( 61.43,138.57) --
	( 61.85,124.70) --
	( 62.28,128.77) --
	( 62.70,135.69) --
	( 63.12,118.69) --
	( 63.54,121.53) --
	( 63.97,121.53) --
	( 64.39,125.16) --
	( 64.81,130.49) --
	( 65.23,130.49) --
	( 65.66,126.26) --
	( 66.08,125.40) --
	( 66.50,126.65) --
	( 66.92,117.28) --
	( 67.35,117.28) --
	( 67.77,121.71) --
	( 68.19,121.71) --
	( 68.61,122.10) --
	( 69.03,122.51) --
	( 69.46,122.51) --
	( 69.88,122.51) --
	( 70.30,125.33) --
	( 70.72,124.88) --
	( 71.15,125.41) --
	( 71.57,126.10) --
	( 71.99,127.50) --
	( 72.41,128.81) --
	( 72.84,128.19) --
	( 73.26,131.47) --
	( 73.68,123.63) --
	( 74.10,123.63) --
	( 74.53,124.43) --
	( 74.95,124.89) --
	( 75.37,127.53) --
	( 75.79,127.48) --
	( 76.22,127.48) --
	( 76.64,127.18) --
	( 77.06,127.28) --
	( 77.48,127.71) --
	( 77.91,128.05) --
	( 78.33,128.29) --
	( 78.75,129.04) --
	( 79.17,129.03) --
	( 79.60,129.03) --
	( 80.02,134.15) --
	( 80.44,135.99) --
	( 80.86,135.99) --
	( 81.29,135.46) --
	( 81.71,135.35) --
	( 82.13,137.05) --
	( 82.55,138.12) --
	( 82.97,138.13) --
	( 83.40,138.58) --
	( 83.82,138.65) --
	( 84.24,138.65) --
	( 84.66,138.48) --
	( 85.09,137.26) --
	( 85.51,137.26) --
	( 85.93,137.80) --
	( 86.35,137.45) --
	( 86.78,138.16) --
	( 87.20,138.16) --
	( 87.62,138.33) --
	( 88.04,137.55) --
	( 88.47,137.51) --
	( 88.89,137.51) --
	( 89.31,137.61) --
	( 89.73,136.26) --
	( 90.16,140.74) --
	( 90.58,140.74) --
	( 91.00,140.74) --
	( 91.42,140.32) --
	( 91.85,140.32) --
	( 92.27,140.32) --
	( 92.69,140.32) --
	( 93.11,140.32) --
	( 93.54,140.32) --
	( 93.96,140.32) --
	( 94.38,139.10) --
	( 94.80,139.57) --
	( 95.22,139.57) --
	( 95.65,139.57) --
	( 96.07,139.19) --
	( 96.49,139.19) --
	( 96.91,139.19) --
	( 97.34,138.92) --
	( 97.76,138.92) --
	( 98.18,138.92) --
	( 98.60,138.92) --
	( 99.03,138.92) --
	( 99.45,139.22) --
	( 99.87,139.22) --
	(100.29,139.13) --
	(100.72,139.13) --
	(101.14,139.27) --
	(101.56,139.27) --
	(101.98,138.87) --
	(102.41,138.83) --
	(102.83,138.83) --
	(103.25,138.83) --
	(103.67,138.83) --
	(104.10,138.83) --
	(104.52,138.83) --
	(104.94,138.83) --
	(105.36,138.88) --
	(105.79,138.42) --
	(106.21,138.42) --
	(106.63,138.42) --
	(107.05,138.42) --
	(107.48,138.42) --
	(107.90,138.42) --
	(108.32,138.43) --
	(108.74,137.68) --
	(109.16,137.68) --
	(109.59,137.68) --
	(110.01,137.50) --
	(110.43,137.62) --
	(110.85,137.62) --
	(111.28,137.62) --
	(111.70,137.62) --
	(112.12,137.62) --
	(112.54,137.62) --
	(112.97,137.62) --
	(113.39,137.62) --
	(113.81,137.47) --
	(114.23,137.08) --
	(114.66,137.08) --
	(115.08,137.08) --
	(115.50,137.08) --
	(115.92,137.08) --
	(116.35,137.08) --
	(116.77,137.60) --
	(117.19,137.60) --
	(117.61,137.60) --
	(118.04,137.60) --
	(118.46,137.55) --
	(118.88,137.55) --
	(119.30,137.55) --
	(119.73,137.55) --
	(120.15,135.19) --
	(120.57,135.19) --
	(120.99,135.19) --
	(121.42,135.19) --
	(121.84,135.19) --
	(122.26,135.19) --
	(122.68,135.19) --
	(123.10,134.64) --
	(123.53,134.89) --
	(123.95,134.89) --
	(124.37,134.89) --
	(124.79,134.89) --
	(125.22,134.25) --
	(125.64,134.25) --
	(126.06,134.25) --
	(126.48,134.25) --
	(126.91,134.25) --
	(127.33,134.25) --
	(127.75,134.25) --
	(128.17,134.25) --
	(128.60,134.20) --
	(129.02,134.12) --
	(129.44,134.12) --
	(129.86,134.12) --
	(130.29,134.12) --
	(130.71,134.12) --
	(131.13,134.12) --
	(131.55,134.12) --
	(131.98,134.12) --
	(132.40,133.35) --
	(132.82,133.35) --
	(133.24,133.35) --
	(133.67,133.35) --
	(134.09,133.35) --
	(134.51,133.35) --
	(134.93,133.35) --
	(135.35,133.35) --
	(135.78,133.35) --
	(136.20,133.35) --
	(136.62,133.35) --
	(137.04,133.35) --
	(137.47,133.35) --
	(137.89,133.35) --
	(138.31,133.35) --
	(138.73,133.35) --
	(139.16,133.35) --
	(139.58,133.35) --
	(140.00,133.35) --
	(140.42,133.35) --
	(140.85,133.35) --
	(141.27,133.35) --
	(141.69,133.35) --
	(142.11,133.35) --
	(142.54,133.35) --
	(142.96,133.35) --
	(143.38,133.35) --
	(143.80,133.35) --
	(144.23,133.35) --
	(144.65,133.35) --
	(145.07,133.35) --
	(145.49,133.35) --
	(145.92,133.35) --
	(146.34,133.35) --
	(146.76,133.35) --
	(147.18,133.35) --
	(147.61,133.10) --
	(148.03,133.10) --
	(148.45,133.10) --
	(148.87,133.10) --
	(149.29,133.10) --
	(149.72,133.10) --
	(150.14,133.10) --
	(150.56,133.10) --
	(150.98,133.10) --
	(151.41,133.10) --
	(151.83,133.10) --
	(152.25,133.10) --
	(152.67,133.10) --
	(153.10,133.10) --
	(153.52,133.10) --
	(153.94,133.10) --
	(154.36,133.10) --
	(154.79,133.10) --
	(155.21,133.10) --
	(155.63,133.10) --
	(156.05,133.10) --
	(156.48,133.10) --
	(156.90,133.10) --
	(157.32,133.10) --
	(157.74,133.10) --
	(158.17,133.10) --
	(158.59,133.10) --
	(159.01,133.10) --
	(159.43,133.10) --
	(159.86,133.10) --
	(160.28,133.10) --
	(160.70,133.10) --
	(161.12,133.10) --
	(161.55,133.10) --
	(161.97,133.10) --
	(162.39,133.10) --
	(162.81,133.10) --
	(163.23,133.10) --
	(163.66,133.10) --
	(164.08,133.65) --
	(164.50,133.65) --
	(164.92,133.65) --
	(165.35,133.65) --
	(165.77,133.65) --
	(166.19,133.65) --
	(166.61,133.65) --
	(167.04,133.65) --
	(167.46,133.65) --
	(167.88,133.65) --
	(168.30,133.65) --
	(168.73,133.65) --
	(169.15,133.65) --
	(169.57,133.65) --
	(169.99,133.65) --
	(170.42,133.65) --
	(170.84,133.65) --
	(171.26,133.65) --
	(171.68,133.65) --
	(172.11,133.65) --
	(172.53,133.65) --
	(172.95,133.65) --
	(173.37,133.65) --
	(173.80,133.65) --
	(174.22,133.65) --
	(174.64,133.65) --
	(175.06,133.65) --
	(175.48,133.65) --
	(175.91,133.65) --
	(176.33,133.65) --
	(176.75,133.65) --
	(177.17,133.65) --
	(177.60,133.65) --
	(178.02,133.65) --
	(178.44,133.65) --
	(178.86,133.65) --
	(179.29,133.65) --
	(179.71,133.65) --
	(180.13,133.65) --
	(180.55,133.65) --
	(180.98,133.65) --
	(181.40,133.65) --
	(181.82,133.65) --
	(182.24,133.65) --
	(182.67,133.65) --
	(183.09,133.65) --
	(183.51,133.65) --
	(183.93,133.65) --
	(184.36,131.00) --
	(184.78,131.00) --
	(185.20,131.00) --
	(185.62,131.00) --
	(186.05,131.00) --
	(186.47,131.00) --
	(186.89,131.00) --
	(187.31,131.00) --
	(187.74,131.00) --
	(188.16,132.39) --
	(188.58,132.39) --
	(189.00,132.39) --
	(189.42,132.39) --
	(189.85,132.39) --
	(190.27,132.39) --
	(190.69,132.39) --
	(191.11,132.39) --
	(191.54,132.39) --
	(191.96,132.39) --
	(192.38,132.39) --
	(192.80,132.39) --
	(193.23,132.39) --
	(193.65,132.39) --
	(194.07,132.39) --
	(194.49,132.39) --
	(194.92,132.39) --
	(195.34,132.39) --
	(195.76,132.39) --
	(196.18,132.39) --
	(196.61,132.39) --
	(197.03,132.39) --
	(197.45,132.39) --
	(197.87,132.39) --
	(198.30,132.39) --
	(198.72,132.39) --
	(199.14,132.39) --
	(199.56,132.39) --
	(199.99,132.39) --
	(200.41,132.39) --
	(200.83,132.39) --
	(201.25,132.39) --
	(201.68,132.39) --
	(202.10,132.39) --
	(202.52,132.39) --
	(202.94,132.16) --
	(203.36,132.16) --
	(203.79,132.16) --
	(204.21,132.16) --
	(204.63,132.16) --
	(205.05,130.04) --
	(205.48,130.04) --
	(205.90,130.04) --
	(206.32,130.04) --
	(206.74,130.04) --
	(207.17,130.04) --
	(207.59,130.04) --
	(208.01,129.78) --
	(208.43,129.78) --
	(208.86,129.78) --
	(209.28,129.78) --
	(209.70,129.78) --
	(210.12,129.78) --
	(210.55,129.78) --
	(210.97,129.61) --
	(211.39,129.61) --
	(211.81,129.61) --
	(212.24,129.61) --
	(212.66,129.61) --
	(213.08,129.61) --
	(213.50,129.61) --
	(213.93,129.61) --
	(214.35,129.61) --
	(214.77,129.61) --
	(215.19,129.61) --
	(215.61,129.61) --
	(216.04,129.61) --
	(216.46,129.61) --
	(216.88,129.61) --
	(217.30,129.61) --
	(217.73,129.61) --
	(218.15,129.61) --
	(218.57,129.61) --
	(218.99,129.61) --
	(219.42,129.61) --
	(219.84,129.61) --
	(220.26,129.61) --
	(220.68,129.61) --
	(221.11,129.61) --
	(221.53,129.61) --
	(221.95,129.61) --
	(222.37,129.61) --
	(222.80,129.61) --
	(223.22,129.61) --
	(223.64,129.61) --
	(224.06,129.61) --
	(224.49,129.61) --
	(224.91,129.61) --
	(225.33,129.61) --
	(225.75,129.61) --
	(226.18,129.61) --
	(226.60,129.61) --
	(227.02,129.61) --
	(227.44,129.61) --
	(227.87,129.61) --
	(228.29,129.61) --
	(228.71,129.61) --
	(229.13,129.61) --
	(229.55,129.61) --
	(229.98,129.61) --
	(230.40,129.61) --
	(230.82,129.61) --
	(231.24,129.61) --
	(231.67,129.61) --
	(232.09,129.61) --
	(232.51,129.61) --
	(232.93,129.61) --
	(233.36,129.61) --
	(233.78,129.61) --
	(234.20,129.61) --
	(234.62,129.61) --
	(235.05,129.61) --
	(235.47,129.61) --
	(235.89,129.61) --
	(236.31,129.61) --
	(236.74,129.61) --
	(237.16,129.61) --
	(237.58,129.61) --
	(238.00,129.61) --
	(238.43,129.61) --
	(238.85,129.61) --
	(239.27,129.61) --
	(239.69,129.61) --
	(240.12,129.61) --
	(240.54,129.61) --
	(240.96,129.61) --
	(241.38,129.61) --
	(241.80,129.61) --
	(242.23,129.61) --
	(242.65,129.61) --
	(243.07,129.61) --
	(243.49,129.61) --
	(243.92,129.61) --
	(244.34,129.61) --
	(244.76,129.61) --
	(245.18,129.61) --
	(245.61,129.61) --
	(246.03,129.61) --
	(246.45,129.61) --
	(246.87,129.61) --
	(247.30,129.61) --
	(247.72,129.61) --
	(248.14,129.61) --
	(248.56,129.61) --
	(248.99,129.61) --
	(249.41,129.61) --
	(249.83,129.61) --
	(250.25,129.61) --
	(250.68,129.61) --
	(251.10,129.61) --
	(251.52,129.61) --
	(251.94,129.61) --
	(252.37,129.61) --
	(252.79,129.61) --
	(253.21,129.61) --
	(253.63,129.61) --
	(254.06,129.61) --
	(254.48,129.61) --
	(254.90,129.61) --
	(255.32,129.61) --
	(255.74,129.61) --
	(256.17,129.61) --
	(256.59,129.61) --
	(257.01,129.61) --
	(257.43,129.61) --
	(257.86,129.61) --
	(258.28,129.61) --
	(258.70,129.61) --
	(259.12,129.61) --
	(259.55,129.61) --
	(259.97,129.61) --
	(260.39,129.61) --
	(260.81,129.61) --
	(261.24,129.61) --
	(261.66,129.61) --
	(262.08,129.61) --
	(262.50,129.61) --
	(262.93,129.61) --
	(263.35,129.61) --
	(263.77,129.61) --
	(264.19,129.61) --
	(264.62,129.67) --
	(265.04,129.67) --
	(265.46,129.67) --
	(265.88,129.67) --
	(266.31,129.67) --
	(266.73,129.67) --
	(267.15,129.67) --
	(267.57,129.67) --
	(268.00,129.67) --
	(268.42,129.67) --
	(268.84,129.67) --
	(269.26,129.67) --
	(269.68,129.67) --
	(270.11,129.67) --
	(270.53,129.67) --
	(270.95,129.67) --
	(271.37,129.67) --
	(271.80,129.67) --
	(272.22,129.67) --
	(272.64,129.67) --
	(273.06,129.67) --
	(273.49,129.67) --
	(273.91,129.67) --
	(274.33,129.67) --
	(274.75,129.67) --
	(275.18,129.67) --
	(275.60,129.67) --
	(276.02,129.67) --
	(276.44,129.67) --
	(276.87,129.67) --
	(277.29,129.67) --
	(277.71,129.67) --
	(278.13,129.67) --
	(278.56,129.67) --
	(278.98,129.67) --
	(279.40,129.67) --
	(279.82,129.67) --
	(280.25,129.67) --
	(280.67,129.67) --
	(281.09,129.67) --
	(281.51,129.67) --
	(281.93,129.67) --
	(282.36,129.67) --
	(282.78,129.67) --
	(283.20,129.67) --
	(283.62,129.67) --
	(284.05,129.67) --
	(284.47,129.67) --
	(284.89,129.67) --
	(285.31,129.67) --
	(285.74,129.67) --
	(286.16,129.67) --
	(286.58,129.67) --
	(287.00,129.67) --
	(287.43,129.67) --
	(287.85,129.67) --
	(288.27,129.67) --
	(288.69,129.67) --
	(289.12,129.67) --
	(289.54,129.67) --
	(289.96,129.67) --
	(290.38,129.67) --
	(290.81,129.67) --
	(291.23,129.67) --
	(291.65,129.67) --
	(292.07,129.67) --
	(292.50,129.67) --
	(292.92,129.67) --
	(293.34,129.67) --
	(293.76,129.67) --
	(294.19,129.67) --
	(294.61,129.67) --
	(295.03,129.67) --
	(295.45,129.67) --
	(295.87,129.67) --
	(296.30,129.67) --
	(296.72,129.67) --
	(297.14,129.67) --
	(297.56,129.67) --
	(297.99,129.67) --
	(298.41,129.67) --
	(298.83,129.67) --
	(299.25,129.67) --
	(299.68,129.67) --
	(300.10,129.67) --
	(300.52,129.67) --
	(300.94,129.67) --
	(301.37,129.67) --
	(301.79,129.67) --
	(302.21,129.67) --
	(302.63,129.67) --
	(303.06,129.67) --
	(303.48,129.67) --
	(303.90,129.67) --
	(304.32,129.67) --
	(304.75,129.67) --
	(305.17,129.67) --
	(305.59,129.67) --
	(306.01,129.67) --
	(306.44,129.67) --
	(306.86,129.67) --
	(307.28,129.67) --
	(307.70,129.67) --
	(308.12,129.67) --
	(308.55,129.67) --
	(308.97,129.67) --
	(309.39,129.67) --
	(309.81,129.67) --
	(310.24,129.67) --
	(310.66,129.67) --
	(311.08,129.67) --
	(311.50,129.67) --
	(311.93,129.67) --
	(312.35,129.67) --
	(312.77,129.67) --
	(313.19,129.67) --
	(313.62,129.67) --
	(314.04,129.67) --
	(314.46,129.67) --
	(314.88,129.67) --
	(315.31,129.67) --
	(315.73,129.67) --
	(316.15,129.67) --
	(316.57,129.67) --
	(317.00,129.67) --
	(317.42,129.67) --
	(317.84,129.67) --
	(318.26,129.67) --
	(318.69,129.67) --
	(319.11,129.67) --
	(319.53,129.67) --
	(319.95,129.67) --
	(320.38,129.67) --
	(320.80,129.67) --
	(321.22,129.67) --
	(321.64,129.67) --
	(322.06,129.67) --
	(322.49,129.67) --
	(322.91,129.67) --
	(323.33,129.67) --
	(323.75,129.67) --
	(324.18,129.67) --
	(324.60,129.67) --
	(325.02,129.67) --
	(325.44,129.67) --
	(325.87,129.67) --
	(326.29,129.67) --
	(326.71,129.67) --
	(327.13,129.67) --
	(327.56,129.67) --
	(327.98,129.67) --
	(328.40,129.67) --
	(328.82,129.67) --
	(329.25,129.67) --
	(329.67,129.67) --
	(330.09,129.67) --
	(330.51,129.61) --
	(330.94,129.61) --
	(331.36,129.61) --
	(331.78,129.61) --
	(332.20,129.61) --
	(332.63,129.61) --
	(333.05,129.61) --
	(333.47,129.61) --
	(333.89,129.61) --
	(334.32,129.61) --
	(334.74,129.61) --
	(335.16,129.61) --
	(335.58,129.61) --
	(336.00,129.61) --
	(336.43,129.61) --
	(336.85,129.61) --
	(337.27,129.61) --
	(337.69,129.61) --
	(338.12,129.61) --
	(338.54,129.61) --
	(338.96,129.61) --
	(339.38,129.61) --
	(339.81,129.61) --
	(340.23,129.61) --
	(340.65,129.61) --
	(341.07,129.61) --
	(341.50,129.61) --
	(341.92,129.61) --
	(342.34,129.61) --
	(342.76,129.61) --
	(343.19,129.61) --
	(343.61,129.61) --
	(344.03,129.61) --
	(344.45,129.61) --
	(344.88,129.61) --
	(345.30,129.61) --
	(345.72,129.61) --
	(346.14,129.61) --
	(346.57,129.61) --
	(346.99,129.61) --
	(347.41,129.61) --
	(347.83,129.61) --
	(348.25,129.61) --
	(348.68,129.61) --
	(349.10,129.61) --
	(349.52,129.61);
\end{scope}
\end{tikzpicture}
}
	\end{adjustbox}
	
	\caption[Einfluss des \gls{rmse}-Grenzwerts der \gls{ssauf} auf den Zusammenhang zwischen der Asymptote, der Steigung und dem \gls{zwert} des \gls{bist}s]{Einfluss des \gls{rmse}-Grenzwerts auf die Zusammenhänge der aus der \gls{ssauf} mit einer exponentiellen Regression abgeleiteten Aufgabenparameter ($a$) der Asymptote  und ($b$) der Steigung  mit dem \gls{zwert} des \gls{bist}s. Die durchgezogene Linie kennzeichnet den Verlauf des Zusammenhangs. Der graue Bereich beschreibt das $95\,\%$-Konfidenzintervall.}
	\label{fig:spatial_suppression_asymtote_slope_zscore}
\end{figure}











%Abschliessend zur zweiten Fragestellung kann festgehalten werden, dass die Quantifizierung der Wahrnehmungsleistungsverschlechterung durch den \gls{si} (als Differenzmass) nicht mit Nachteilen verbunden ist. Die Steigung der exponentiellen Regression liefert in Bezug auf die Vorhersage von psychometrischer Intelligenz gewissermassen dieselbe Information wie der \gls{si}. Weiter konnte die Asymptote der exponentiellen Regression psychometrische Intelligenz nicht vorhersagen. Die Analyse  der \gls{ssauf} auf manifester Ebene hat ergeben, dass weder der \gls{si} noch die aus der exponentiellen Regression abgeleiteten Aufgabenparameter mit psychometrischer Intelligenz zusammenhängen.
















\clearpage
\section{3. Fragestellung \label{sec:3Fragestellung}}

Mit der dritten Fragestellung sollte der Zusammenhang zwischen der \gls{ssauf} und psychometrischer Intelligenz auf latenter Ebene untersucht werden. 
%Mit der dritten Fragestellung sollte der prädiktive Wert der \gls{ssauf} in Bezug auf \textit{g}, der latenten Operationalisierung psychometrischer Intelligenz, untersucht werden.
Alle konfirmatorischen Faktorenanalysen wurden mit der Satorra-Bentler Maximum-Likelihood Schätzmethode \citep{Satorra1994} berechnet, weil diese bei nicht-normal\-ver\-teilten, intervallskalierten Daten empfohlen wird \citep[z.B.][]{Curran1996, Finney2006}.
Um die aus den Aufgaben extrahierten Faktoren auf latenter Ebene miteinander in Verbindung zu bringen, wurden als erstes für jede Aufgabe einzeln kongenerische Messmodelle \citep{Joereskog1971} gerechnet. Diese dem Strukturgleichungsmodell vorausgehende Prüfung der Modellannahmen erlaubte es, allfällige Fehlspezifikationen bereits auf Aufgabenebene zu erkennen.

Das kongenerische Messmodell der \gls{ssauf} (Modell 1; siehe \autoref{fig:spatial_suppression_congeneric_model}) bildete die empirischen Varianzen und Kovarianzen schlecht ab.  Der \gls{cst} zeigte eine überzufällig hohe Abweichung zwischen der theoretische und der empirischen Var\-ianz-Ko\-var\-ianz\-ma\-trix an und der \gls{cfi} und \gls{rmsea} lagen weit weg vom akzeptablen Bereich, $\upchi^2(2)=103.13$, $p<.001$, $\textnormal{CFI}=.78$, $\textnormal{RMSEA}=.53$, $\textnormal{SRMR}=.06$.

Um den \gls{gfaktor} aus dem \gls{bist} zu bilden, wurden die gemittelten \textit{z}-Werte der Operationen \gls{k}, \gls{b} und \gls{M} als Indikatoren verwendet \citep[für ein gleiches Vorgehen siehe][]{Stauffer2014}. Weil dieses kongenerische Messmodell mit drei Indikatoren genau identifiziert war, konnte es nicht getestet werden \citep[][S. 125]{Kline2011}.

\begin{figure}[htbp]
	\centering
	\begin{tikzpicture}
	[font=\sffamily, scale=2, inner sep=0pt,
	latent/.style	= {circle, draw, inner sep=0pt, minimum size=12mm},
	manifest/.style	= {rectangle, draw, inner sep=0pt, minimum width=12mm, minimum height=12mm},
	paths/.style	= {->, >=stealth, shorten >= 1pt},
	error/.style	= {circle, draw=none, fill=white, minimum size=5mm},
	covar/.style	= {<->, >=stealth, shorten >= 1pt, shorten <= 1pt}]
	
	\node at (0, 1.7)		[latent]	(sup)	{S};
	
	\node at (-1.5, 2.9)	[manifest]	(s1)	{1.8$^{\circ}$};
	\node at (-1.5, 2.1)	[manifest]	(s2)	{3.6$^{\circ}$};
	\node at (-1.5, 1.3)	[manifest]	(s3)	{5.4$^{\circ}$};
	\node at (-1.5, 0.5)	[manifest]	(s4)	{7.2$^{\circ}$};
	
	\node at (-2.3, 2.9)	[error]		(e1)	{\footnotesize .45};
	\node at (-2.3, 2.1)	[error]		(e2)	{\footnotesize .23};
	\node at (-2.3, 1.3)	[error]		(e3)	{\footnotesize .02};
	\node at (-2.3, 0.5)	[error]		(e4)	{\footnotesize .24};
	
	\draw [paths] (sup.west) -- (s1.east) node[minimum size = 4mm, draw=none, fill=white, midway] {\footnotesize .75{$^{1}$}\hphantom{$^**$}};	
	\draw [paths] (sup.west) -- (s2.east) node[minimum size = 4mm, draw=none, fill=white, midway] {\footnotesize .88{$^{***}$}};	
	\draw [paths] (sup.west) -- (s3.east) node[minimum size = 4mm, draw=none, fill=white, midway] {\footnotesize .99{$^{***}$}};	
	\draw [paths] (sup.west) -- (s4.east) node[minimum size = 4mm, draw=none, fill=white, midway] {\footnotesize .87{$^{***}$}};
	
	\draw [paths] (e1) -- (s1.west) {};
	\draw [paths] (e2) -- (s2.west) {};
	\draw [paths] (e3) -- (s3.west) {};
	\draw [paths] (e4) -- (s4.west) {};
	\end{tikzpicture}
	
	\vspace{.2cm}
	\caption[Modell 1: Kongenerisches Messmodell der \gls{ssauf}]{Modell 1: Kongenerisches Messmodell der \gls{ssauf} (\textsf{S}). Eingezeichnet sind die standardisierten Koeffizienten.\\
	$^1$Um die Identifizierung der Varianz der latenten Variable zu ermöglichen, wurde diese unstandardisierte Faktorladung auf $1$ fixiert.\\
	$^{***}p~<~.001$.}
	\label{fig:spatial_suppression_congeneric_model}
\end{figure}



\begin{figure}[htbp]
	\centering
	\begin{tikzpicture}
	[font=\sffamily, scale=2, inner sep=0pt,
	latent/.style	= {circle,draw,inner sep=0pt,minimum size=12mm},
	manifest/.style	= {rectangle,draw,inner sep=0pt,minimum width=12mm,minimum height=12mm},
	paths/.style	= {->, >=stealth, shorten >= 1pt},
	error/.style	= {circle, draw=none, fill=white, minimum size=5mm},
	covar/.style	= {<->, >=stealth, shorten >= 1pt, shorten <= 1pt}]
	
	\node at (0, 0)			[latent]	(sup)	{S};
	\node at (1.5, 0)		[latent]	(g)		{\textrm{\textit{g}}};
	
	\node at (-1.5, 1.2)	[manifest]	(s1)	{1.8$^{\circ}$};
	\node at (-1.5, 0.4)	[manifest]	(s2)	{3.6$^{\circ}$};
	\node at (-1.5, -.4)	[manifest]	(s3)	{5.4$^{\circ}$};
	\node at (-1.5, -1.2)	[manifest]	(s4)	{7.2$^{\circ}$};
	
	\node at (3, .8)		[manifest]	(k)		{K};
	\node at (3, 0)			[manifest]	(b)		{B};
	\node at (3, -.8)		[manifest]	(m)		{M};
	
	\node at (-2.3, 1.2)	[error]		(e1)	{\footnotesize .44};
	\node at (-2.3, 0.4)	[error]		(e2)	{\footnotesize .23};
	\node at (-2.3, -.4)	[error]		(e3)	{\footnotesize .02};
	\node at (-2.3, -1.2)	[error]		(e4)	{\footnotesize .24};
	
	\node at (3.8, .8)		[error]		(e9)	{\footnotesize .32};
	\node at (3.8, 0)		[error]		(e10)	{\footnotesize .46};
	\node at (3.8, -.8)		[error]		(e11)	{\footnotesize .71};
	
	\node at (1.5, 0.8)		[error]		(e12)	{\footnotesize .95};
	
	\draw [paths] (sup.west) -- (s1.east) node[minimum size = 4mm, draw=none, fill=white, midway] {\footnotesize .75{$^{1}$}\hphantom{$^**$}};
	\draw [paths] (sup.west) -- (s2.east) node[minimum size = 4mm, draw=none, fill=white, midway] {\footnotesize .88{$^{***}$}};
	\draw [paths] (sup.west) -- (s3.east) node[minimum size = 4mm, draw=none, fill=white, midway] {\footnotesize .99{$^{***}$}};
	\draw [paths] (sup.west) -- (s4.east) node[minimum size = 4mm, draw=none, fill=white, midway] {\footnotesize .87{$^{***}$}};
	
	\draw [paths] (g.east) -- (k.west) node[minimum size = 4mm, draw=none, fill=white, midway] {\footnotesize .83{$^{1}$}\hphantom{$^**$}};
	\draw [paths] (g.east) -- (b.west) node[minimum size = 4mm, draw=none, fill=white, midway] {\footnotesize .74{$^{***}$}};
	\draw [paths] (g.east) -- (m.west) node[minimum size = 4mm, draw=none, fill=white, midway] {\footnotesize .54{$^{***}$}};
	
	\draw [paths] (e1) -- (s1.west);
	\draw [paths] (e2) -- (s2.west);
	\draw [paths] (e3) -- (s3.west);
	\draw [paths] (e4) -- (s4.west);
	
	\draw [paths] (e9)  -- (k.east);
	\draw [paths] (e10) -- (b.east);
	\draw [paths] (e11) -- (m.east);
	
	\draw [paths] (e12) -- (g);
	
	\draw [paths] (sup)  -- (g.west) node[minimum size = 4mm, draw=none,fill=white,midway] {\footnotesize \hspace{.5em}--.23{$^{**}$}};
	\end{tikzpicture}
	
	\vspace{.2cm}
	\caption[Modell 2: Strukturgleichungsmodell zur Vorhersage des \gls{gfaktor}s durch die \gls{ssauf}]{Modell 2: Latenter Zusammenhang zwischen der \gls{ssauf} (\textsf{S}) und dem \gls{gfaktor} des \gls{bist}. Eingezeichnet sind die standardisierten Koeffizienten. \textsf{K} = Kapazität; \textsf{B} = Bearbeitungsgeschwindigkeit; \textsf{M} = Merkfähigkeit.\\
	$^1$Um die Identifizierung der Varianz der latenten Variable zu ermöglichen, wurde diese unstandardisierte Faktorladung auf $1$ fixiert.\\
	$^{**}p~=~.01$. $^{***}p~<~.001$.}
	\label{fig:spatial_suppression_g_model}
\end{figure} 

Trotz des schlechten kongenerischen Modell-Fits der \gls{ssauf} wurden die beiden Messmodelle in einem Strukturgleichungsmodell miteinander in Verbindung gebracht. Das theoretische Modell (Modell 2; siehe \autoref{fig:spatial_suppression_g_model}) bildete die empirischen Daten erneut schlecht ab.  Der \gls{cst} zeigte eine überzufällig hohe Abweichung zwischen der theoretische und der empirischen Var\-ianz-Ko\-var\-ianz\-ma\-trix an und der \gls{cfi} und \gls{rmsea} lagen nicht im akzeptablen Bereich, $\upchi^2(13)=123.88$, $p<.001$, $\textnormal{CFI}=.85$, $\textnormal{RMSEA}=.22$, $\textnormal{SRMR}=.06$. 
Der standardisierte Regressionskoeffizient zwischen der aus den vier Bedingungen der \gls{ssauf} extrahierten latenten Variable und dem \gls{gfaktor} aus dem \gls{bist} betrug $\upbeta~=~-.23$ ($p~=~.01$).
Die aus der \gls{ssauf} extrahierte latente Variable erklärte damit $5\,\%$ der Varianz im \gls{gfaktor}.

Abschliessend zur dritten Fragestellung kann festgehalten werden, dass sich zwischen der \gls{ssauf} und psychometrischer Intelligenz auf latenter Ebene ein geringer bis mittlerer negativer Zusammenhang zeigte.
Tiefere Faktorwerte auf der aus den vier Bedingungen der \gls{ssauf} extrahierten latenten Variablen waren also tendenziell mit höheren Faktorwerten im \gls{gfaktor} verbunden. Dieser Zusammenhang muss jedoch aufgrund des schlechten theoretischen Modells  mit Vorsicht interpretiert werden.


















\section{4. Fragestellung \label{sec:4Fragestellung}}

Mit der vierten Fragestellung sollte versucht werden, die \gls{ssauf} mit einem \gls{flm} zu beschreiben und die zwei aus der Aufgabe abgeleiteten latenten Variablen mit dem \gls{gfaktor} des \gls{bist} in Verbindung zu bringen.


\subsection{Fixed-Links-Messmodell \label{subsec:spatial_suppression_fixed_links_messmodell}}

Weil die \gls{ssauf} noch nie mit einem \gls{flm} beschrieben wurde, sind unterschiedliche Modelle getestet und miteinander verglichen worden. Bei allen berechneten Modellen wurden zwei voneinander unabhängige latente Variablen angenommen: 

Die erste latente Variable beinhaltete aufgabenrelevante Prozesse, deren Einflüsse sich über die vier Bedingungen hinweg nicht verändert haben. In den Messmodellen wurde dieser gleichbleibende Einfluss hergestellt, indem die unstandardisierten Faktorladungen aller manifesten Variablen auf den Wert 1 fixiert wurden. Diese latente Variable wird im Folgenden \textit{konstante} latente Variable genannt. 

Die zweite latente Variable beinhaltete aufgabenrelevante Prozesse, die durch die vier Bedingungen systematisch manipuliert wurden. Der unterschiedlich starke Einfluss der in der latenten Variable abgebildeten Prozesse auf die $82\,\%$-$\log_{10}$-Er\-ken\-nungs\-schwel\-len der \gls{ssauf} wurde in den Messmodellen durch sich unterscheidende unstandardisierte Faktorladungen hergestellt. Diese latente Variable wird im Folgenden \textit{dynamische} latente Variable genannt.

Die konstante latente Variable wurde in allen Messmodellen unabhängig von der dynamischen latenten Variable gehalten. Diese Unabhängigkeit der beiden extrahierten Variablen ist im Rahmen der Anwendung von \gls{flm}en üblich \citep[z.B.][]{Wagner2014, Schweizer2007, Wang2015}, weil sich dadurch die Interpretation der latenten Variablen vereinfacht.
Alle Modell-Fits der in den folgenden Paragraphen berichteten \gls{flm}e sind in \autoref{tab:spatial_suppression_fixedlinks_measurement_models} aufgeführt.

Das erste berechnete \gls{flm} (Modell 3) berücksichtige das Ergebnis der exponentiellen Regression (siehe \autoref{sec:2Fragestellung}), welches auf manifester Ebene eine Steigung von $e^{0.103x}$ ergeben hat. Die unstandardisierten Faktorladungen der dynamischen latenten Variable wurden deshalb mit diesem Parameter ($y=e^{0.103x},\,x\in\{1, 2, 3, 4\}$) gebildet. Modell 3 bildete die empirischen Varianzen und Kovarianzen der \gls{ssauf} nicht gut ab. Der \gls{cst} war hochsignifikant und der \gls{cfi}, der \gls{rmsea} und das \gls{srmr} lagen nicht im akzeptablen Bereich.

Modell 4 beachtete die Tatsache, dass die den \glspl{vp} vorgelegten Mustergrössen ($1.8^{\circ}$, $3.6^{\circ}$, $5.4^{\circ}$, $7.2^{\circ}$) ein Vielfaches von $1.8$ waren.
Die unstandardisierten Faktorladungen der dynamischen latenten Variable in Modell 4 wurden deshalb linear ansteigend ($y=x,\,x\in\{1, 2, 3, 4\}$) fixiert. Modell 4 bildete die empirischen Varianzen und Kovarianzen der \gls{ssauf} ebenfalls nicht gut ab. Der $\upchi^2$-Wert reduzierte sich im Vergleich zu Modell 3 zwar beträchtlich, war aber immer noch hochsignifikant. Die schlechte Passung des Modells wurde weiter durch einen hohen \gls{rmsea} und ein hohes \gls{srmr} angezeigt.


\begin{table}[ht]
	%\flushleft
	\centering
	\captionsetup{labelsep = none}
	\caption[Modell-Fits der Fixed-Links-Messmodelle der \gls{ssauf}]{\newline  \textit{Modell-Fits der berichteten \gls{flm}e der \gls{ssauf}. Der Ladungsverlauf bezieht sich auf die unstandardisierten Faktorladungen der dynamischen latenten Variable. Die unstandardisierten Faktorladungen der konstanten latenten Variable betrugen immer 1} \vspace{.2cm}}
	\label{tab:spatial_suppression_fixedlinks_measurement_models}
	\begin{adjustbox}{width=1\textwidth}
		\begin{threeparttable}
	%		\sisetup{table-space-text-post = $^{*}$} % not needed
	%		\sisetup{table-text-alignment=center}
			\begin{tabular}{
					l
%					S[table-format = 1.0]
					l
					S[table-format = 2.2]
					S[table-format = 1.0]
					S[table-format = <0.3, add-integer-zero=false]
					S[table-format = 0.3, add-integer-zero=false]
					S[table-format = 0.3, add-integer-zero=false]
					S[table-format = 0.3, add-integer-zero=false]
				}
				\hline
				\multicolumn{1}{c}{Modell}	& Ladungsverlauf	& 	{$\upchi^2$}	& \textit{df}	& {\textit{p}}	&	{CFI} 	&	{RMSEA}	&	{SRMR}	\\
				\hline
				3			&	$y=e^{0.103x}$	&	68.43			&	4			&	<.001		&	.861	&	.302	&	.084	\\
				4			&	$y=x$			&	22.67			&	4			&	<.001		&	.960	&	.162	&	.317	\\
				5{$^*$}		&	$y=2^x$			&	16.70			&	4			&	.001		&	.973	&	.134	&	.182	\\
				6			&	$y=\log_e x$	&	14.13			&	4			&	.007		&	.978	&	.120	&	.215	\\
				7			&	$y=x^2$			&	9.20			&	4			&	.056		&	.989	&	.086	&	.127	\\
				8			&	$y=x$			&	6.09			&	4			&	.193		&	.995	&	.054	&	.123	\\
				\hline
			\end{tabular}
	
			\begin{tablenotes}[flushleft]
				\footnotesize				% font size
				\setlength\labelsep{0pt}	% no indent on second line
				\item \textit{Anmerkungen.} Es gilt für alle Funktionen $x\in\{1,2,3,4\}$ (ausgenommen Modell 8, in welchem $x\in\{0,1,2,3\}$). $\upchi^2 =$ Satorra-Bentler \citeyearpar{Satorra1994} korrigierter $\upchi^2$-Wert; \textit{df} = Freiheitsgrade; \gls{cfi} = comparative fit index; \gls{rmsea} = root mean square error of approximation; \gls{srmr} = standardized root mean square residual.
				\item {$^*$}Das Modell konnte nicht interpretiert werden, weil die Fehlervarianz der $7.2^{\circ}$-Bedingung negativ geschätzt wurde.
			\end{tablenotes}
		\end{threeparttable}
	\end{adjustbox}
\end{table}

Nach diesen zwei Modellen, welche klare Annahmen über den Verlauf der Faktorladungen der dynamischen latenten Variable beinhalteten, wurden Verläufe von Faktorladungen gesucht, welche die empirischen Daten bestmöglich beschreiben. 
Die unstandardisierten Faktorladungen der dynamische latente Variable von Modell 5 wurden mit einer exponentiellen Funktion ($y=2^x,\,x\in\{1, 2, 3, 4\}$) bestimmt. Dieses Modell konnte nicht interpretiert werden, weil die Fehlervarianz der $7.2^{\circ}$-Bedingung negativ geschätzt wurde. 

In Modell 6 wiesen die unstandardisierten Faktorladungen der dynamischen latenten Variable einen logarithmischen Verlauf ($y=\log_{e}x,\,x\in\{1, 2, 3, 4\}$) auf. Das Modell bildete die empirischen Varianzen und Kovarianzen der \gls{ssauf} nicht adäquat ab. Zwar reduzierte sich der $\upchi^2$-Wert im Vergleich zu Modell 4 erneut, der \gls{cst} war aber immer noch signifikant. Weiter deuteten der \gls{rmsea} und das \gls{srmr} mit Werten ausserhalb des akzeptablen Bereichs auf eine schlechte Modellpassung hin.

Die unstandardisierten Faktorladungen der dynamischen latenten Variable von Modell 7 wurden mit einer quadratischen Funktion ($y=x^2,\,x\in\{1, 2, 3, 4\}$) bestimmt. Der \gls{cst} erkannte keine signifikante Abweichung zwischen der von Modell 7 implizierten und der empirischen Var\-ianz-Ko\-var\-ianz\-ma\-trix. Obwohl der \gls{cfi} im akzeptablen Bereich lag, deuteten der \gls{rmsea} und das \gls{srmr} auf eine schlechte Passung des Modells hin.

In Modell 8 (siehe \autoref{fig:spatial_suppression_fixedlinks_measurement_model}) wurden die unstandardisierten Faktorladungen der dynamischen latenten Variable erneut linear ansteigend fixiert.
Im Gegensatz zu Modell 4 wurde die Faktorladung der ersten Bedingung aber auf 0 gesetzt ($y=x,\,x\in\{0, 1, 2, 3\}$).
Verglichen mit den Modellen 3 bis 7 wich die von Modell 8 implizierte Var\-ianz-Ko\-var\-ianz\-ma\-trix am wenigsten von der empirische Var\-ianz-Ko\-var\-ianz\-ma\-trix ab. Der \gls{cst} war nicht signifikant und der \gls{cfi} und \gls{rmsea} deuteten auf eine gute Modellpassung hin. 
Das \gls{srmr}  lag nicht unter dem von \citet{Hu1999} vorgegebenen Wert von $\leq.08$, fiel aber deshalb  nicht tiefer aus, weil die beiden latenten Variablen unabhängig voneinander gehalten wurden\footnote{Gestützt wurde diese Erklärung durch der Tatsache, dass das \gls{srmr} deutlich tiefer ausfiel, als die Unabhängigkeit zwischen der konstanten latenten Variable und der dynamischen latenten Variable aufgehoben wurde, $\upchi^2(3)=1.98$, $p=.58$, $\textnormal{CFI}>.999$, $\textnormal{RMSEA}~=~.036$, $\textnormal{SRMR}~=~.023$. Die beiden latenten Variablen korrelierten in diesem Fall mit $r=-.22$ ($p=.02$). Das \gls{srmr} wurde bei der Beurteilung der folgenden Modelle deshalb nicht mehr berücksichtigt.}. 
Die Varianz der konstanten latenten Variable betrug $0.018$ ($z~=~8.45$, $p~<.001$) und die Varianz der dynamischen latenten Variable betrug $0.002$ ($z~=~5.53$, $p~<~.001$). 
Der relative Anteil dieser beiden Varianzen an der in den manifesten Variablen erklärten Varianz liess sich aufgrund der in konfirmatorischen Faktorenanalysen gegebenen multiplikativen Verknüpfung von Faktorladungen und Varianzen nicht direkt ermitteln. Um die Varianzen miteinander vergleichen zu können, wurde der Einfluss der Faktorladungen auf die Varianzen deshalb mit der Methode von  \citet{Schweizer2011a} kontrolliert. Die Skalierung der Varianzen hat ergeben, dass die konstante latente Variable 72\,\% und die dynamische latente Variable 28\,\% von der in den manifesten Variablen gemeinsamen Varianz band.

\begin{figure}[htbp]
	\centering
	
	\begin{tikzpicture}
	[font=\sffamily, scale=2, inner sep=0pt,
	latent/.style	= {circle,draw,inner sep=0pt,minimum size=12mm},
	manifest/.style	= {rectangle,draw,inner sep=0pt,minimum width=12mm,minimum height=12mm},
	paths/.style	= {->, >=stealth, shorten >= 1pt},
	error/.style	= {circle, draw=none, fill=white, minimum size=5mm},
	covar/.style	= {<->, >=stealth, shorten >= 1pt, shorten <= 1pt}]
	
	\node at (1, 2.4)		[latent]	(sk)	{S\textsubscript{kon}};
	\node at (1, 1)			[latent]	(sd)	{S\textsubscript{dyn}};
	
	\node at (-1.5, 2.9)	[manifest]	(s1)	{1.8$^{\circ}$};
	\node at (-1.5, 2.1)	[manifest]	(s2)	{3.6$^{\circ}$};
	\node at (-1.5, 1.3)	[manifest]	(s3)	{5.4$^{\circ}$};
	\node at (-1.5, 0.5)	[manifest]	(s4)	{7.2$^{\circ}$};
	
	
	\node at (-2.3, 2.9)	[error]		(e1)	{\footnotesize .10};
	\node at (-2.3, 2.1)	[error]		(e2)	{\footnotesize .10};
	\node at (-2.3, 1.3)	[error]		(e3)	{\footnotesize .05};
	\node at (-2.3, .5)		[error]		(e4)	{\footnotesize .12};
	
	\draw [paths] (sk.west) -- (s1.east) node[minimum size = 4mm, draw=none, fill=white, near start] {\footnotesize .95{$^{1}$}};
	\draw [paths] (sk.west) -- (s2.east) node[minimum size = 4mm, draw=none, fill=white, near start] {\footnotesize .90{$^{1}$}};
	\draw [paths] (sk.west) -- (s3.east) node[minimum size = 4mm, draw=none, fill=white, near start] {\footnotesize .81{$^{1}$}};
	\draw [paths] (sk.west) -- (s4.east) node[minimum size = 4mm, draw=none, fill=white, near start] {\footnotesize .66{$^{1}$}};
	
	\draw [paths] (sd.west) -- (s1.east) node[minimum size = 4mm, draw=none, fill=white, near start] {\footnotesize .00{$^{0}$}};
	\draw [paths] (sd.west) -- (s2.east) node[minimum size = 4mm, draw=none, fill=white, near start] {\footnotesize .30{$^{1}$}};
	\draw [paths] (sd.west) -- (s3.east) node[minimum size = 4mm, draw=none, fill=white, near start] {\footnotesize .54{$^{2}$}};
	\draw [paths] (sd.west) -- (s4.east) node[minimum size = 4mm, draw=none, fill=white, near start] {\footnotesize .67{$^{3}$}};
	
	\draw [paths] (e1) -- (s1.west);
	\draw [paths] (e2) -- (s2.west);
	\draw [paths] (e3) -- (s3.west);
	\draw [paths] (e4) -- (s4.west);
	
	\end{tikzpicture}
	
	\vspace{.2cm}
	\caption[Modell 8: Fixed-Links-Messmodell der \gls{ssauf}]{Modell 8: Fixed-Links-Messmodell der \gls{ssauf} (\textsf{S}). Eingezeichnet sind die standardisierten Koeffizienten. Hochgestellt sind die fixierten unstandardisierten Faktorladungen. \textsf{\textsubscript{kon}} = konstante latente Variable; \textsf{\textsubscript{dyn}} = dynamische latente Variable.}
	\label{fig:spatial_suppression_fixedlinks_measurement_model}
\end{figure} 

Im Vergleich zum kongenerischen Messmodell (Modell 1) vermochte das Fixed-Links-Messmodell (Modell 8) die empirischen Daten deutlich besser abzubilden. Die bessere Passung von Modell 8 äusserte sich im Vergleich zu Modell 1 in einem nicht-signifikanten $\upchi^2$-Wert, im akzeptablen \gls{cfi} und \gls{rmsea} sowie in zwei zusätzlichen Freiheitsgraden. Modell 8 war Modell 1 also aufgrund adäquaterer Abbildung der empirischen Daten und höherer Sparsamkeit vorzuziehen.



\subsection{Fixed-Links-Strukturgleichungsmodell}

Als nächstes wurde Modell 8 mit dem \gls{gfaktor} aus dem \gls{bist} in Verbindung gebracht (Modell 9; siehe \autoref{fig:spatial_suppression_fixedlinks_sem}).
Das Modell bildete die empirischen Varianzen und Kovarianzen gut ab. Der \gls{cst} war nicht signifikant und der \gls{cfi} und \gls{rmsea} lagen im akzeptablen Bereich, $\upchi^2(14)=19.06$, $p=.16$, $\textnormal{CFI}=.99$, $\textnormal{RMSEA}~=~.05$, $\textnormal{SRMR}~=~.09$. 
Der standardisierte Regressionskoeffizient zwischen der konstanten latenten Variable und dem \gls{gfaktor} betrug $\upbeta~=~-.25$ ($p~=~.02$). Der standardisierte Regressionskoeffizient zwischen der dynamischen latenten Variable und dem \gls{gfaktor} betrug $\upbeta~=~-.08$ ($p~=~.43$).
Gemeinsam erklärten die konstante und die dynamische latente Variable der \gls{ssauf} $7\,\%$ der Varianz im \gls{gfaktor}.

\begin{figure}[htbp]
	\centering
	\begin{adjustbox}{width=1\textwidth}
		\begin{tikzpicture}
		[font=\sffamily, scale=2, inner sep=0pt,
		latent/.style	= {circle,draw,inner sep=0pt,minimum size=12mm},
		manifest/.style	= {rectangle,draw,inner sep=0pt,minimum width=12mm,minimum height=12mm},
		paths/.style	= {->, >=stealth, shorten >= 1pt},
		error/.style	= {circle, draw=none, fill=white, minimum size=5mm},
		covar/.style	= {<->, >=stealth, shorten >= 1pt, shorten <= 1pt}]
		
		\node at (1, .7)		[latent]	(sk)	{S\textsubscript{kon}};
		\node at (1, -.7)		[latent]	(sd)	{S\textsubscript{dyn}};
		\node at (2.5, 0)		[latent]	(g)		{\textrm{\textit{g}}};
		
		\node at (-1.5, 1.2)	[manifest]	(s1)	{1.8$^{\circ}$};
		\node at (-1.5, 0.4)	[manifest]	(s2)	{3.6$^{\circ}$};
		\node at (-1.5, -.4)	[manifest]	(s3)	{5.4$^{\circ}$};
		\node at (-1.5, -1.2)	[manifest]	(s4)	{7.2$^{\circ}$};
		
		
		\node at (4, .8)		[manifest]	(k)		{K};
		\node at (4, 0)			[manifest]	(b)		{B};
		\node at (4, -.8)		[manifest]	(m)		{M};
		
		\node at (-2.3, 1.2)	[error]		(e1)	{\footnotesize .10};
		\node at (-2.3, 0.4)	[error]		(e2)	{\footnotesize .10};
		\node at (-2.3, -.4)	[error]		(e3)	{\footnotesize .05};
		\node at (-2.3, -1.2)	[error]		(e4)	{\footnotesize .12};
		
		
		
		\node at (4.8, .8)		[error]		(e9)	{\footnotesize .29};
		\node at (4.8, 0)		[error]		(e10)	{\footnotesize .47};
		\node at (4.8, -.8)		[error]		(e11)	{\footnotesize .71};
		
		\node at (2.5, 0.8)		[error]		(e12)	{\footnotesize .93};
		
		\draw [paths] (sk.west) -- (s1.east) node[minimum size = 4mm, draw=none, fill=white, near start] {\footnotesize .95{$^{1}$}};
		\draw [paths] (sk.west) -- (s2.east) node[minimum size = 4mm, draw=none, fill=white, near start] {\footnotesize .90{$^{1}$}};
		\draw [paths] (sk.west) -- (s3.east) node[minimum size = 4mm, draw=none, fill=white, near start] {\footnotesize .81{$^{1}$}};
		\draw [paths] (sk.west) -- (s4.east) node[minimum size = 4mm, draw=none, fill=white, near start] {\footnotesize .66{$^{1}$}};
		
		\draw [paths] (sd.west) -- (s1.east) node[minimum size = 4mm, draw=none, fill=white, near start] {\footnotesize .00{$^{0}$}};
		\draw [paths] (sd.west) -- (s2.east) node[minimum size = 4mm, draw=none, fill=white, near start] {\footnotesize .30{$^{1}$}};
		\draw [paths] (sd.west) -- (s3.east) node[minimum size = 4mm, draw=none, fill=white, near start] {\footnotesize .54{$^{2}$}};
		\draw [paths] (sd.west) -- (s4.east) node[minimum size = 4mm, draw=none, fill=white, near start] {\footnotesize .67{$^{3}$}};
		
		
		\draw [paths] (g.east) -- (k.west) node[minimum size = 4mm, draw=none, fill=white, midway] {\footnotesize .84{$^{1}$}\hphantom{$^**$}};	
		\draw [paths] (g.east) -- (b.west) node[minimum size = 4mm, draw=none, fill=white, midway] {\footnotesize .73{$^{***}$}};	
		\draw [paths] (g.east) -- (m.west) node[minimum size = 4mm, draw=none, fill=white, midway] {\footnotesize .54{$^{***}$}};	
		
		\draw [paths] (e1) -- (s1.west);
		\draw [paths] (e2) -- (s2.west);
		\draw [paths] (e3) -- (s3.west);
		\draw [paths] (e4) -- (s4.west);
		
		
		\draw [paths] (e9) -- (k.east);
		\draw [paths] (e10) -- (b.east);
		\draw [paths] (e11) -- (m.east);
		
		\draw [paths] (e12) -- (g.north);
		
		
		\draw [paths] (sk)  -- (g.west) node[minimum size = 4mm, draw=none, fill=white, midway] {\footnotesize --.25{$^{*}$}};
		\draw [paths] (sd)  -- (g.west) node[minimum size = 4mm, draw=none, fill=white, midway] {\footnotesize --.08\hphantom{$^{*}$}};
		
		\end{tikzpicture}
	\end{adjustbox}
	
	\vspace{.2cm}
	\caption[Modell 9: Fixed-Links-Strukturgleichungsmodell zur Vorhersage des \gls{gfaktor}s durch die \gls{ssauf}]{Modell 9: Latenter Zusammenhang zwischen dem Fixed-Links-Messmodell (Modell 8) der \gls{ssauf} (\textsf{S}) und dem \gls{gfaktor} aus dem \gls{bist}. Eingezeichnet sind die standardisierten Koeffizienten. Hochgestellt sind die fixierten unstandardisierten Faktorladungen. \textsf{\textsubscript{kon}} = konstante latente Variable; \textsf{\textsubscript{dyn}} = dynamische latente Variable. \textsf{K} = Kapazität; \textsf{B} = Bearbeitungsgeschwindigkeit; \textsf{M} = Merkfähigkeit.\\
	$^{*}p~<~.05$. $^{***}p~<~.001$.
	}
	\label{fig:spatial_suppression_fixedlinks_sem}
\end{figure} 

Im Vergleich zum herkömmlichen Strukturgleichungsmodell (Modell 2) bildete das Fixed-Links-Strukturgleichungsmodell (Modell 9) die empirischen Daten deutlich besser ab. Die bessere Passung von Modell 9 äusserte sich im Vergleich zu Modell 2 in einem nicht-signifikanten $\upchi^2$-Wert, im akzeptablen \gls{cfi} und \gls{rmsea} sowie in einem zusätzlichen Freiheitsgrad. Bezüglich der Varianzaufklärung im \gls{gfaktor} waren sich Modell 2 ($5\,\%$) und Modell 9 ($7\,\%$)  vergleichsweise ähnlich. Modell 9 war Modell 2 folglich aufgrund adäquaterer Abbildung der empirischen Daten und höherer Sparsamkeit vorzuziehen. 

Abschliessend zur vierten Fragestellung kann Folgendes festgehalten werden: Auf Messmodellebene vermochte das \gls{flm} (Modell 8) die empirischen Daten der \gls{ssauf} besser zu beschreiben als das kongenerische Messmodell (Modell 1). Auch im Zusammenhang mit dem \gls{gfaktor} war die Beschreibung der empirischen Daten mittels Fixed-Links-Strukturgleichungsmodell (Modell 9) dem herkömmlichen Strukturgleichungsmodell (Modell 2) klar überlegen. 
In Modell 9 zeigte sich zwischen der konstanten latenten Variable der \gls{ssauf} und dem \gls{gfaktor} ein (geringer bis) mittlerer negativer Zusammenhang. Tiefere Faktorwerte auf der konstanten latenten Variable waren also tendenziell mit höheren Faktorwerten im \gls{gfaktor} verbunden.
Zwischen der dynamischen latenten Variable der \gls{ssauf} und dem \gls{gfaktor} bestand ein so geringer Zusammenhang, dass er bei gewählten Irrtumswahrscheinlichkeit von $5\,\%$ nicht von 0 unterschieden werden konnte. 














\section{5. Fragestellung}

Mit der fünften Fragestellung sollte die Frage geklärt werden, ob die \gls{ssauf} zur Aufklärung individueller Intelligenzunterschiede neuartige Erklärungsmöglichkeiten bietet oder ob die \gls{ha} den Zusammenhang zwischen der \gls{ssauf} und psychometrischer Intelligenz vollständig zu erklären vermag. Geprüft wurde diese Frage auf manifester und auf latenter Ebene.

\subsection{Analyse auf manifester Ebene}

\subsubsection*{Die Vorhersage psychometrischer Intelligenz durch die Aufgabenbedingungen der Hick- und \gls{ssauf}}

Die korrelative Analyse der Aufgaben in \autoref{subsec:Zusammenhänge}  hat gezeigt, dass alle vier Bedingungen der \gls{ha} und drei von vier Bedingungen der \gls{ssauf} mit dem \gls{zwert} des \gls{bist}s zusammenhingen. Auch zwischen den Bedingungen der Aufgaben bestanden signifikante Zusammenhänge. 
Um diese Abhängigkeiten bei der Vorhersage des \gls{zwert}s zu berücksichtigen, wurden die Bedingungen in Gruppen zusammengefasst und nacheinander blockweise in eine multiple Regressionsanalyse aufgenommen.

Ausgangslage für die Beantwortung der Fragestellung bildete Modell 10 (siehe \autoref{tab:multiple_regression_all_conditions}), in welchem der \gls{zwert} des \gls{bist}s alleinig mit den vier Bedingungen der \gls{ha} vorhergesagt wurde. Die Regressionsanalyse hat ergeben, dass 
bei einer Kontrolle für die Zusammenhänge zwischen den Bedingungen keiner der Prädiktoren den \gls{zwert} signifikant vorhersagte (alle $p\textnormal{s}>.22$). Gemeinsam sagten die Prädiktoren den \gls{zwert} jedoch signifikant vorher und erklärten $9\,\%$ der Varianz im \gls{zwert}, $F(4,\,172)=4.40$, $p=.002$, $R^2=.09$.
Der Umstand, dass die einzelnen Bedingungen nicht signifikante Regressionskoeffizienten aufwiesen, das gesamte Regressionsmodell hingegen einen signifikanten Varianzanteil im \gls{zwert} erklärte, konnte durch die hohen Abhängigkeiten zwischen den Prädiktoren (Multikollinearität) erklärt werden \citep[S. 686]{Eid2013}. Während Multikollinearität die Interpretation der einzelnen Regressionskoeffizienten erschwert, ist sie bei einer reinen Prädiktion eines Kriteriums (wie sie hier vorlag) unproblematisch.


\begin{table}[tb]
	\centering
	\captionsetup{labelsep = none}
	\caption[Multiple Regression zur Vorhersage des \gls{zwert}s des \gls{bist}s durch die Bedingungen der Spa\-ti\-al-Sup\-pres\-sion- und der \gls{ha}]{\newline  \textit{Multiple Regression zur Vorhersage des \gls{zwert}s des \gls{bist}s durch die Bedingungen der \gls{ha} (Modell 10) respektive durch die Bedingungen der Hick- und der \gls{ssauf} (Modell 11)} \vspace{.2cm}}
	\label{tab:multiple_regression_all_conditions}
	\newcommand{\rowgroup}[1]{\hspace{-1em}#1}
	\newcommand\Tstrut{\rule{0pt}{2.1ex}}       % top strut http://tex.stackexchange.com/questions/65919/space-between-rows-in-a-table - not implemented!
	\begin{threeparttable}
		\begin{tabular}{
				>{\quad}
				l
				S[table-format = 1.4, add-integer-zero=false]
				S[table-format = 1.4, add-integer-zero=false]
				S[table-format = 1.2, add-integer-zero=false]
				S[table-format = 0.2, add-integer-zero=false]
				p{.001cm}
				S[table-format = 1.2, add-integer-zero=false,table-space-text-post = $^{**}$]
				S[table-format = 0.2, add-integer-zero=false]
				S[table-format = 1.2, add-integer-zero=false]
				S[table-format = 0.2, add-integer-zero=false]
				>{\centering\arraybackslash}p{1.2cm}
			}
			\hline
			
			{Prädiktor}	&	{\textit{B}}	&	{\textit{SE}(\textit{B})}	&	{$\upbeta$}	&	{$p$}	& &	{$F$}	&	{$R^2$}	& {$\Delta F$} & {$\Delta R^2$}	\\
			
			\hline
			
			\rowgroup{Modell 10}	&		&			&			&			&	&	4.40{$^{**}$}	&	.09		&					\\
			0-bit				&	0.0008	&	0.0020	&	.04		&	.70		&	&					&			&					\\
			1-bit				&	-0.0027	&	0.0022	&	-.16	&	.22		&	&					&			&					\\
			2-bit				&	-0.0008	&	0.0014	&	-.08	&	.56		&	&					&			&					\\
			2.58-bit			&	-0.0010	&	0.0010	&	-.12	&	.36		&	&					&			&					\\
			
			\rule{0pt}{4ex}			%  EXTRA vertical height
			
			\rowgroup{Modell 11}	&		&			&			&			&	&	2.71{$^{**}$}	&	.11		& 1.02	&	.02		\\
			0-bit				&	0.0018	&	0.0021	&	.10		&	.40		&	&					&			&					\\
			1-bit				&	-0.0031	&	0.0022	&	-.19	&	.16		&	&					&			&					\\
			2-bit				&	-0.0009	&	0.0014	&	-.09	&	.51		&	&					&			&					\\
			2.58-bit			&	-0.0008	&	0.0010	&	-.11	&	.41		&	&					&			&					\\
			$1.8^{\circ}$		&	-0.0536	&	0.5444	&	-.01	&	.92		&	&					&			&					\\
			$3.6^{\circ}$		&	-0.4192	&	0.7062	&	-.11	&	.55		&	&					&			&					\\
			$5.4^{\circ}$		&	-0.1077	&	0.7183	&	-.03	&	.88		&	&					&			&					\\
			$7.2^{\circ}$		&	0.0157	&	0.4532	&	.01		&	.97		&	&					&			&					\\
			\hline
		\end{tabular}
		
		\begin{tablenotes}[flushleft]
			\footnotesize				% font size
			\setlength\labelsep{0pt}	% no indent on second line
			\item \textit{Anmerkungen}. $B$ = unstandardisiertes Regressionsgewicht; $\upbeta$ = standardisiertes Regressionsgewicht; $F$ = $F$-Wert des Regressionsmodells; $R^2$ = erklärte Varianz; $\Delta F$ = $F$-Wert der Veränderung der erklärten Varianz; $\Delta R^2$ = zusätzlich erklärte Varianz.
			\item {$^{**}$}$p<.01$ (zweiseitig).
		\end{tablenotes}
	\end{threeparttable}
\end{table}

Modell 11 (siehe \autoref{tab:multiple_regression_all_conditions}) beinhaltete als Prädiktoren sowohl die Bedingungen der Hick- als auch der \gls{ssauf}. 
Erneut sagte bei einer Kontrolle für die Abhängigkeiten zwischen den acht Bedingungen keiner der Prädiktoren den \gls{zwert} signifikant vorher (alle $p\textnormal{s}>.16$). 
Zusammen aber sagten die Prädiktoren den \gls{zwert} signifikant vorher und erklärten $11\,\%$ der Varianz im \gls{zwert}, $F(8,\,168)=2.71$, $p=.008$, $R^2=.11$. 

Um zu prüfen, ob die Bedingungen der \gls{ssauf} einen inkrementellen Beitrag zur Varianzaufklärung im \gls{zwert} des \gls{bist}s leisteten, wurde der Zuwachs an erklärter Varianz im \gls{zwert} zwischen Modell 10 und Modell 11 auf Signifikanz getestet. 
Dabei hat sich ergeben, dass $\Delta R^2=.02$ kein signifikanter Zuwachs an erklärter Varianz darstellte, $F(4,\,168)=1.02$, $p=.40$.
Die Bedingungen der \gls{ssauf} haben also auf Ebene der Aufgabenbedingungen keinen inkrementellen Beitrag zur Aufklärung individueller Intelligenzunterschiede geleistet.


%Auf eine schrittweise Aufnahme in die Regressionsanalyse wurde bewusst verzichtet, weil sie zu Verzerrungen führt \citep[S. 68]{Harrell2015}.
% Multikollinearität \citep[S. 686]{Eid2013}









\subsubsection*{Die Vorhersage psychometrischer Intelligenz durch die Aufgabenparameter der Hick- und \gls{ssauf}}

Um den \gls{zwert} des \gls{bist}s mit den abgeleiteten Aufgabenparametern beider Aufgaben vorherzusagen, mussten die Aufgabenparameter der \gls{ha} noch bestimmt werden (für die Bestimmung der Aufgabenparameter der \gls{ssauf} siehe \autoref{sec:2Fragestellung}). 
Die Reaktionszeit (RZ) in einer Auswahlaufgabe (wie sie die \gls{ha} darstellt) kann gemäss \citet[S. 105]{Jensen1987a} mit der linearen Funktion $RZ=a+b\log_{2}n$ beschrieben werden, wobei $a$ durch den y-Achsenabschnitt, $b$ durch die Steigung der Regressionsgeraden und $\log_{2}\,n$ durch den Logarithmus zur Basis 2 der Anzahl Antwortalternativen ($n$) bestimmt ist.
Das Produkt $\log_{2}\,n$ wurde von \citet{Hick1952} als \textit{Bit} bezeichnet und entspricht derjenigen Menge an Information, welche die Entscheidung zwischen zwei gleich wahrscheinlichen Antwortalternativen ermöglicht\footnote{Entsprechend dieser Definition gab das \textit{Bit} den Bedingungen der \gls{ha} ihre Namen: In der 0-bit-Bedingung steht eine Antwortalternative zur Verfügung ($\log_{2}\,1=0$), in der 1-bit-Bedingung stehen zwei Antwortalternativen zur Verfügung ($\log_{2}\,2=1$), in der 2-bit-Bedingung stehen vier Antwortalternativen zur Verfügung ($\log_{2}\,4=2$) und in der 2.58-bit-Bedingung stehen sechs Antwortalternative zur Verfügung ($\log_{2}\,6=2.58$).} \citep[siehe auch][S. 27]{Jensen2006}.

Die Reaktionszeiten der \gls{ha} wurden für jede Person mit einer linearen Regression der Form $y=a+b\log_{2}n$  vorhergesagt (siehe \autoref{fig:hick_linear_model}). Deskriptive Angaben zu den daraus resultierenden Parametern, dem y-Achsenabschnitt $a$ und der Steigung $b$, sind in \autoref{tab:hick_linear_model} zu finden. Wie bei der Analyse der \gls{ssauf} wurde als Mass für die Anpassungsgüte des Modells an die Daten für jede Person der \gls{rmse} berechnet. Dabei hat sich gezeigt, dass sich ein lineares Modell zur Beschreibung der Daten für einen grossen Teil der \glspl{vp} gut eignete (siehe \autoref{fig:hick_rmse_density}). Der Median betrug $12$~ms und das dritte Quartil lag bei $19$~ms (Minimum $=0.96$~ms, Maximum $=54.23$~ms).


\begin{figure}[tb]
	\centering
	\begin{adjustbox}{width=1\textwidth}
		% Created by tikzDevice version 0.10.1 on 2016-08-16 09:19:32
% !TEX encoding = UTF-8 Unicode
\begin{tikzpicture}[x=1pt,y=1pt]
\definecolor{fillColor}{RGB}{255,255,255}
\path[use as bounding box,fill=fillColor,fill opacity=0.00] (0,0) rectangle (505.89,505.89);
\begin{scope}
\path[clip] ( 54.00, 51.00) rectangle (505.89,502.89);
\definecolor{drawColor}{RGB}{0,0,0}

\path[draw=drawColor,line width= 0.4pt,line join=round,line cap=round] (  0.00,134.98) --
	(  1.13,135.73) --
	(  2.75,136.79) --
	(  4.37,137.86) --
	(  5.99,138.92) --
	(  7.61,139.98) --
	(  9.23,141.04) --
	( 10.85,142.11) --
	( 12.47,143.17) --
	( 14.08,144.23) --
	( 15.70,145.30) --
	( 17.32,146.36) --
	( 18.94,147.42) --
	( 20.56,148.48) --
	( 22.18,149.55) --
	( 23.80,150.61) --
	( 25.41,151.67) --
	( 27.03,152.74) --
	( 28.65,153.80) --
	( 30.27,154.86) --
	( 31.89,155.92) --
	( 33.51,156.99) --
	( 35.13,158.05) --
	( 36.74,159.11) --
	( 38.36,160.18) --
	( 39.98,161.24) --
	( 41.60,162.30) --
	( 43.22,163.36) --
	( 44.84,164.43) --
	( 46.46,165.49) --
	( 48.08,166.55) --
	( 49.69,167.62) --
	( 51.31,168.68) --
	( 52.93,169.74) --
	( 54.55,170.81) --
	( 56.17,171.87) --
	( 57.79,172.93) --
	( 59.41,173.99) --
	( 61.02,175.06) --
	( 62.64,176.12) --
	( 64.26,177.18) --
	( 65.88,178.25) --
	( 67.50,179.31) --
	( 69.12,180.37) --
	( 70.74,181.43) --
	( 72.36,182.50) --
	( 73.97,183.56) --
	( 75.59,184.62) --
	( 77.21,185.69) --
	( 78.83,186.75) --
	( 80.45,187.81) --
	( 82.07,188.87) --
	( 83.69,189.94) --
	( 85.30,191.00) --
	( 86.92,192.06) --
	( 88.54,193.13) --
	( 90.16,194.19) --
	( 91.78,195.25) --
	( 93.40,196.31) --
	( 95.02,197.38) --
	( 96.64,198.44) --
	( 98.25,199.50) --
	( 99.87,200.57) --
	(101.49,201.63) --
	(103.11,202.69) --
	(104.73,203.76) --
	(106.35,204.82) --
	(107.97,205.88) --
	(109.58,206.94) --
	(111.20,208.01) --
	(112.82,209.07) --
	(114.44,210.13) --
	(116.06,211.20) --
	(117.68,212.26) --
	(119.30,213.32) --
	(120.92,214.38) --
	(122.53,215.45) --
	(124.15,216.51) --
	(125.77,217.57) --
	(127.39,218.64) --
	(129.01,219.70) --
	(130.63,220.76) --
	(132.25,221.82) --
	(133.86,222.89) --
	(135.48,223.95) --
	(137.10,225.01) --
	(138.72,226.08) --
	(140.34,227.14) --
	(141.96,228.20) --
	(143.58,229.26) --
	(145.19,230.33) --
	(146.81,231.39) --
	(148.43,232.45) --
	(150.05,233.52) --
	(151.67,234.58) --
	(153.29,235.64) --
	(154.91,236.71) --
	(156.53,237.77) --
	(158.14,238.83) --
	(159.76,239.89) --
	(161.38,240.96) --
	(163.00,242.02) --
	(164.62,243.08) --
	(166.24,244.15) --
	(167.86,245.21) --
	(169.47,246.27) --
	(171.09,247.33) --
	(172.71,248.40) --
	(174.33,249.46) --
	(175.95,250.52) --
	(177.57,251.59) --
	(179.19,252.65) --
	(180.81,253.71) --
	(182.42,254.77) --
	(184.04,255.84) --
	(185.66,256.90) --
	(187.28,257.96) --
	(188.90,259.03) --
	(190.52,260.09) --
	(192.14,261.15) --
	(193.75,262.21) --
	(195.37,263.28) --
	(196.99,264.34) --
	(198.61,265.40) --
	(200.23,266.47) --
	(201.85,267.53) --
	(203.47,268.59) --
	(205.09,269.66) --
	(206.70,270.72) --
	(208.32,271.78) --
	(209.94,272.84) --
	(211.56,273.91) --
	(213.18,274.97) --
	(214.80,276.03) --
	(216.42,277.10) --
	(218.03,278.16) --
	(219.65,279.22) --
	(221.27,280.28) --
	(222.89,281.35) --
	(224.51,282.41) --
	(226.13,283.47) --
	(227.75,284.54) --
	(229.37,285.60) --
	(230.98,286.66) --
	(232.60,287.72) --
	(234.22,288.79) --
	(235.84,289.85) --
	(237.46,290.91) --
	(239.08,291.98) --
	(240.70,293.04) --
	(242.31,294.10) --
	(243.93,295.16) --
	(245.55,296.23) --
	(247.17,297.29) --
	(248.79,298.35) --
	(250.41,299.42) --
	(252.03,300.48) --
	(253.64,301.54) --
	(255.26,302.61) --
	(256.88,303.67) --
	(258.50,304.73) --
	(260.12,305.79) --
	(261.74,306.86) --
	(263.36,307.92) --
	(264.98,308.98) --
	(266.59,310.05) --
	(268.21,311.11) --
	(269.83,312.17) --
	(271.45,313.23) --
	(273.07,314.30) --
	(274.69,315.36) --
	(276.31,316.42) --
	(277.92,317.49) --
	(279.54,318.55) --
	(281.16,319.61) --
	(282.78,320.67) --
	(284.40,321.74) --
	(286.02,322.80) --
	(287.64,323.86) --
	(289.26,324.93) --
	(290.87,325.99) --
	(292.49,327.05) --
	(294.11,328.11) --
	(295.73,329.18) --
	(297.35,330.24) --
	(298.97,331.30) --
	(300.59,332.37) --
	(302.20,333.43) --
	(303.82,334.49) --
	(305.44,335.56) --
	(307.06,336.62) --
	(308.68,337.68) --
	(310.30,338.74) --
	(311.92,339.81) --
	(313.54,340.87) --
	(315.15,341.93) --
	(316.77,343.00) --
	(318.39,344.06) --
	(320.01,345.12) --
	(321.63,346.18) --
	(323.25,347.25) --
	(324.87,348.31) --
	(326.48,349.37) --
	(328.10,350.44) --
	(329.72,351.50) --
	(331.34,352.56) --
	(332.96,353.62) --
	(334.58,354.69) --
	(336.20,355.75) --
	(337.82,356.81) --
	(339.43,357.88) --
	(341.05,358.94) --
	(342.67,360.00) --
	(344.29,361.06) --
	(345.91,362.13) --
	(347.53,363.19) --
	(349.15,364.25) --
	(350.76,365.32) --
	(352.38,366.38) --
	(354.00,367.44) --
	(355.62,368.51) --
	(357.24,369.57) --
	(358.86,370.63) --
	(360.48,371.69) --
	(362.09,372.76) --
	(363.71,373.82) --
	(365.33,374.88) --
	(366.95,375.95) --
	(368.57,377.01) --
	(370.19,378.07) --
	(371.81,379.13) --
	(373.43,380.20) --
	(375.04,381.26) --
	(376.66,382.32) --
	(378.28,383.39) --
	(379.90,384.45) --
	(381.52,385.51) --
	(383.14,386.57) --
	(384.76,387.64) --
	(386.37,388.70) --
	(387.99,389.76) --
	(389.61,390.83) --
	(391.23,391.89) --
	(392.85,392.95) --
	(394.47,394.01) --
	(396.09,395.08) --
	(397.71,396.14) --
	(399.32,397.20) --
	(400.94,398.27) --
	(402.56,399.33) --
	(404.18,400.39) --
	(405.80,401.46) --
	(407.42,402.52) --
	(409.04,403.58) --
	(410.65,404.64) --
	(412.27,405.71) --
	(413.89,406.77) --
	(415.51,407.83) --
	(417.13,408.90) --
	(418.75,409.96) --
	(420.37,411.02) --
	(421.99,412.08) --
	(423.60,413.15) --
	(425.22,414.21) --
	(426.84,415.27) --
	(428.46,416.34) --
	(430.08,417.40) --
	(431.70,418.46) --
	(433.32,419.52) --
	(434.93,420.59) --
	(436.55,421.65) --
	(438.17,422.71) --
	(439.79,423.78) --
	(441.41,424.84) --
	(443.03,425.90) --
	(444.65,426.97) --
	(446.27,428.03) --
	(447.88,429.09) --
	(449.50,430.15) --
	(451.12,431.22) --
	(452.74,432.28) --
	(454.36,433.34) --
	(455.98,434.41) --
	(457.60,435.47) --
	(459.21,436.53) --
	(460.83,437.59) --
	(462.45,438.66) --
	(464.07,439.72) --
	(465.69,440.78) --
	(467.31,441.85) --
	(468.93,442.91) --
	(470.54,443.97) --
	(472.16,445.03) --
	(473.78,446.10) --
	(475.40,447.16) --
	(477.02,448.22) --
	(478.64,449.29) --
	(480.26,450.35) --
	(481.88,451.41) --
	(483.49,452.47) --
	(485.11,453.54) --
	(486.73,454.60) --
	(488.35,455.66) --
	(489.97,456.73) --
	(491.59,457.79) --
	(493.21,458.85) --
	(494.82,459.92) --
	(496.44,460.98) --
	(498.06,462.04) --
	(499.68,463.10) --
	(501.30,464.17) --
	(502.92,465.23) --
	(504.54,466.29) --
	(505.89,467.18);
\end{scope}
\begin{scope}
\path[clip] (  0.00,  0.00) rectangle (505.89,505.89);
\definecolor{drawColor}{RGB}{0,0,0}

\node[text=drawColor,anchor=base,inner sep=0pt, outer sep=0pt, scale=  1.20] at (279.95,  5.40) {Bit (log\textsubscript{2}\,n)};

\node[text=drawColor,rotate= 90.00,anchor=base,inner sep=0pt, outer sep=0pt, scale=  1.20] at ( 15.60,276.94) {Reaktionszeit (ms)};
\end{scope}
\begin{scope}
\path[clip] (  0.00,  0.00) rectangle (505.89,505.89);
\definecolor{drawColor}{RGB}{0,0,0}

\path[draw=drawColor,line width= 0.4pt,line join=round,line cap=round] ( 70.74, 51.00) -- (489.15, 51.00);

\path[draw=drawColor,line width= 0.4pt,line join=round,line cap=round] ( 70.74, 51.00) -- ( 70.74, 45.00);

\path[draw=drawColor,line width= 0.4pt,line join=round,line cap=round] (232.60, 51.00) -- (232.60, 45.00);

\path[draw=drawColor,line width= 0.4pt,line join=round,line cap=round] (394.47, 51.00) -- (394.47, 45.00);

\path[draw=drawColor,line width= 0.4pt,line join=round,line cap=round] (489.15, 51.00) -- (489.15, 45.00);

\node[text=drawColor,anchor=base,inner sep=0pt, outer sep=0pt, scale=  1.20] at ( 70.74, 33.00) {0};

\node[text=drawColor,anchor=base,inner sep=0pt, outer sep=0pt, scale=  1.20] at (232.60, 33.00) {1};

\node[text=drawColor,anchor=base,inner sep=0pt, outer sep=0pt, scale=  1.20] at (394.47, 33.00) {2};

\node[text=drawColor,anchor=base,inner sep=0pt, outer sep=0pt, scale=  1.20] at (489.15, 33.00) {2.58};

\path[draw=drawColor,line width= 0.4pt,line join=round,line cap=round] ( 54.00, 67.74) -- ( 54.00,486.15);

\path[draw=drawColor,line width= 0.4pt,line join=round,line cap=round] ( 54.00, 67.74) -- ( 48.00, 67.74);

\path[draw=drawColor,line width= 0.4pt,line join=round,line cap=round] ( 54.00,137.47) -- ( 48.00,137.47);

\path[draw=drawColor,line width= 0.4pt,line join=round,line cap=round] ( 54.00,207.21) -- ( 48.00,207.21);

\path[draw=drawColor,line width= 0.4pt,line join=round,line cap=round] ( 54.00,276.94) -- ( 48.00,276.94);

\path[draw=drawColor,line width= 0.4pt,line join=round,line cap=round] ( 54.00,346.68) -- ( 48.00,346.68);

\path[draw=drawColor,line width= 0.4pt,line join=round,line cap=round] ( 54.00,416.42) -- ( 48.00,416.42);

\path[draw=drawColor,line width= 0.4pt,line join=round,line cap=round] ( 54.00,486.15) -- ( 48.00,486.15);

\node[text=drawColor,anchor=base east,inner sep=0pt, outer sep=0pt, scale=  1.20] at ( 45.60, 63.60) {0};

\node[text=drawColor,anchor=base east,inner sep=0pt, outer sep=0pt, scale=  1.20] at ( 45.60,133.34) {200};

\node[text=drawColor,anchor=base east,inner sep=0pt, outer sep=0pt, scale=  1.20] at ( 45.60,203.08) {250};

\node[text=drawColor,anchor=base east,inner sep=0pt, outer sep=0pt, scale=  1.20] at ( 45.60,272.81) {300};

\node[text=drawColor,anchor=base east,inner sep=0pt, outer sep=0pt, scale=  1.20] at ( 45.60,342.55) {350};

\node[text=drawColor,anchor=base east,inner sep=0pt, outer sep=0pt, scale=  1.20] at ( 45.60,412.28) {400};

\node[text=drawColor,anchor=base east,inner sep=0pt, outer sep=0pt, scale=  1.20] at ( 45.60,482.02) {450};
\end{scope}
\begin{scope}
\path[clip] (  0.00,  0.00) rectangle (505.89,505.89);
\definecolor{drawColor}{RGB}{255,255,255}
\definecolor{fillColor}{RGB}{255,255,255}

\path[draw=drawColor,line width= 0.4pt,line join=round,line cap=round,fill=fillColor] ( 50.61, 99.22) rectangle ( 57.39,105.99);
\definecolor{drawColor}{RGB}{0,0,0}

\path[draw=drawColor,line width= 0.4pt,line join=round,line cap=round] ( 50.61, 95.83) -- ( 57.39,102.60);

\path[draw=drawColor,line width= 0.4pt,line join=round,line cap=round] ( 50.61,102.60) -- ( 57.39,109.38);
\end{scope}
\begin{scope}
\path[clip] ( 54.00, 51.00) rectangle (505.89,502.89);
\definecolor{drawColor}{RGB}{0,0,0}
\definecolor{fillColor}{RGB}{0,0,0}

\path[draw=drawColor,line width= 0.4pt,line join=round,line cap=round,fill=fillColor] ( 70.74,193.27) circle (  2.25);

\path[draw=drawColor,line width= 0.4pt,line join=round,line cap=round,fill=fillColor] (232.60,271.89) circle (  2.25);

\path[draw=drawColor,line width= 0.4pt,line join=round,line cap=round,fill=fillColor] (394.47,384.35) circle (  2.25);

\path[draw=drawColor,line width= 0.4pt,line join=round,line cap=round,fill=fillColor] (489.15,469.70) circle (  2.25);

\path[draw=drawColor,line width= 0.4pt,line join=round,line cap=round] ( 70.74,190.24) -- ( 70.74,196.30);

\path[draw=drawColor,line width= 0.4pt,line join=round,line cap=round] ( 67.12,190.24) --
	( 70.74,190.24) --
	( 74.35,190.24);

\path[draw=drawColor,line width= 0.4pt,line join=round,line cap=round] ( 74.35,196.30) --
	( 70.74,196.30) --
	( 67.12,196.30);

\path[draw=drawColor,line width= 0.4pt,line join=round,line cap=round] (232.60,268.58) -- (232.60,275.19);

\path[draw=drawColor,line width= 0.4pt,line join=round,line cap=round] (228.99,268.58) --
	(232.60,268.58) --
	(236.22,268.58);

\path[draw=drawColor,line width= 0.4pt,line join=round,line cap=round] (236.22,275.19) --
	(232.60,275.19) --
	(228.99,275.19);

\path[draw=drawColor,line width= 0.4pt,line join=round,line cap=round] (394.47,378.74) -- (394.47,389.97);

\path[draw=drawColor,line width= 0.4pt,line join=round,line cap=round] (390.85,378.74) --
	(394.47,378.74) --
	(398.08,378.74);

\path[draw=drawColor,line width= 0.4pt,line join=round,line cap=round] (398.08,389.97) --
	(394.47,389.97) --
	(390.85,389.97);

\path[draw=drawColor,line width= 0.4pt,line join=round,line cap=round] (489.15,462.64) -- (489.15,476.76);

\path[draw=drawColor,line width= 0.4pt,line join=round,line cap=round] (485.54,462.64) --
	(489.15,462.64) --
	(492.77,462.64);

\path[draw=drawColor,line width= 0.4pt,line join=round,line cap=round] (492.77,476.76) --
	(489.15,476.76) --
	(485.54,476.76);
\end{scope}
\end{tikzpicture}

	\end{adjustbox}
	\caption[Lineares Modell zur Vorhersage der Reaktionszeit durch das Bit der \gls{ha}]{Linearer Einfluss des Bits auf die Reaktionszeit in der \gls{ha}. Eingezeichnet sind die Mittelwerte $\pm$ Standardfehler der Mittelwerte. $y = 232 + 76\log_{2}\,n$. n = Anzahl Antwortalternativen.}
	\label{fig:hick_linear_model}
\end{figure}

\begin{table}[htbp]
	\centering
	\captionsetup{labelsep = none}
	\caption[Deskriptive Angaben zur linearen Regression für die Vorhersage der Reaktionszeiten durch die Bits der \gls{ha}]{\newline  \textit{Deskriptive Angaben zur linearen Regression ($y=a + b\log_{2}\,n$) für die Vorhersage der Reaktionszeiten durch die Anzahl Antwortalternativen n der \gls{ha} und Kennwerte zur Verteilungsform der Daten} \vspace{.2cm}}
	\label{tab:hick_linear_model}
		\begin{threeparttable}
			\begin{tabular}{
					l
					S[table-format = 3.0]
					S[table-format = 2.0]
					S[table-format = 3.0]
					S[table-format = 3.0]
					S[table-format = 1.2]
					S[table-format = 1.2]
					S[table-format = <0.3, add-integer-zero=false]
				}
			\hline
			\multicolumn{1}{c}{Parameter}	& 	\textit{M}	& \textit{SD}	&	{Min}	&	{Max} 	&	{Schiefe}	&	{Kurtosis} & {S-W \textit{p}-Wert}	\\
			\hline
			$a$			&	232			&	28			&	168		&	347		&	1.18		&	2.95		& 		<.001			\\
			$b$			&	76			&	22			&	33		&	142		&	0.53		&	-0.12		& 		.003			\\
			\hline
			\end{tabular}

			\begin{tablenotes}[flushleft]
				\footnotesize				% font size
				\setlength\labelsep{0pt}	% no indent on second line
				\item \textit{Anmerkungen.} \textit{a}~=~y-Achsenabschnitt (in ms); \textit{b}~=~Steigung; Min~=~Minimum; Max~=Maximum; S-W~= Shapiro-Wilk-Test.
			\end{tablenotes}
		\end{threeparttable}
\end{table}

%Über alle \glspl{vp} gemittelt betrug $R^2=.96$, wobei im Gegensatz zur regressionsanalytischen Analyse der \gls{ssauf} (siehe \autoref{sec:2Fragestellung}) geringere individuelle Variabilität (\gls{sd} $=.04$, Min = .732, Max = .999) bestand.

\begin{figure}[t]
	\centering
	%	\captionsetup{font = small}
	\begin{adjustbox}{width=1\textwidth}
		% Created by tikzDevice version 0.10.1 on 2016-09-01 13:39:19
% !TEX encoding = UTF-8 Unicode
\begin{tikzpicture}[x=1pt,y=1pt]
\definecolor{fillColor}{RGB}{255,255,255}
\path[use as bounding box,fill=fillColor,fill opacity=0.00] (0,0) rectangle (361.35,144.54);
\begin{scope}
\path[clip] ( 42.00, 48.00) rectangle (361.35,138.54);
\definecolor{drawColor}{RGB}{0,0,0}

\path[draw=drawColor,line width= 0.4pt,line join=round,line cap=round] ( 49.01, 51.38) --
	( 49.49, 51.39) --
	( 49.96, 51.41) --
	( 50.44, 51.44) --
	( 50.91, 51.48) --
	( 51.39, 51.54) --
	( 51.86, 51.62) --
	( 52.34, 51.73) --
	( 52.81, 51.86) --
	( 53.29, 52.03) --
	( 53.76, 52.23) --
	( 54.24, 52.47) --
	( 54.71, 52.73) --
	( 55.19, 53.03) --
	( 55.66, 53.34) --
	( 56.14, 53.67) --
	( 56.61, 54.00) --
	( 57.09, 54.32) --
	( 57.56, 54.64) --
	( 58.04, 54.94) --
	( 58.51, 55.22) --
	( 58.99, 55.49) --
	( 59.46, 55.77) --
	( 59.94, 56.06) --
	( 60.41, 56.39) --
	( 60.89, 56.78) --
	( 61.36, 57.22) --
	( 61.84, 57.74) --
	( 62.31, 58.34) --
	( 62.79, 59.00) --
	( 63.26, 59.72) --
	( 63.74, 60.47) --
	( 64.21, 61.22) --
	( 64.69, 61.94) --
	( 65.16, 62.58) --
	( 65.64, 63.14) --
	( 66.12, 63.59) --
	( 66.59, 63.93) --
	( 67.07, 64.14) --
	( 67.54, 64.24) --
	( 68.02, 64.25) --
	( 68.49, 64.18) --
	( 68.97, 64.06) --
	( 69.44, 63.93) --
	( 69.92, 63.82) --
	( 70.39, 63.78) --
	( 70.87, 63.82) --
	( 71.34, 63.98) --
	( 71.82, 64.27) --
	( 72.29, 64.71) --
	( 72.77, 65.30) --
	( 73.24, 66.03) --
	( 73.72, 66.88) --
	( 74.19, 67.81) --
	( 74.67, 68.78) --
	( 75.14, 69.74) --
	( 75.62, 70.65) --
	( 76.09, 71.47) --
	( 76.57, 72.16) --
	( 77.04, 72.71) --
	( 77.52, 73.10) --
	( 77.99, 73.33) --
	( 78.47, 73.42) --
	( 78.94, 73.39) --
	( 79.42, 73.26) --
	( 79.89, 73.06) --
	( 80.37, 72.84) --
	( 80.84, 72.65) --
	( 81.32, 72.53) --
	( 81.79, 72.51) --
	( 82.27, 72.64) --
	( 82.74, 72.92) --
	( 83.22, 73.39) --
	( 83.69, 74.05) --
	( 84.17, 74.87) --
	( 84.64, 75.85) --
	( 85.12, 76.95) --
	( 85.59, 78.13) --
	( 86.07, 79.32) --
	( 86.54, 80.49) --
	( 87.02, 81.61) --
	( 87.49, 82.64) --
	( 87.97, 83.56) --
	( 88.44, 84.36) --
	( 88.92, 85.03) --
	( 89.39, 85.58) --
	( 89.87, 86.02) --
	( 90.34, 86.34) --
	( 90.82, 86.56) --
	( 91.29, 86.71) --
	( 91.77, 86.80) --
	( 92.24, 86.86) --
	( 92.72, 86.92) --
	( 93.19, 87.00) --
	( 93.67, 87.14) --
	( 94.14, 87.36) --
	( 94.62, 87.69) --
	( 95.09, 88.13) --
	( 95.57, 88.68) --
	( 96.04, 89.32) --
	( 96.52, 90.01) --
	( 97.00, 90.71) --
	( 97.47, 91.35) --
	( 97.95, 91.90) --
	( 98.42, 92.31) --
	( 98.90, 92.56) --
	( 99.37, 92.62) --
	( 99.85, 92.50) --
	(100.32, 92.20) --
	(100.80, 91.74) --
	(101.27, 91.12) --
	(101.75, 90.38) --
	(102.22, 89.52) --
	(102.70, 88.56) --
	(103.17, 87.50) --
	(103.65, 86.36) --
	(104.12, 85.14) --
	(104.60, 83.88) --
	(105.07, 82.59) --
	(105.55, 81.31) --
	(106.02, 80.09) --
	(106.50, 78.98) --
	(106.97, 78.02) --
	(107.45, 77.23) --
	(107.92, 76.63) --
	(108.40, 76.21) --
	(108.87, 75.94) --
	(109.35, 75.79) --
	(109.82, 75.71) --
	(110.30, 75.63) --
	(110.77, 75.51) --
	(111.25, 75.31) --
	(111.72, 74.97) --
	(112.20, 74.49) --
	(112.67, 73.88) --
	(113.15, 73.18) --
	(113.62, 72.42) --
	(114.10, 71.65) --
	(114.57, 70.94) --
	(115.05, 70.33) --
	(115.52, 69.87) --
	(116.00, 69.59) --
	(116.47, 69.51) --
	(116.95, 69.65) --
	(117.42, 69.99) --
	(117.90, 70.51) --
	(118.37, 71.19) --
	(118.85, 71.99) --
	(119.32, 72.89) --
	(119.80, 73.86) --
	(120.27, 74.89) --
	(120.75, 75.93) --
	(121.22, 76.98) --
	(121.70, 77.99) --
	(122.17, 78.95) --
	(122.65, 79.79) --
	(123.12, 80.49) --
	(123.60, 81.01) --
	(124.07, 81.32) --
	(124.55, 81.40) --
	(125.02, 81.22) --
	(125.50, 80.79) --
	(125.97, 80.10) --
	(126.45, 79.16) --
	(126.93, 78.02) --
	(127.40, 76.69) --
	(127.88, 75.23) --
	(128.35, 73.67) --
	(128.83, 72.09) --
	(129.30, 70.54) --
	(129.78, 69.06) --
	(130.25, 67.70) --
	(130.73, 66.49) --
	(131.20, 65.47) --
	(131.68, 64.66) --
	(132.15, 64.06) --
	(132.63, 63.69) --
	(133.10, 63.54) --
	(133.58, 63.60) --
	(134.05, 63.83) --
	(134.53, 64.19) --
	(135.00, 64.66) --
	(135.48, 65.21) --
	(135.95, 65.80) --
	(136.43, 66.40) --
	(136.90, 66.99) --
	(137.38, 67.54) --
	(137.85, 68.03) --
	(138.33, 68.43) --
	(138.80, 68.73) --
	(139.28, 68.91) --
	(139.75, 68.97) --
	(140.23, 68.90) --
	(140.70, 68.70) --
	(141.18, 68.39) --
	(141.65, 67.96) --
	(142.13, 67.44) --
	(142.60, 66.85) --
	(143.08, 66.22) --
	(143.55, 65.56) --
	(144.03, 64.91) --
	(144.50, 64.29) --
	(144.98, 63.71) --
	(145.45, 63.18) --
	(145.93, 62.70) --
	(146.40, 62.30) --
	(146.88, 61.95) --
	(147.35, 61.67) --
	(147.83, 61.45) --
	(148.30, 61.28) --
	(148.78, 61.15) --
	(149.25, 61.04) --
	(149.73, 60.95) --
	(150.20, 60.85) --
	(150.68, 60.73) --
	(151.15, 60.59) --
	(151.63, 60.42) --
	(152.10, 60.23) --
	(152.58, 60.02) --
	(153.05, 59.81) --
	(153.53, 59.63) --
	(154.00, 59.49) --
	(154.48, 59.43) --
	(154.95, 59.46) --
	(155.43, 59.60) --
	(155.90, 59.84) --
	(156.38, 60.17) --
	(156.86, 60.58) --
	(157.33, 61.06) --
	(157.81, 61.57) --
	(158.28, 62.10) --
	(158.76, 62.62) --
	(159.23, 63.11) --
	(159.71, 63.56) --
	(160.18, 63.93) --
	(160.66, 64.24) --
	(161.13, 64.45) --
	(161.61, 64.58) --
	(162.08, 64.62) --
	(162.56, 64.56) --
	(163.03, 64.40) --
	(163.51, 64.14) --
	(163.98, 63.80) --
	(164.46, 63.37) --
	(164.93, 62.87) --
	(165.41, 62.31) --
	(165.88, 61.70) --
	(166.36, 61.06) --
	(166.83, 60.41) --
	(167.31, 59.75) --
	(167.78, 59.10) --
	(168.26, 58.47) --
	(168.73, 57.86) --
	(169.21, 57.28) --
	(169.68, 56.74) --
	(170.16, 56.23) --
	(170.63, 55.76) --
	(171.11, 55.33) --
	(171.58, 54.93) --
	(172.06, 54.55) --
	(172.53, 54.20) --
	(173.01, 53.86) --
	(173.48, 53.54) --
	(173.96, 53.24) --
	(174.43, 52.96) --
	(174.91, 52.70) --
	(175.38, 52.46) --
	(175.86, 52.25) --
	(176.33, 52.07) --
	(176.81, 51.91) --
	(177.28, 51.79) --
	(177.76, 51.70) --
	(178.23, 51.64) --
	(178.71, 51.60) --
	(179.18, 51.59) --
	(179.66, 51.61) --
	(180.13, 51.66) --
	(180.61, 51.73) --
	(181.08, 51.83) --
	(181.56, 51.95) --
	(182.03, 52.10) --
	(182.51, 52.27) --
	(182.98, 52.44) --
	(183.46, 52.62) --
	(183.93, 52.80) --
	(184.41, 52.96) --
	(184.88, 53.11) --
	(185.36, 53.23) --
	(185.83, 53.32) --
	(186.31, 53.39) --
	(186.78, 53.42) --
	(187.26, 53.44) --
	(187.74, 53.44) --
	(188.21, 53.44) --
	(188.69, 53.43) --
	(189.16, 53.43) --
	(189.64, 53.43) --
	(190.11, 53.44) --
	(190.59, 53.44) --
	(191.06, 53.44) --
	(191.54, 53.42) --
	(192.01, 53.37) --
	(192.49, 53.30) --
	(192.96, 53.20) --
	(193.44, 53.08) --
	(193.91, 52.93) --
	(194.39, 52.76) --
	(194.86, 52.58) --
	(195.34, 52.40) --
	(195.81, 52.23) --
	(196.29, 52.06) --
	(196.76, 51.92) --
	(197.24, 51.80) --
	(197.71, 51.70) --
	(198.19, 51.63) --
	(198.66, 51.58) --
	(199.14, 51.55) --
	(199.61, 51.55) --
	(200.09, 51.57) --
	(200.56, 51.62) --
	(201.04, 51.70) --
	(201.51, 51.80) --
	(201.99, 51.94) --
	(202.46, 52.10) --
	(202.94, 52.30) --
	(203.41, 52.52) --
	(203.89, 52.78) --
	(204.36, 53.05) --
	(204.84, 53.34) --
	(205.31, 53.64) --
	(205.79, 53.95) --
	(206.26, 54.25) --
	(206.74, 54.55) --
	(207.21, 54.84) --
	(207.69, 55.11) --
	(208.16, 55.34) --
	(208.64, 55.54) --
	(209.11, 55.70) --
	(209.59, 55.81) --
	(210.06, 55.87) --
	(210.54, 55.88) --
	(211.01, 55.83) --
	(211.49, 55.73) --
	(211.96, 55.60) --
	(212.44, 55.43) --
	(212.91, 55.24) --
	(213.39, 55.03) --
	(213.86, 54.83) --
	(214.34, 54.64) --
	(214.81, 54.45) --
	(215.29, 54.28) --
	(215.76, 54.12) --
	(216.24, 53.97) --
	(216.71, 53.83) --
	(217.19, 53.69) --
	(217.67, 53.57) --
	(218.14, 53.46) --
	(218.62, 53.38) --
	(219.09, 53.31) --
	(219.57, 53.28) --
	(220.04, 53.27) --
	(220.52, 53.29) --
	(220.99, 53.34) --
	(221.47, 53.40) --
	(221.94, 53.47) --
	(222.42, 53.55) --
	(222.89, 53.63) --
	(223.37, 53.70) --
	(223.84, 53.75) --
	(224.32, 53.80) --
	(224.79, 53.83) --
	(225.27, 53.84) --
	(225.74, 53.84) --
	(226.22, 53.83) --
	(226.69, 53.80) --
	(227.17, 53.76) --
	(227.64, 53.69) --
	(228.12, 53.61) --
	(228.59, 53.50) --
	(229.07, 53.36) --
	(229.54, 53.20) --
	(230.02, 53.02) --
	(230.49, 52.83) --
	(230.97, 52.64) --
	(231.44, 52.45) --
	(231.92, 52.26) --
	(232.39, 52.09) --
	(232.87, 51.94) --
	(233.34, 51.80) --
	(233.82, 51.69) --
	(234.29, 51.60) --
	(234.77, 51.53) --
	(235.24, 51.48) --
	(235.72, 51.44) --
	(236.19, 51.41) --
	(236.67, 51.39) --
	(237.14, 51.38) --
	(237.62, 51.37) --
	(238.09, 51.36) --
	(238.57, 51.36) --
	(239.04, 51.36) --
	(239.52, 51.35) --
	(239.99, 51.35) --
	(240.47, 51.35) --
	(240.94, 51.35) --
	(241.42, 51.35) --
	(241.89, 51.35) --
	(242.37, 51.35) --
	(242.84, 51.35) --
	(243.32, 51.35) --
	(243.79, 51.35) --
	(244.27, 51.35) --
	(244.74, 51.35) --
	(245.22, 51.35) --
	(245.69, 51.35) --
	(246.17, 51.36) --
	(246.64, 51.36) --
	(247.12, 51.36) --
	(247.60, 51.36) --
	(248.07, 51.37) --
	(248.55, 51.38) --
	(249.02, 51.40) --
	(249.50, 51.42) --
	(249.97, 51.45) --
	(250.45, 51.49) --
	(250.92, 51.55) --
	(251.40, 51.63) --
	(251.87, 51.73) --
	(252.35, 51.84) --
	(252.82, 51.98) --
	(253.30, 52.13) --
	(253.77, 52.30) --
	(254.25, 52.47) --
	(254.72, 52.64) --
	(255.20, 52.80) --
	(255.67, 52.94) --
	(256.15, 53.05) --
	(256.62, 53.12) --
	(257.10, 53.15) --
	(257.57, 53.13) --
	(258.05, 53.07) --
	(258.52, 52.97) --
	(259.00, 52.84) --
	(259.47, 52.69) --
	(259.95, 52.54) --
	(260.42, 52.39) --
	(260.90, 52.25) --
	(261.37, 52.13) --
	(261.85, 52.05) --
	(262.32, 52.00) --
	(262.80, 52.00) --
	(263.27, 52.03) --
	(263.75, 52.10) --
	(264.22, 52.20) --
	(264.70, 52.33) --
	(265.17, 52.48) --
	(265.65, 52.64) --
	(266.12, 52.79) --
	(266.60, 52.93) --
	(267.07, 53.04) --
	(267.55, 53.11) --
	(268.02, 53.15) --
	(268.50, 53.14) --
	(268.97, 53.08) --
	(269.45, 52.99) --
	(269.92, 52.86) --
	(270.40, 52.70) --
	(270.87, 52.54) --
	(271.35, 52.36) --
	(271.82, 52.19) --
	(272.30, 52.04) --
	(272.77, 51.90) --
	(273.25, 51.78) --
	(273.72, 51.68) --
	(274.20, 51.61) --
	(274.67, 51.56) --
	(275.15, 51.53) --
	(275.62, 51.52) --
	(276.10, 51.53) --
	(276.57, 51.57) --
	(277.05, 51.63) --
	(277.52, 51.71) --
	(278.00, 51.82) --
	(278.48, 51.95) --
	(278.95, 52.10) --
	(279.43, 52.26) --
	(279.90, 52.43) --
	(280.38, 52.60) --
	(280.85, 52.76) --
	(281.33, 52.91) --
	(281.80, 53.02) --
	(282.28, 53.10) --
	(282.75, 53.14) --
	(283.23, 53.13) --
	(283.70, 53.08) --
	(284.18, 52.99) --
	(284.65, 52.86) --
	(285.13, 52.71) --
	(285.60, 52.54) --
	(286.08, 52.37) --
	(286.55, 52.20) --
	(287.03, 52.04) --
	(287.50, 51.90) --
	(287.98, 51.77) --
	(288.45, 51.67) --
	(288.93, 51.58) --
	(289.40, 51.52) --
	(289.88, 51.47) --
	(290.35, 51.43) --
	(290.83, 51.41) --
	(291.30, 51.39) --
	(291.78, 51.37);
\end{scope}
\begin{scope}
\path[clip] (  0.00,  0.00) rectangle (361.35,144.54);
\definecolor{drawColor}{RGB}{0,0,0}

\node[text=drawColor,anchor=base,inner sep=0pt, outer sep=0pt, scale=  1.00] at (201.68,  8.40) {\textit{RMSE} (ms)};

\node[text=drawColor,rotate= 90.00,anchor=base,inner sep=0pt, outer sep=0pt, scale=  1.00] at (  9.60, 93.27) {Dichte};
\end{scope}
\begin{scope}
\path[clip] (  0.00,  0.00) rectangle (361.35,144.54);
\definecolor{drawColor}{RGB}{0,0,0}

\path[draw=drawColor,line width= 0.4pt,line join=round,line cap=round] ( 53.83, 48.00) -- (349.52, 48.00);

\path[draw=drawColor,line width= 0.4pt,line join=round,line cap=round] ( 53.83, 48.00) -- ( 53.83, 42.00);

\path[draw=drawColor,line width= 0.4pt,line join=round,line cap=round] ( 96.07, 48.00) -- ( 96.07, 42.00);

\path[draw=drawColor,line width= 0.4pt,line join=round,line cap=round] (138.31, 48.00) -- (138.31, 42.00);

\path[draw=drawColor,line width= 0.4pt,line join=round,line cap=round] (180.55, 48.00) -- (180.55, 42.00);

\path[draw=drawColor,line width= 0.4pt,line join=round,line cap=round] (222.80, 48.00) -- (222.80, 42.00);

\path[draw=drawColor,line width= 0.4pt,line join=round,line cap=round] (265.04, 48.00) -- (265.04, 42.00);

\path[draw=drawColor,line width= 0.4pt,line join=round,line cap=round] (307.28, 48.00) -- (307.28, 42.00);

\path[draw=drawColor,line width= 0.4pt,line join=round,line cap=round] (349.52, 48.00) -- (349.52, 42.00);

\node[text=drawColor,anchor=base,inner sep=0pt, outer sep=0pt, scale=  1.00] at ( 53.83, 30.00) {0};

\node[text=drawColor,anchor=base,inner sep=0pt, outer sep=0pt, scale=  1.00] at ( 96.07, 30.00) {10};

\node[text=drawColor,anchor=base,inner sep=0pt, outer sep=0pt, scale=  1.00] at (138.31, 30.00) {20};

\node[text=drawColor,anchor=base,inner sep=0pt, outer sep=0pt, scale=  1.00] at (180.55, 30.00) {30};

\node[text=drawColor,anchor=base,inner sep=0pt, outer sep=0pt, scale=  1.00] at (222.80, 30.00) {40};

\node[text=drawColor,anchor=base,inner sep=0pt, outer sep=0pt, scale=  1.00] at (265.04, 30.00) {50};

\node[text=drawColor,anchor=base,inner sep=0pt, outer sep=0pt, scale=  1.00] at (307.28, 30.00) {60};

\node[text=drawColor,anchor=base,inner sep=0pt, outer sep=0pt, scale=  1.00] at (349.52, 30.00) {70};

\path[draw=drawColor,line width= 0.4pt,line join=round,line cap=round] ( 42.00, 51.35) -- ( 42.00,135.19);

\path[draw=drawColor,line width= 0.4pt,line join=round,line cap=round] ( 42.00, 51.35) -- ( 36.00, 51.35);

\path[draw=drawColor,line width= 0.4pt,line join=round,line cap=round] ( 42.00, 79.30) -- ( 36.00, 79.30);

\path[draw=drawColor,line width= 0.4pt,line join=round,line cap=round] ( 42.00,107.24) -- ( 36.00,107.24);

\path[draw=drawColor,line width= 0.4pt,line join=round,line cap=round] ( 42.00,135.19) -- ( 36.00,135.19);

\node[text=drawColor,anchor=base east,inner sep=0pt, outer sep=0pt, scale=  1.00] at ( 33.60, 47.91) {0.00};

\node[text=drawColor,anchor=base east,inner sep=0pt, outer sep=0pt, scale=  1.00] at ( 33.60, 75.85) {0.05};

\node[text=drawColor,anchor=base east,inner sep=0pt, outer sep=0pt, scale=  1.00] at ( 33.60,103.80) {0.10};

\node[text=drawColor,anchor=base east,inner sep=0pt, outer sep=0pt, scale=  1.00] at ( 33.60,131.74) {0.15};

\path[draw=drawColor,line width= 0.2pt,line join=round,line cap=round] ( 57.88, 48.00) -- ( 57.88, 57.05);

\path[draw=drawColor,line width= 0.2pt,line join=round,line cap=round] ( 60.08, 48.00) -- ( 60.08, 57.05);

\path[draw=drawColor,line width= 0.2pt,line join=round,line cap=round] ( 64.77, 48.00) -- ( 64.77, 57.05);

\path[draw=drawColor,line width= 0.2pt,line join=round,line cap=round] ( 65.06, 48.00) -- ( 65.06, 57.05);

\path[draw=drawColor,line width= 0.2pt,line join=round,line cap=round] ( 65.82, 48.00) -- ( 65.82, 57.05);

\path[draw=drawColor,line width= 0.2pt,line join=round,line cap=round] ( 66.08, 48.00) -- ( 66.08, 57.05);

\path[draw=drawColor,line width= 0.2pt,line join=round,line cap=round] ( 66.84, 48.00) -- ( 66.84, 57.05);

\path[draw=drawColor,line width= 0.2pt,line join=round,line cap=round] ( 68.82, 48.00) -- ( 68.82, 57.05);

\path[draw=drawColor,line width= 0.2pt,line join=round,line cap=round] ( 69.58, 48.00) -- ( 69.58, 57.05);

\path[draw=drawColor,line width= 0.2pt,line join=round,line cap=round] ( 69.92, 48.00) -- ( 69.92, 57.05);

\path[draw=drawColor,line width= 0.2pt,line join=round,line cap=round] ( 71.36, 48.00) -- ( 71.36, 57.05);

\path[draw=drawColor,line width= 0.2pt,line join=round,line cap=round] ( 73.94, 48.00) -- ( 73.94, 57.05);

\path[draw=drawColor,line width= 0.2pt,line join=round,line cap=round] ( 74.82, 48.00) -- ( 74.82, 57.05);

\path[draw=drawColor,line width= 0.2pt,line join=round,line cap=round] ( 76.05, 48.00) -- ( 76.05, 57.05);

\path[draw=drawColor,line width= 0.2pt,line join=round,line cap=round] ( 76.05, 48.00) -- ( 76.05, 57.05);

\path[draw=drawColor,line width= 0.2pt,line join=round,line cap=round] ( 76.13, 48.00) -- ( 76.13, 57.05);

\path[draw=drawColor,line width= 0.2pt,line join=round,line cap=round] ( 76.26, 48.00) -- ( 76.26, 57.05);

\path[draw=drawColor,line width= 0.2pt,line join=round,line cap=round] ( 77.36, 48.00) -- ( 77.36, 57.05);

\path[draw=drawColor,line width= 0.2pt,line join=round,line cap=round] ( 77.44, 48.00) -- ( 77.44, 57.05);

\path[draw=drawColor,line width= 0.2pt,line join=round,line cap=round] ( 77.69, 48.00) -- ( 77.69, 57.05);

\path[draw=drawColor,line width= 0.2pt,line join=round,line cap=round] ( 79.64, 48.00) -- ( 79.64, 57.05);

\path[draw=drawColor,line width= 0.2pt,line join=round,line cap=round] ( 79.85, 48.00) -- ( 79.85, 57.05);

\path[draw=drawColor,line width= 0.2pt,line join=round,line cap=round] ( 79.98, 48.00) -- ( 79.98, 57.05);

\path[draw=drawColor,line width= 0.2pt,line join=round,line cap=round] ( 80.31, 48.00) -- ( 80.31, 57.05);

\path[draw=drawColor,line width= 0.2pt,line join=round,line cap=round] ( 81.03, 48.00) -- ( 81.03, 57.05);

\path[draw=drawColor,line width= 0.2pt,line join=round,line cap=round] ( 81.67, 48.00) -- ( 81.67, 57.05);

\path[draw=drawColor,line width= 0.2pt,line join=round,line cap=round] ( 82.43, 48.00) -- ( 82.43, 57.05);

\path[draw=drawColor,line width= 0.2pt,line join=round,line cap=round] ( 84.07, 48.00) -- ( 84.07, 57.05);

\path[draw=drawColor,line width= 0.2pt,line join=round,line cap=round] ( 84.79, 48.00) -- ( 84.79, 57.05);

\path[draw=drawColor,line width= 0.2pt,line join=round,line cap=round] ( 85.59, 48.00) -- ( 85.59, 57.05);

\path[draw=drawColor,line width= 0.2pt,line join=round,line cap=round] ( 86.06, 48.00) -- ( 86.06, 57.05);

\path[draw=drawColor,line width= 0.2pt,line join=round,line cap=round] ( 86.31, 48.00) -- ( 86.31, 57.05);

\path[draw=drawColor,line width= 0.2pt,line join=round,line cap=round] ( 86.65, 48.00) -- ( 86.65, 57.05);

\path[draw=drawColor,line width= 0.2pt,line join=round,line cap=round] ( 86.69, 48.00) -- ( 86.69, 57.05);

\path[draw=drawColor,line width= 0.2pt,line join=round,line cap=round] ( 87.33, 48.00) -- ( 87.33, 57.05);

\path[draw=drawColor,line width= 0.2pt,line join=round,line cap=round] ( 87.33, 48.00) -- ( 87.33, 57.05);

\path[draw=drawColor,line width= 0.2pt,line join=round,line cap=round] ( 87.92, 48.00) -- ( 87.92, 57.05);

\path[draw=drawColor,line width= 0.2pt,line join=round,line cap=round] ( 88.51, 48.00) -- ( 88.51, 57.05);

\path[draw=drawColor,line width= 0.2pt,line join=round,line cap=round] ( 88.64, 48.00) -- ( 88.64, 57.05);

\path[draw=drawColor,line width= 0.2pt,line join=round,line cap=round] ( 89.02, 48.00) -- ( 89.02, 57.05);

\path[draw=drawColor,line width= 0.2pt,line join=round,line cap=round] ( 89.44, 48.00) -- ( 89.44, 57.05);

\path[draw=drawColor,line width= 0.2pt,line join=round,line cap=round] ( 89.99, 48.00) -- ( 89.99, 57.05);

\path[draw=drawColor,line width= 0.2pt,line join=round,line cap=round] ( 90.16, 48.00) -- ( 90.16, 57.05);

\path[draw=drawColor,line width= 0.2pt,line join=round,line cap=round] ( 91.00, 48.00) -- ( 91.00, 57.05);

\path[draw=drawColor,line width= 0.2pt,line join=round,line cap=round] ( 91.04, 48.00) -- ( 91.04, 57.05);

\path[draw=drawColor,line width= 0.2pt,line join=round,line cap=round] ( 91.38, 48.00) -- ( 91.38, 57.05);

\path[draw=drawColor,line width= 0.2pt,line join=round,line cap=round] ( 91.89, 48.00) -- ( 91.89, 57.05);

\path[draw=drawColor,line width= 0.2pt,line join=round,line cap=round] ( 92.23, 48.00) -- ( 92.23, 57.05);

\path[draw=drawColor,line width= 0.2pt,line join=round,line cap=round] ( 92.77, 48.00) -- ( 92.77, 57.05);

\path[draw=drawColor,line width= 0.2pt,line join=round,line cap=round] ( 92.86, 48.00) -- ( 92.86, 57.05);

\path[draw=drawColor,line width= 0.2pt,line join=round,line cap=round] ( 93.03, 48.00) -- ( 93.03, 57.05);

\path[draw=drawColor,line width= 0.2pt,line join=round,line cap=round] ( 93.07, 48.00) -- ( 93.07, 57.05);

\path[draw=drawColor,line width= 0.2pt,line join=round,line cap=round] ( 93.70, 48.00) -- ( 93.70, 57.05);

\path[draw=drawColor,line width= 0.2pt,line join=round,line cap=round] ( 93.87, 48.00) -- ( 93.87, 57.05);

\path[draw=drawColor,line width= 0.2pt,line join=round,line cap=round] ( 94.38, 48.00) -- ( 94.38, 57.05);

\path[draw=drawColor,line width= 0.2pt,line join=round,line cap=round] ( 94.59, 48.00) -- ( 94.59, 57.05);

\path[draw=drawColor,line width= 0.2pt,line join=round,line cap=round] ( 96.45, 48.00) -- ( 96.45, 57.05);

\path[draw=drawColor,line width= 0.2pt,line join=round,line cap=round] ( 96.75, 48.00) -- ( 96.75, 57.05);

\path[draw=drawColor,line width= 0.2pt,line join=round,line cap=round] ( 97.00, 48.00) -- ( 97.00, 57.05);

\path[draw=drawColor,line width= 0.2pt,line join=round,line cap=round] ( 97.25, 48.00) -- ( 97.25, 57.05);

\path[draw=drawColor,line width= 0.2pt,line join=round,line cap=round] ( 97.34, 48.00) -- ( 97.34, 57.05);

\path[draw=drawColor,line width= 0.2pt,line join=round,line cap=round] ( 97.59, 48.00) -- ( 97.59, 57.05);

\path[draw=drawColor,line width= 0.2pt,line join=round,line cap=round] ( 97.68, 48.00) -- ( 97.68, 57.05);

\path[draw=drawColor,line width= 0.2pt,line join=round,line cap=round] ( 98.18, 48.00) -- ( 98.18, 57.05);

\path[draw=drawColor,line width= 0.2pt,line join=round,line cap=round] ( 98.31, 48.00) -- ( 98.31, 57.05);

\path[draw=drawColor,line width= 0.2pt,line join=round,line cap=round] ( 98.48, 48.00) -- ( 98.48, 57.05);

\path[draw=drawColor,line width= 0.2pt,line join=round,line cap=round] ( 98.86, 48.00) -- ( 98.86, 57.05);

\path[draw=drawColor,line width= 0.2pt,line join=round,line cap=round] ( 98.94, 48.00) -- ( 98.94, 57.05);

\path[draw=drawColor,line width= 0.2pt,line join=round,line cap=round] ( 99.28, 48.00) -- ( 99.28, 57.05);

\path[draw=drawColor,line width= 0.2pt,line join=round,line cap=round] ( 99.32, 48.00) -- ( 99.32, 57.05);

\path[draw=drawColor,line width= 0.2pt,line join=round,line cap=round] ( 99.96, 48.00) -- ( 99.96, 57.05);

\path[draw=drawColor,line width= 0.2pt,line join=round,line cap=round] (100.00, 48.00) -- (100.00, 57.05);

\path[draw=drawColor,line width= 0.2pt,line join=round,line cap=round] (100.29, 48.00) -- (100.29, 57.05);

\path[draw=drawColor,line width= 0.2pt,line join=round,line cap=round] (100.63, 48.00) -- (100.63, 57.05);

\path[draw=drawColor,line width= 0.2pt,line join=round,line cap=round] (100.84, 48.00) -- (100.84, 57.05);

\path[draw=drawColor,line width= 0.2pt,line join=round,line cap=round] (101.98, 48.00) -- (101.98, 57.05);

\path[draw=drawColor,line width= 0.2pt,line join=round,line cap=round] (102.24, 48.00) -- (102.24, 57.05);

\path[draw=drawColor,line width= 0.2pt,line join=round,line cap=round] (102.41, 48.00) -- (102.41, 57.05);

\path[draw=drawColor,line width= 0.2pt,line join=round,line cap=round] (103.04, 48.00) -- (103.04, 57.05);

\path[draw=drawColor,line width= 0.2pt,line join=round,line cap=round] (103.25, 48.00) -- (103.25, 57.05);

\path[draw=drawColor,line width= 0.2pt,line join=round,line cap=round] (103.55, 48.00) -- (103.55, 57.05);

\path[draw=drawColor,line width= 0.2pt,line join=round,line cap=round] (103.93, 48.00) -- (103.93, 57.05);

\path[draw=drawColor,line width= 0.2pt,line join=round,line cap=round] (104.22, 48.00) -- (104.22, 57.05);

\path[draw=drawColor,line width= 0.2pt,line join=round,line cap=round] (104.22, 48.00) -- (104.22, 57.05);

\path[draw=drawColor,line width= 0.2pt,line join=round,line cap=round] (104.60, 48.00) -- (104.60, 57.05);

\path[draw=drawColor,line width= 0.2pt,line join=round,line cap=round] (104.77, 48.00) -- (104.77, 57.05);

\path[draw=drawColor,line width= 0.2pt,line join=round,line cap=round] (105.32, 48.00) -- (105.32, 57.05);

\path[draw=drawColor,line width= 0.2pt,line join=round,line cap=round] (105.41, 48.00) -- (105.41, 57.05);

\path[draw=drawColor,line width= 0.2pt,line join=round,line cap=round] (105.70, 48.00) -- (105.70, 57.05);

\path[draw=drawColor,line width= 0.2pt,line join=round,line cap=round] (107.35, 48.00) -- (107.35, 57.05);

\path[draw=drawColor,line width= 0.2pt,line join=round,line cap=round] (108.45, 48.00) -- (108.45, 57.05);

\path[draw=drawColor,line width= 0.2pt,line join=round,line cap=round] (109.50, 48.00) -- (109.50, 57.05);

\path[draw=drawColor,line width= 0.2pt,line join=round,line cap=round] (109.59, 48.00) -- (109.59, 57.05);

\path[draw=drawColor,line width= 0.2pt,line join=round,line cap=round] (109.76, 48.00) -- (109.76, 57.05);

\path[draw=drawColor,line width= 0.2pt,line join=round,line cap=round] (110.18, 48.00) -- (110.18, 57.05);

\path[draw=drawColor,line width= 0.2pt,line join=round,line cap=round] (111.61, 48.00) -- (111.61, 57.05);

\path[draw=drawColor,line width= 0.2pt,line join=round,line cap=round] (111.83, 48.00) -- (111.83, 57.05);

\path[draw=drawColor,line width= 0.2pt,line join=round,line cap=round] (112.00, 48.00) -- (112.00, 57.05);

\path[draw=drawColor,line width= 0.2pt,line join=round,line cap=round] (112.29, 48.00) -- (112.29, 57.05);

\path[draw=drawColor,line width= 0.2pt,line join=round,line cap=round] (112.33, 48.00) -- (112.33, 57.05);

\path[draw=drawColor,line width= 0.2pt,line join=round,line cap=round] (112.92, 48.00) -- (112.92, 57.05);

\path[draw=drawColor,line width= 0.2pt,line join=round,line cap=round] (113.26, 48.00) -- (113.26, 57.05);

\path[draw=drawColor,line width= 0.2pt,line join=round,line cap=round] (113.26, 48.00) -- (113.26, 57.05);

\path[draw=drawColor,line width= 0.2pt,line join=round,line cap=round] (116.13, 48.00) -- (116.13, 57.05);

\path[draw=drawColor,line width= 0.2pt,line join=round,line cap=round] (117.49, 48.00) -- (117.49, 57.05);

\path[draw=drawColor,line width= 0.2pt,line join=round,line cap=round] (117.49, 48.00) -- (117.49, 57.05);

\path[draw=drawColor,line width= 0.2pt,line join=round,line cap=round] (118.71, 48.00) -- (118.71, 57.05);

\path[draw=drawColor,line width= 0.2pt,line join=round,line cap=round] (119.43, 48.00) -- (119.43, 57.05);

\path[draw=drawColor,line width= 0.2pt,line join=round,line cap=round] (119.56, 48.00) -- (119.56, 57.05);

\path[draw=drawColor,line width= 0.2pt,line join=round,line cap=round] (119.85, 48.00) -- (119.85, 57.05);

\path[draw=drawColor,line width= 0.2pt,line join=round,line cap=round] (120.44, 48.00) -- (120.44, 57.05);

\path[draw=drawColor,line width= 0.2pt,line join=round,line cap=round] (121.88, 48.00) -- (121.88, 57.05);

\path[draw=drawColor,line width= 0.2pt,line join=round,line cap=round] (122.56, 48.00) -- (122.56, 57.05);

\path[draw=drawColor,line width= 0.2pt,line join=round,line cap=round] (122.85, 48.00) -- (122.85, 57.05);

\path[draw=drawColor,line width= 0.2pt,line join=round,line cap=round] (122.85, 48.00) -- (122.85, 57.05);

\path[draw=drawColor,line width= 0.2pt,line join=round,line cap=round] (122.98, 48.00) -- (122.98, 57.05);

\path[draw=drawColor,line width= 0.2pt,line join=round,line cap=round] (123.99, 48.00) -- (123.99, 57.05);

\path[draw=drawColor,line width= 0.2pt,line join=round,line cap=round] (124.08, 48.00) -- (124.08, 57.05);

\path[draw=drawColor,line width= 0.2pt,line join=round,line cap=round] (124.25, 48.00) -- (124.25, 57.05);

\path[draw=drawColor,line width= 0.2pt,line join=round,line cap=round] (124.92, 48.00) -- (124.92, 57.05);

\path[draw=drawColor,line width= 0.2pt,line join=round,line cap=round] (125.22, 48.00) -- (125.22, 57.05);

\path[draw=drawColor,line width= 0.2pt,line join=round,line cap=round] (125.30, 48.00) -- (125.30, 57.05);

\path[draw=drawColor,line width= 0.2pt,line join=round,line cap=round] (125.55, 48.00) -- (125.55, 57.05);

\path[draw=drawColor,line width= 0.2pt,line join=round,line cap=round] (126.23, 48.00) -- (126.23, 57.05);

\path[draw=drawColor,line width= 0.2pt,line join=round,line cap=round] (126.44, 48.00) -- (126.44, 57.05);

\path[draw=drawColor,line width= 0.2pt,line join=round,line cap=round] (127.24, 48.00) -- (127.24, 57.05);

\path[draw=drawColor,line width= 0.2pt,line join=round,line cap=round] (127.62, 48.00) -- (127.62, 57.05);

\path[draw=drawColor,line width= 0.2pt,line join=round,line cap=round] (127.75, 48.00) -- (127.75, 57.05);

\path[draw=drawColor,line width= 0.2pt,line join=round,line cap=round] (128.30, 48.00) -- (128.30, 57.05);

\path[draw=drawColor,line width= 0.2pt,line join=round,line cap=round] (129.31, 48.00) -- (129.31, 57.05);

\path[draw=drawColor,line width= 0.2pt,line join=round,line cap=round] (129.95, 48.00) -- (129.95, 57.05);

\path[draw=drawColor,line width= 0.2pt,line join=round,line cap=round] (131.47, 48.00) -- (131.47, 57.05);

\path[draw=drawColor,line width= 0.2pt,line join=round,line cap=round] (134.17, 48.00) -- (134.17, 57.05);

\path[draw=drawColor,line width= 0.2pt,line join=round,line cap=round] (134.89, 48.00) -- (134.89, 57.05);

\path[draw=drawColor,line width= 0.2pt,line join=round,line cap=round] (135.90, 48.00) -- (135.90, 57.05);

\path[draw=drawColor,line width= 0.2pt,line join=round,line cap=round] (136.54, 48.00) -- (136.54, 57.05);

\path[draw=drawColor,line width= 0.2pt,line join=round,line cap=round] (137.72, 48.00) -- (137.72, 57.05);

\path[draw=drawColor,line width= 0.2pt,line join=round,line cap=round] (138.23, 48.00) -- (138.23, 57.05);

\path[draw=drawColor,line width= 0.2pt,line join=round,line cap=round] (138.61, 48.00) -- (138.61, 57.05);

\path[draw=drawColor,line width= 0.2pt,line join=round,line cap=round] (140.04, 48.00) -- (140.04, 57.05);

\path[draw=drawColor,line width= 0.2pt,line join=round,line cap=round] (140.38, 48.00) -- (140.38, 57.05);

\path[draw=drawColor,line width= 0.2pt,line join=round,line cap=round] (140.97, 48.00) -- (140.97, 57.05);

\path[draw=drawColor,line width= 0.2pt,line join=round,line cap=round] (141.44, 48.00) -- (141.44, 57.05);

\path[draw=drawColor,line width= 0.2pt,line join=round,line cap=round] (141.86, 48.00) -- (141.86, 57.05);

\path[draw=drawColor,line width= 0.2pt,line join=round,line cap=round] (142.24, 48.00) -- (142.24, 57.05);

\path[draw=drawColor,line width= 0.2pt,line join=round,line cap=round] (144.44, 48.00) -- (144.44, 57.05);

\path[draw=drawColor,line width= 0.2pt,line join=round,line cap=round] (145.45, 48.00) -- (145.45, 57.05);

\path[draw=drawColor,line width= 0.2pt,line join=round,line cap=round] (146.46, 48.00) -- (146.46, 57.05);

\path[draw=drawColor,line width= 0.2pt,line join=round,line cap=round] (147.65, 48.00) -- (147.65, 57.05);

\path[draw=drawColor,line width= 0.2pt,line join=round,line cap=round] (150.01, 48.00) -- (150.01, 57.05);

\path[draw=drawColor,line width= 0.2pt,line join=round,line cap=round] (151.53, 48.00) -- (151.53, 57.05);

\path[draw=drawColor,line width= 0.2pt,line join=round,line cap=round] (152.00, 48.00) -- (152.00, 57.05);

\path[draw=drawColor,line width= 0.2pt,line join=round,line cap=round] (152.38, 48.00) -- (152.38, 57.05);

\path[draw=drawColor,line width= 0.2pt,line join=round,line cap=round] (156.56, 48.00) -- (156.56, 57.05);

\path[draw=drawColor,line width= 0.2pt,line join=round,line cap=round] (158.59, 48.00) -- (158.59, 57.05);

\path[draw=drawColor,line width= 0.2pt,line join=round,line cap=round] (158.88, 48.00) -- (158.88, 57.05);

\path[draw=drawColor,line width= 0.2pt,line join=round,line cap=round] (159.01, 48.00) -- (159.01, 57.05);

\path[draw=drawColor,line width= 0.2pt,line join=round,line cap=round] (161.04, 48.00) -- (161.04, 57.05);

\path[draw=drawColor,line width= 0.2pt,line join=round,line cap=round] (162.26, 48.00) -- (162.26, 57.05);

\path[draw=drawColor,line width= 0.2pt,line join=round,line cap=round] (162.69, 48.00) -- (162.69, 57.05);

\path[draw=drawColor,line width= 0.2pt,line join=round,line cap=round] (163.40, 48.00) -- (163.40, 57.05);

\path[draw=drawColor,line width= 0.2pt,line join=round,line cap=round] (163.87, 48.00) -- (163.87, 57.05);

\path[draw=drawColor,line width= 0.2pt,line join=round,line cap=round] (165.35, 48.00) -- (165.35, 57.05);

\path[draw=drawColor,line width= 0.2pt,line join=round,line cap=round] (167.29, 48.00) -- (167.29, 57.05);

\path[draw=drawColor,line width= 0.2pt,line join=round,line cap=round] (167.92, 48.00) -- (167.92, 57.05);

\path[draw=drawColor,line width= 0.2pt,line join=round,line cap=round] (172.11, 48.00) -- (172.11, 57.05);

\path[draw=drawColor,line width= 0.2pt,line join=round,line cap=round] (186.00, 48.00) -- (186.00, 57.05);

\path[draw=drawColor,line width= 0.2pt,line join=round,line cap=round] (192.21, 48.00) -- (192.21, 57.05);

\path[draw=drawColor,line width= 0.2pt,line join=round,line cap=round] (206.62, 48.00) -- (206.62, 57.05);

\path[draw=drawColor,line width= 0.2pt,line join=round,line cap=round] (210.38, 48.00) -- (210.38, 57.05);

\path[draw=drawColor,line width= 0.2pt,line join=round,line cap=round] (211.31, 48.00) -- (211.31, 57.05);

\path[draw=drawColor,line width= 0.2pt,line join=round,line cap=round] (216.42, 48.00) -- (216.42, 57.05);

\path[draw=drawColor,line width= 0.2pt,line join=round,line cap=round] (223.26, 48.00) -- (223.26, 57.05);

\path[draw=drawColor,line width= 0.2pt,line join=round,line cap=round] (228.37, 48.00) -- (228.37, 57.05);

\path[draw=drawColor,line width= 0.2pt,line join=round,line cap=round] (257.14, 48.00) -- (257.14, 57.05);

\path[draw=drawColor,line width= 0.2pt,line join=round,line cap=round] (268.16, 48.00) -- (268.16, 57.05);

\path[draw=drawColor,line width= 0.2pt,line join=round,line cap=round] (282.91, 48.00) -- (282.91, 57.05);
\end{scope}
\end{tikzpicture}

	\end{adjustbox}
	\caption[Dichtefunktion des aus der \gls{ha} mit einer linearen Regression abgeleiteten \gls{rmse}]{Dichtefunktion des aus der \gls{ha} mit einer linearen Regression abgeleiteten Root Mean Square Error (\gls{rmse}; in Millisekunden).  Alle Datenpunkte sind auf der x-Achse mit vertikalen Strichen markiert.}
	\label{fig:hick_rmse_density}
\end{figure}

Wie bei der \gls{ssauf} (siehe \autoref{sec:2Fragestellung}) wurde der Zusammenhang zwischen den Aufgabenparametern (y-Achsenabschnitt und Steigung) und dem \gls{zwert} des \gls{bist}s in Abhängigkeit des \gls{rmse} betrachtet. Die Analysen haben ergeben, dass der y-Achsenabschnitt und der \gls{zwert} bei einem \gls{rmse}-Grenzwert zwischen $9.6$ und $10.4$~ms ($r = -.28$ bis $ -.29$, alle $p\textnormal{s} < .03$), zwischen $11$ und $15.7$~ms ($r = -.24$ bis $ -.31$, alle $p\textnormal{s} < .04$), zwischen $17.2$ und $22.2$~ms ($r = -.17$ bis $ -.20$, alle $p\textnormal{s} < .04$) und ab $23.3$~ms ($r = -.15$ bis $ -.17$, alle $p\textnormal{s} < .049$) signifikant miteinander korrelierten (siehe \autoref{fig:hick_rmse_cutoff}a). In den erwähnten Bereichen war ein tiefer y-Achsenabschnitt also tendenziell mit einem hohen \gls{zwert} verbunden. Eine visuelle Inspektion des Verlaufs liess keine Aussage darüber zu, ob der \gls{rmse}-Grenzwert einen positiven oder negativen Einfluss auf die Höhe des Zusammenhangs ausübte.


\begin{figure}[htbp]
	\centering
	%	\captionsetup{font = small}
	\begin{adjustbox}{width=1\textwidth}
		\subfloat[Test][Zusammenhang ($r$) zwischen dem y-Achsenabschnitt und dem \gls{zwert} des \gls{bist}s.]{% Created by tikzDevice version 0.10.1 on 2016-08-19 08:52:31
% !TEX encoding = UTF-8 Unicode
\begin{tikzpicture}[x=1pt,y=1pt]
\definecolor{fillColor}{RGB}{255,255,255}
\path[use as bounding box,fill=fillColor,fill opacity=0.00] (0,0) rectangle (361.35,216.81);
\begin{scope}
\path[clip] (  0.00,  0.00) rectangle (361.35,216.81);
\definecolor{drawColor}{RGB}{0,0,0}

\node[text=drawColor,anchor=base,inner sep=0pt, outer sep=0pt, scale=  1.00] at (201.68,  8.40) {\textit{RMSE}-Grenzwert (ms)};

\node[text=drawColor,rotate= 90.00,anchor=base,inner sep=0pt, outer sep=0pt, scale=  1.00] at (  9.60,129.40) {\textit{r}};
\end{scope}
\begin{scope}
\path[clip] (  0.00,  0.00) rectangle (361.35,216.81);
\definecolor{drawColor}{RGB}{0,0,0}

\path[draw=drawColor,line width= 0.4pt,line join=round,line cap=round] ( 66.50, 48.00) -- (349.52, 48.00);

\path[draw=drawColor,line width= 0.4pt,line join=round,line cap=round] ( 66.50, 48.00) -- ( 66.50, 42.00);

\path[draw=drawColor,line width= 0.4pt,line join=round,line cap=round] ( 96.07, 48.00) -- ( 96.07, 42.00);

\path[draw=drawColor,line width= 0.4pt,line join=round,line cap=round] (138.31, 48.00) -- (138.31, 42.00);

\path[draw=drawColor,line width= 0.4pt,line join=round,line cap=round] (180.55, 48.00) -- (180.55, 42.00);

\path[draw=drawColor,line width= 0.4pt,line join=round,line cap=round] (222.80, 48.00) -- (222.80, 42.00);

\path[draw=drawColor,line width= 0.4pt,line join=round,line cap=round] (265.04, 48.00) -- (265.04, 42.00);

\path[draw=drawColor,line width= 0.4pt,line join=round,line cap=round] (307.28, 48.00) -- (307.28, 42.00);

\path[draw=drawColor,line width= 0.4pt,line join=round,line cap=round] (349.52, 48.00) -- (349.52, 42.00);

\node[text=drawColor,anchor=base,inner sep=0pt, outer sep=0pt, scale=  1.00] at ( 66.50, 30.00) {3};

\node[text=drawColor,anchor=base,inner sep=0pt, outer sep=0pt, scale=  1.00] at ( 96.07, 30.00) {10};

\node[text=drawColor,anchor=base,inner sep=0pt, outer sep=0pt, scale=  1.00] at (138.31, 30.00) {20};

\node[text=drawColor,anchor=base,inner sep=0pt, outer sep=0pt, scale=  1.00] at (180.55, 30.00) {30};

\node[text=drawColor,anchor=base,inner sep=0pt, outer sep=0pt, scale=  1.00] at (222.80, 30.00) {40};

\node[text=drawColor,anchor=base,inner sep=0pt, outer sep=0pt, scale=  1.00] at (265.04, 30.00) {50};

\node[text=drawColor,anchor=base,inner sep=0pt, outer sep=0pt, scale=  1.00] at (307.28, 30.00) {60};

\node[text=drawColor,anchor=base,inner sep=0pt, outer sep=0pt, scale=  1.00] at (349.52, 30.00) {70};

\path[draw=drawColor,line width= 0.4pt,line join=round,line cap=round] ( 42.00, 54.03) -- ( 42.00,204.78);

\path[draw=drawColor,line width= 0.4pt,line join=round,line cap=round] ( 42.00, 54.03) -- ( 36.00, 54.03);

\path[draw=drawColor,line width= 0.4pt,line join=round,line cap=round] ( 42.00, 72.87) -- ( 36.00, 72.87);

\path[draw=drawColor,line width= 0.4pt,line join=round,line cap=round] ( 42.00, 91.72) -- ( 36.00, 91.72);

\path[draw=drawColor,line width= 0.4pt,line join=round,line cap=round] ( 42.00,110.56) -- ( 36.00,110.56);

\path[draw=drawColor,line width= 0.4pt,line join=round,line cap=round] ( 42.00,129.40) -- ( 36.00,129.40);

\path[draw=drawColor,line width= 0.4pt,line join=round,line cap=round] ( 42.00,148.25) -- ( 36.00,148.25);

\path[draw=drawColor,line width= 0.4pt,line join=round,line cap=round] ( 42.00,167.09) -- ( 36.00,167.09);

\path[draw=drawColor,line width= 0.4pt,line join=round,line cap=round] ( 42.00,185.94) -- ( 36.00,185.94);

\path[draw=drawColor,line width= 0.4pt,line join=round,line cap=round] ( 42.00,204.78) -- ( 36.00,204.78);

\node[text=drawColor,anchor=base east,inner sep=0pt, outer sep=0pt, scale=  1.00] at ( 33.60, 50.59) {--1.00};

\node[text=drawColor,anchor=base east,inner sep=0pt, outer sep=0pt, scale=  1.00] at ( 33.60, 69.43) {--.75};

\node[text=drawColor,anchor=base east,inner sep=0pt, outer sep=0pt, scale=  1.00] at ( 33.60, 88.27) {--.50};

\node[text=drawColor,anchor=base east,inner sep=0pt, outer sep=0pt, scale=  1.00] at ( 33.60,107.12) {--.25};

\node[text=drawColor,anchor=base east,inner sep=0pt, outer sep=0pt, scale=  1.00] at ( 33.60,125.96) {.00};

\node[text=drawColor,anchor=base east,inner sep=0pt, outer sep=0pt, scale=  1.00] at ( 33.60,144.81) {.25};

\node[text=drawColor,anchor=base east,inner sep=0pt, outer sep=0pt, scale=  1.00] at ( 33.60,163.65) {.50};

\node[text=drawColor,anchor=base east,inner sep=0pt, outer sep=0pt, scale=  1.00] at ( 33.60,182.49) {.75};

\node[text=drawColor,anchor=base east,inner sep=0pt, outer sep=0pt, scale=  1.00] at ( 33.60,201.34) {1.00};
\end{scope}
\begin{scope}
\path[clip] ( 42.00, 48.00) rectangle (361.35,210.81);
\definecolor{fillColor}{RGB}{190,190,190}

\path[fill=fillColor] ( 66.50, 70.76) --
	( 66.92, 71.26) --
	( 67.35, 71.26) --
	( 67.77, 71.26) --
	( 68.19, 71.26) --
	( 68.61, 71.26) --
	( 69.03, 77.23) --
	( 69.46, 77.23) --
	( 69.88, 83.79) --
	( 70.30, 87.67) --
	( 70.72, 87.67) --
	( 71.15, 87.67) --
	( 71.57, 83.51) --
	( 71.99, 83.51) --
	( 72.41, 83.51) --
	( 72.84, 83.51) --
	( 73.26, 83.51) --
	( 73.68, 83.51) --
	( 74.10, 85.45) --
	( 74.53, 85.45) --
	( 74.95, 87.08) --
	( 75.37, 87.08) --
	( 75.79, 87.08) --
	( 76.22, 86.82) --
	( 76.64, 91.02) --
	( 77.06, 91.02) --
	( 77.48, 98.00) --
	( 77.91, 93.15) --
	( 78.33, 93.15) --
	( 78.75, 93.15) --
	( 79.17, 93.15) --
	( 79.60, 93.15) --
	( 80.02, 92.25) --
	( 80.44, 93.30) --
	( 80.86, 93.30) --
	( 81.29, 93.62) --
	( 81.71, 87.17) --
	( 82.13, 87.17) --
	( 82.55, 87.51) --
	( 82.97, 87.51) --
	( 83.40, 87.51) --
	( 83.82, 87.51) --
	( 84.24, 83.73) --
	( 84.66, 83.73) --
	( 85.09, 84.10) --
	( 85.51, 84.10) --
	( 85.93, 84.07) --
	( 86.35, 88.39) --
	( 86.78, 90.47) --
	( 87.20, 90.47) --
	( 87.62, 93.00) --
	( 88.04, 93.74) --
	( 88.47, 93.74) --
	( 88.89, 93.10) --
	( 89.31, 91.80) --
	( 89.73, 91.85) --
	( 90.16, 93.97) --
	( 90.58, 93.97) --
	( 91.00, 92.53) --
	( 91.42, 91.88) --
	( 91.85, 91.88) --
	( 92.27, 90.22) --
	( 92.69, 90.22) --
	( 93.11, 91.32) --
	( 93.54, 91.32) --
	( 93.96, 92.10) --
	( 94.38, 89.96) --
	( 94.80, 90.41) --
	( 95.22, 90.41) --
	( 95.65, 90.41) --
	( 96.07, 90.41) --
	( 96.49, 90.43) --
	( 96.91, 91.32) --
	( 97.34, 91.37) --
	( 97.76, 92.02) --
	( 98.18, 95.79) --
	( 98.60, 96.40) --
	( 99.03, 97.00) --
	( 99.45, 96.20) --
	( 99.87, 96.20) --
	(100.29, 95.66) --
	(100.72, 94.32) --
	(101.14, 94.23) --
	(101.56, 94.23) --
	(101.98, 94.32) --
	(102.41, 94.11) --
	(102.83, 94.11) --
	(103.25, 93.72) --
	(103.67, 93.68) --
	(104.10, 93.75) --
	(104.52, 92.75) --
	(104.94, 92.54) --
	(105.36, 92.29) --
	(105.79, 93.05) --
	(106.21, 93.05) --
	(106.63, 93.05) --
	(107.05, 93.05) --
	(107.48, 93.12) --
	(107.90, 93.12) --
	(108.32, 93.12) --
	(108.74, 93.75) --
	(109.16, 93.75) --
	(109.59, 94.27) --
	(110.01, 95.10) --
	(110.43, 95.33) --
	(110.85, 95.33) --
	(111.28, 95.33) --
	(111.70, 95.56) --
	(112.12, 95.33) --
	(112.54, 96.29) --
	(112.97, 96.48) --
	(113.39, 96.71) --
	(113.81, 96.71) --
	(114.23, 96.71) --
	(114.66, 96.71) --
	(115.08, 96.71) --
	(115.50, 96.71) --
	(115.92, 96.71) --
	(116.35, 97.52) --
	(116.77, 97.52) --
	(117.19, 97.52) --
	(117.61, 95.24) --
	(118.04, 95.24) --
	(118.46, 95.24) --
	(118.88, 96.08) --
	(119.30, 96.08) --
	(119.73, 96.46) --
	(120.15,101.86) --
	(120.57,103.06) --
	(120.99,103.06) --
	(121.42,103.06) --
	(121.84,103.06) --
	(122.26,103.36) --
	(122.68,103.41) --
	(123.10,103.50) --
	(123.53,103.50) --
	(123.95,103.50) --
	(124.37,105.90) --
	(124.79,105.90) --
	(125.22,107.15) --
	(125.64,106.44) --
	(126.06,106.44) --
	(126.48,102.27) --
	(126.91,102.27) --
	(127.33,102.30) --
	(127.75,103.10) --
	(128.17,103.10) --
	(128.60,103.33) --
	(129.02,103.33) --
	(129.44,103.34) --
	(129.86,103.34) --
	(130.29,103.39) --
	(130.71,103.39) --
	(131.13,103.39) --
	(131.55,103.82) --
	(131.98,103.82) --
	(132.40,103.82) --
	(132.82,103.82) --
	(133.24,103.82) --
	(133.67,103.82) --
	(134.09,103.82) --
	(134.51,103.98) --
	(134.93,103.91) --
	(135.35,103.91) --
	(135.78,103.91) --
	(136.20,103.94) --
	(136.62,104.03) --
	(137.04,104.03) --
	(137.47,104.03) --
	(137.89,103.58) --
	(138.31,103.55) --
	(138.73,103.50) --
	(139.16,103.50) --
	(139.58,103.50) --
	(140.00,103.50) --
	(140.42,104.43) --
	(140.85,104.43) --
	(141.27,103.18) --
	(141.69,102.93) --
	(142.11,103.48) --
	(142.54,103.37) --
	(142.96,103.37) --
	(143.38,103.37) --
	(143.80,103.37) --
	(144.23,103.37) --
	(144.65,103.91) --
	(145.07,103.91) --
	(145.49,105.03) --
	(145.92,105.03) --
	(146.34,105.03) --
	(146.76,105.25) --
	(147.18,105.25) --
	(147.61,105.25) --
	(148.03,105.86) --
	(148.45,105.86) --
	(148.87,105.86) --
	(149.29,105.86) --
	(149.72,105.86) --
	(150.14,105.89) --
	(150.56,105.89) --
	(150.98,105.89) --
	(151.41,105.89) --
	(151.83,106.22) --
	(152.25,105.94) --
	(152.67,106.03) --
	(153.10,106.03) --
	(153.52,106.03) --
	(153.94,106.03) --
	(154.36,106.03) --
	(154.79,106.03) --
	(155.21,106.03) --
	(155.63,106.03) --
	(156.05,106.03) --
	(156.48,106.03) --
	(156.90,105.64) --
	(157.32,105.64) --
	(157.74,105.64) --
	(158.17,105.64) --
	(158.59,105.55) --
	(159.01,106.23) --
	(159.43,106.23) --
	(159.86,106.23) --
	(160.28,106.23) --
	(160.70,106.23) --
	(161.12,106.24) --
	(161.55,106.24) --
	(161.97,106.24) --
	(162.39,105.89) --
	(162.81,106.04) --
	(163.23,106.04) --
	(163.66,106.06) --
	(164.08,106.22) --
	(164.50,106.22) --
	(164.92,106.22) --
	(165.35,106.38) --
	(165.77,106.38) --
	(166.19,106.38) --
	(166.61,106.38) --
	(167.04,106.38) --
	(167.46,106.07) --
	(167.88,106.07) --
	(168.30,106.09) --
	(168.73,106.09) --
	(169.15,106.09) --
	(169.57,106.09) --
	(169.99,106.09) --
	(170.42,106.09) --
	(170.84,106.09) --
	(171.26,106.09) --
	(171.68,106.09) --
	(172.11,106.07) --
	(172.53,106.07) --
	(172.95,106.07) --
	(173.37,106.07) --
	(173.80,106.07) --
	(174.22,106.07) --
	(174.64,106.07) --
	(175.06,106.07) --
	(175.48,106.07) --
	(175.91,106.07) --
	(176.33,106.07) --
	(176.75,106.07) --
	(177.17,106.07) --
	(177.60,106.07) --
	(178.02,106.07) --
	(178.44,106.07) --
	(178.86,106.07) --
	(179.29,106.07) --
	(179.71,106.07) --
	(180.13,106.07) --
	(180.55,106.07) --
	(180.98,106.07) --
	(181.40,106.07) --
	(181.82,106.07) --
	(182.24,106.07) --
	(182.67,106.07) --
	(183.09,106.07) --
	(183.51,106.07) --
	(183.93,106.07) --
	(184.36,106.07) --
	(184.78,106.07) --
	(185.20,106.07) --
	(185.62,106.07) --
	(186.05,105.82) --
	(186.47,105.82) --
	(186.89,105.82) --
	(187.31,105.82) --
	(187.74,105.82) --
	(188.16,105.82) --
	(188.58,105.82) --
	(189.00,105.82) --
	(189.42,105.82) --
	(189.85,105.82) --
	(190.27,105.82) --
	(190.69,105.82) --
	(191.11,105.82) --
	(191.54,105.82) --
	(191.96,105.82) --
	(192.38,106.49) --
	(192.80,106.49) --
	(193.23,106.49) --
	(193.65,106.49) --
	(194.07,106.49) --
	(194.49,106.49) --
	(194.92,106.49) --
	(195.34,106.49) --
	(195.76,106.49) --
	(196.18,106.49) --
	(196.61,106.49) --
	(197.03,106.49) --
	(197.45,106.49) --
	(197.87,106.49) --
	(198.30,106.49) --
	(198.72,106.49) --
	(199.14,106.49) --
	(199.56,106.49) --
	(199.99,106.49) --
	(200.41,106.49) --
	(200.83,106.49) --
	(201.25,106.49) --
	(201.68,106.49) --
	(202.10,106.49) --
	(202.52,106.49) --
	(202.94,106.49) --
	(203.36,106.49) --
	(203.79,106.49) --
	(204.21,106.49) --
	(204.63,106.49) --
	(205.05,106.49) --
	(205.48,106.49) --
	(205.90,106.49) --
	(206.32,106.49) --
	(206.74,106.38) --
	(207.17,106.38) --
	(207.59,106.38) --
	(208.01,106.38) --
	(208.43,106.38) --
	(208.86,106.38) --
	(209.28,106.38) --
	(209.70,106.38) --
	(210.12,106.38) --
	(210.55,106.59) --
	(210.97,106.59) --
	(211.39,106.58) --
	(211.81,106.58) --
	(212.24,106.58) --
	(212.66,106.58) --
	(213.08,106.58) --
	(213.50,106.58) --
	(213.93,106.58) --
	(214.35,106.58) --
	(214.77,106.58) --
	(215.19,106.58) --
	(215.61,106.58) --
	(216.04,106.58) --
	(216.46,107.29) --
	(216.88,107.29) --
	(217.30,107.29) --
	(217.73,107.29) --
	(218.15,107.29) --
	(218.57,107.29) --
	(218.99,107.29) --
	(219.42,107.29) --
	(219.84,107.29) --
	(220.26,107.29) --
	(220.68,107.29) --
	(221.11,107.29) --
	(221.53,107.29) --
	(221.95,107.29) --
	(222.37,107.29) --
	(222.80,107.29) --
	(223.22,107.29) --
	(223.64,107.33) --
	(224.06,107.33) --
	(224.49,107.33) --
	(224.91,107.33) --
	(225.33,107.33) --
	(225.75,107.33) --
	(226.18,107.33) --
	(226.60,107.33) --
	(227.02,107.33) --
	(227.44,107.33) --
	(227.87,107.33) --
	(228.29,107.33) --
	(228.71,107.41) --
	(229.13,107.41) --
	(229.55,107.41) --
	(229.98,107.41) --
	(230.40,107.41) --
	(230.82,107.41) --
	(231.24,107.41) --
	(231.67,107.41) --
	(232.09,107.41) --
	(232.51,107.41) --
	(232.93,107.41) --
	(233.36,107.41) --
	(233.78,107.41) --
	(234.20,107.41) --
	(234.62,107.41) --
	(235.05,107.41) --
	(235.47,107.41) --
	(235.89,107.41) --
	(236.31,107.41) --
	(236.74,107.41) --
	(237.16,107.41) --
	(237.58,107.41) --
	(238.00,107.41) --
	(238.43,107.41) --
	(238.85,107.41) --
	(239.27,107.41) --
	(239.69,107.41) --
	(240.12,107.41) --
	(240.54,107.41) --
	(240.96,107.41) --
	(241.38,107.41) --
	(241.80,107.41) --
	(242.23,107.41) --
	(242.65,107.41) --
	(243.07,107.41) --
	(243.49,107.41) --
	(243.92,107.41) --
	(244.34,107.41) --
	(244.76,107.41) --
	(245.18,107.41) --
	(245.61,107.41) --
	(246.03,107.41) --
	(246.45,107.41) --
	(246.87,107.41) --
	(247.30,107.41) --
	(247.72,107.41) --
	(248.14,107.41) --
	(248.56,107.41) --
	(248.99,107.41) --
	(249.41,107.41) --
	(249.83,107.41) --
	(250.25,107.41) --
	(250.68,107.41) --
	(251.10,107.41) --
	(251.52,107.41) --
	(251.94,107.41) --
	(252.37,107.41) --
	(252.79,107.41) --
	(253.21,107.41) --
	(253.63,107.41) --
	(254.06,107.41) --
	(254.48,107.41) --
	(254.90,107.41) --
	(255.32,107.41) --
	(255.74,107.41) --
	(256.17,107.41) --
	(256.59,107.41) --
	(257.01,107.41) --
	(257.43,107.42) --
	(257.86,107.42) --
	(258.28,107.42) --
	(258.70,107.42) --
	(259.12,107.42) --
	(259.55,107.42) --
	(259.97,107.42) --
	(260.39,107.42) --
	(260.81,107.42) --
	(261.24,107.42) --
	(261.66,107.42) --
	(262.08,107.42) --
	(262.50,107.42) --
	(262.93,107.42) --
	(263.35,107.42) --
	(263.77,107.42) --
	(264.19,107.42) --
	(264.62,107.42) --
	(265.04,107.42) --
	(265.46,107.42) --
	(265.88,107.42) --
	(266.31,107.42) --
	(266.73,107.42) --
	(267.15,107.42) --
	(267.57,107.42) --
	(268.00,107.42) --
	(268.42,107.33) --
	(268.84,107.33) --
	(269.26,107.33) --
	(269.68,107.33) --
	(270.11,107.33) --
	(270.53,107.33) --
	(270.95,107.33) --
	(271.37,107.33) --
	(271.80,107.33) --
	(272.22,107.33) --
	(272.64,107.33) --
	(273.06,107.33) --
	(273.49,107.33) --
	(273.91,107.33) --
	(274.33,107.33) --
	(274.75,107.33) --
	(275.18,107.33) --
	(275.60,107.33) --
	(276.02,107.33) --
	(276.44,107.33) --
	(276.87,107.33) --
	(277.29,107.33) --
	(277.71,107.33) --
	(278.13,107.33) --
	(278.56,107.33) --
	(278.98,107.33) --
	(279.40,107.33) --
	(279.82,107.33) --
	(280.25,107.33) --
	(280.67,107.33) --
	(281.09,107.33) --
	(281.51,107.33) --
	(281.93,107.33) --
	(282.36,107.33) --
	(282.78,107.33) --
	(283.20,105.92) --
	(283.62,105.92) --
	(284.05,105.92) --
	(284.47,105.92) --
	(284.89,105.92) --
	(285.31,105.92) --
	(285.74,105.92) --
	(286.16,105.92) --
	(286.58,105.92) --
	(287.00,105.92) --
	(287.43,105.92) --
	(287.85,105.92) --
	(288.27,105.92) --
	(288.69,105.92) --
	(289.12,105.92) --
	(289.54,105.92) --
	(289.96,105.92) --
	(290.38,105.92) --
	(290.81,105.92) --
	(291.23,105.92) --
	(291.65,105.92) --
	(292.07,105.92) --
	(292.50,105.92) --
	(292.92,105.92) --
	(293.34,105.92) --
	(293.76,105.92) --
	(294.19,105.92) --
	(294.61,105.92) --
	(295.03,105.92) --
	(295.45,105.92) --
	(295.87,105.92) --
	(296.30,105.92) --
	(296.72,105.92) --
	(297.14,105.92) --
	(297.56,105.92) --
	(297.99,105.92) --
	(298.41,105.92) --
	(298.83,105.92) --
	(299.25,105.92) --
	(299.68,105.92) --
	(300.10,105.92) --
	(300.52,105.92) --
	(300.94,105.92) --
	(301.37,105.92) --
	(301.79,105.92) --
	(302.21,105.92) --
	(302.63,105.92) --
	(303.06,105.92) --
	(303.48,105.92) --
	(303.90,105.92) --
	(304.32,105.92) --
	(304.75,105.92) --
	(305.17,105.92) --
	(305.59,105.92) --
	(306.01,105.92) --
	(306.44,105.92) --
	(306.86,105.92) --
	(307.28,105.92) --
	(307.70,105.92) --
	(308.12,105.92) --
	(308.55,105.92) --
	(308.97,105.92) --
	(309.39,105.92) --
	(309.81,105.92) --
	(310.24,105.92) --
	(310.66,105.92) --
	(311.08,105.92) --
	(311.50,105.92) --
	(311.93,105.92) --
	(312.35,105.92) --
	(312.77,105.92) --
	(313.19,105.92) --
	(313.62,105.92) --
	(314.04,105.92) --
	(314.46,105.92) --
	(314.88,105.92) --
	(315.31,105.92) --
	(315.73,105.92) --
	(316.15,105.92) --
	(316.57,105.92) --
	(317.00,105.92) --
	(317.42,105.92) --
	(317.84,105.92) --
	(318.26,105.92) --
	(318.69,105.92) --
	(319.11,105.92) --
	(319.53,105.92) --
	(319.95,105.92) --
	(320.38,105.92) --
	(320.80,105.92) --
	(321.22,105.92) --
	(321.64,105.92) --
	(322.06,105.92) --
	(322.49,105.92) --
	(322.91,105.92) --
	(323.33,105.92) --
	(323.75,105.92) --
	(324.18,105.92) --
	(324.60,105.92) --
	(325.02,105.92) --
	(325.44,105.92) --
	(325.87,105.92) --
	(326.29,105.92) --
	(326.71,105.92) --
	(327.13,105.92) --
	(327.56,105.92) --
	(327.98,105.92) --
	(328.40,105.92) --
	(328.82,105.92) --
	(329.25,105.92) --
	(329.67,105.92) --
	(330.09,105.92) --
	(330.51,105.92) --
	(330.94,105.92) --
	(331.36,105.92) --
	(331.78,105.92) --
	(332.20,105.92) --
	(332.63,105.92) --
	(333.05,105.92) --
	(333.47,105.92) --
	(333.89,105.92) --
	(334.32,105.92) --
	(334.74,105.92) --
	(335.16,105.92) --
	(335.58,105.92) --
	(336.00,105.92) --
	(336.43,105.92) --
	(336.85,105.92) --
	(337.27,105.92) --
	(337.69,105.92) --
	(338.12,105.92) --
	(338.54,105.92) --
	(338.96,105.92) --
	(339.38,105.92) --
	(339.81,105.92) --
	(340.23,105.92) --
	(340.65,105.92) --
	(341.07,105.92) --
	(341.50,105.92) --
	(341.92,105.92) --
	(342.34,105.92) --
	(342.76,105.92) --
	(343.19,105.92) --
	(343.61,105.92) --
	(344.03,105.92) --
	(344.45,105.92) --
	(344.88,105.92) --
	(345.30,105.92) --
	(345.72,105.92) --
	(346.14,105.92) --
	(346.57,105.92) --
	(346.99,105.92) --
	(347.41,105.92) --
	(347.83,105.92) --
	(348.25,105.92) --
	(348.68,105.92) --
	(349.10,105.92) --
	(349.52,105.92) --
	(349.52,127.52) --
	(349.10,127.52) --
	(348.68,127.52) --
	(348.25,127.52) --
	(347.83,127.52) --
	(347.41,127.52) --
	(346.99,127.52) --
	(346.57,127.52) --
	(346.14,127.52) --
	(345.72,127.52) --
	(345.30,127.52) --
	(344.88,127.52) --
	(344.45,127.52) --
	(344.03,127.52) --
	(343.61,127.52) --
	(343.19,127.52) --
	(342.76,127.52) --
	(342.34,127.52) --
	(341.92,127.52) --
	(341.50,127.52) --
	(341.07,127.52) --
	(340.65,127.52) --
	(340.23,127.52) --
	(339.81,127.52) --
	(339.38,127.52) --
	(338.96,127.52) --
	(338.54,127.52) --
	(338.12,127.52) --
	(337.69,127.52) --
	(337.27,127.52) --
	(336.85,127.52) --
	(336.43,127.52) --
	(336.00,127.52) --
	(335.58,127.52) --
	(335.16,127.52) --
	(334.74,127.52) --
	(334.32,127.52) --
	(333.89,127.52) --
	(333.47,127.52) --
	(333.05,127.52) --
	(332.63,127.52) --
	(332.20,127.52) --
	(331.78,127.52) --
	(331.36,127.52) --
	(330.94,127.52) --
	(330.51,127.52) --
	(330.09,127.52) --
	(329.67,127.52) --
	(329.25,127.52) --
	(328.82,127.52) --
	(328.40,127.52) --
	(327.98,127.52) --
	(327.56,127.52) --
	(327.13,127.52) --
	(326.71,127.52) --
	(326.29,127.52) --
	(325.87,127.52) --
	(325.44,127.52) --
	(325.02,127.52) --
	(324.60,127.52) --
	(324.18,127.52) --
	(323.75,127.52) --
	(323.33,127.52) --
	(322.91,127.52) --
	(322.49,127.52) --
	(322.06,127.52) --
	(321.64,127.52) --
	(321.22,127.52) --
	(320.80,127.52) --
	(320.38,127.52) --
	(319.95,127.52) --
	(319.53,127.52) --
	(319.11,127.52) --
	(318.69,127.52) --
	(318.26,127.52) --
	(317.84,127.52) --
	(317.42,127.52) --
	(317.00,127.52) --
	(316.57,127.52) --
	(316.15,127.52) --
	(315.73,127.52) --
	(315.31,127.52) --
	(314.88,127.52) --
	(314.46,127.52) --
	(314.04,127.52) --
	(313.62,127.52) --
	(313.19,127.52) --
	(312.77,127.52) --
	(312.35,127.52) --
	(311.93,127.52) --
	(311.50,127.52) --
	(311.08,127.52) --
	(310.66,127.52) --
	(310.24,127.52) --
	(309.81,127.52) --
	(309.39,127.52) --
	(308.97,127.52) --
	(308.55,127.52) --
	(308.12,127.52) --
	(307.70,127.52) --
	(307.28,127.52) --
	(306.86,127.52) --
	(306.44,127.52) --
	(306.01,127.52) --
	(305.59,127.52) --
	(305.17,127.52) --
	(304.75,127.52) --
	(304.32,127.52) --
	(303.90,127.52) --
	(303.48,127.52) --
	(303.06,127.52) --
	(302.63,127.52) --
	(302.21,127.52) --
	(301.79,127.52) --
	(301.37,127.52) --
	(300.94,127.52) --
	(300.52,127.52) --
	(300.10,127.52) --
	(299.68,127.52) --
	(299.25,127.52) --
	(298.83,127.52) --
	(298.41,127.52) --
	(297.99,127.52) --
	(297.56,127.52) --
	(297.14,127.52) --
	(296.72,127.52) --
	(296.30,127.52) --
	(295.87,127.52) --
	(295.45,127.52) --
	(295.03,127.52) --
	(294.61,127.52) --
	(294.19,127.52) --
	(293.76,127.52) --
	(293.34,127.52) --
	(292.92,127.52) --
	(292.50,127.52) --
	(292.07,127.52) --
	(291.65,127.52) --
	(291.23,127.52) --
	(290.81,127.52) --
	(290.38,127.52) --
	(289.96,127.52) --
	(289.54,127.52) --
	(289.12,127.52) --
	(288.69,127.52) --
	(288.27,127.52) --
	(287.85,127.52) --
	(287.43,127.52) --
	(287.00,127.52) --
	(286.58,127.52) --
	(286.16,127.52) --
	(285.74,127.52) --
	(285.31,127.52) --
	(284.89,127.52) --
	(284.47,127.52) --
	(284.05,127.52) --
	(283.62,127.52) --
	(283.20,127.52) --
	(282.78,129.13) --
	(282.36,129.13) --
	(281.93,129.13) --
	(281.51,129.13) --
	(281.09,129.13) --
	(280.67,129.13) --
	(280.25,129.13) --
	(279.82,129.13) --
	(279.40,129.13) --
	(278.98,129.13) --
	(278.56,129.13) --
	(278.13,129.13) --
	(277.71,129.13) --
	(277.29,129.13) --
	(276.87,129.13) --
	(276.44,129.13) --
	(276.02,129.13) --
	(275.60,129.13) --
	(275.18,129.13) --
	(274.75,129.13) --
	(274.33,129.13) --
	(273.91,129.13) --
	(273.49,129.13) --
	(273.06,129.13) --
	(272.64,129.13) --
	(272.22,129.13) --
	(271.80,129.13) --
	(271.37,129.13) --
	(270.95,129.13) --
	(270.53,129.13) --
	(270.11,129.13) --
	(269.68,129.13) --
	(269.26,129.13) --
	(268.84,129.13) --
	(268.42,129.13) --
	(268.00,129.29) --
	(267.57,129.29) --
	(267.15,129.29) --
	(266.73,129.29) --
	(266.31,129.29) --
	(265.88,129.29) --
	(265.46,129.29) --
	(265.04,129.29) --
	(264.62,129.29) --
	(264.19,129.29) --
	(263.77,129.29) --
	(263.35,129.29) --
	(262.93,129.29) --
	(262.50,129.29) --
	(262.08,129.29) --
	(261.66,129.29) --
	(261.24,129.29) --
	(260.81,129.29) --
	(260.39,129.29) --
	(259.97,129.29) --
	(259.55,129.29) --
	(259.12,129.29) --
	(258.70,129.29) --
	(258.28,129.29) --
	(257.86,129.29) --
	(257.43,129.29) --
	(257.01,129.34) --
	(256.59,129.34) --
	(256.17,129.34) --
	(255.74,129.34) --
	(255.32,129.34) --
	(254.90,129.34) --
	(254.48,129.34) --
	(254.06,129.34) --
	(253.63,129.34) --
	(253.21,129.34) --
	(252.79,129.34) --
	(252.37,129.34) --
	(251.94,129.34) --
	(251.52,129.34) --
	(251.10,129.34) --
	(250.68,129.34) --
	(250.25,129.34) --
	(249.83,129.34) --
	(249.41,129.34) --
	(248.99,129.34) --
	(248.56,129.34) --
	(248.14,129.34) --
	(247.72,129.34) --
	(247.30,129.34) --
	(246.87,129.34) --
	(246.45,129.34) --
	(246.03,129.34) --
	(245.61,129.34) --
	(245.18,129.34) --
	(244.76,129.34) --
	(244.34,129.34) --
	(243.92,129.34) --
	(243.49,129.34) --
	(243.07,129.34) --
	(242.65,129.34) --
	(242.23,129.34) --
	(241.80,129.34) --
	(241.38,129.34) --
	(240.96,129.34) --
	(240.54,129.34) --
	(240.12,129.34) --
	(239.69,129.34) --
	(239.27,129.34) --
	(238.85,129.34) --
	(238.43,129.34) --
	(238.00,129.34) --
	(237.58,129.34) --
	(237.16,129.34) --
	(236.74,129.34) --
	(236.31,129.34) --
	(235.89,129.34) --
	(235.47,129.34) --
	(235.05,129.34) --
	(234.62,129.34) --
	(234.20,129.34) --
	(233.78,129.34) --
	(233.36,129.34) --
	(232.93,129.34) --
	(232.51,129.34) --
	(232.09,129.34) --
	(231.67,129.34) --
	(231.24,129.34) --
	(230.82,129.34) --
	(230.40,129.34) --
	(229.98,129.34) --
	(229.55,129.34) --
	(229.13,129.34) --
	(228.71,129.34) --
	(228.29,129.33) --
	(227.87,129.33) --
	(227.44,129.33) --
	(227.02,129.33) --
	(226.60,129.33) --
	(226.18,129.33) --
	(225.75,129.33) --
	(225.33,129.33) --
	(224.91,129.33) --
	(224.49,129.33) --
	(224.06,129.33) --
	(223.64,129.33) --
	(223.22,129.35) --
	(222.80,129.35) --
	(222.37,129.35) --
	(221.95,129.35) --
	(221.53,129.35) --
	(221.11,129.35) --
	(220.68,129.35) --
	(220.26,129.35) --
	(219.84,129.35) --
	(219.42,129.35) --
	(218.99,129.35) --
	(218.57,129.35) --
	(218.15,129.35) --
	(217.73,129.35) --
	(217.30,129.35) --
	(216.88,129.35) --
	(216.46,129.35) --
	(216.04,128.64) --
	(215.61,128.64) --
	(215.19,128.64) --
	(214.77,128.64) --
	(214.35,128.64) --
	(213.93,128.64) --
	(213.50,128.64) --
	(213.08,128.64) --
	(212.66,128.64) --
	(212.24,128.64) --
	(211.81,128.64) --
	(211.39,128.64) --
	(210.97,128.72) --
	(210.55,128.72) --
	(210.12,128.55) --
	(209.70,128.55) --
	(209.28,128.55) --
	(208.86,128.55) --
	(208.43,128.55) --
	(208.01,128.55) --
	(207.59,128.55) --
	(207.17,128.55) --
	(206.74,128.55) --
	(206.32,128.75) --
	(205.90,128.75) --
	(205.48,128.75) --
	(205.05,128.75) --
	(204.63,128.75) --
	(204.21,128.75) --
	(203.79,128.75) --
	(203.36,128.75) --
	(202.94,128.75) --
	(202.52,128.75) --
	(202.10,128.75) --
	(201.68,128.75) --
	(201.25,128.75) --
	(200.83,128.75) --
	(200.41,128.75) --
	(199.99,128.75) --
	(199.56,128.75) --
	(199.14,128.75) --
	(198.72,128.75) --
	(198.30,128.75) --
	(197.87,128.75) --
	(197.45,128.75) --
	(197.03,128.75) --
	(196.61,128.75) --
	(196.18,128.75) --
	(195.76,128.75) --
	(195.34,128.75) --
	(194.92,128.75) --
	(194.49,128.75) --
	(194.07,128.75) --
	(193.65,128.75) --
	(193.23,128.75) --
	(192.80,128.75) --
	(192.38,128.75) --
	(191.96,128.07) --
	(191.54,128.07) --
	(191.11,128.07) --
	(190.69,128.07) --
	(190.27,128.07) --
	(189.85,128.07) --
	(189.42,128.07) --
	(189.00,128.07) --
	(188.58,128.07) --
	(188.16,128.07) --
	(187.74,128.07) --
	(187.31,128.07) --
	(186.89,128.07) --
	(186.47,128.07) --
	(186.05,128.07) --
	(185.62,128.42) --
	(185.20,128.42) --
	(184.78,128.42) --
	(184.36,128.42) --
	(183.93,128.42) --
	(183.51,128.42) --
	(183.09,128.42) --
	(182.67,128.42) --
	(182.24,128.42) --
	(181.82,128.42) --
	(181.40,128.42) --
	(180.98,128.42) --
	(180.55,128.42) --
	(180.13,128.42) --
	(179.71,128.42) --
	(179.29,128.42) --
	(178.86,128.42) --
	(178.44,128.42) --
	(178.02,128.42) --
	(177.60,128.42) --
	(177.17,128.42) --
	(176.75,128.42) --
	(176.33,128.42) --
	(175.91,128.42) --
	(175.48,128.42) --
	(175.06,128.42) --
	(174.64,128.42) --
	(174.22,128.42) --
	(173.80,128.42) --
	(173.37,128.42) --
	(172.95,128.42) --
	(172.53,128.42) --
	(172.11,128.42) --
	(171.68,128.52) --
	(171.26,128.52) --
	(170.84,128.52) --
	(170.42,128.52) --
	(169.99,128.52) --
	(169.57,128.52) --
	(169.15,128.52) --
	(168.73,128.52) --
	(168.30,128.52) --
	(167.88,128.56) --
	(167.46,128.56) --
	(167.04,128.98) --
	(166.61,128.98) --
	(166.19,128.98) --
	(165.77,128.98) --
	(165.35,128.98) --
	(164.92,128.87) --
	(164.50,128.87) --
	(164.08,128.87) --
	(163.66,128.77) --
	(163.23,128.82) --
	(162.81,128.82) --
	(162.39,128.74) --
	(161.97,129.20) --
	(161.55,129.20) --
	(161.12,129.20) --
	(160.70,129.27) --
	(160.28,129.27) --
	(159.86,129.27) --
	(159.43,129.27) --
	(159.01,129.27) --
	(158.59,128.67) --
	(158.17,128.84) --
	(157.74,128.84) --
	(157.32,128.84) --
	(156.90,128.84) --
	(156.48,129.36) --
	(156.05,129.36) --
	(155.63,129.36) --
	(155.21,129.36) --
	(154.79,129.36) --
	(154.36,129.36) --
	(153.94,129.36) --
	(153.52,129.36) --
	(153.10,129.36) --
	(152.67,129.36) --
	(152.25,129.34) --
	(151.83,129.73) --
	(151.41,129.44) --
	(150.98,129.44) --
	(150.56,129.44) --
	(150.14,129.44) --
	(149.72,129.50) --
	(149.29,129.50) --
	(148.87,129.50) --
	(148.45,129.50) --
	(148.03,129.50) --
	(147.61,128.91) --
	(147.18,128.91) --
	(146.76,128.91) --
	(146.34,128.74) --
	(145.92,128.74) --
	(145.49,128.74) --
	(145.07,127.57) --
	(144.65,127.57) --
	(144.23,127.05) --
	(143.80,127.05) --
	(143.38,127.05) --
	(142.96,127.05) --
	(142.54,127.05) --
	(142.11,127.27) --
	(141.69,126.73) --
	(141.27,127.10) --
	(140.85,128.60) --
	(140.42,128.60) --
	(140.00,127.73) --
	(139.58,127.73) --
	(139.16,127.73) --
	(138.73,127.73) --
	(138.31,127.88) --
	(137.89,128.02) --
	(137.47,128.63) --
	(137.04,128.63) --
	(136.62,128.63) --
	(136.20,128.62) --
	(135.78,128.68) --
	(135.35,128.68) --
	(134.93,128.68) --
	(134.51,128.85) --
	(134.09,128.77) --
	(133.67,128.77) --
	(133.24,128.77) --
	(132.82,128.77) --
	(132.40,128.77) --
	(131.98,128.77) --
	(131.55,128.77) --
	(131.13,128.39) --
	(130.71,128.39) --
	(130.29,128.39) --
	(129.86,128.44) --
	(129.44,128.44) --
	(129.02,128.53) --
	(128.60,128.53) --
	(128.17,128.37) --
	(127.75,128.37) --
	(127.33,127.67) --
	(126.91,127.75) --
	(126.48,127.75) --
	(126.06,132.66) --
	(125.64,132.66) --
	(125.22,133.66) --
	(124.79,132.52) --
	(124.37,132.52) --
	(123.95,130.19) --
	(123.53,130.19) --
	(123.10,130.19) --
	(122.68,130.47) --
	(122.26,130.55) --
	(121.84,130.33) --
	(121.42,130.33) --
	(120.99,130.33) --
	(120.57,130.33) --
	(120.15,129.09) --
	(119.73,122.80) --
	(119.30,122.60) --
	(118.88,122.60) --
	(118.46,121.70) --
	(118.04,121.70) --
	(117.61,121.70) --
	(117.19,124.79) --
	(116.77,124.79) --
	(116.35,124.79) --
	(115.92,123.95) --
	(115.50,123.95) --
	(115.08,123.95) --
	(114.66,123.95) --
	(114.23,123.95) --
	(113.81,123.95) --
	(113.39,123.95) --
	(112.97,123.97) --
	(112.54,123.88) --
	(112.12,123.00) --
	(111.70,123.62) --
	(111.28,123.50) --
	(110.85,123.50) --
	(110.43,123.50) --
	(110.01,123.38) --
	(109.59,122.49) --
	(109.16,122.18) --
	(108.74,122.18) --
	(108.32,121.55) --
	(107.90,121.55) --
	(107.48,121.55) --
	(107.05,121.65) --
	(106.63,121.65) --
	(106.21,121.65) --
	(105.79,121.65) --
	(105.36,121.03) --
	(104.94,121.55) --
	(104.52,122.21) --
	(104.10,123.91) --
	(103.67,124.04) --
	(103.25,124.31) --
	(102.83,125.25) --
	(102.41,125.25) --
	(101.98,125.98) --
	(101.56,126.10) --
	(101.14,126.10) --
	(100.72,126.47) --
	(100.29,128.41) --
	( 99.87,129.85) --
	( 99.45,129.85) --
	( 99.03,131.40) --
	( 98.60,131.23) --
	( 98.18,131.08) --
	( 97.76,126.55) --
	( 97.34,126.33) --
	( 96.91,127.31) --
	( 96.49,126.48) --
	( 96.07,126.82) --
	( 95.65,126.82) --
	( 95.22,126.82) --
	( 94.80,126.82) --
	( 94.38,126.60) --
	( 93.96,129.88) --
	( 93.54,129.68) --
	( 93.11,129.68) --
	( 92.69,130.03) --
	( 92.27,130.03) --
	( 91.85,133.28) --
	( 91.42,133.28) --
	( 91.00,135.20) --
	( 90.58,137.63) --
	( 90.16,137.63) --
	( 89.73,136.09) --
	( 89.31,136.67) --
	( 88.89,139.01) --
	( 88.47,141.23) --
	( 88.04,141.23) --
	( 87.62,141.04) --
	( 87.20,139.33) --
	( 86.78,139.33) --
	( 86.35,138.25) --
	( 85.93,133.84) --
	( 85.51,134.98) --
	( 85.09,134.98) --
	( 84.66,135.53) --
	( 84.24,135.53) --
	( 83.82,142.34) --
	( 83.40,142.34) --
	( 82.97,142.34) --
	( 82.55,142.34) --
	( 82.13,143.12) --
	( 81.71,143.12) --
	( 81.29,152.70) --
	( 80.86,153.66) --
	( 80.44,153.66) --
	( 80.02,153.86) --
	( 79.60,159.74) --
	( 79.17,159.74) --
	( 78.75,159.74) --
	( 78.33,159.74) --
	( 77.91,159.74) --
	( 77.48,166.35) --
	( 77.06,163.40) --
	( 76.64,163.40) --
	( 76.22,161.02) --
	( 75.79,170.13) --
	( 75.37,170.13) --
	( 74.95,170.13) --
	( 74.53,171.95) --
	( 74.10,171.95) --
	( 73.68,173.93) --
	( 73.26,173.93) --
	( 72.84,173.93) --
	( 72.41,173.93) --
	( 71.99,173.93) --
	( 71.57,173.93) --
	( 71.15,181.80) --
	( 70.72,181.80) --
	( 70.30,181.80) --
	( 69.88,183.36) --
	( 69.46,183.43) --
	( 69.03,183.43) --
	( 68.61,184.68) --
	( 68.19,184.68) --
	( 67.77,184.68) --
	( 67.35,184.68) --
	( 66.92,184.68) --
	( 66.50,192.75) --
	cycle;
\definecolor{drawColor}{RGB}{0,0,0}

\path[draw=drawColor,line width= 0.4pt,dash pattern=on 7pt off 3pt ,line join=round,line cap=round] ( 42.00,129.40) -- (361.35,129.40);

\path[draw=drawColor,line width= 0.4pt,line join=round,line cap=round] ( 66.50,136.25) --
	( 66.92,126.09) --
	( 67.35,126.09) --
	( 67.77,126.09) --
	( 68.19,126.09) --
	( 68.61,126.09) --
	( 69.03,131.25) --
	( 69.46,131.25) --
	( 69.88,136.84) --
	( 70.30,138.19) --
	( 70.72,138.19) --
	( 71.15,138.19) --
	( 71.57,128.33) --
	( 71.99,128.33) --
	( 72.41,128.33) --
	( 72.84,128.33) --
	( 73.26,128.33) --
	( 73.68,128.33) --
	( 74.10,128.35) --
	( 74.53,128.35) --
	( 74.95,128.26) --
	( 75.37,128.26) --
	( 75.79,128.26) --
	( 76.22,122.15) --
	( 76.64,126.55) --
	( 77.06,126.55) --
	( 77.48,132.89) --
	( 77.91,125.73) --
	( 78.33,125.73) --
	( 78.75,125.73) --
	( 79.17,125.73) --
	( 79.60,125.73) --
	( 80.02,121.77) --
	( 80.44,122.34) --
	( 80.86,122.34) --
	( 81.29,122.01) --
	( 81.71,112.75) --
	( 82.13,112.75) --
	( 82.55,112.60) --
	( 82.97,112.60) --
	( 83.40,112.60) --
	( 83.82,112.60) --
	( 84.24,106.73) --
	( 84.66,106.73) --
	( 85.09,106.74) --
	( 85.51,106.74) --
	( 85.93,106.20) --
	( 86.35,111.22) --
	( 86.78,113.12) --
	( 87.20,113.12) --
	( 87.62,115.57) --
	( 88.04,116.13) --
	( 88.47,116.13) --
	( 88.89,114.64) --
	( 89.31,112.69) --
	( 89.73,112.44) --
	( 90.16,114.50) --
	( 90.58,114.50) --
	( 91.00,112.44) --
	( 91.42,111.12) --
	( 91.85,111.12) --
	( 92.27,108.56) --
	( 92.69,108.56) --
	( 93.11,109.09) --
	( 93.54,109.09) --
	( 93.96,109.66) --
	( 94.38,106.82) --
	( 94.80,107.20) --
	( 95.22,107.20) --
	( 95.65,107.20) --
	( 96.07,107.20) --
	( 96.49,107.06) --
	( 96.91,108.00) --
	( 97.34,107.57) --
	( 97.76,108.07) --
	( 98.18,112.46) --
	( 98.60,112.89) --
	( 99.03,113.32) --
	( 99.45,112.12) --
	( 99.87,112.12) --
	(100.29,111.12) --
	(100.72,109.42) --
	(101.14,109.20) --
	(101.56,109.20) --
	(101.98,109.19) --
	(102.41,108.73) --
	(102.83,108.73) --
	(103.25,108.06) --
	(103.67,107.92) --
	(104.10,107.89) --
	(104.52,106.52) --
	(104.94,106.09) --
	(105.36,105.71) --
	(105.79,106.44) --
	(106.21,106.44) --
	(106.63,106.44) --
	(107.05,106.44) --
	(107.48,106.44) --
	(107.90,106.44) --
	(108.32,106.44) --
	(108.74,107.10) --
	(109.16,107.10) --
	(109.59,107.55) --
	(110.01,108.44) --
	(110.43,108.63) --
	(110.85,108.63) --
	(111.28,108.63) --
	(111.70,108.82) --
	(112.12,108.40) --
	(112.54,109.36) --
	(112.97,109.52) --
	(113.39,109.64) --
	(113.81,109.64) --
	(114.23,109.64) --
	(114.66,109.64) --
	(115.08,109.64) --
	(115.50,109.64) --
	(115.92,109.64) --
	(116.35,110.49) --
	(116.77,110.49) --
	(117.19,110.49) --
	(117.61,107.75) --
	(118.04,107.75) --
	(118.46,107.75) --
	(118.88,108.64) --
	(119.30,108.64) --
	(119.73,108.96) --
	(120.15,114.99) --
	(120.57,116.25) --
	(120.99,116.25) --
	(121.42,116.25) --
	(121.84,116.25) --
	(122.26,116.52) --
	(122.68,116.51) --
	(123.10,116.43) --
	(123.53,116.43) --
	(123.95,116.43) --
	(124.37,118.87) --
	(124.79,118.87) --
	(125.22,120.11) --
	(125.64,119.24) --
	(126.06,119.24) --
	(126.48,114.57) --
	(126.91,114.57) --
	(127.33,114.55) --
	(127.75,115.33) --
	(128.17,115.33) --
	(128.60,115.52) --
	(129.02,115.52) --
	(129.44,115.49) --
	(129.86,115.49) --
	(130.29,115.49) --
	(130.71,115.49) --
	(131.13,115.49) --
	(131.55,115.91) --
	(131.98,115.91) --
	(132.40,115.91) --
	(132.82,115.91) --
	(133.24,115.91) --
	(133.67,115.91) --
	(134.09,115.91) --
	(134.51,116.04) --
	(134.93,115.92) --
	(135.35,115.92) --
	(135.78,115.92) --
	(136.20,115.91) --
	(136.62,115.96) --
	(137.04,115.96) --
	(137.47,115.96) --
	(137.89,115.42) --
	(138.31,115.34) --
	(138.73,115.23) --
	(139.16,115.23) --
	(139.58,115.23) --
	(140.00,115.23) --
	(140.42,116.16) --
	(140.85,116.16) --
	(141.27,114.76) --
	(141.69,114.44) --
	(142.11,115.00) --
	(142.54,114.84) --
	(142.96,114.84) --
	(143.38,114.84) --
	(143.80,114.84) --
	(144.23,114.84) --
	(144.65,115.38) --
	(145.07,115.38) --
	(145.49,116.56) --
	(145.92,116.56) --
	(146.34,116.56) --
	(146.76,116.76) --
	(147.18,116.76) --
	(147.61,116.76) --
	(148.03,117.38) --
	(148.45,117.38) --
	(148.87,117.38) --
	(149.29,117.38) --
	(149.72,117.38) --
	(150.14,117.36) --
	(150.56,117.36) --
	(150.98,117.36) --
	(151.41,117.36) --
	(151.83,117.68) --
	(152.25,117.34) --
	(152.67,117.40) --
	(153.10,117.40) --
	(153.52,117.40) --
	(153.94,117.40) --
	(154.36,117.40) --
	(154.79,117.40) --
	(155.21,117.40) --
	(155.63,117.40) --
	(156.05,117.40) --
	(156.48,117.40) --
	(156.90,116.94) --
	(157.32,116.94) --
	(157.74,116.94) --
	(158.17,116.94) --
	(158.59,116.80) --
	(159.01,117.46) --
	(159.43,117.46) --
	(159.86,117.46) --
	(160.28,117.46) --
	(160.70,117.46) --
	(161.12,117.44) --
	(161.55,117.44) --
	(161.97,117.44) --
	(162.39,117.02) --
	(162.81,117.14) --
	(163.23,117.14) --
	(163.66,117.13) --
	(164.08,117.26) --
	(164.50,117.26) --
	(164.92,117.26) --
	(165.35,117.41) --
	(165.77,117.41) --
	(166.19,117.41) --
	(166.61,117.41) --
	(167.04,117.41) --
	(167.46,117.03) --
	(167.88,117.03) --
	(168.30,117.02) --
	(168.73,117.02) --
	(169.15,117.02) --
	(169.57,117.02) --
	(169.99,117.02) --
	(170.42,117.02) --
	(170.84,117.02) --
	(171.26,117.02) --
	(171.68,117.02) --
	(172.11,116.96) --
	(172.53,116.96) --
	(172.95,116.96) --
	(173.37,116.96) --
	(173.80,116.96) --
	(174.22,116.96) --
	(174.64,116.96) --
	(175.06,116.96) --
	(175.48,116.96) --
	(175.91,116.96) --
	(176.33,116.96) --
	(176.75,116.96) --
	(177.17,116.96) --
	(177.60,116.96) --
	(178.02,116.96) --
	(178.44,116.96) --
	(178.86,116.96) --
	(179.29,116.96) --
	(179.71,116.96) --
	(180.13,116.96) --
	(180.55,116.96) --
	(180.98,116.96) --
	(181.40,116.96) --
	(181.82,116.96) --
	(182.24,116.96) --
	(182.67,116.96) --
	(183.09,116.96) --
	(183.51,116.96) --
	(183.93,116.96) --
	(184.36,116.96) --
	(184.78,116.96) --
	(185.20,116.96) --
	(185.62,116.96) --
	(186.05,116.66) --
	(186.47,116.66) --
	(186.89,116.66) --
	(187.31,116.66) --
	(187.74,116.66) --
	(188.16,116.66) --
	(188.58,116.66) --
	(189.00,116.66) --
	(189.42,116.66) --
	(189.85,116.66) --
	(190.27,116.66) --
	(190.69,116.66) --
	(191.11,116.66) --
	(191.54,116.66) --
	(191.96,116.66) --
	(192.38,117.35) --
	(192.80,117.35) --
	(193.23,117.35) --
	(193.65,117.35) --
	(194.07,117.35) --
	(194.49,117.35) --
	(194.92,117.35) --
	(195.34,117.35) --
	(195.76,117.35) --
	(196.18,117.35) --
	(196.61,117.35) --
	(197.03,117.35) --
	(197.45,117.35) --
	(197.87,117.35) --
	(198.30,117.35) --
	(198.72,117.35) --
	(199.14,117.35) --
	(199.56,117.35) --
	(199.99,117.35) --
	(200.41,117.35) --
	(200.83,117.35) --
	(201.25,117.35) --
	(201.68,117.35) --
	(202.10,117.35) --
	(202.52,117.35) --
	(202.94,117.35) --
	(203.36,117.35) --
	(203.79,117.35) --
	(204.21,117.35) --
	(204.63,117.35) --
	(205.05,117.35) --
	(205.48,117.35) --
	(205.90,117.35) --
	(206.32,117.35) --
	(206.74,117.19) --
	(207.17,117.19) --
	(207.59,117.19) --
	(208.01,117.19) --
	(208.43,117.19) --
	(208.86,117.19) --
	(209.28,117.19) --
	(209.70,117.19) --
	(210.12,117.19) --
	(210.55,117.39) --
	(210.97,117.39) --
	(211.39,117.35) --
	(211.81,117.35) --
	(212.24,117.35) --
	(212.66,117.35) --
	(213.08,117.35) --
	(213.50,117.35) --
	(213.93,117.35) --
	(214.35,117.35) --
	(214.77,117.35) --
	(215.19,117.35) --
	(215.61,117.35) --
	(216.04,117.35) --
	(216.46,118.07) --
	(216.88,118.07) --
	(217.30,118.07) --
	(217.73,118.07) --
	(218.15,118.07) --
	(218.57,118.07) --
	(218.99,118.07) --
	(219.42,118.07) --
	(219.84,118.07) --
	(220.26,118.07) --
	(220.68,118.07) --
	(221.11,118.07) --
	(221.53,118.07) --
	(221.95,118.07) --
	(222.37,118.07) --
	(222.80,118.07) --
	(223.22,118.07) --
	(223.64,118.09) --
	(224.06,118.09) --
	(224.49,118.09) --
	(224.91,118.09) --
	(225.33,118.09) --
	(225.75,118.09) --
	(226.18,118.09) --
	(226.60,118.09) --
	(227.02,118.09) --
	(227.44,118.09) --
	(227.87,118.09) --
	(228.29,118.09) --
	(228.71,118.13) --
	(229.13,118.13) --
	(229.55,118.13) --
	(229.98,118.13) --
	(230.40,118.13) --
	(230.82,118.13) --
	(231.24,118.13) --
	(231.67,118.13) --
	(232.09,118.13) --
	(232.51,118.13) --
	(232.93,118.13) --
	(233.36,118.13) --
	(233.78,118.13) --
	(234.20,118.13) --
	(234.62,118.13) --
	(235.05,118.13) --
	(235.47,118.13) --
	(235.89,118.13) --
	(236.31,118.13) --
	(236.74,118.13) --
	(237.16,118.13) --
	(237.58,118.13) --
	(238.00,118.13) --
	(238.43,118.13) --
	(238.85,118.13) --
	(239.27,118.13) --
	(239.69,118.13) --
	(240.12,118.13) --
	(240.54,118.13) --
	(240.96,118.13) --
	(241.38,118.13) --
	(241.80,118.13) --
	(242.23,118.13) --
	(242.65,118.13) --
	(243.07,118.13) --
	(243.49,118.13) --
	(243.92,118.13) --
	(244.34,118.13) --
	(244.76,118.13) --
	(245.18,118.13) --
	(245.61,118.13) --
	(246.03,118.13) --
	(246.45,118.13) --
	(246.87,118.13) --
	(247.30,118.13) --
	(247.72,118.13) --
	(248.14,118.13) --
	(248.56,118.13) --
	(248.99,118.13) --
	(249.41,118.13) --
	(249.83,118.13) --
	(250.25,118.13) --
	(250.68,118.13) --
	(251.10,118.13) --
	(251.52,118.13) --
	(251.94,118.13) --
	(252.37,118.13) --
	(252.79,118.13) --
	(253.21,118.13) --
	(253.63,118.13) --
	(254.06,118.13) --
	(254.48,118.13) --
	(254.90,118.13) --
	(255.32,118.13) --
	(255.74,118.13) --
	(256.17,118.13) --
	(256.59,118.13) --
	(257.01,118.13) --
	(257.43,118.11) --
	(257.86,118.11) --
	(258.28,118.11) --
	(258.70,118.11) --
	(259.12,118.11) --
	(259.55,118.11) --
	(259.97,118.11) --
	(260.39,118.11) --
	(260.81,118.11) --
	(261.24,118.11) --
	(261.66,118.11) --
	(262.08,118.11) --
	(262.50,118.11) --
	(262.93,118.11) --
	(263.35,118.11) --
	(263.77,118.11) --
	(264.19,118.11) --
	(264.62,118.11) --
	(265.04,118.11) --
	(265.46,118.11) --
	(265.88,118.11) --
	(266.31,118.11) --
	(266.73,118.11) --
	(267.15,118.11) --
	(267.57,118.11) --
	(268.00,118.11) --
	(268.42,117.98) --
	(268.84,117.98) --
	(269.26,117.98) --
	(269.68,117.98) --
	(270.11,117.98) --
	(270.53,117.98) --
	(270.95,117.98) --
	(271.37,117.98) --
	(271.80,117.98) --
	(272.22,117.98) --
	(272.64,117.98) --
	(273.06,117.98) --
	(273.49,117.98) --
	(273.91,117.98) --
	(274.33,117.98) --
	(274.75,117.98) --
	(275.18,117.98) --
	(275.60,117.98) --
	(276.02,117.98) --
	(276.44,117.98) --
	(276.87,117.98) --
	(277.29,117.98) --
	(277.71,117.98) --
	(278.13,117.98) --
	(278.56,117.98) --
	(278.98,117.98) --
	(279.40,117.98) --
	(279.82,117.98) --
	(280.25,117.98) --
	(280.67,117.98) --
	(281.09,117.98) --
	(281.51,117.98) --
	(281.93,117.98) --
	(282.36,117.98) --
	(282.78,117.98) --
	(283.20,116.45) --
	(283.62,116.45) --
	(284.05,116.45) --
	(284.47,116.45) --
	(284.89,116.45) --
	(285.31,116.45) --
	(285.74,116.45) --
	(286.16,116.45) --
	(286.58,116.45) --
	(287.00,116.45) --
	(287.43,116.45) --
	(287.85,116.45) --
	(288.27,116.45) --
	(288.69,116.45) --
	(289.12,116.45) --
	(289.54,116.45) --
	(289.96,116.45) --
	(290.38,116.45) --
	(290.81,116.45) --
	(291.23,116.45) --
	(291.65,116.45) --
	(292.07,116.45) --
	(292.50,116.45) --
	(292.92,116.45) --
	(293.34,116.45) --
	(293.76,116.45) --
	(294.19,116.45) --
	(294.61,116.45) --
	(295.03,116.45) --
	(295.45,116.45) --
	(295.87,116.45) --
	(296.30,116.45) --
	(296.72,116.45) --
	(297.14,116.45) --
	(297.56,116.45) --
	(297.99,116.45) --
	(298.41,116.45) --
	(298.83,116.45) --
	(299.25,116.45) --
	(299.68,116.45) --
	(300.10,116.45) --
	(300.52,116.45) --
	(300.94,116.45) --
	(301.37,116.45) --
	(301.79,116.45) --
	(302.21,116.45) --
	(302.63,116.45) --
	(303.06,116.45) --
	(303.48,116.45) --
	(303.90,116.45) --
	(304.32,116.45) --
	(304.75,116.45) --
	(305.17,116.45) --
	(305.59,116.45) --
	(306.01,116.45) --
	(306.44,116.45) --
	(306.86,116.45) --
	(307.28,116.45) --
	(307.70,116.45) --
	(308.12,116.45) --
	(308.55,116.45) --
	(308.97,116.45) --
	(309.39,116.45) --
	(309.81,116.45) --
	(310.24,116.45) --
	(310.66,116.45) --
	(311.08,116.45) --
	(311.50,116.45) --
	(311.93,116.45) --
	(312.35,116.45) --
	(312.77,116.45) --
	(313.19,116.45) --
	(313.62,116.45) --
	(314.04,116.45) --
	(314.46,116.45) --
	(314.88,116.45) --
	(315.31,116.45) --
	(315.73,116.45) --
	(316.15,116.45) --
	(316.57,116.45) --
	(317.00,116.45) --
	(317.42,116.45) --
	(317.84,116.45) --
	(318.26,116.45) --
	(318.69,116.45) --
	(319.11,116.45) --
	(319.53,116.45) --
	(319.95,116.45) --
	(320.38,116.45) --
	(320.80,116.45) --
	(321.22,116.45) --
	(321.64,116.45) --
	(322.06,116.45) --
	(322.49,116.45) --
	(322.91,116.45) --
	(323.33,116.45) --
	(323.75,116.45) --
	(324.18,116.45) --
	(324.60,116.45) --
	(325.02,116.45) --
	(325.44,116.45) --
	(325.87,116.45) --
	(326.29,116.45) --
	(326.71,116.45) --
	(327.13,116.45) --
	(327.56,116.45) --
	(327.98,116.45) --
	(328.40,116.45) --
	(328.82,116.45) --
	(329.25,116.45) --
	(329.67,116.45) --
	(330.09,116.45) --
	(330.51,116.45) --
	(330.94,116.45) --
	(331.36,116.45) --
	(331.78,116.45) --
	(332.20,116.45) --
	(332.63,116.45) --
	(333.05,116.45) --
	(333.47,116.45) --
	(333.89,116.45) --
	(334.32,116.45) --
	(334.74,116.45) --
	(335.16,116.45) --
	(335.58,116.45) --
	(336.00,116.45) --
	(336.43,116.45) --
	(336.85,116.45) --
	(337.27,116.45) --
	(337.69,116.45) --
	(338.12,116.45) --
	(338.54,116.45) --
	(338.96,116.45) --
	(339.38,116.45) --
	(339.81,116.45) --
	(340.23,116.45) --
	(340.65,116.45) --
	(341.07,116.45) --
	(341.50,116.45) --
	(341.92,116.45) --
	(342.34,116.45) --
	(342.76,116.45) --
	(343.19,116.45) --
	(343.61,116.45) --
	(344.03,116.45) --
	(344.45,116.45) --
	(344.88,116.45) --
	(345.30,116.45) --
	(345.72,116.45) --
	(346.14,116.45) --
	(346.57,116.45) --
	(346.99,116.45) --
	(347.41,116.45) --
	(347.83,116.45) --
	(348.25,116.45) --
	(348.68,116.45) --
	(349.10,116.45) --
	(349.52,116.45);
\end{scope}
\end{tikzpicture}
}
	\end{adjustbox}
	\newline
	\begin{adjustbox}{width=1\textwidth}
		\subfloat[Test][Zusammenhang ($r$) zwischen der Steigung und dem \gls{zwert} des \gls{bist}s.]{% Created by tikzDevice version 0.10.1 on 2016-08-19 08:52:45
% !TEX encoding = UTF-8 Unicode
\begin{tikzpicture}[x=1pt,y=1pt]
\definecolor{fillColor}{RGB}{255,255,255}
\path[use as bounding box,fill=fillColor,fill opacity=0.00] (0,0) rectangle (361.35,216.81);
\begin{scope}
\path[clip] (  0.00,  0.00) rectangle (361.35,216.81);
\definecolor{drawColor}{RGB}{0,0,0}

\node[text=drawColor,anchor=base,inner sep=0pt, outer sep=0pt, scale=  1.00] at (201.68,  8.40) {\textit{RMSE}-Grenzwert (ms)};

\node[text=drawColor,rotate= 90.00,anchor=base,inner sep=0pt, outer sep=0pt, scale=  1.00] at (  9.60,129.40) {\textit{r}};
\end{scope}
\begin{scope}
\path[clip] (  0.00,  0.00) rectangle (361.35,216.81);
\definecolor{drawColor}{RGB}{0,0,0}

\path[draw=drawColor,line width= 0.4pt,line join=round,line cap=round] ( 66.50, 48.00) -- (349.52, 48.00);

\path[draw=drawColor,line width= 0.4pt,line join=round,line cap=round] ( 66.50, 48.00) -- ( 66.50, 42.00);

\path[draw=drawColor,line width= 0.4pt,line join=round,line cap=round] ( 96.07, 48.00) -- ( 96.07, 42.00);

\path[draw=drawColor,line width= 0.4pt,line join=round,line cap=round] (138.31, 48.00) -- (138.31, 42.00);

\path[draw=drawColor,line width= 0.4pt,line join=round,line cap=round] (180.55, 48.00) -- (180.55, 42.00);

\path[draw=drawColor,line width= 0.4pt,line join=round,line cap=round] (222.80, 48.00) -- (222.80, 42.00);

\path[draw=drawColor,line width= 0.4pt,line join=round,line cap=round] (265.04, 48.00) -- (265.04, 42.00);

\path[draw=drawColor,line width= 0.4pt,line join=round,line cap=round] (307.28, 48.00) -- (307.28, 42.00);

\path[draw=drawColor,line width= 0.4pt,line join=round,line cap=round] (349.52, 48.00) -- (349.52, 42.00);

\node[text=drawColor,anchor=base,inner sep=0pt, outer sep=0pt, scale=  1.00] at ( 66.50, 30.00) {3};

\node[text=drawColor,anchor=base,inner sep=0pt, outer sep=0pt, scale=  1.00] at ( 96.07, 30.00) {10};

\node[text=drawColor,anchor=base,inner sep=0pt, outer sep=0pt, scale=  1.00] at (138.31, 30.00) {20};

\node[text=drawColor,anchor=base,inner sep=0pt, outer sep=0pt, scale=  1.00] at (180.55, 30.00) {30};

\node[text=drawColor,anchor=base,inner sep=0pt, outer sep=0pt, scale=  1.00] at (222.80, 30.00) {40};

\node[text=drawColor,anchor=base,inner sep=0pt, outer sep=0pt, scale=  1.00] at (265.04, 30.00) {50};

\node[text=drawColor,anchor=base,inner sep=0pt, outer sep=0pt, scale=  1.00] at (307.28, 30.00) {60};

\node[text=drawColor,anchor=base,inner sep=0pt, outer sep=0pt, scale=  1.00] at (349.52, 30.00) {70};

\path[draw=drawColor,line width= 0.4pt,line join=round,line cap=round] ( 42.00, 54.03) -- ( 42.00,204.78);

\path[draw=drawColor,line width= 0.4pt,line join=round,line cap=round] ( 42.00, 54.03) -- ( 36.00, 54.03);

\path[draw=drawColor,line width= 0.4pt,line join=round,line cap=round] ( 42.00, 72.87) -- ( 36.00, 72.87);

\path[draw=drawColor,line width= 0.4pt,line join=round,line cap=round] ( 42.00, 91.72) -- ( 36.00, 91.72);

\path[draw=drawColor,line width= 0.4pt,line join=round,line cap=round] ( 42.00,110.56) -- ( 36.00,110.56);

\path[draw=drawColor,line width= 0.4pt,line join=round,line cap=round] ( 42.00,129.40) -- ( 36.00,129.40);

\path[draw=drawColor,line width= 0.4pt,line join=round,line cap=round] ( 42.00,148.25) -- ( 36.00,148.25);

\path[draw=drawColor,line width= 0.4pt,line join=round,line cap=round] ( 42.00,167.09) -- ( 36.00,167.09);

\path[draw=drawColor,line width= 0.4pt,line join=round,line cap=round] ( 42.00,185.94) -- ( 36.00,185.94);

\path[draw=drawColor,line width= 0.4pt,line join=round,line cap=round] ( 42.00,204.78) -- ( 36.00,204.78);

\node[text=drawColor,anchor=base east,inner sep=0pt, outer sep=0pt, scale=  1.00] at ( 33.60, 50.59) {--1.00};

\node[text=drawColor,anchor=base east,inner sep=0pt, outer sep=0pt, scale=  1.00] at ( 33.60, 69.43) {--.75};

\node[text=drawColor,anchor=base east,inner sep=0pt, outer sep=0pt, scale=  1.00] at ( 33.60, 88.27) {--.50};

\node[text=drawColor,anchor=base east,inner sep=0pt, outer sep=0pt, scale=  1.00] at ( 33.60,107.12) {--.25};

\node[text=drawColor,anchor=base east,inner sep=0pt, outer sep=0pt, scale=  1.00] at ( 33.60,125.96) {.00};

\node[text=drawColor,anchor=base east,inner sep=0pt, outer sep=0pt, scale=  1.00] at ( 33.60,144.81) {.25};

\node[text=drawColor,anchor=base east,inner sep=0pt, outer sep=0pt, scale=  1.00] at ( 33.60,163.65) {.50};

\node[text=drawColor,anchor=base east,inner sep=0pt, outer sep=0pt, scale=  1.00] at ( 33.60,182.49) {.75};

\node[text=drawColor,anchor=base east,inner sep=0pt, outer sep=0pt, scale=  1.00] at ( 33.60,201.34) {1.00};
\end{scope}
\begin{scope}
\path[clip] ( 42.00, 48.00) rectangle (361.35,210.81);
\definecolor{fillColor}{RGB}{190,190,190}

\path[fill=fillColor] ( 66.50, 56.83) --
	( 66.92, 57.49) --
	( 67.35, 57.49) --
	( 67.77, 57.49) --
	( 68.19, 57.49) --
	( 68.61, 57.49) --
	( 69.03, 60.15) --
	( 69.46, 60.15) --
	( 69.88, 67.81) --
	( 70.30, 69.89) --
	( 70.72, 69.89) --
	( 71.15, 69.89) --
	( 71.57, 68.80) --
	( 71.99, 68.80) --
	( 72.41, 68.80) --
	( 72.84, 68.80) --
	( 73.26, 68.80) --
	( 73.68, 68.80) --
	( 74.10, 69.88) --
	( 74.53, 69.88) --
	( 74.95, 71.10) --
	( 75.37, 71.10) --
	( 75.79, 71.10) --
	( 76.22, 79.25) --
	( 76.64, 85.81) --
	( 77.06, 85.81) --
	( 77.48, 80.61) --
	( 77.91, 80.58) --
	( 78.33, 80.58) --
	( 78.75, 80.58) --
	( 79.17, 80.58) --
	( 79.60, 80.58) --
	( 80.02, 86.03) --
	( 80.44, 87.46) --
	( 80.86, 87.46) --
	( 81.29, 88.64) --
	( 81.71, 87.01) --
	( 82.13, 87.01) --
	( 82.55, 84.57) --
	( 82.97, 84.57) --
	( 83.40, 84.57) --
	( 83.82, 84.57) --
	( 84.24, 87.38) --
	( 84.66, 87.38) --
	( 85.09, 87.95) --
	( 85.51, 87.95) --
	( 85.93, 89.19) --
	( 86.35, 89.82) --
	( 86.78, 87.68) --
	( 87.20, 87.68) --
	( 87.62, 87.56) --
	( 88.04, 87.82) --
	( 88.47, 87.82) --
	( 88.89, 87.81) --
	( 89.31, 87.06) --
	( 89.73, 87.56) --
	( 90.16, 89.60) --
	( 90.58, 89.60) --
	( 91.00, 88.76) --
	( 91.42, 87.95) --
	( 91.85, 87.95) --
	( 92.27, 88.66) --
	( 92.69, 88.66) --
	( 93.11, 88.78) --
	( 93.54, 88.78) --
	( 93.96, 85.89) --
	( 94.38, 89.10) --
	( 94.80, 89.16) --
	( 95.22, 89.16) --
	( 95.65, 89.16) --
	( 96.07, 89.16) --
	( 96.49, 90.30) --
	( 96.91, 90.08) --
	( 97.34, 91.06) --
	( 97.76, 91.36) --
	( 98.18, 92.15) --
	( 98.60, 94.80) --
	( 99.03, 95.23) --
	( 99.45, 96.93) --
	( 99.87, 96.93) --
	(100.29, 99.84) --
	(100.72,100.26) --
	(101.14,100.96) --
	(101.56,100.96) --
	(101.98,101.07) --
	(102.41,102.90) --
	(102.83,102.90) --
	(103.25,105.39) --
	(103.67,105.59) --
	(104.10,105.65) --
	(104.52,104.91) --
	(104.94,104.12) --
	(105.36,104.31) --
	(105.79,104.15) --
	(106.21,104.15) --
	(106.63,104.15) --
	(107.05,104.15) --
	(107.48,104.18) --
	(107.90,104.18) --
	(108.32,104.18) --
	(108.74,103.48) --
	(109.16,103.48) --
	(109.59,102.48) --
	(110.01,102.36) --
	(110.43,101.88) --
	(110.85,101.88) --
	(111.28,101.88) --
	(111.70,101.97) --
	(112.12,103.04) --
	(112.54,102.59) --
	(112.97,102.75) --
	(113.39,103.07) --
	(113.81,103.07) --
	(114.23,103.07) --
	(114.66,103.07) --
	(115.08,103.07) --
	(115.50,103.07) --
	(115.92,103.07) --
	(116.35,103.12) --
	(116.77,103.12) --
	(117.19,103.12) --
	(117.61,104.61) --
	(118.04,104.61) --
	(118.46,104.61) --
	(118.88,103.40) --
	(119.30,103.40) --
	(119.73,103.30) --
	(120.15,101.88) --
	(120.57,101.41) --
	(120.99,101.41) --
	(121.42,101.41) --
	(121.84,101.41) --
	(122.26,101.39) --
	(122.68,100.47) --
	(123.10,100.98) --
	(123.53,100.98) --
	(123.95,100.98) --
	(124.37,100.93) --
	(124.79,100.93) --
	(125.22, 99.50) --
	(125.64, 98.97) --
	(126.06, 98.97) --
	(126.48,101.12) --
	(126.91,101.12) --
	(127.33,101.74) --
	(127.75,102.03) --
	(128.17,102.03) --
	(128.60,102.04) --
	(129.02,102.04) --
	(129.44,102.26) --
	(129.86,102.26) --
	(130.29,102.22) --
	(130.71,102.22) --
	(131.13,102.22) --
	(131.55,102.14) --
	(131.98,102.14) --
	(132.40,102.14) --
	(132.82,102.14) --
	(133.24,102.14) --
	(133.67,102.14) --
	(134.09,102.14) --
	(134.51,102.07) --
	(134.93,101.99) --
	(135.35,101.99) --
	(135.78,101.99) --
	(136.20,102.03) --
	(136.62,102.08) --
	(137.04,102.08) --
	(137.47,102.08) --
	(137.89,102.03) --
	(138.31,101.99) --
	(138.73,102.89) --
	(139.16,102.89) --
	(139.58,102.89) --
	(140.00,102.89) --
	(140.42,102.02) --
	(140.85,102.02) --
	(141.27,101.55) --
	(141.69,101.97) --
	(142.11,101.54) --
	(142.54,101.42) --
	(142.96,101.42) --
	(143.38,101.42) --
	(143.80,101.42) --
	(144.23,101.42) --
	(144.65,101.46) --
	(145.07,101.46) --
	(145.49,100.99) --
	(145.92,100.99) --
	(146.34,100.99) --
	(146.76,100.81) --
	(147.18,100.81) --
	(147.61,100.81) --
	(148.03, 99.79) --
	(148.45, 99.79) --
	(148.87, 99.79) --
	(149.29, 99.79) --
	(149.72, 99.79) --
	(150.14, 99.88) --
	(150.56, 99.88) --
	(150.98, 99.88) --
	(151.41, 99.88) --
	(151.83,100.22) --
	(152.25,100.02) --
	(152.67,100.05) --
	(153.10,100.05) --
	(153.52,100.05) --
	(153.94,100.05) --
	(154.36,100.05) --
	(154.79,100.05) --
	(155.21,100.05) --
	(155.63,100.05) --
	(156.05,100.05) --
	(156.48,100.05) --
	(156.90, 99.16) --
	(157.32, 99.16) --
	(157.74, 99.16) --
	(158.17, 99.16) --
	(158.59, 99.17) --
	(159.01, 98.22) --
	(159.43, 98.22) --
	(159.86, 98.22) --
	(160.28, 98.22) --
	(160.70, 98.22) --
	(161.12, 98.23) --
	(161.55, 98.23) --
	(161.97, 98.23) --
	(162.39, 99.11) --
	(162.81, 99.85) --
	(163.23, 99.85) --
	(163.66,100.00) --
	(164.08, 99.24) --
	(164.50, 99.24) --
	(164.92, 99.24) --
	(165.35, 99.29) --
	(165.77, 99.29) --
	(166.19, 99.29) --
	(166.61, 99.29) --
	(167.04, 99.29) --
	(167.46, 99.83) --
	(167.88, 99.83) --
	(168.30,100.02) --
	(168.73,100.02) --
	(169.15,100.02) --
	(169.57,100.02) --
	(169.99,100.02) --
	(170.42,100.02) --
	(170.84,100.02) --
	(171.26,100.02) --
	(171.68,100.02) --
	(172.11,100.02) --
	(172.53,100.02) --
	(172.95,100.02) --
	(173.37,100.02) --
	(173.80,100.02) --
	(174.22,100.02) --
	(174.64,100.02) --
	(175.06,100.02) --
	(175.48,100.02) --
	(175.91,100.02) --
	(176.33,100.02) --
	(176.75,100.02) --
	(177.17,100.02) --
	(177.60,100.02) --
	(178.02,100.02) --
	(178.44,100.02) --
	(178.86,100.02) --
	(179.29,100.02) --
	(179.71,100.02) --
	(180.13,100.02) --
	(180.55,100.02) --
	(180.98,100.02) --
	(181.40,100.02) --
	(181.82,100.02) --
	(182.24,100.02) --
	(182.67,100.02) --
	(183.09,100.02) --
	(183.51,100.02) --
	(183.93,100.02) --
	(184.36,100.02) --
	(184.78,100.02) --
	(185.20,100.02) --
	(185.62,100.02) --
	(186.05,100.77) --
	(186.47,100.77) --
	(186.89,100.77) --
	(187.31,100.77) --
	(187.74,100.77) --
	(188.16,100.77) --
	(188.58,100.77) --
	(189.00,100.77) --
	(189.42,100.77) --
	(189.85,100.77) --
	(190.27,100.77) --
	(190.69,100.77) --
	(191.11,100.77) --
	(191.54,100.77) --
	(191.96,100.77) --
	(192.38,100.90) --
	(192.80,100.90) --
	(193.23,100.90) --
	(193.65,100.90) --
	(194.07,100.90) --
	(194.49,100.90) --
	(194.92,100.90) --
	(195.34,100.90) --
	(195.76,100.90) --
	(196.18,100.90) --
	(196.61,100.90) --
	(197.03,100.90) --
	(197.45,100.90) --
	(197.87,100.90) --
	(198.30,100.90) --
	(198.72,100.90) --
	(199.14,100.90) --
	(199.56,100.90) --
	(199.99,100.90) --
	(200.41,100.90) --
	(200.83,100.90) --
	(201.25,100.90) --
	(201.68,100.90) --
	(202.10,100.90) --
	(202.52,100.90) --
	(202.94,100.90) --
	(203.36,100.90) --
	(203.79,100.90) --
	(204.21,100.90) --
	(204.63,100.90) --
	(205.05,100.90) --
	(205.48,100.90) --
	(205.90,100.90) --
	(206.32,100.90) --
	(206.74,100.75) --
	(207.17,100.75) --
	(207.59,100.75) --
	(208.01,100.75) --
	(208.43,100.75) --
	(208.86,100.75) --
	(209.28,100.75) --
	(209.70,100.75) --
	(210.12,100.75) --
	(210.55,100.17) --
	(210.97,100.17) --
	(211.39,100.36) --
	(211.81,100.36) --
	(212.24,100.36) --
	(212.66,100.36) --
	(213.08,100.36) --
	(213.50,100.36) --
	(213.93,100.36) --
	(214.35,100.36) --
	(214.77,100.36) --
	(215.19,100.36) --
	(215.61,100.36) --
	(216.04,100.36) --
	(216.46,100.13) --
	(216.88,100.13) --
	(217.30,100.13) --
	(217.73,100.13) --
	(218.15,100.13) --
	(218.57,100.13) --
	(218.99,100.13) --
	(219.42,100.13) --
	(219.84,100.13) --
	(220.26,100.13) --
	(220.68,100.13) --
	(221.11,100.13) --
	(221.53,100.13) --
	(221.95,100.13) --
	(222.37,100.13) --
	(222.80,100.13) --
	(223.22,100.13) --
	(223.64,100.68) --
	(224.06,100.68) --
	(224.49,100.68) --
	(224.91,100.68) --
	(225.33,100.68) --
	(225.75,100.68) --
	(226.18,100.68) --
	(226.60,100.68) --
	(227.02,100.68) --
	(227.44,100.68) --
	(227.87,100.68) --
	(228.29,100.68) --
	(228.71,101.66) --
	(229.13,101.66) --
	(229.55,101.66) --
	(229.98,101.66) --
	(230.40,101.66) --
	(230.82,101.66) --
	(231.24,101.66) --
	(231.67,101.66) --
	(232.09,101.66) --
	(232.51,101.66) --
	(232.93,101.66) --
	(233.36,101.66) --
	(233.78,101.66) --
	(234.20,101.66) --
	(234.62,101.66) --
	(235.05,101.66) --
	(235.47,101.66) --
	(235.89,101.66) --
	(236.31,101.66) --
	(236.74,101.66) --
	(237.16,101.66) --
	(237.58,101.66) --
	(238.00,101.66) --
	(238.43,101.66) --
	(238.85,101.66) --
	(239.27,101.66) --
	(239.69,101.66) --
	(240.12,101.66) --
	(240.54,101.66) --
	(240.96,101.66) --
	(241.38,101.66) --
	(241.80,101.66) --
	(242.23,101.66) --
	(242.65,101.66) --
	(243.07,101.66) --
	(243.49,101.66) --
	(243.92,101.66) --
	(244.34,101.66) --
	(244.76,101.66) --
	(245.18,101.66) --
	(245.61,101.66) --
	(246.03,101.66) --
	(246.45,101.66) --
	(246.87,101.66) --
	(247.30,101.66) --
	(247.72,101.66) --
	(248.14,101.66) --
	(248.56,101.66) --
	(248.99,101.66) --
	(249.41,101.66) --
	(249.83,101.66) --
	(250.25,101.66) --
	(250.68,101.66) --
	(251.10,101.66) --
	(251.52,101.66) --
	(251.94,101.66) --
	(252.37,101.66) --
	(252.79,101.66) --
	(253.21,101.66) --
	(253.63,101.66) --
	(254.06,101.66) --
	(254.48,101.66) --
	(254.90,101.66) --
	(255.32,101.66) --
	(255.74,101.66) --
	(256.17,101.66) --
	(256.59,101.66) --
	(257.01,101.66) --
	(257.43,101.78) --
	(257.86,101.78) --
	(258.28,101.78) --
	(258.70,101.78) --
	(259.12,101.78) --
	(259.55,101.78) --
	(259.97,101.78) --
	(260.39,101.78) --
	(260.81,101.78) --
	(261.24,101.78) --
	(261.66,101.78) --
	(262.08,101.78) --
	(262.50,101.78) --
	(262.93,101.78) --
	(263.35,101.78) --
	(263.77,101.78) --
	(264.19,101.78) --
	(264.62,101.78) --
	(265.04,101.78) --
	(265.46,101.78) --
	(265.88,101.78) --
	(266.31,101.78) --
	(266.73,101.78) --
	(267.15,101.78) --
	(267.57,101.78) --
	(268.00,101.78) --
	(268.42,102.25) --
	(268.84,102.25) --
	(269.26,102.25) --
	(269.68,102.25) --
	(270.11,102.25) --
	(270.53,102.25) --
	(270.95,102.25) --
	(271.37,102.25) --
	(271.80,102.25) --
	(272.22,102.25) --
	(272.64,102.25) --
	(273.06,102.25) --
	(273.49,102.25) --
	(273.91,102.25) --
	(274.33,102.25) --
	(274.75,102.25) --
	(275.18,102.25) --
	(275.60,102.25) --
	(276.02,102.25) --
	(276.44,102.25) --
	(276.87,102.25) --
	(277.29,102.25) --
	(277.71,102.25) --
	(278.13,102.25) --
	(278.56,102.25) --
	(278.98,102.25) --
	(279.40,102.25) --
	(279.82,102.25) --
	(280.25,102.25) --
	(280.67,102.25) --
	(281.09,102.25) --
	(281.51,102.25) --
	(281.93,102.25) --
	(282.36,102.25) --
	(282.78,102.25) --
	(283.20,101.86) --
	(283.62,101.86) --
	(284.05,101.86) --
	(284.47,101.86) --
	(284.89,101.86) --
	(285.31,101.86) --
	(285.74,101.86) --
	(286.16,101.86) --
	(286.58,101.86) --
	(287.00,101.86) --
	(287.43,101.86) --
	(287.85,101.86) --
	(288.27,101.86) --
	(288.69,101.86) --
	(289.12,101.86) --
	(289.54,101.86) --
	(289.96,101.86) --
	(290.38,101.86) --
	(290.81,101.86) --
	(291.23,101.86) --
	(291.65,101.86) --
	(292.07,101.86) --
	(292.50,101.86) --
	(292.92,101.86) --
	(293.34,101.86) --
	(293.76,101.86) --
	(294.19,101.86) --
	(294.61,101.86) --
	(295.03,101.86) --
	(295.45,101.86) --
	(295.87,101.86) --
	(296.30,101.86) --
	(296.72,101.86) --
	(297.14,101.86) --
	(297.56,101.86) --
	(297.99,101.86) --
	(298.41,101.86) --
	(298.83,101.86) --
	(299.25,101.86) --
	(299.68,101.86) --
	(300.10,101.86) --
	(300.52,101.86) --
	(300.94,101.86) --
	(301.37,101.86) --
	(301.79,101.86) --
	(302.21,101.86) --
	(302.63,101.86) --
	(303.06,101.86) --
	(303.48,101.86) --
	(303.90,101.86) --
	(304.32,101.86) --
	(304.75,101.86) --
	(305.17,101.86) --
	(305.59,101.86) --
	(306.01,101.86) --
	(306.44,101.86) --
	(306.86,101.86) --
	(307.28,101.86) --
	(307.70,101.86) --
	(308.12,101.86) --
	(308.55,101.86) --
	(308.97,101.86) --
	(309.39,101.86) --
	(309.81,101.86) --
	(310.24,101.86) --
	(310.66,101.86) --
	(311.08,101.86) --
	(311.50,101.86) --
	(311.93,101.86) --
	(312.35,101.86) --
	(312.77,101.86) --
	(313.19,101.86) --
	(313.62,101.86) --
	(314.04,101.86) --
	(314.46,101.86) --
	(314.88,101.86) --
	(315.31,101.86) --
	(315.73,101.86) --
	(316.15,101.86) --
	(316.57,101.86) --
	(317.00,101.86) --
	(317.42,101.86) --
	(317.84,101.86) --
	(318.26,101.86) --
	(318.69,101.86) --
	(319.11,101.86) --
	(319.53,101.86) --
	(319.95,101.86) --
	(320.38,101.86) --
	(320.80,101.86) --
	(321.22,101.86) --
	(321.64,101.86) --
	(322.06,101.86) --
	(322.49,101.86) --
	(322.91,101.86) --
	(323.33,101.86) --
	(323.75,101.86) --
	(324.18,101.86) --
	(324.60,101.86) --
	(325.02,101.86) --
	(325.44,101.86) --
	(325.87,101.86) --
	(326.29,101.86) --
	(326.71,101.86) --
	(327.13,101.86) --
	(327.56,101.86) --
	(327.98,101.86) --
	(328.40,101.86) --
	(328.82,101.86) --
	(329.25,101.86) --
	(329.67,101.86) --
	(330.09,101.86) --
	(330.51,101.86) --
	(330.94,101.86) --
	(331.36,101.86) --
	(331.78,101.86) --
	(332.20,101.86) --
	(332.63,101.86) --
	(333.05,101.86) --
	(333.47,101.86) --
	(333.89,101.86) --
	(334.32,101.86) --
	(334.74,101.86) --
	(335.16,101.86) --
	(335.58,101.86) --
	(336.00,101.86) --
	(336.43,101.86) --
	(336.85,101.86) --
	(337.27,101.86) --
	(337.69,101.86) --
	(338.12,101.86) --
	(338.54,101.86) --
	(338.96,101.86) --
	(339.38,101.86) --
	(339.81,101.86) --
	(340.23,101.86) --
	(340.65,101.86) --
	(341.07,101.86) --
	(341.50,101.86) --
	(341.92,101.86) --
	(342.34,101.86) --
	(342.76,101.86) --
	(343.19,101.86) --
	(343.61,101.86) --
	(344.03,101.86) --
	(344.45,101.86) --
	(344.88,101.86) --
	(345.30,101.86) --
	(345.72,101.86) --
	(346.14,101.86) --
	(346.57,101.86) --
	(346.99,101.86) --
	(347.41,101.86) --
	(347.83,101.86) --
	(348.25,101.86) --
	(348.68,101.86) --
	(349.10,101.86) --
	(349.52,101.86) --
	(349.52,122.94) --
	(349.10,122.94) --
	(348.68,122.94) --
	(348.25,122.94) --
	(347.83,122.94) --
	(347.41,122.94) --
	(346.99,122.94) --
	(346.57,122.94) --
	(346.14,122.94) --
	(345.72,122.94) --
	(345.30,122.94) --
	(344.88,122.94) --
	(344.45,122.94) --
	(344.03,122.94) --
	(343.61,122.94) --
	(343.19,122.94) --
	(342.76,122.94) --
	(342.34,122.94) --
	(341.92,122.94) --
	(341.50,122.94) --
	(341.07,122.94) --
	(340.65,122.94) --
	(340.23,122.94) --
	(339.81,122.94) --
	(339.38,122.94) --
	(338.96,122.94) --
	(338.54,122.94) --
	(338.12,122.94) --
	(337.69,122.94) --
	(337.27,122.94) --
	(336.85,122.94) --
	(336.43,122.94) --
	(336.00,122.94) --
	(335.58,122.94) --
	(335.16,122.94) --
	(334.74,122.94) --
	(334.32,122.94) --
	(333.89,122.94) --
	(333.47,122.94) --
	(333.05,122.94) --
	(332.63,122.94) --
	(332.20,122.94) --
	(331.78,122.94) --
	(331.36,122.94) --
	(330.94,122.94) --
	(330.51,122.94) --
	(330.09,122.94) --
	(329.67,122.94) --
	(329.25,122.94) --
	(328.82,122.94) --
	(328.40,122.94) --
	(327.98,122.94) --
	(327.56,122.94) --
	(327.13,122.94) --
	(326.71,122.94) --
	(326.29,122.94) --
	(325.87,122.94) --
	(325.44,122.94) --
	(325.02,122.94) --
	(324.60,122.94) --
	(324.18,122.94) --
	(323.75,122.94) --
	(323.33,122.94) --
	(322.91,122.94) --
	(322.49,122.94) --
	(322.06,122.94) --
	(321.64,122.94) --
	(321.22,122.94) --
	(320.80,122.94) --
	(320.38,122.94) --
	(319.95,122.94) --
	(319.53,122.94) --
	(319.11,122.94) --
	(318.69,122.94) --
	(318.26,122.94) --
	(317.84,122.94) --
	(317.42,122.94) --
	(317.00,122.94) --
	(316.57,122.94) --
	(316.15,122.94) --
	(315.73,122.94) --
	(315.31,122.94) --
	(314.88,122.94) --
	(314.46,122.94) --
	(314.04,122.94) --
	(313.62,122.94) --
	(313.19,122.94) --
	(312.77,122.94) --
	(312.35,122.94) --
	(311.93,122.94) --
	(311.50,122.94) --
	(311.08,122.94) --
	(310.66,122.94) --
	(310.24,122.94) --
	(309.81,122.94) --
	(309.39,122.94) --
	(308.97,122.94) --
	(308.55,122.94) --
	(308.12,122.94) --
	(307.70,122.94) --
	(307.28,122.94) --
	(306.86,122.94) --
	(306.44,122.94) --
	(306.01,122.94) --
	(305.59,122.94) --
	(305.17,122.94) --
	(304.75,122.94) --
	(304.32,122.94) --
	(303.90,122.94) --
	(303.48,122.94) --
	(303.06,122.94) --
	(302.63,122.94) --
	(302.21,122.94) --
	(301.79,122.94) --
	(301.37,122.94) --
	(300.94,122.94) --
	(300.52,122.94) --
	(300.10,122.94) --
	(299.68,122.94) --
	(299.25,122.94) --
	(298.83,122.94) --
	(298.41,122.94) --
	(297.99,122.94) --
	(297.56,122.94) --
	(297.14,122.94) --
	(296.72,122.94) --
	(296.30,122.94) --
	(295.87,122.94) --
	(295.45,122.94) --
	(295.03,122.94) --
	(294.61,122.94) --
	(294.19,122.94) --
	(293.76,122.94) --
	(293.34,122.94) --
	(292.92,122.94) --
	(292.50,122.94) --
	(292.07,122.94) --
	(291.65,122.94) --
	(291.23,122.94) --
	(290.81,122.94) --
	(290.38,122.94) --
	(289.96,122.94) --
	(289.54,122.94) --
	(289.12,122.94) --
	(288.69,122.94) --
	(288.27,122.94) --
	(287.85,122.94) --
	(287.43,122.94) --
	(287.00,122.94) --
	(286.58,122.94) --
	(286.16,122.94) --
	(285.74,122.94) --
	(285.31,122.94) --
	(284.89,122.94) --
	(284.47,122.94) --
	(284.05,122.94) --
	(283.62,122.94) --
	(283.20,122.94) --
	(282.78,123.45) --
	(282.36,123.45) --
	(281.93,123.45) --
	(281.51,123.45) --
	(281.09,123.45) --
	(280.67,123.45) --
	(280.25,123.45) --
	(279.82,123.45) --
	(279.40,123.45) --
	(278.98,123.45) --
	(278.56,123.45) --
	(278.13,123.45) --
	(277.71,123.45) --
	(277.29,123.45) --
	(276.87,123.45) --
	(276.44,123.45) --
	(276.02,123.45) --
	(275.60,123.45) --
	(275.18,123.45) --
	(274.75,123.45) --
	(274.33,123.45) --
	(273.91,123.45) --
	(273.49,123.45) --
	(273.06,123.45) --
	(272.64,123.45) --
	(272.22,123.45) --
	(271.80,123.45) --
	(271.37,123.45) --
	(270.95,123.45) --
	(270.53,123.45) --
	(270.11,123.45) --
	(269.68,123.45) --
	(269.26,123.45) --
	(268.84,123.45) --
	(268.42,123.45) --
	(268.00,122.97) --
	(267.57,122.97) --
	(267.15,122.97) --
	(266.73,122.97) --
	(266.31,122.97) --
	(265.88,122.97) --
	(265.46,122.97) --
	(265.04,122.97) --
	(264.62,122.97) --
	(264.19,122.97) --
	(263.77,122.97) --
	(263.35,122.97) --
	(262.93,122.97) --
	(262.50,122.97) --
	(262.08,122.97) --
	(261.66,122.97) --
	(261.24,122.97) --
	(260.81,122.97) --
	(260.39,122.97) --
	(259.97,122.97) --
	(259.55,122.97) --
	(259.12,122.97) --
	(258.70,122.97) --
	(258.28,122.97) --
	(257.86,122.97) --
	(257.43,122.97) --
	(257.01,122.91) --
	(256.59,122.91) --
	(256.17,122.91) --
	(255.74,122.91) --
	(255.32,122.91) --
	(254.90,122.91) --
	(254.48,122.91) --
	(254.06,122.91) --
	(253.63,122.91) --
	(253.21,122.91) --
	(252.79,122.91) --
	(252.37,122.91) --
	(251.94,122.91) --
	(251.52,122.91) --
	(251.10,122.91) --
	(250.68,122.91) --
	(250.25,122.91) --
	(249.83,122.91) --
	(249.41,122.91) --
	(248.99,122.91) --
	(248.56,122.91) --
	(248.14,122.91) --
	(247.72,122.91) --
	(247.30,122.91) --
	(246.87,122.91) --
	(246.45,122.91) --
	(246.03,122.91) --
	(245.61,122.91) --
	(245.18,122.91) --
	(244.76,122.91) --
	(244.34,122.91) --
	(243.92,122.91) --
	(243.49,122.91) --
	(243.07,122.91) --
	(242.65,122.91) --
	(242.23,122.91) --
	(241.80,122.91) --
	(241.38,122.91) --
	(240.96,122.91) --
	(240.54,122.91) --
	(240.12,122.91) --
	(239.69,122.91) --
	(239.27,122.91) --
	(238.85,122.91) --
	(238.43,122.91) --
	(238.00,122.91) --
	(237.58,122.91) --
	(237.16,122.91) --
	(236.74,122.91) --
	(236.31,122.91) --
	(235.89,122.91) --
	(235.47,122.91) --
	(235.05,122.91) --
	(234.62,122.91) --
	(234.20,122.91) --
	(233.78,122.91) --
	(233.36,122.91) --
	(232.93,122.91) --
	(232.51,122.91) --
	(232.09,122.91) --
	(231.67,122.91) --
	(231.24,122.91) --
	(230.82,122.91) --
	(230.40,122.91) --
	(229.98,122.91) --
	(229.55,122.91) --
	(229.13,122.91) --
	(228.71,122.91) --
	(228.29,121.84) --
	(227.87,121.84) --
	(227.44,121.84) --
	(227.02,121.84) --
	(226.60,121.84) --
	(226.18,121.84) --
	(225.75,121.84) --
	(225.33,121.84) --
	(224.91,121.84) --
	(224.49,121.84) --
	(224.06,121.84) --
	(223.64,121.84) --
	(223.22,121.26) --
	(222.80,121.26) --
	(222.37,121.26) --
	(221.95,121.26) --
	(221.53,121.26) --
	(221.11,121.26) --
	(220.68,121.26) --
	(220.26,121.26) --
	(219.84,121.26) --
	(219.42,121.26) --
	(218.99,121.26) --
	(218.57,121.26) --
	(218.15,121.26) --
	(217.73,121.26) --
	(217.30,121.26) --
	(216.88,121.26) --
	(216.46,121.26) --
	(216.04,121.60) --
	(215.61,121.60) --
	(215.19,121.60) --
	(214.77,121.60) --
	(214.35,121.60) --
	(213.93,121.60) --
	(213.50,121.60) --
	(213.08,121.60) --
	(212.66,121.60) --
	(212.24,121.60) --
	(211.81,121.60) --
	(211.39,121.60) --
	(210.97,121.45) --
	(210.55,121.45) --
	(210.12,122.19) --
	(209.70,122.19) --
	(209.28,122.19) --
	(208.86,122.19) --
	(208.43,122.19) --
	(208.01,122.19) --
	(207.59,122.19) --
	(207.17,122.19) --
	(206.74,122.19) --
	(206.32,122.43) --
	(205.90,122.43) --
	(205.48,122.43) --
	(205.05,122.43) --
	(204.63,122.43) --
	(204.21,122.43) --
	(203.79,122.43) --
	(203.36,122.43) --
	(202.94,122.43) --
	(202.52,122.43) --
	(202.10,122.43) --
	(201.68,122.43) --
	(201.25,122.43) --
	(200.83,122.43) --
	(200.41,122.43) --
	(199.99,122.43) --
	(199.56,122.43) --
	(199.14,122.43) --
	(198.72,122.43) --
	(198.30,122.43) --
	(197.87,122.43) --
	(197.45,122.43) --
	(197.03,122.43) --
	(196.61,122.43) --
	(196.18,122.43) --
	(195.76,122.43) --
	(195.34,122.43) --
	(194.92,122.43) --
	(194.49,122.43) --
	(194.07,122.43) --
	(193.65,122.43) --
	(193.23,122.43) --
	(192.80,122.43) --
	(192.38,122.43) --
	(191.96,122.36) --
	(191.54,122.36) --
	(191.11,122.36) --
	(190.69,122.36) --
	(190.27,122.36) --
	(189.85,122.36) --
	(189.42,122.36) --
	(189.00,122.36) --
	(188.58,122.36) --
	(188.16,122.36) --
	(187.74,122.36) --
	(187.31,122.36) --
	(186.89,122.36) --
	(186.47,122.36) --
	(186.05,122.36) --
	(185.62,121.55) --
	(185.20,121.55) --
	(184.78,121.55) --
	(184.36,121.55) --
	(183.93,121.55) --
	(183.51,121.55) --
	(183.09,121.55) --
	(182.67,121.55) --
	(182.24,121.55) --
	(181.82,121.55) --
	(181.40,121.55) --
	(180.98,121.55) --
	(180.55,121.55) --
	(180.13,121.55) --
	(179.71,121.55) --
	(179.29,121.55) --
	(178.86,121.55) --
	(178.44,121.55) --
	(178.02,121.55) --
	(177.60,121.55) --
	(177.17,121.55) --
	(176.75,121.55) --
	(176.33,121.55) --
	(175.91,121.55) --
	(175.48,121.55) --
	(175.06,121.55) --
	(174.64,121.55) --
	(174.22,121.55) --
	(173.80,121.55) --
	(173.37,121.55) --
	(172.95,121.55) --
	(172.53,121.55) --
	(172.11,121.55) --
	(171.68,121.62) --
	(171.26,121.62) --
	(170.84,121.62) --
	(170.42,121.62) --
	(169.99,121.62) --
	(169.57,121.62) --
	(169.15,121.62) --
	(168.73,121.62) --
	(168.30,121.62) --
	(167.88,121.47) --
	(167.46,121.47) --
	(167.04,120.91) --
	(166.61,120.91) --
	(166.19,120.91) --
	(165.77,120.91) --
	(165.35,120.91) --
	(164.92,120.93) --
	(164.50,120.93) --
	(164.08,120.93) --
	(163.66,121.88) --
	(163.23,121.78) --
	(162.81,121.78) --
	(162.39,120.99) --
	(161.97,120.02) --
	(161.55,120.02) --
	(161.12,120.02) --
	(160.70,120.09) --
	(160.28,120.09) --
	(159.86,120.09) --
	(159.43,120.09) --
	(159.01,120.09) --
	(158.59,121.37) --
	(158.17,121.43) --
	(157.74,121.43) --
	(157.32,121.43) --
	(156.90,121.43) --
	(156.48,122.56) --
	(156.05,122.56) --
	(155.63,122.56) --
	(155.21,122.56) --
	(154.79,122.56) --
	(154.36,122.56) --
	(153.94,122.56) --
	(153.52,122.56) --
	(153.10,122.56) --
	(152.67,122.56) --
	(152.25,122.60) --
	(151.83,122.91) --
	(151.41,122.60) --
	(150.98,122.60) --
	(150.56,122.60) --
	(150.14,122.60) --
	(149.72,122.58) --
	(149.29,122.58) --
	(148.87,122.58) --
	(148.45,122.58) --
	(148.03,122.58) --
	(147.61,123.86) --
	(147.18,123.86) --
	(146.76,123.86) --
	(146.34,124.14) --
	(145.92,124.14) --
	(145.49,124.14) --
	(145.07,124.77) --
	(144.65,124.77) --
	(144.23,124.81) --
	(143.80,124.81) --
	(143.38,124.81) --
	(142.96,124.81) --
	(142.54,124.81) --
	(142.11,125.04) --
	(141.69,125.62) --
	(141.27,125.23) --
	(140.85,125.86) --
	(140.42,125.86) --
	(140.00,127.05) --
	(139.58,127.05) --
	(139.16,127.05) --
	(138.73,127.05) --
	(138.31,126.10) --
	(137.89,126.25) --
	(137.47,126.40) --
	(137.04,126.40) --
	(136.62,126.40) --
	(136.20,126.44) --
	(135.78,126.49) --
	(135.35,126.49) --
	(134.93,126.49) --
	(134.51,126.69) --
	(134.09,126.87) --
	(133.67,126.87) --
	(133.24,126.87) --
	(132.82,126.87) --
	(132.40,126.87) --
	(131.98,126.87) --
	(131.55,126.87) --
	(131.13,127.05) --
	(130.71,127.05) --
	(130.29,127.05) --
	(129.86,127.20) --
	(129.44,127.20) --
	(129.02,127.06) --
	(128.60,127.06) --
	(128.17,127.15) --
	(127.75,127.15) --
	(127.33,127.03) --
	(126.91,126.42) --
	(126.48,126.42) --
	(126.06,124.12) --
	(125.64,124.12) --
	(125.22,124.97) --
	(124.79,126.87) --
	(124.37,126.87) --
	(123.95,127.30) --
	(123.53,127.30) --
	(123.10,127.30) --
	(122.68,127.08) --
	(122.26,128.28) --
	(121.84,128.44) --
	(121.42,128.44) --
	(120.99,128.44) --
	(120.57,128.44) --
	(120.15,129.11) --
	(119.73,130.87) --
	(119.30,131.25) --
	(118.88,131.25) --
	(118.46,132.77) --
	(118.04,132.77) --
	(117.61,132.77) --
	(117.19,131.37) --
	(116.77,131.37) --
	(116.35,131.37) --
	(115.92,131.46) --
	(115.50,131.46) --
	(115.08,131.46) --
	(114.66,131.46) --
	(114.23,131.46) --
	(113.81,131.46) --
	(113.39,131.46) --
	(112.97,131.39) --
	(112.54,131.36) --
	(112.12,132.19) --
	(111.70,131.29) --
	(111.28,131.35) --
	(110.85,131.35) --
	(110.43,131.35) --
	(110.01,132.08) --
	(109.59,132.39) --
	(109.16,133.87) --
	(108.74,133.87) --
	(108.32,134.84) --
	(107.90,134.84) --
	(107.48,134.84) --
	(107.05,134.99) --
	(106.63,134.99) --
	(106.21,134.99) --
	(105.79,134.99) --
	(105.36,135.54) --
	(104.94,135.51) --
	(104.52,136.79) --
	(104.10,138.02) --
	(103.67,138.16) --
	(103.25,138.15) --
	(102.83,135.82) --
	(102.41,135.82) --
	(101.98,134.19) --
	(101.56,134.30) --
	(101.14,134.30) --
	(100.72,133.73) --
	(100.29,133.48) --
	( 99.87,130.76) --
	( 99.45,130.76) --
	( 99.03,129.19) --
	( 98.60,129.24) --
	( 98.18,126.41) --
	( 97.76,125.67) --
	( 97.34,125.91) --
	( 96.91,125.63) --
	( 96.49,126.30) --
	( 96.07,125.09) --
	( 95.65,125.09) --
	( 95.22,125.09) --
	( 94.80,125.09) --
	( 94.38,125.40) --
	( 93.96,121.18) --
	( 93.54,126.18) --
	( 93.11,126.18) --
	( 92.69,127.85) --
	( 92.27,127.85) --
	( 91.85,127.85) --
	( 91.42,127.85) --
	( 91.00,130.09) --
	( 90.58,131.83) --
	( 90.16,131.83) --
	( 89.73,130.17) --
	( 89.31,130.08) --
	( 88.89,131.84) --
	( 88.47,133.28) --
	( 88.04,133.28) --
	( 87.62,133.66) --
	( 87.20,135.47) --
	( 86.78,135.47) --
	( 86.35,140.22) --
	( 85.93,141.31) --
	( 85.51,140.66) --
	( 85.09,140.66) --
	( 84.66,140.97) --
	( 84.24,140.97) --
	( 83.82,138.04) --
	( 83.40,138.04) --
	( 82.97,138.04) --
	( 82.55,138.04) --
	( 82.13,142.90) --
	( 81.71,142.90) --
	( 81.29,146.47) --
	( 80.86,146.28) --
	( 80.44,146.28) --
	( 80.02,145.79) --
	( 79.60,142.78) --
	( 79.17,142.78) --
	( 78.75,142.78) --
	( 78.33,142.78) --
	( 77.91,142.78) --
	( 77.48,144.95) --
	( 77.06,157.24) --
	( 76.64,157.24) --
	( 76.22,150.30) --
	( 75.79,145.04) --
	( 75.37,145.04) --
	( 74.95,145.04) --
	( 74.53,146.87) --
	( 74.10,146.87) --
	( 73.68,149.68) --
	( 73.26,149.68) --
	( 72.84,149.68) --
	( 72.41,149.68) --
	( 71.99,149.68) --
	( 71.57,149.68) --
	( 71.15,158.78) --
	( 70.72,158.78) --
	( 70.30,158.78) --
	( 69.88,161.33) --
	( 69.46,142.24) --
	( 69.03,142.24) --
	( 68.61,135.78) --
	( 68.19,135.78) --
	( 67.77,135.78) --
	( 67.35,135.78) --
	( 66.92,135.78) --
	( 66.50,149.96) --
	cycle;
\definecolor{drawColor}{RGB}{0,0,0}

\path[draw=drawColor,line width= 0.4pt,dash pattern=on 7pt off 3pt ,line join=round,line cap=round] ( 42.00,129.40) -- (361.35,129.40);

\path[draw=drawColor,line width= 0.4pt,line join=round,line cap=round] ( 66.50, 77.25) --
	( 66.92, 75.59) --
	( 67.35, 75.59) --
	( 67.77, 75.59) --
	( 68.19, 75.59) --
	( 68.61, 75.59) --
	( 69.03, 83.63) --
	( 69.46, 83.63) --
	( 69.88,104.18) --
	( 70.30,105.44) --
	( 70.72,105.44) --
	( 71.15,105.44) --
	( 71.57, 99.67) --
	( 71.99, 99.67) --
	( 72.41, 99.67) --
	( 72.84, 99.67) --
	( 73.26, 99.67) --
	( 73.68, 99.67) --
	( 74.10, 99.66) --
	( 74.53, 99.66) --
	( 74.95,100.17) --
	( 75.37,100.17) --
	( 75.79,100.17) --
	( 76.22,110.32) --
	( 76.64,119.20) --
	( 77.06,119.20) --
	( 77.48,108.78) --
	( 77.91,107.73) --
	( 78.33,107.73) --
	( 78.75,107.73) --
	( 79.17,107.73) --
	( 79.60,107.73) --
	( 80.02,113.27) --
	( 80.44,114.53) --
	( 80.86,114.53) --
	( 81.29,115.44) --
	( 81.71,112.54) --
	( 82.13,112.54) --
	( 82.55,108.48) --
	( 82.97,108.48) --
	( 83.40,108.48) --
	( 83.82,108.48) --
	( 84.24,111.85) --
	( 84.66,111.85) --
	( 85.09,112.09) --
	( 85.51,112.09) --
	( 85.93,113.23) --
	( 86.35,113.12) --
	( 86.78,109.43) --
	( 87.20,109.43) --
	( 87.62,108.51) --
	( 88.04,108.51) --
	( 88.47,108.51) --
	( 88.89,107.84) --
	( 89.31,106.54) --
	( 89.73,106.91) --
	( 90.16,109.00) --
	( 90.58,109.00) --
	( 91.00,107.65) --
	( 91.42,106.11) --
	( 91.85,106.11) --
	( 92.27,106.57) --
	( 92.69,106.57) --
	( 93.11,105.89) --
	( 93.54,105.89) --
	( 93.96,101.80) --
	( 94.38,105.74) --
	( 94.80,105.63) --
	( 95.22,105.63) --
	( 95.65,105.63) --
	( 96.07,105.63) --
	( 96.49,106.90) --
	( 96.91,106.46) --
	( 97.34,107.19) --
	( 97.76,107.27) --
	( 98.18,108.09) --
	( 98.60,111.00) --
	( 99.03,111.23) --
	( 99.45,112.98) --
	( 99.87,112.98) --
	(100.29,115.97) --
	(100.72,116.33) --
	(101.14,117.01) --
	(101.56,117.01) --
	(101.98,117.02) --
	(102.41,118.85) --
	(102.83,118.85) --
	(103.25,121.39) --
	(103.67,121.50) --
	(104.10,121.46) --
	(104.52,120.45) --
	(104.94,119.37) --
	(105.36,119.50) --
	(105.79,119.14) --
	(106.21,119.14) --
	(106.63,119.14) --
	(107.05,119.14) --
	(107.48,119.08) --
	(107.90,119.08) --
	(108.32,119.08) --
	(108.74,118.21) --
	(109.16,118.21) --
	(109.59,116.93) --
	(110.01,116.71) --
	(110.43,116.09) --
	(110.85,116.09) --
	(111.28,116.09) --
	(111.70,116.11) --
	(112.12,117.14) --
	(112.54,116.49) --
	(112.97,116.59) --
	(113.39,116.81) --
	(113.81,116.81) --
	(114.23,116.81) --
	(114.66,116.81) --
	(115.08,116.81) --
	(115.50,116.81) --
	(115.92,116.81) --
	(116.35,116.79) --
	(116.77,116.79) --
	(117.19,116.79) --
	(117.61,118.29) --
	(118.04,118.29) --
	(118.46,118.29) --
	(118.88,116.89) --
	(119.30,116.89) --
	(119.73,116.65) --
	(120.15,115.01) --
	(120.57,114.43) --
	(120.99,114.43) --
	(121.42,114.43) --
	(121.84,114.43) --
	(122.26,114.34) --
	(122.68,113.25) --
	(123.10,113.64) --
	(123.53,113.64) --
	(123.95,113.64) --
	(124.37,113.40) --
	(124.79,113.40) --
	(125.22,111.70) --
	(125.64,111.00) --
	(126.06,111.00) --
	(126.48,113.30) --
	(126.91,113.30) --
	(127.33,113.94) --
	(127.75,114.15) --
	(128.17,114.15) --
	(128.60,114.11) --
	(129.02,114.11) --
	(129.44,114.30) --
	(129.86,114.30) --
	(130.29,114.21) --
	(130.71,114.21) --
	(131.13,114.21) --
	(131.55,114.08) --
	(131.98,114.08) --
	(132.40,114.08) --
	(132.82,114.08) --
	(133.24,114.08) --
	(133.67,114.08) --
	(134.09,114.08) --
	(134.51,113.95) --
	(134.93,113.81) --
	(135.35,113.81) --
	(135.78,113.81) --
	(136.20,113.81) --
	(136.62,113.82) --
	(137.04,113.82) --
	(137.47,113.82) --
	(137.89,113.72) --
	(138.31,113.62) --
	(138.73,114.57) --
	(139.16,114.57) --
	(139.58,114.57) --
	(140.00,114.57) --
	(140.42,113.53) --
	(140.85,113.53) --
	(141.27,112.97) --
	(141.69,113.38) --
	(142.11,112.87) --
	(142.54,112.69) --
	(142.96,112.69) --
	(143.38,112.69) --
	(143.80,112.69) --
	(144.23,112.69) --
	(144.65,112.70) --
	(145.07,112.70) --
	(145.49,112.14) --
	(145.92,112.14) --
	(146.34,112.14) --
	(146.76,111.91) --
	(147.18,111.91) --
	(147.61,111.91) --
	(148.03,110.73) --
	(148.45,110.73) --
	(148.87,110.73) --
	(149.29,110.73) --
	(149.72,110.73) --
	(150.14,110.79) --
	(150.56,110.79) --
	(150.98,110.79) --
	(151.41,110.79) --
	(151.83,111.13) --
	(152.25,110.86) --
	(152.67,110.87) --
	(153.10,110.87) --
	(153.52,110.87) --
	(153.94,110.87) --
	(154.36,110.87) --
	(154.79,110.87) --
	(155.21,110.87) --
	(155.63,110.87) --
	(156.05,110.87) --
	(156.48,110.87) --
	(156.90,109.84) --
	(157.32,109.84) --
	(157.74,109.84) --
	(158.17,109.84) --
	(158.59,109.81) --
	(159.01,108.68) --
	(159.43,108.68) --
	(159.86,108.68) --
	(160.28,108.68) --
	(160.70,108.68) --
	(161.12,108.66) --
	(161.55,108.66) --
	(161.97,108.66) --
	(162.39,109.60) --
	(162.81,110.38) --
	(163.23,110.38) --
	(163.66,110.52) --
	(164.08,109.65) --
	(164.50,109.65) --
	(164.92,109.65) --
	(165.35,109.67) --
	(165.77,109.67) --
	(166.19,109.67) --
	(166.61,109.67) --
	(167.04,109.67) --
	(167.46,110.23) --
	(167.88,110.23) --
	(168.30,110.41) --
	(168.73,110.41) --
	(169.15,110.41) --
	(169.57,110.41) --
	(169.99,110.41) --
	(170.42,110.41) --
	(170.84,110.41) --
	(171.26,110.41) --
	(171.68,110.41) --
	(172.11,110.37) --
	(172.53,110.37) --
	(172.95,110.37) --
	(173.37,110.37) --
	(173.80,110.37) --
	(174.22,110.37) --
	(174.64,110.37) --
	(175.06,110.37) --
	(175.48,110.37) --
	(175.91,110.37) --
	(176.33,110.37) --
	(176.75,110.37) --
	(177.17,110.37) --
	(177.60,110.37) --
	(178.02,110.37) --
	(178.44,110.37) --
	(178.86,110.37) --
	(179.29,110.37) --
	(179.71,110.37) --
	(180.13,110.37) --
	(180.55,110.37) --
	(180.98,110.37) --
	(181.40,110.37) --
	(181.82,110.37) --
	(182.24,110.37) --
	(182.67,110.37) --
	(183.09,110.37) --
	(183.51,110.37) --
	(183.93,110.37) --
	(184.36,110.37) --
	(184.78,110.37) --
	(185.20,110.37) --
	(185.62,110.37) --
	(186.05,111.17) --
	(186.47,111.17) --
	(186.89,111.17) --
	(187.31,111.17) --
	(187.74,111.17) --
	(188.16,111.17) --
	(188.58,111.17) --
	(189.00,111.17) --
	(189.42,111.17) --
	(189.85,111.17) --
	(190.27,111.17) --
	(190.69,111.17) --
	(191.11,111.17) --
	(191.54,111.17) --
	(191.96,111.17) --
	(192.38,111.27) --
	(192.80,111.27) --
	(193.23,111.27) --
	(193.65,111.27) --
	(194.07,111.27) --
	(194.49,111.27) --
	(194.92,111.27) --
	(195.34,111.27) --
	(195.76,111.27) --
	(196.18,111.27) --
	(196.61,111.27) --
	(197.03,111.27) --
	(197.45,111.27) --
	(197.87,111.27) --
	(198.30,111.27) --
	(198.72,111.27) --
	(199.14,111.27) --
	(199.56,111.27) --
	(199.99,111.27) --
	(200.41,111.27) --
	(200.83,111.27) --
	(201.25,111.27) --
	(201.68,111.27) --
	(202.10,111.27) --
	(202.52,111.27) --
	(202.94,111.27) --
	(203.36,111.27) --
	(203.79,111.27) --
	(204.21,111.27) --
	(204.63,111.27) --
	(205.05,111.27) --
	(205.48,111.27) --
	(205.90,111.27) --
	(206.32,111.27) --
	(206.74,111.07) --
	(207.17,111.07) --
	(207.59,111.07) --
	(208.01,111.07) --
	(208.43,111.07) --
	(208.86,111.07) --
	(209.28,111.07) --
	(209.70,111.07) --
	(210.12,111.07) --
	(210.55,110.40) --
	(210.97,110.40) --
	(211.39,110.58) --
	(211.81,110.58) --
	(212.24,110.58) --
	(212.66,110.58) --
	(213.08,110.58) --
	(213.50,110.58) --
	(213.93,110.58) --
	(214.35,110.58) --
	(214.77,110.58) --
	(215.19,110.58) --
	(215.61,110.58) --
	(216.04,110.58) --
	(216.46,110.29) --
	(216.88,110.29) --
	(217.30,110.29) --
	(217.73,110.29) --
	(218.15,110.29) --
	(218.57,110.29) --
	(218.99,110.29) --
	(219.42,110.29) --
	(219.84,110.29) --
	(220.26,110.29) --
	(220.68,110.29) --
	(221.11,110.29) --
	(221.53,110.29) --
	(221.95,110.29) --
	(222.37,110.29) --
	(222.80,110.29) --
	(223.22,110.29) --
	(223.64,110.88) --
	(224.06,110.88) --
	(224.49,110.88) --
	(224.91,110.88) --
	(225.33,110.88) --
	(225.75,110.88) --
	(226.18,110.88) --
	(226.60,110.88) --
	(227.02,110.88) --
	(227.44,110.88) --
	(227.87,110.88) --
	(228.29,110.88) --
	(228.71,111.92) --
	(229.13,111.92) --
	(229.55,111.92) --
	(229.98,111.92) --
	(230.40,111.92) --
	(230.82,111.92) --
	(231.24,111.92) --
	(231.67,111.92) --
	(232.09,111.92) --
	(232.51,111.92) --
	(232.93,111.92) --
	(233.36,111.92) --
	(233.78,111.92) --
	(234.20,111.92) --
	(234.62,111.92) --
	(235.05,111.92) --
	(235.47,111.92) --
	(235.89,111.92) --
	(236.31,111.92) --
	(236.74,111.92) --
	(237.16,111.92) --
	(237.58,111.92) --
	(238.00,111.92) --
	(238.43,111.92) --
	(238.85,111.92) --
	(239.27,111.92) --
	(239.69,111.92) --
	(240.12,111.92) --
	(240.54,111.92) --
	(240.96,111.92) --
	(241.38,111.92) --
	(241.80,111.92) --
	(242.23,111.92) --
	(242.65,111.92) --
	(243.07,111.92) --
	(243.49,111.92) --
	(243.92,111.92) --
	(244.34,111.92) --
	(244.76,111.92) --
	(245.18,111.92) --
	(245.61,111.92) --
	(246.03,111.92) --
	(246.45,111.92) --
	(246.87,111.92) --
	(247.30,111.92) --
	(247.72,111.92) --
	(248.14,111.92) --
	(248.56,111.92) --
	(248.99,111.92) --
	(249.41,111.92) --
	(249.83,111.92) --
	(250.25,111.92) --
	(250.68,111.92) --
	(251.10,111.92) --
	(251.52,111.92) --
	(251.94,111.92) --
	(252.37,111.92) --
	(252.79,111.92) --
	(253.21,111.92) --
	(253.63,111.92) --
	(254.06,111.92) --
	(254.48,111.92) --
	(254.90,111.92) --
	(255.32,111.92) --
	(255.74,111.92) --
	(256.17,111.92) --
	(256.59,111.92) --
	(257.01,111.92) --
	(257.43,112.01) --
	(257.86,112.01) --
	(258.28,112.01) --
	(258.70,112.01) --
	(259.12,112.01) --
	(259.55,112.01) --
	(259.97,112.01) --
	(260.39,112.01) --
	(260.81,112.01) --
	(261.24,112.01) --
	(261.66,112.01) --
	(262.08,112.01) --
	(262.50,112.01) --
	(262.93,112.01) --
	(263.35,112.01) --
	(263.77,112.01) --
	(264.19,112.01) --
	(264.62,112.01) --
	(265.04,112.01) --
	(265.46,112.01) --
	(265.88,112.01) --
	(266.31,112.01) --
	(266.73,112.01) --
	(267.15,112.01) --
	(267.57,112.01) --
	(268.00,112.01) --
	(268.42,112.50) --
	(268.84,112.50) --
	(269.26,112.50) --
	(269.68,112.50) --
	(270.11,112.50) --
	(270.53,112.50) --
	(270.95,112.50) --
	(271.37,112.50) --
	(271.80,112.50) --
	(272.22,112.50) --
	(272.64,112.50) --
	(273.06,112.50) --
	(273.49,112.50) --
	(273.91,112.50) --
	(274.33,112.50) --
	(274.75,112.50) --
	(275.18,112.50) --
	(275.60,112.50) --
	(276.02,112.50) --
	(276.44,112.50) --
	(276.87,112.50) --
	(277.29,112.50) --
	(277.71,112.50) --
	(278.13,112.50) --
	(278.56,112.50) --
	(278.98,112.50) --
	(279.40,112.50) --
	(279.82,112.50) --
	(280.25,112.50) --
	(280.67,112.50) --
	(281.09,112.50) --
	(281.51,112.50) --
	(281.93,112.50) --
	(282.36,112.50) --
	(282.78,112.50) --
	(283.20,112.04) --
	(283.62,112.04) --
	(284.05,112.04) --
	(284.47,112.04) --
	(284.89,112.04) --
	(285.31,112.04) --
	(285.74,112.04) --
	(286.16,112.04) --
	(286.58,112.04) --
	(287.00,112.04) --
	(287.43,112.04) --
	(287.85,112.04) --
	(288.27,112.04) --
	(288.69,112.04) --
	(289.12,112.04) --
	(289.54,112.04) --
	(289.96,112.04) --
	(290.38,112.04) --
	(290.81,112.04) --
	(291.23,112.04) --
	(291.65,112.04) --
	(292.07,112.04) --
	(292.50,112.04) --
	(292.92,112.04) --
	(293.34,112.04) --
	(293.76,112.04) --
	(294.19,112.04) --
	(294.61,112.04) --
	(295.03,112.04) --
	(295.45,112.04) --
	(295.87,112.04) --
	(296.30,112.04) --
	(296.72,112.04) --
	(297.14,112.04) --
	(297.56,112.04) --
	(297.99,112.04) --
	(298.41,112.04) --
	(298.83,112.04) --
	(299.25,112.04) --
	(299.68,112.04) --
	(300.10,112.04) --
	(300.52,112.04) --
	(300.94,112.04) --
	(301.37,112.04) --
	(301.79,112.04) --
	(302.21,112.04) --
	(302.63,112.04) --
	(303.06,112.04) --
	(303.48,112.04) --
	(303.90,112.04) --
	(304.32,112.04) --
	(304.75,112.04) --
	(305.17,112.04) --
	(305.59,112.04) --
	(306.01,112.04) --
	(306.44,112.04) --
	(306.86,112.04) --
	(307.28,112.04) --
	(307.70,112.04) --
	(308.12,112.04) --
	(308.55,112.04) --
	(308.97,112.04) --
	(309.39,112.04) --
	(309.81,112.04) --
	(310.24,112.04) --
	(310.66,112.04) --
	(311.08,112.04) --
	(311.50,112.04) --
	(311.93,112.04) --
	(312.35,112.04) --
	(312.77,112.04) --
	(313.19,112.04) --
	(313.62,112.04) --
	(314.04,112.04) --
	(314.46,112.04) --
	(314.88,112.04) --
	(315.31,112.04) --
	(315.73,112.04) --
	(316.15,112.04) --
	(316.57,112.04) --
	(317.00,112.04) --
	(317.42,112.04) --
	(317.84,112.04) --
	(318.26,112.04) --
	(318.69,112.04) --
	(319.11,112.04) --
	(319.53,112.04) --
	(319.95,112.04) --
	(320.38,112.04) --
	(320.80,112.04) --
	(321.22,112.04) --
	(321.64,112.04) --
	(322.06,112.04) --
	(322.49,112.04) --
	(322.91,112.04) --
	(323.33,112.04) --
	(323.75,112.04) --
	(324.18,112.04) --
	(324.60,112.04) --
	(325.02,112.04) --
	(325.44,112.04) --
	(325.87,112.04) --
	(326.29,112.04) --
	(326.71,112.04) --
	(327.13,112.04) --
	(327.56,112.04) --
	(327.98,112.04) --
	(328.40,112.04) --
	(328.82,112.04) --
	(329.25,112.04) --
	(329.67,112.04) --
	(330.09,112.04) --
	(330.51,112.04) --
	(330.94,112.04) --
	(331.36,112.04) --
	(331.78,112.04) --
	(332.20,112.04) --
	(332.63,112.04) --
	(333.05,112.04) --
	(333.47,112.04) --
	(333.89,112.04) --
	(334.32,112.04) --
	(334.74,112.04) --
	(335.16,112.04) --
	(335.58,112.04) --
	(336.00,112.04) --
	(336.43,112.04) --
	(336.85,112.04) --
	(337.27,112.04) --
	(337.69,112.04) --
	(338.12,112.04) --
	(338.54,112.04) --
	(338.96,112.04) --
	(339.38,112.04) --
	(339.81,112.04) --
	(340.23,112.04) --
	(340.65,112.04) --
	(341.07,112.04) --
	(341.50,112.04) --
	(341.92,112.04) --
	(342.34,112.04) --
	(342.76,112.04) --
	(343.19,112.04) --
	(343.61,112.04) --
	(344.03,112.04) --
	(344.45,112.04) --
	(344.88,112.04) --
	(345.30,112.04) --
	(345.72,112.04) --
	(346.14,112.04) --
	(346.57,112.04) --
	(346.99,112.04) --
	(347.41,112.04) --
	(347.83,112.04) --
	(348.25,112.04) --
	(348.68,112.04) --
	(349.10,112.04) --
	(349.52,112.04);
\end{scope}
\end{tikzpicture}
}
	\end{adjustbox}
	
	\caption[Einfluss des \gls{rmse}-Grenzwerts der \gls{ha} auf den Zusammenhang zwischen dem y-Achsenabschnitt, der Steigung und dem \gls{zwert} des \gls{bist}s]{Einfluss des \gls{rmse}-Grenzwerts auf die Zusammenhänge der aus der \gls{ha} mit einer linearen Regression abgeleiteten Aufgabenparameter dem ($a$) y-Achsenabschnitt und der ($b$) Steigung mit dem \gls{zwert} des \gls{bist}s. Die durchgezogene Linie kennzeichnet den Verlauf des Zusammenhangs. Der graue Bereich beschreibt das $95\,\%$-Konfidenzintervall.}
	\label{fig:hick_rmse_cutoff}
\end{figure}

Die Steigung und der \gls{zwert} des \gls{bist}s korrelierten bei einem \gls{rmse}-Grenzewert zwischen $8.9$ und $10.7$~ms ($r = -.24$ bis $ -.37$, alle $p\textnormal{s} < .048$) und ab $15.7$~ms ($r = -.19$ bis $ -.28$, alle $p\textnormal{s} < .046$) signifikant miteinander (siehe \autoref{fig:hick_rmse_cutoff}b). In diesen Bereichen waren geringe Steigungen also tendenziell mit höheren \gls{zwert}en verbunden. Wie beim Zusammenhang zwischen dem y-Achsenabschnitt und dem \gls{zwert} konnte eine visuelle Inspektion des Verlaufs keine klaren Hinweise dafür liefern, ob der \gls{rmse}-Grenzwert einen positiven oder negativen Einfluss auf die Höhe des Zusammenhangs zwischen der Steigung und dem \gls{zwert} ausübte.
Betrachtet man die mit der Gesamtstichprobe ermittelten Ergebnisse, kann festgehalten werden, dass sowohl der y-Achsenabschnitt ($r=-.17$, $p=.02$) als auch die Steigung ($r=-.23$, $p=.002$) der \gls{ha} leicht negativ mit dem \gls{zwert} des \gls{bist}s korrelierte. %Die Zusammenhänge der Aufgabenparameter der Hick- und der \gls{ssauf} mit dem \gls{zwert} des \gls{bist}s sind in  zu finden.
%Diese Zusammenhänge waren vergleichbar mit den von \citet{Jensen1987a} berichteten Ergebnissen.

Nachdem die Aufgabenparameter der \gls{ha} bestimmt waren, konnte der \gls{zwert} des \gls{bist}s mit den Aufgabenparameter der Hick- und der \gls{ssauf} vorhergesagt werden. Für diese multiple Regressionsanalyse wurde die Gesamtstichprobe verwendet, weil die Analysen zum Einfluss des \gls{rmse}-Grenzwerts auf die Zusammenhänge der Aufgabenparameter mit dem \gls{zwert} kein eindeutiges Ergebnis lieferten (siehe \autoref{fig:spatial_suppression_asymtote_slope_zscore} und \autoref{fig:hick_rmse_cutoff}). Um die Abhängigkeiten der Aufgabenparameter zwischen den beiden Aufgaben (siehe \autoref{tab:suppression_hick_regression_correlations}) bei der Vorhersage des \gls{zwert}s zu berücksichtigen, wurden die Aufgabenparameter in Gruppen zusammengefasst und nacheinander blockweise in die multiple Regressionsanalyse aufgenommen. 

\begin{table}[htbp] % see http://tex.stackexchange.com/questions/247921/different-column-widths-under-a-multicolumn-prevent-appropriate-centering
	\centering
	\captionsetup{labelsep = none}
	\caption[Produkt-Moment-Korrelationen zwischen dem \gls{zwert} des \gls{bist}s und den Aufgabenparameter der Spa\-ti\-al-Sup\-pres\-sion- und der \gls{ha}]{\newline  \textit{Produkt-Moment-Korrelationen zwischen den aus der Spatial-Suppression- und der \gls{ha} regressionsanalytisch abgeleiteten Aufgabenparametern und dem \textit{z}-Wert des \gls{bist}s} \vspace{.2cm}}
	\label{tab:suppression_hick_regression_correlations}
	\begin{adjustbox}{width=1\textwidth}
		\begin{threeparttable}
			\newlength{\tempdima}
			\settowidth{\tempdima}{\gls{ha}}% compute width needed
			\addtolength{\tempdima}{-2\tabcolsep}% minus default column sep
			\newlength{\tempdimb}
			\settowidth{\tempdimb}{\gls{ssauf}}% compute width needed
			\addtolength{\tempdimb}{-2\tabcolsep}% minus default column sep
			\begin{tabular}{
%				p{.1cm}
				l
				l
				S[table-format = 1.2, add-integer-zero=false, table-space-text-post = $^{*}$]
				S[table-format = 1.2, add-integer-zero=false, table-space-text-post = $^{**}$]
				p{.001cm}
				S[table-format = 1.2, add-integer-zero=false,table-space-text-post = $^{***}$]
				S[table-format = 0.2, add-integer-zero=false]
				p{.001cm}
				S[table-format = 0.0, add-integer-zero=false]
				>{\centering\arraybackslash}p{1.2cm}
				}
			\hline
				&		& 	\multicolumn{2}{c}{\gls{ha}}	&	&	\multicolumn{2}{c}{\gls{ssauf}}	&	&	\multicolumn{1}{c}{\gls{bist}}	\\
			\cline{3-4}
			\cline{6-7}
			\cline{9-9}

			& \multicolumn{1}{c}{Parameter} & {\makebox[0.5\tempdima]{1}} & {\makebox[0.5\tempdima]{2}} && {\makebox[0.5\tempdimb]{3}} & {\makebox[0.5\tempdimb]{4}} && {5}\\
			\hline
		1	&	y-Achsenabschnitt	&				&					&& 					&		&&	\\
		2	&	Steigung			&	-.08		&					&& 					&		&&	\\
		\rule{0pt}{4ex}%  EXTRA vertical height
		3	&	Asymptote			&	.16{$^{*}$}	&	.03				&&					&		&&	\\
		4	&	Steigung			&	-.02		&	-.03			&&	-.57{$^{***}$}	&		&&	\\
		\rule{0pt}{4ex}%  EXTRA vertical height
		5	&	\textit{z}-Wert		&	-.17{$^{*}$}&	-.23{$^{**}$}	&&	-.16{$^{*}$}	&	.00	&&	\\
			\hline
			\end{tabular}

			\begin{tablenotes}[flushleft]
				\footnotesize				% font size
				\setlength\labelsep{0pt}	% no indent on second line
%				\item \textit{Anmerkungen.} \textit{a}~=~Asymptote bei der \gls{ssauf}, Achsenabschnitt bei der \gls{ha}; \textit{b}~=~Steigung. \\
				\item $^{*}p~<~.05$. $^{**}p~<~.01$. $^{***}p~<~.001$ (zweiseitig).
			\end{tablenotes}
		\end{threeparttable}
	\end{adjustbox}
\end{table}



Grundlage für die Beantwortung der Fragestellung bildete Modell 12, in welchem der \gls{zwert} des \gls{bist}s alleinig mit den Aufgabenparametern der \gls{ha} vorhergesagt wurde (siehe \autoref{tab:multiple_regression_all_parameters}).
Die Regressionsanalyse hat ergeben, dass bei einer Kontrolle für den Zusammenhang zwischen den Aufgabenparametern sowohl der y-Achsenabschnitt ($\upbeta=-.19$, $p=.009$) als auch die Steigung ($\upbeta=-.24$, $p<.001$) den \gls{zwert} signifikant vorhersagten. Gemeinsam erklärten sie mit $9\,\%$ einen signifikanten Varianzanteil im \gls{zwert}, $F(2,\,174)=8.52$, $p<.001$, $R^2=.09$. Tiefere y-Achsenabschnitte und geringere Steigungen gingen also tendenziell mit höheren \gls{zwert}en einher.

%Gemeinsam ekrlärten sie einen signifkanten Varianzanteil von 9%

\begin{table}[htbp]
	\centering
	\captionsetup{labelsep = none}
	\caption[Multiple Regression zur Vorhersage des \gls{zwert}s des \gls{bist}s durch die Aufgabenparameter der Spa\-ti\-al-Sup\-pres\-sion- und der \gls{ha}]{\newline  \textit{Multiple Regression zur Vorhersage des \gls{zwert}s des \gls{bist}s durch die Aufgabenparameter der \gls{ha} (Modell 12) respektive durch die Aufgabenparameter der Hick- und der \gls{ssauf} (Modell 13)} \vspace{.2cm}}
	\label{tab:multiple_regression_all_parameters}
	\newcommand{\rowgroup}[1]{\hspace{-1em}#1}
	\newcommand\Tstrut{\rule{0pt}{2.1ex}}       % top strut http://tex.stackexchange.com/questions/65919/space-between-rows-in-a-table - not implemented!
	\begin{adjustbox}{width=1\textwidth}
		\begin{threeparttable}
			\begin{tabular}{
				>{\quad}
				l
				S[table-format = 1.4]
				S[table-format = 0.4]
				S[table-format = 1.2, add-integer-zero=false]
				S[table-format = <0.3, add-integer-zero=false]
				p{.001cm}
				S[table-format = 1.2, table-space-text-post = $^{***}$]
				S[table-format = 0.2, add-integer-zero=false]
				S[table-format = 1.2]
				S[table-format = 0.2, add-integer-zero=false]
				>{\centering\arraybackslash}p{1.2cm}
				}
			\hline

			\multicolumn{1}{c}{Prädiktor}	&	{\textit{B}}	&	{\textit{SE}(\textit{B})}	&	{$\upbeta$}	&	{$p$}	& &	{$F$}	&	{$R^2$}	& {$\Delta F$} & {$\Delta R^2$}	\\

			\hline

			\rowgroup{Modell 12}	&			&			&			&			&	&	8.52{$^{***}$}	&	.09		&					\\
			H-y-Achsenabschnitt		&	-0.0037	&	0.0014	&	-.19	&	.009	&	&					&			&					\\
			H-Steigung				&	-0.0058	&	0.0017	&	-.24	&	<.001	&	&					&			&					\\

			\rule{0pt}{4ex}			%  EXTRA vertical height

			\rowgroup{Modell 13}	&			&			&			&			&	&	5.58{$^{***}$}	&	.12		& 2.49	&	.03		\\
			H-y-Achsenabschnitt		&	-0.0031	&	0.0014	&	-.16	&	.03		&	&					&			&					\\
			H-Steigung				&	-0.0057	&	0.0017	&	-.24	&	.001	&	&					&			&					\\
			S-Asymptote				&	-0.0037	&	0.0017	&	-.20	&	.03		&	&					&			&					\\
			S-Steigung				&	-0.7887	&	0.5695	&	-.12	&	.17		&	&					&			&					\\
			\hline
		\end{tabular}

		\begin{tablenotes}[flushleft]
			\footnotesize				% font size
			\setlength\labelsep{0pt}	% no indent on second line
			\item \textit{Anmerkungen}. $B$ = unstandardisiertes Regressionsgewicht; $\upbeta$ = standardisiertes Regressionsgewicht; $F$ = $F$-Wert des Regressionsmodells; $R^2$ = erklärte Varianz; $\Delta F$ = $F$-Wert der Veränderung der erklärten Varianz; $\Delta R^2$ = zusätzlich erklärte Varianz; H = \gls{ha}; S = \gls{ssauf}.
			\item {$^{**}$}$p<.01$ (zweiseitig).
		\end{tablenotes}
	\end{threeparttable}
	\end{adjustbox}
\end{table}

Model 13 beinhaltete als Prädiktoren die Aufgabenparameter der Hick- und der \gls{ssauf} (siehe \autoref{tab:multiple_regression_all_parameters}). 
Die Regressionsanalyse hat ergeben, dass bei einer Kontrolle für die Zusammenhänge zwischen den Aufgabenparametern der y-Achsenabschnitt der \gls{ha}  ($\upbeta~=~-.16$, $p~=~.03$), die Steigung der \gls{ha} ($\upbeta~=~-.24$, $p~=~.001$) und die Asymptote der \gls{ssauf} ($\upbeta~=~-.20$, $p~=~.03$) den \gls{zwert} signifikant vorhersagten. Die Steigung der \gls{ssauf} war mit ($\upbeta~=~-.12$, $p~=~.17$) kein signifikanter Prädiktor des \gls{zwert}s.
Gemeinsam erklärten die Prädiktoren mit $12\,\%$ einen signifikanten Varianzanteil im \gls{zwert}, $F(4,\,172)=5.58$, $p<.001$, $R^2=.12$. 
Tiefere y-Achsenabschnitte und geringere Steigungen in der \gls{ha} sowie tiefere Asymptoten in der \gls{ssauf} gingen also tendenziell mit höheren \gls{zwert}en einher.

Um zu prüfen, ob die Aufgabenparameter der \gls{ssauf} einen inkrementellen Beitrag zur Varianzaufklärung im \gls{zwert} des \gls{bist}s leisteten, wurde der Zuwachs an erklärter Varianz im \gls{zwert} zwischen Modell 12 und Modell 13 auf Signifikanz getestet. 
Dabei hat sich ergeben, dass $\Delta R^2=.03$ kein signifikanter Zuwachs an erklärter Varianz darstellte, $F(2,\,172)=2.49$, $p=.09$.
Obwohl die Asymptote der \gls{ssauf} ein signifikanter Prädiktor des \gls{zwert}s war, haben die Asymptote und die Steigung der \gls{ssauf} auf Ebene der Aufgabenparameter also keinen inkrementellen Beitrag zur Aufklärung individueller Intelligenzunterschiede geleistet.








\subsection{Analyse auf latenter Ebene}

\subsubsection*{Mess- und Strukturgleichungsmodell}

Bevor die \gls{ha} mit der \gls{ssauf} und dem \gls{gfaktor} in Verbindung gesetzt werden konnte, musst für die \gls{ha} das kongenerische Messmodell bestimmt werden. Modell 14 (siehe \autoref{fig:hick_congeneric_model}) 
\begin{figure}[!t]
	\centering
	\begin{tikzpicture}
	[font=\sffamily, scale=2, inner sep=0pt,
	latent/.style	= {circle, draw, inner sep=0pt, minimum size=12mm},
	manifest/.style	= {rectangle, draw, inner sep=0pt, minimum width=12mm, minimum height=12mm},
	paths/.style	= {->, >=stealth, shorten >= 1pt},
	error/.style	= {circle, draw=none, fill=white, minimum size=5mm},
	covar/.style	= {<->, >=stealth, shorten >= 1pt, shorten <= 1pt}]
	
	\node at (0, 1.7)		[latent]	(hick)	{H};
	
	\node at (-1.5, 2.9)	[manifest]	(h0)	{0-bit};
	\node at (-1.5, 2.1)	[manifest]	(h1)	{1-bit};
	\node at (-1.5, 1.3)	[manifest]	(h2)	{2-bit};
	\node at (-1.5, 0.5)	[manifest]	(h3)	{2.58-bit};
	
	\node at (-2.3, 2.9)	[error]		(e1)	{\footnotesize .55};
	\node at (-2.3, 2.1)	[error]		(e2)	{\footnotesize .35};
	\node at (-2.3, 1.3)	[error]		(e3)	{\footnotesize .15};
	\node at (-2.3, 0.5)	[error]		(e4)	{\footnotesize .25};
	
	\draw [paths] (hick.west) -- (h0.east) node[minimum size = 4mm, draw=none, fill=white, midway] {\footnotesize .67{$^{1}$}\hphantom{$^**$}};	
	\draw [paths] (hick.west) -- (h1.east) node[minimum size = 4mm, draw=none, fill=white, midway] {\footnotesize .80{$^{***}$}};	
	\draw [paths] (hick.west) -- (h2.east) node[minimum size = 4mm, draw=none, fill=white, midway] {\footnotesize .92{$^{***}$}};	
	\draw [paths] (hick.west) -- (h3.east) node[minimum size = 4mm, draw=none, fill=white, midway] {\footnotesize .87{$^{***}$}};
	
	\draw [paths] (e1) -- (h0.west) {};
	\draw [paths] (e2) -- (h1.west) {};
	\draw [paths] (e3) -- (h2.west) {};
	\draw [paths] (e4) -- (h3.west) {};
	\end{tikzpicture}
	
	\vspace{.2cm}
	\caption[Modell 14: Kongenerisches Messmodell der \gls{ha}]{Modell 14: Kongenerisches Messmodell der \gls{ha} (\textsf{H}). Eingezeichnet sind die standardisierten Koeffizienten.\\
		$^1$Um die Identifizierung der Varianz der latenten Variable zu ermöglichen, wurde diese unstandardisierte Faktorladung auf $1$ fixiert.\\
		$^{***}p~<~.001$.}
	\label{fig:hick_congeneric_model}
\end{figure}
bildete die empirischen Varianzen und Kovarianzen der \gls{ha} schlecht ab.  Der \gls{cst} zeigte eine überzufällig hohe Abweichung zwischen der theoretische und der empirischen Var\-ianz-Ko\-var\-ianz\-ma\-trix an und der \gls{cfi} und \gls{rmsea} lagen weit weg vom akzeptablen Bereich, $\upchi^2(2)=42.58$, $p<.001$, $\textnormal{CFI}=.87$, $\textnormal{RMSEA}=.33$, $\textnormal{SRMR}=.06$.






\begin{figure}[p]
	\begin{adjustbox}{width=1\textwidth, keepaspectratio}
		\begin{tikzpicture}
		[font=\sffamily, scale=2, inner sep=0pt,
		latent/.style	= {circle,draw,inner sep=0pt,minimum size=12mm},
		manifest/.style	= {rectangle,draw,inner sep=0pt,minimum width=12mm,minimum height=12mm},
		paths/.style	= {->, >=stealth, shorten >= 1pt},
		error/.style	= {circle, draw=none, fill=white, minimum size=5mm},
		covar/.style	= {<->, >=stealth, shorten >= 1pt, shorten <= 1pt}]
		
		
		\node at (0, 1.7)		[latent]	(sup)	{S};
		\node at (0, -1.7)		[latent]	(hick)	{H};
		\node at (1.5, 0)		[latent]	(g)		{\textrm{\textit{g}}};
		
		\node at (-1.5, 2.9)	[manifest]	(s1)	{1.8$^{\circ}$};
		\node at (-1.5, 2.1)	[manifest]	(s2)	{3.6$^{\circ}$};
		\node at (-1.5, 1.3)	[manifest]	(s3)	{5.4$^{\circ}$};
		\node at (-1.5, 0.5)	[manifest]	(s4)	{7.2$^{\circ}$};
		
		\node at (-1.5, -0.5)	[manifest]	(h0)	{0-bit};
		\node at (-1.5, -1.3)	[manifest]	(h1)	{1-bit};
		\node at (-1.5, -2.1)	[manifest]	(h2)	{2-bit};
		\node at (-1.5, -2.9)	[manifest]	(h3)	{2.58-bit};
		
		\node at (3, .8)		[manifest]	(k)		{K};
		\node at (3, 0)			[manifest]	(b)		{B};
		\node at (3, -.8)		[manifest]	(m)		{M};
		
		\node at (-2.3, 2.9)	[error]		(e1)	{\footnotesize .44};
		\node at (-2.3, 2.1)	[error]		(e2)	{\footnotesize .23};
		\node at (-2.3, 1.3)	[error]		(e3)	{\footnotesize .02};
		\node at (-2.3, .5)		[error]		(e4)	{\footnotesize .24};
		
		\node at (-2.3, -.5)	[error]		(e5)	{\footnotesize .54};
		\node at (-2.3, -1.3)	[error]		(e6)	{\footnotesize .35};
		\node at (-2.3, -2.1)	[error]		(e7)	{\footnotesize .16};
		\node at (-2.3, -2.9)	[error]		(e8)	{\footnotesize .25};
		
		\node at (3.8, .8)		[error]		(e9)	{\footnotesize .31};
		\node at (3.8, 0)		[error]		(e10)	{\footnotesize .47};
		\node at (3.8, -.8)		[error]		(e11)	{\footnotesize .71};
		
		\node at (1.5, 0.8)		[error]		(e12)	{\footnotesize .84};
		
		\draw [paths] (sup.west) -- (s1.east) node[minimum size = 4mm, draw=none, fill=white, midway] {\footnotesize .75{$^{1}$}\hphantom{$^**$}};	
		\draw [paths] (sup.west) -- (s2.east) node[minimum size = 4mm, draw=none, fill=white, midway] {\footnotesize .88{$^{***}$}};	
		\draw [paths] (sup.west) -- (s3.east) node[minimum size = 4mm, draw=none, fill=white, midway] {\footnotesize .99{$^{***}$}};	
		\draw [paths] (sup.west) -- (s4.east) node[minimum size = 4mm, draw=none, fill=white, midway] {\footnotesize .87{$^{***}$}};
		
		\draw [paths] (hick.west) -- (h0.east) node[minimum size = 4mm, draw=none, fill=white, midway] {\footnotesize .68{$^{1}$}\hphantom{$^**$}};	
		\draw [paths] (hick.west) -- (h1.east) node[minimum size = 4mm, draw=none, fill=white, midway] {\footnotesize .81{$^{***}$}};	
		\draw [paths] (hick.west) -- (h2.east) node[minimum size = 4mm, draw=none, fill=white, midway] {\footnotesize .92{$^{***}$}};	
		\draw [paths] (hick.west) -- (h3.east) node[minimum size = 4mm, draw=none, fill=white, midway] {\footnotesize .87{$^{***}$}};
		
		\draw [paths] (g.east) -- (k.west) node[minimum size = 4mm, draw=none, fill=white, midway] {\footnotesize .83{$^{1}$}\hphantom{$^**$}};	
		\draw [paths] (g.east) -- (b.west) node[minimum size = 4mm, draw=none, fill=white, midway] {\footnotesize .73{$^{***}$}};	
		\draw [paths] (g.east) -- (m.west) node[minimum size = 4mm, draw=none, fill=white, midway] {\footnotesize .54{$^{***}$}};	
		
		\draw [paths] (e1) -- (s1.west);
		\draw [paths] (e2) -- (s2.west);
		\draw [paths] (e3) -- (s3.west);
		\draw [paths] (e4) -- (s4.west);
		
		\draw [paths] (e5) -- (h0.west);
		\draw [paths] (e6) -- (h1.west);
		\draw [paths] (e7) -- (h2.west);
		\draw [paths] (e8) -- (h3.west);
		
		\draw [paths] (e9) -- (k.east);
		\draw [paths] (e10) -- (b.east);
		\draw [paths] (e11) -- (m.east);
		
		\draw [paths] (e12) -- (g.north);
		
		\path [covar] (hick.north) edge [bend left=45] node[minimum size = 4mm, draw=none,fill=white,midway]  {\footnotesize .14} (sup.south);
		\draw [paths] (sup)  -- (g.west) node[minimum size = 4mm, draw=none,fill=white,midway] {\footnotesize --.19{$^{*}$}\hphantom{$^**$}};
		\draw [paths] (hick) -- (g.west) node[minimum size = 4mm, draw=none,fill=white,midway] {\footnotesize --.33{$^{***}$}};
		
		\end{tikzpicture}
	\end{adjustbox}
	\vspace{.2cm}
	\caption[Modell 15: Strukturgleichungsmodell zur Vorhersage des \gls{gfaktor}s durch die Spa\-ti\-al-Sup\-pres\-sion- und die \gls{ha}]{Modell 15: Latenter Zusammenhang zwischen der \gls{ssauf} (\textsf{S}), der \gls{ha} (\textsf{H}) und dem \gls{gfaktor} des \gls{bist}. \textsf{K} = Kapazität; \textsf{B} = Bearbeitungsgeschwindigkeit; \textsf{M} = Merkfähigkeit.\\
		$^1$Um die Identifizierung der Varianz der latenten Variable zu ermöglichen, wurde diese unstandardisierte Faktorladung auf $1$ fixiert.\\
		$^{*}p~<~.05$. $^{***}p~<~.001$.}
	\label{fig:spatial_suppression_hick_g_model}
\end{figure}

Trotz des schlechten kongenerischen Modell-Fits der \gls{ha} wurde Modell 14 in einem Strukturgleichungsmodell mit dem kongenerischen Messmodell der \gls{ssauf} (Modell 1; siehe \autoref{fig:spatial_suppression_congeneric_model}) und dem \gls{gfaktor} des \gls{bist} in Verbindung gebracht. Das theoretische Modell (Modell 15; siehe \autoref{fig:spatial_suppression_hick_g_model}) bildete die empirischen Daten erneut schlecht ab.  Der \gls{cst} zeigte eine überzufällig hohe Abweichung zwischen der theoretische und der empirischen Var\-ianz-Ko\-var\-ianz\-ma\-trix an und der \gls{cfi} und \gls{rmsea} lagen nicht im akzeptablen Bereich, $\upchi^2(41)=205.68$, $p<.001$, $\textnormal{CFI}=.86$, $\textnormal{RMSEA}=.15$, $\textnormal{SRMR}=.06$. 
Der standardisierte Regressionskoeffizient zwischen der aus den vier Bedingungen der \gls{ssauf} extrahierten latenten Variable und dem \gls{gfaktor} betrug $\upbeta~=~-.19$ ($p~=~.03$). Der standardisierte Regressionskoeffizient zwischen der aus den vier Bedingungen der \gls{ha} extrahierten latenten Variable und dem \gls{gfaktor} betrug $\upbeta~=~-.33$ ($p~<~.001$). Der Korrelationskoeffizient zwischen den aus den vier Bedingungen der Spatial-Suppression- und \gls{ha} extrahierten latenten Variablen betrug $r~=~.14$ ($p~=~.16$). Gemeinsam erklärten diese beiden latenten Variablen $16\,\%$ der Varianz im \gls{gfaktor}.




\subsubsection*{Fixed-Links-Mess- und Strukturgleichungsmodell}

Für die Analyse der Zusammenhänge auf latenter Ebene mittels Fixed-Links-Modellen musste für die \gls{ha} zuerst ein Fixed-Links\--Mess\-modell gefunden werden. Das Vorgehen bei der Bestimmung des Fixed-Links-Mess\-mod\-el\-ls war dabei identisch mit dem Vorgehen bei der Bestimmung des Fixed-Links-Mess\-mod\-el\-ls für die \gls{ssauf} (siehe \autoref{subsec:spatial_suppression_fixed_links_messmodell}). Alle Modell-Fits der in den folgenden Paragraphen berichteten \gls{flm}e sind in \autoref{tab:hick_fixedlinks_measurement_models} aufgeführt.

\begin{table}[htbp]
	%\flushleft
	\centering
	\captionsetup{labelsep = none}
	\caption[Modell-Fits der Fixed-Links-Messmodelle der \gls{ha}]{\newline  \textit{Modell-Fits der Fixed-Links-Messmodelle der \gls{ha}. Der Ladungsverlauf bezieht sich auf die unstandardisierten Faktorladungen der dynamischen latenten Variable. Die unstandardisierten Faktorladungen der konstanten latenten Variable betrugen immer 1} \vspace{.2cm}}
	\label{tab:hick_fixedlinks_measurement_models}
	\begin{adjustbox}{width=1\textwidth, keepaspectratio}
		\begin{threeparttable}
			
			{\renewcommand{\arraystretch}{1.0} % <- modify value to suit your needs: line spacing inside table
				\begin{tabular}{
						%						S[table-format = 2.0, table-space-text-post = $^{*a}$]
						l
						l
						S[table-format = 2.2]
						S[table-format = 1.0]
						S[table-format = <0.3, add-integer-zero=false]
						S[table-format = 0.3, add-integer-zero=false]
						S[table-format = 0.3, add-integer-zero=false]
						S[table-format = 0.3, add-integer-zero=false]
						%				S[table-format = 1.2, add-integer-zero=false]
					}
					
					\hline
					\multicolumn{1}{c}{Modell}		& \multicolumn{1}{c}{Ladungsverlauf}	&	{$\upchi^2$}	& \textit{df}	& {\textit{p}}	&	{\textnormal{CFI}} 	&	{\textnormal{RMSEA}}	&	{\textnormal{SRMR}}\\
					\hline
					16{$^{*}$}	&	$y=\log_{e}x$					&	57.55	&	4	&	<.001	&	.825	&	.275	&	.197	\\
					17			&	$y=x$							&	37.60	&	4	&	<.001	&	.890	&	.218	&	.169	\\
					18			&	$y=\log_{2}x$					&	32.20	&	4	&	<.001	&	.908	&	.200	&	.136	\\
					19			&	$y=2^x$							&	13.33	&	4	&	.010	&	.970	&	.115	&	.072	\\
					20			&	$y=x^2$							&	11.37	&	4	&	.023	&	.976	&	.102	&	.070	\\
					21			&	$y=x$							&	8.76	&	4	&	.067	&	.984	&	.082	&	.089	\\
					22			&	$y=\dfrac{1}{1+e^{(-x/.8)}}$	&	4.50	&	4	&	.342	&	.998	&	.027	&	.076	\\
					%								\rule{0pt}{1ex} \\%  EXTRA vertical height
					\hline
					
					
				\end{tabular}%
			}
			\begin{tablenotes}[flushleft]
				\footnotesize				% font size
				\setlength\labelsep{0pt}	% no indent on second line
				\item \textit{Anmerkungen.} Für Modelle 16, 17, 19 und 20 gilt $x\in\{1,2,3,4\}$. Für Modelle 18 und 21 gilt $x\in\{1,2,4,6\}$ und für Modell 21 $x\in\{-3,-1,1,3\}$. $\upchi^2 =$ Satorra-Bentler \citeyearpar{Satorra1994} korrigierter $\upchi^2$-Wert; \textit{df} = Freiheitsgrade; \gls{cfi} = comparative fit index; \gls{rmsea} = root mean square error of approximation; \gls{srmr} = standardized root mean square residual.
				\item {$^{*}$}Das Modell konnte nicht interpretiert werden, weil die Fehlervarianz der 0-bit-Bedingung negativ geschätzt wurde.
				
			\end{tablenotes}%
		\end{threeparttable}
	\end{adjustbox}	
\end{table}



Das erste berechnete Fixed-Links-Modell (Modell 16) berücksichtige die von Blank \citep[1934; zitiert nach][S. 103]{Jensen1987a} formulierte logarithmische Beziehung zwischen der Anzahl Antwortalternativen und der Reaktionszeit. Die unstandardisierten Faktorladungen der dynamischen latenten Variablen wurden deshalb mit einer logarithmischen Funktion ($y=\log_{e}x,\,x\in\{1, 2, 3, 4\}$) bestimmt. Dieses Modell konnte nicht interpretiert werden, weil die Fehlervarianz der 0-bit-Bedingung negativ geschätzt wurde.

\citet{Schweizer2006a} hat in seiner Arbeit für die dynamische latenten Variable der \gls{ha} unter anderen Verläufen auch einen linearen Verlauf eingesetzt. 
Die unstandardisierten Faktorladungen der dynamischen latenten Variable in Modell 17 wurden deshalb linear ansteigend ($y=x,\,x\in\{1, 2, 3, 4\}$) fixiert. Das Modell bildete die empirischen Varianzen und Kovarianzen der \gls{ha} nicht gut ab. Der $\upchi^2$-Wert war hochsignifikant und der \gls{cfi}, der \gls{rmsea} und das \gls{srmr} lagen weit weg vom akzeptablen Bereich.

In Modell 18 wiesen die unstandardisierten Faktorladungen der dynamischen latenten Variablen einen Verlauf entsprechend den verwendeten Bit-Bedingungen auf ($y=\log_{2}x,\,x\in\{1, 2, 4, 6\}$). Das Modell bildete die empirischen Varianzen und Kovarianzen der \gls{ha} nicht adäquat ab. Zwar reduzierte sich der $\upchi^2$-Wert im Vergleich zu Modell 17 leicht, der \gls{cst} war aber immer noch signifikant. Weiter deuteten der \gls{cfi}, der \gls{rmsea} und das \gls{srmr} mit Werten ausserhalb des akzeptablen Bereichs auf eine schlechte Modellpassung hin.

Die unstandardisierten Faktorladungen der dynamischen latenten Variablen von Modell 19 wurden mit einer exponentiellen Funktion ($y=2^x,\,x\in\{1, 2, 3, 4\}$) bestimmt. Der \gls{cst} zeigte eine überzufällig hohe Abweichung zwischen der theoretische und der empirischen Var\-ianz-Ko\-var\-ianz\-ma\-trix an. Gemeinsam mit dem hohen \gls{rmsea} deutete dies darauf hin, dass das Modell die empirischen Varianzen und Kovarianzen der \gls{ha} nicht adäquat abbildete.

In Modell 20 wurden die unstandardisierten Faktorladungen der dynamischen latenten Variablen mit dem von \citet{Schweizer2006a} verwendeten quadratischen Verlauf ($y=x^2,\,x\in\{1, 2, 3, 4\}$) gebildet. Der $\upchi^2$-Wert reduzierte sich im Vergleich zu Modell 19 zwar leicht, war aber immer noch signifikant. Die schlechte Passung des Modells wurde weiter durch einen hohen \gls{rmsea} angezeigt.

Modell 21 testete die Annahme, dass die Ladungen der unstandardisierten Faktorladungen der dynamischen latenten Variable entsprechend der Anzahl Antwortalternativen verlaufen ($y=x,\,x\in\{1, 2, 4, 6\}$). Der \gls{cst} erkannte keine signifikante Abweichung zwischen der von Modell 21 implizierten und der empirischen Var\-ianz-Ko\-var\-ianz\-ma\-trix. Der \gls{rmsea} und das \gls{srmr} hingegen lagen ausserhalb des akzeptablen Bereichs.

Die unstandardisierten Faktorladungen der dynamischen latenten Variable von Modell 22 (siehe \autoref{fig:hick_fixedlinks_measurement_model}) wurden mit einer logistischen Funktion bestimmt $\{y={1}/[{1 + e^{(-x/.8)}}],\,x\in\{-3,-1,1,3\}\}$. Verglichen mit den Modellen 17 bis 21 wich die von Modell 22 implizierte Var\-ianz-Ko\-var\-ianz\-ma\-trix am wenigsten von der empirischen Var\-ianz-Ko\-var\-ianz\-ma\-trix ab. Der \gls{cst} war nicht signifikant und der \gls{cfi}, der \gls{rmsea} und das \gls{srmr} deuteten auf eine gute Modellpassung hin. 
Die Varianz der konstanten latenten Variable betrug $653.98$ ($z=5.75$, $p<.001$) und die Varianz der dynamischen latenten Variable betrug $2573.97$ ($z=6.90$, $p<.001$). Die Skalierung der Varianzen \citep{Schweizer2011a} hat ergeben, dass die konstante latente Variable $39\,\%$ und die dynamische latente Variable $61\,\%$ von der in den mainfesten Variablen gemeinsamen Varianz band.


\begin{figure}[htbp]
	\centering
	\begin{tikzpicture}
	[font=\sffamily, scale=2, inner sep=0pt,
	latent/.style	= {circle,draw,inner sep=0pt,minimum size=12mm},
	manifest/.style	= {rectangle,draw,inner sep=0pt,minimum width=12mm,minimum height=12mm},
	paths/.style	= {->, >=stealth, shorten >= 1pt},
	error/.style	= {circle, draw=none, fill=white, minimum size=5mm},
	covar/.style	= {<->, >=stealth, shorten >= 1pt, shorten <= 1pt}]
	
	\node at (1, 2.4)	[latent]	(hk)	{H\textsubscript{kon}};
	\node at (1, 1)		[latent]	(hd)	{H\textsubscript{dyn}};
	
	\node at (-1.5, 2.9)	[manifest]	(h0)	{0-bit};
	\node at (-1.5, 2.1)	[manifest]	(h1)	{1-bit};
	\node at (-1.5, 1.3)	[manifest]	(h2)	{2-bit};
	\node at (-1.5, 0.5)	[manifest]	(h3)	{2.58-bit};
	
	\node at (-2.3, 2.9)	[error]		(e1)	{\footnotesize .19};
	\node at (-2.3, 2.1)	[error]		(e2)	{\footnotesize .21};
	\node at (-2.3, 1.3)	[error]		(e3)	{\footnotesize .14};
	\node at (-2.3, .5)		[error]		(e4)	{\footnotesize .24};
	
	\draw [paths] (hk.west) -- (h0.east) node[minimum size = 4mm, draw=none, fill=white, near start] {\footnotesize .90{$^{1}$}};
	\draw [paths] (hk.west) -- (h1.east) node[minimum size = 4mm, draw=none, fill=white, near start] {\footnotesize .81{$^{1}$}};
	\draw [paths] (hk.west) -- (h2.east) node[minimum size = 4mm, draw=none, fill=white, near start] {\footnotesize .50{$^{1}$}};
	\draw [paths] (hk.west) -- (h3.east) node[minimum size = 4mm, draw=none, fill=white, near start] {\footnotesize .40{$^{1}$}};
	
	\draw [paths] (hd.west) -- (h0.east) node[minimum size = 4mm, draw=none, fill=white, near start] {\footnotesize .04{$^{0.02}$}};
	\draw [paths] (hd.west) -- (h1.east) node[minimum size = 4mm, draw=none, fill=white, near start] {\footnotesize .36{$^{0.22}$}};
	\draw [paths] (hd.west) -- (h2.east) node[minimum size = 4mm, draw=none, fill=white, near start] {\footnotesize .78{$^{0.78}$}};
	\draw [paths] (hd.west) -- (h3.east) node[minimum size = 4mm, draw=none, fill=white, near start] {\footnotesize .78{$^{0.98}$}};
	
	\draw [paths] (e1) -- (h0.west);
	\draw [paths] (e2) -- (h1.west);
	\draw [paths] (e3) -- (h2.west);
	\draw [paths] (e4) -- (h3.west);
	\end{tikzpicture}
	
	\vspace{.2cm}
	\caption[Modell 22: Fixed-Links-Messmodell der \gls{ha}]{Modell 22: Fixed-Links-Messmodell der \gls{ha} (\textsf{H}). Eingezeichnet sind die standardisierten Koeffizienten. Hochgestellt sind die fixierten unstandardisierten Faktorladungen. \textsf{\textsubscript{kon}} = konstante latente Variable; \textsf{\textsubscript{dyn}} = dynamische latente Variable.}
	\label{fig:hick_fixedlinks_measurement_model}
\end{figure}

Im Vergleich zum kongenerischen Messmodell der \gls{ha} (Modell 14) vermochte das Fixed-Links-Messmodell (Modell 22) die empirischen Daten deutlich besser abzubilden. Die bessere Passung von Modell 22 äusserte sich im Vergleich zu Modell 14 in einem nicht-signifikanten $\upchi^2$-Wert, im akzeptablen \gls{cfi} und \gls{rmsea} sowie in zwei zusätzlichen Freiheitsgraden. Modell 8 war Modell 1 also aufgrund adäquaterer Abbildung der empirischen Daten und höherer Sparsamkeit vorzuziehen.

In einem letzten Schritt wurde das Fixed-Links-Messmodell der \gls{ssauf} (Modell 8) mit dem Fixed-Links-Messmodell der \gls{ha} (Modell 22) und dem \gls{gfaktor} aus dem \gls{bist} in Verbindung gebracht (Modell 23; siehe \autoref{fig:spatialsuppression_hick_fixedlinks_sem}). Das Modell bildete die empirischen Varianzen und Kovarianzen gut ab. Der \gls{cst} war nicht signifikant und der \gls{cfi}, der \gls{rmsea} und das \gls{srmr} lagen im akzeptablen Bereich, $\upchi^2(40)=48.81$, $p=.16$, $\textnormal{CFI}=.99$, $\textnormal{RMSEA}~=~.04$, $\textnormal{SRMR}~=~.08$. 
Die standardisierten Regressionskoeffizienten betrugen zwischen der konstanten latenten Variable der \gls{ssauf} und dem \gls{gfaktor} $\upbeta~=~-.21$ ($p~=~.06$), zwischen der dynamischen latenten Variable der \gls{ssauf} und dem \gls{gfaktor} $\upbeta~=~-.08$ ($p~=~.38$), zwischen der konstanten latenten Variable der \gls{ha} und dem \gls{gfaktor} $\upbeta~=~-.17$ ($p~=~.06$) und zwischen der dynamischen latenten Variable der \gls{ha} und dem \gls{gfaktor} $\upbeta~=~-.26$ ($p~=~.002$). Gemeinsam erklärten die konstanten und dynamischen latenten Variablen der Spatial-Suppression- und \gls{ha} $16\,\%$ der Varianz im \gls{gfaktor}. Die Korrelationskoeffizient betrug zwischen den beiden konstanten latenten Variablen $r=.21$ ($p=.005$), zwischen den beiden dynamischen latenten Variablen $r=-.13$ ($p=.11$), zwischen der konstanten latenten Variable der \gls{ha} und der dynamischen latenten Variable der \gls{ssauf} $r=.15$ ($p=.09$) und zwischen der konstanten latenten Variable der \gls{ssauf} und der dynamischen latenten Variable der \gls{ha} $r=.00$ ($p=.97$).

\begin{figure}[t]
	\centering
	\begin{tikzpicture}
	[font=\sffamily, scale=2, inner sep=0pt,
	latent/.style	= {circle,draw,inner sep=0pt,minimum size=12mm},
	manifest/.style	= {rectangle,draw,inner sep=0pt,minimum width=12mm,minimum height=12mm},
	paths/.style	= {->, >=stealth, shorten >= 1pt},
	error/.style	= {circle, draw=none, fill=white, minimum size=5mm},
	covar/.style	= {<->, >=stealth, shorten >= 1pt, shorten <= 1pt}]
	
	\node at (1, 2.4)		[latent]	(sk)	{S\textsubscript{kon}};
	\node at (1, 1)			[latent]	(sd)	{S\textsubscript{dyn}};
	\node at (1, -1)		[latent]	(hk)	{H\textsubscript{kon}};
	\node at (1, -2.4)		[latent]	(hd)	{H\textsubscript{dyn}};
	\node at (2.5, 0)		[latent]	(g)		{\textrm{\textit{g}}};
	\node at (2.5, 0.8)		[error]		(e12)	{\footnotesize .84};
	
	\draw [paths] (e12) -- (g.north);
	\draw [paths] (sk) -- (g.west) node[minimum size = 4mm, draw=none, fill=white, midway] {\footnotesize --.21\hphantom{$^{**}$}};
	\draw [paths] (sd) -- (g.west) node[minimum size = 4mm, draw=none, fill=white, midway] {\footnotesize --.08\hphantom{$^{**}$}};
	\draw [paths] (hk) -- (g.west) node[minimum size = 4mm, draw=none, fill=white, midway] {\footnotesize --.17\hphantom{$^{**}$}};
	\draw [paths] (hd) -- (g.west) node[minimum size = 4mm, draw=none, fill=white, midway] {\footnotesize --.26{$^{**}$}};
	
	\path [covar] (hk.west) edge [bend left=45] node[minimum size = 4mm, draw=none, fill=white, near end]   {\footnotesize .21{$^{**}$}} (sk.west);
	\path [covar] (hk.west) edge [bend left=25] node[minimum size = 4mm, draw=none, fill=white, midway]     {\footnotesize .15} (sd.west);
	\path [covar] (hd.west) edge [bend left=45] node[minimum size = 4mm, draw=none, fill=white, near start] {\footnotesize --.13} (sd.west);
	\path [covar] (hd.west) edge [bend left=65] node[minimum size = 4mm, draw=none, fill=white, midway]     {\footnotesize --.00} (sk.west);
	
	\end{tikzpicture}
	
	\vspace{.2cm}
	\caption[Modell 23: Fixed-Links-Strukturgleichungsmodell zur Vorhersage des \gls{gfaktor}s durch die Spa\-ti\-al-Sup\-pres\-sion- und die \gls{ha}]{Modell 23: Latenter Zusammenhang zwischen dem Fixed-Links-Messmodell der \gls{ssauf} (\textsf{S}; Modell 8), dem Fixed-Links-Messmodell  der \gls{ha} (\textsf{H}; Modell 22) und dem \gls{gfaktor} aus dem \gls{bist}. Abgebildet ist das Strukturmodell. Eingezeichnet sind die standardisierten Koeffizienten. \textsf{\textsubscript{kon}} = konstante latente Variable; \textsf{\textsubscript{dyn}} = dynamische latente Variable.\\
		$^{**}p~<~.01$.}
	\label{fig:spatialsuppression_hick_fixedlinks_sem}
\end{figure}


Im Vergleich zum herkömmlichen Strukturgleichungsmodell (Modell 15) vermochte das Fixed-Links-Strukturgleichungsmodell (Modell 23) die empirischen Daten deutlich besser abzubilden. Die bessere Passung von Modell 23 äusserte sich im Vergleich zu Modell 15 in einem nicht-signifikanten $\upchi^2$-Wert und in einem akzeptablen \gls{cfi} und \gls{rmsea}. Bezüglich der Varianzaufklärung im \gls{gfaktor} waren Modell 15 ($16\,\%$) und Modell 23 ($16\,\%$) identisch. Modell 23 war Modell 15 folglich aufgrund adäquaterer Abbildung der empirischen Daten vorzuziehen. 

Abschliessend zur fünften Fragestellung kann folgendes Festgehalten werden: 
Auf manifester Ebene vermochte die \gls{ssauf} (sowohl auf Ebene der Aufgabenbedingungen als auch auf Ebene der Aufgabenparameter) bei einer Kontrolle des Zusammenhang zwischen der \gls{ha} und psychometrischer Intelligenz keinen inkrementellen Beitrag zur Aufklärung individueller Intelligenzunterschiede leisten.

Auf latenter Ebene zeigte sich in einem herkömmlichen Strukturgleichungsmodell (Modell 15) ein geringer bis mittlerer negativer Zusammenhang zwischen der aus den vier Bedingungen der \gls{ssauf} abgeleiteten latenten Variable und dem \gls{gfaktor}. Tiefere Faktorwerte auf der aus den vier Bedingungen der \gls{ssauf} extrahierten latenten Variable waren also tendenziell mit höheren Faktorwerten im \gls{gfaktor} verbunden. 
Dieser latente Zusammenhang erklärte ungefähr $4\,\%$ der Varianz im \gls{gfaktor} und leistete damit bei Berücksichtigung des Zusammenhangs zwischen der \gls{ha} und dem \gls{gfaktor} einen inkrementellen Beitrag zur Aufklärung individueller Intelligenzunterschiede. Diese Resultate müssen jedoch aufgrund des schlechten Modell-Fits mit Vorsicht interpretiert werden.

Bei der Analyse der Zusammenhänge mittels Fixed-Links-Struk\-tur\-glei\-chungs\-mo\-dell zeigte sich zwischen der konstanten latenten Variable der \gls{ssauf} und der konstanten latenten Variable der \gls{ha} ein geringer bis mittlerer positiver Zusammenhang. Tiefere Faktorwerte in der einen latenten Variable waren also tendenziell mit tieferen Faktorwerten in der anderen latenten Variable verbunden. Signifikanter Prädiktor des \gls{gfaktor}s war bei dem gewählten $\upalpha$-Fehler von $5\,\%$ nur die dynamische latente Variable der \gls{ha}, welche einen (geringen bis) mittleren negativen Zusammenhang mit dem \gls{gfaktor} aufwies. Tiefere Faktorwerte in der dynamischen latenten Variable der \gls{ha} waren also tendenziell mit höheren Faktorwerten im \gls{gfaktor} verbunden.















% =================================================================
% D I S C U S S I O N
% =================================================================
\chapter{Diskussion \label{cha:Diskussion}}
%\ac{MLS}
\begin{itemize}
	\item konsistent über alle modelle, zusammenhang da latent
	\item Anderes Resultate, weil anderer IQ-Test eingesetzt?
	\item Anderes Resultat, weil nicht Projektor eingesetzt?
	\item für diese kurzversion betsehenkeine normen. wir sind aber auch nicht am iq sondern an der varianzaufklärung interessiert. deshalb nicht relevant.
	\item 360-Hz Code auf 144-Hz Monitor
\end{itemize}

p-Wert problematisch:\\
\citet{Gelman2006}\\
\citet{Wasserstein2016}\\
\citet{Nuzzo2014}
\citet{Hayduk2014} Shame on disrespecting evividenc p 6 of 10


fixed-links als Lösung für exponentielle Regression, da hats einfach ein paar Leute mit hohem RMSE gehabt. im sem drückt lässt sich der missfit quantifizieren (modelltest)





% =================================================================
% G L O S S A R Y   &   A C R O N Y M S
% =================================================================
\printglossaries	% put this where you want your list of entries to appear 
% Important:
% 1. compile document with PdfLaTeX
% 2. run 'makeglossaries diss' in terminal
% 3. compile document with pdfLateX
% 4. -> glossaries should appear in pdf


% =================================================================
% R E F E R E N C E S 
% =================================================================
\renewcommand\bibname{Literatur}				% rename bibliography
\bibliography{bibliography.bib}					% provide .bib file
\addcontentsline{toc}{chapter}{Literatur}		% for a toc entry
%../references

% =================================================================
% A P P E N D I X   A
% =================================================================
\appendix
% not sure if this code works -----------
\setcounter{figure}{0}
\renewcommand\thefigure{\Alph{appndx}\@arabic\c@figure}
% works --------------------------
\setcounter{table}{0}
\renewcommand{\thetable}{A\arabic{table}}

\chapter{Anhang \label{cha:Anhang_A}}
Dieser Anhang beschreibt die Vorgehensweise bei der Datenaufbereitung, welche zum Ausschluss von \glspl{vp} geführt hat (vgl. \autoref{sec:Stichprobe}). Am Ende des Anhangs fasst \autoref{tab:Datenbereinigung} die Datenbereinigung zusammen.

\section{Alter}
Trotz sorgfältiger Auswahl der \glspl{vp} hat sich nachträglich bei der Altersberechnung herausgestellt, dass drei \glspl{vp} zum Testzeitpunkt noch nicht $18$~Jahre alt waren. Sie wurden vor der Analyse entfernt.

\section{Spatial-Suppression-Aufgabe}
Zu Beginn der Datenerhebung wurde die \gls{ssauf} mit einem Kontrast von $74\,\%$ dargeboten. Nach Inspektion der Daten der sieben ersten getesteten \glspl{vp} wurde in Absprache mit Duje Tadin entschieden, den Kontrast der Aufgabe auf $99\,\%$ zu erhöhen. Mit dieser Erhöhung des Kontrasts wurde sichergestellt, dass die über die vier Mustergrössen hinweg erwartete Verschlechterung der Wahrnehmungsleistung möglichst gross ausfällt \citep[für den Zusammenhang zwischen Kontrast und Wahrnehmungsleistung siehe][]{Tadin2003}. Den restlichen \glspl{vp} wurde die Aufgabe folglich mit einem Kontrast von $99\,\%$ dargeboten und die Daten der ersten sieben \glspl{vp} wurden von der Analyse ausgeschlossen.

Der Code, welcher die Darbietungszeiten generierte, hatte eine fest-codierte Darbietungszeitlimite von $1000$~ms. Immer wenn der adaptive Alogrithmus des QUEST-Verfahrens \citep{Watson1983} eine Darbietungszeit von $> 1000$~ms ermittelte, wurde den \glspl{vp} deshalb der Stimulus mit einer Darbietungszeit von exakt $1000$~ms präsentiert. 
Die Daten von $12$~\glspl{vp} mussten vor der Analyse entfernt werden, weil sie bei den sechs Schätzungen der $82\,\%$-Er\-ken\-nungs\-schwel\-le innerhalb einer Mustergrösse mehr als ein Mal eine $82\,\%$-Er\-ken\-nungs\-schwel\-le von $> 1000$~ms erzielt hatten. Dasselbe Ausschlussverfahren verwendeten auch \citet{Melnick2013}.

Die Daten von zwei \glspl{vp} wurden von der Analyse ausgeschlossen, weil sie verglichen mit den restlichen \glspl{vp} in der $1.8^{\circ}$-Be\-ding\-ung eine gemittelte $82\,\%$-$\log_{10}$-Er\-ken\-nungs\-schwel\-le hatten, die über das dreifache der \gls{sd} der $82\,\%$-$\log_{10}$-Er\-ken\-nungs\-schwel\-le der Gesamtstichprobe betrug. Diese drei \glspl{vp} wurden nicht zur Grundpopulation gezählt und vor der Analyse entfernt.

%\section{\gls{ha}}
%Bei der \gls{ha} mussten keine \glspl{vp} ausgeschlossen werden.

\section{BIS-Test}
Bei den Subtests \gls{bd}, \gls{kw}, \gls{oe}, \gls{RZ}, \gls{tg}, \gls{uw} und \gls{xg} ist der Rohwert Null im Manual des \gls{bist} \citep{Jaeger1997} keinem Punktwert zugeordnet. Vier \glspl{vp} erzielten beim Subtest \gls{xg} einen Rohwert von Null, was den Subtest nicht auswertbar machte. Die Daten dieser vier \glspl{vp} wurden vor der Analyse aufgrund dieses nicht auswertbaren Subtests entfernt. Eine \gls{vp} wurde von der Analyse ausgeschlossen, weil sie bei den \gls{b}-Subtests deutlich schlechter Abschnitt als der Rest der Stichprobe und damit einen Einfluss auf die berechneten Zusammenhänge gehabt hätte.

\begin{table}[htbp]
	\centering
	\captionsetup{labelsep = none}
	\caption[Übersicht über die Datenbereinigung]{\newline  \textit{Übersicht über die Datenbereinigung} \vspace{.2cm}}
	\label{tab:Datenbereinigung}
	\begin{adjustbox}{width=1\textwidth}
		\begin{threeparttable}
			\begin{tabular}{
					l
					l
					S[table-format = 3.0]
					S[table-format = 2.0]
					S[table-format = 2.0]
					c
					S[table-format = 3.0]
					S[table-format = 2.0]
					S[table-format = 2.0]
					}
				\hline
					&	&	\multicolumn{3}{c}{absolut}	&	&	\multicolumn{3}{c}{relativ (\%)} \\
				\cline{3-5}
				\cline{7-9}
				Beschrieb & \multicolumn{1}{c}{Korrektur für} &{\textit{N}} & {D} & {D kum.}	&	&	{\textit{N}} & {D} & {D kum.}\\
				\hline
				
				Getestet	&	-								&	206	&		&		&&	100	&		&		\\
							&	Alter							&	203	&	-3	&	-3	&&	99	&	-2	&	-2	\\
							&	Spatial-Suppression-Aufgabe		&	182	&	-21	&	-24	&&	88	&	-10	&	-12	\\
							&	BIS-Test						&	177	&	-5	&	-29	&&	86	&	-2	&	-14	\\
				Analysiert	&	-								&	177	&		&		&&	86	&		&		\\
				\hline
			\end{tabular}

			\begin{tablenotes}[flushleft]
				\footnotesize				% font size
				\setlength\labelsep{0pt}	% no indent on second line
				\item \textit{Anmerkungen.} \textit{N} = Stichprobengrösse, D = Differenz, D kum. = kumulierte Differenz.
			\end{tablenotes}
		\end{threeparttable}
	\end{adjustbox}
\end{table}


% =================================================================
% A P P E N D I X   B
% =================================================================
\chapter{Anhang \label{cha:Anhang_B}}

Dieser Anhang beinhaltet Ergebnisse der Deskriptiv- und Inferenzstatistik, welche sich bei der Anwendung nonparametrischer Analyseverfahren ergeben haben. Die Ergebnisse dieser nonparametrischer Analyseverfahren wichen nicht bedeutend von den mit parametrischen Verfahren ermittelten Ergebnissen ab (vgl. \autoref{sec:Deskriptive_Statistik}).

\section{Spatial-Suppression-Aufgabe}

Um zu prüfen, ob die experimentelle Manipulation (die Musterrösse) einen Einfluss auf die abhängige Variable (die $82\,\%$-Er\-ken\-nungs\-schwel\-le) ausübte, wurde ein Friedman-Test durchgeführt. Der Globaltest hat gezeigt, dass die Unterschiede zwischen den Bedingungen signifikant waren, $\upchi^2(3)=345.26$, $p<.001$. 
Um zu erfahren, welche Bedingungen sich voneinander unterschieden, wurden Post-hoc-Tests \citep{Galili2010, Hollander2014} gerechnet. Diese haben ergeben, dass sich von den (durch die vier Bedingungen bestimmten) sechs Einzelvergleichen nur die $1.8^{\circ}$- und $3.6^{\circ}$-Bedingung nicht signifikant voneinander unterschieden ($p=.09$). Die restlichen fünf Einzelvergleiche waren mit $p<.001$ alle statistisch signifikant.

\section{Hick-Aufgabe}

Um zu testen, ob die experimentelle Manipulation (die Anzahl an Antwortalternativen) einen Einfluss auf die abhängige Variable (die Reaktionszeit) ausübte, wurde ein Friedman-Test durchgeführt. Der Globaltest belegte, dass die Unterschiede zwischen den Bedingungen signifikant waren, $\upchi^2(3)=516.12$, $p<.001$. Welche Bedingungen sich voneinander unterschieden, wurde mit Post-hoc-Tests \citep{Galili2010, Hollander2014} geprüft. Diese haben gezeigt, dass sich alle Bedingungen signifikant voneinander unterschieden (alle $p\textnormal{s}<.001$). 

\section{BIS-Test}

Die Zusammenhänge der Subtests wurden mit Spearmans Rangkorrelationen bestimmt und sind in \autoref{tab:bis_subtest_description_subtest_correlations_parametric_and_nonparametric} unterhalb der Diagonale abgetragen. Oberhalb der Diagonale sind die Differenzen zwischen den Produkt-Moment-Korrelationen und Spearmans Rangkorrelationen abgetragen.

\section{Zusammenhangsmasse}

Die Zusammenhänge der Subtests wurden mit Spearmans Rangkorrelationen bestimmt und sind in \autoref{tab:correlations_between_manifest_parametric_and_nonparametric} unterhalb der Diagonale abgetragen. Oberhalb der Diagonale sind die Differenzen zwischen den Produkt-Moment-Korrelationen und Spearmans Rangkorrelationen abgetragen.

\begin{sidewaystable}[htbp]
	\captionsetup{labelsep = none}
	\caption[Spearmans Rangkorrelationen zwischen den Subtests des \gls{bist}s]{\newline  \textit{Spearmans Rangkorrelationen (unterhalb der Diagonale) zwischen den Subtests des \gls{bist}. Oberhalb der Diagonale sind die Differenzen zwischen der Produkt-Moment-Korrelation und Spearmans Rangkorrelation abgetragen} \vspace{.2cm}}
	\label{tab:bis_subtest_description_subtest_correlations_parametric_and_nonparametric}
	\sisetup{table-space-text-post = $^{*ab}$  }
	\begin{adjustbox}{width=1\textwidth,totalheight=1\textheight,keepaspectratio}
		\begin{threeparttable}
			\begin{tabular}{
					l
					l
					S[table-format = 0.2, add-integer-zero=false]
					S[table-format = 0.2, add-integer-zero=false]
					S[table-format = 0.2, add-integer-zero=false]
					S[table-format = 0.2, add-integer-zero=false]
					S[table-format = 0.2, add-integer-zero=false]
					S[table-format = 0.2, add-integer-zero=false]
					S[table-format = 0.2, add-integer-zero=false]
					S[table-format = 0.2, add-integer-zero=false]
					S[table-format = 0.2, add-integer-zero=false]
					S[table-format = 0.2, add-integer-zero=false]
					S[table-format = 0.2, add-integer-zero=false]
					S[table-format = 0.2, add-integer-zero=false]
					S[table-format = 0.2, add-integer-zero=false]
					S[table-format = 0.2, add-integer-zero=false]
					S[table-format = 0.2, add-integer-zero=false]
					S[table-format = 0.2, add-integer-zero=false]
					S[table-format = 0.2, add-integer-zero=false]
					S[table-format = 0.2, add-integer-zero=false]
					>{\centering\arraybackslash}p{1.2cm}
				}
				\hline
				&	\multicolumn{1}{c}{Subtest}&	{1}	&	{2}	&	{3}	&	{4}	&	{5}	&	{6}	&	{7}	&	{8}	&	{9}	&	{10}&	{11}&	{12}&	{13}&	{14}&	{15}&	{16}&	{17}&	{18}	\\
				\hline
				
1	&	OG	&					&	.02				&	.05			&.00	&.04	&-.05	&.01	&-.12	&-.02	&.04	&.03	&-.01	&-.02	&.00	&.01	&-.02	&.04	&-.03	\\
2	&	ZN	&	.25{$^{***}$}	&					&	.00			&.01	&.00	&.02	&.00	&.00	&.01	&.04	&.02	&-.03	&.01	&.01	&.00	&.03	&.06	&.04	\\
3	&	AN	&	.26{$^{***}$}	&	.42{$^{***}$}	&				&-.03	&.02	&-.02	&.03	&-.12	&-.01	&.01	&.01	&-.03	&-.01	&.05	&.00	&.01	&.05	&.02	\\
4	&	XG	&	.21{$^{**}$}	&	.55{$^{***}$}	&	.35{$^{***}$}	&	&.01	&.01	&-.04	&-.08	&.00	&.01	&.02	&-.05	&.00	&.00	&-.03	&.01	&.01	&.03	\\
5	&	WA	&	.31{$^{***}$}	&	.41{$^{***}$}	&	.47{$^{***}$}	&	.34{$^{***}$}	&&-.01	&.03	&-.03	&.01	&.01	&.01	&-.02	&-.01	&.06	&.01	&.01	&.05&.00\\
6	&	ZP	&	.27{$^{***}$}	&	.15{$^a$}		&	.15{$^{*}$}	&	.30{$^{***}$}	&	.17{$^{*}$}		&	&	-.03&-.04	&.00	&.02	&-.01	&-.03	&-.02	&.01	&.00	&.03	&-.02	&.02	\\
7	&	TM	&	.29{$^{***}$}	&	.26{$^{***}$}	&	.41{$^{***}$}	&	.36{$^{***}$}	&	.48{$^{***}$}	&	.24{$^{**}$}	&	&-.07	&.00	&.02	&.02	&-.04	&-.03	&.05	&.02	&.01	&.01	&.02	\\
8	&	BD	&	.19{$^{*}$}		&	.08				&	.17{$^{*}$}		&	.19{$^{*}$}		&	.02				&	.08				&	.10	&	&-.05	&.02	&-.05	&-.07	&-.05	&-.05	&-.04	&-.02	&-.12	&-.06	\\
9	&	SC	&	.16{$^{*}$}		&	.51{$^{***}$}	&	.35{$^{***}$}	&	.48{$^{***}$}	&	.22{$^{**}$}	&	.17{$^{*}$}		&	.32{$^{***}$}	&	.25{$^{***}$}	&	&.01	&.01	&.01&.00&-.03	&.02&.01	&.01&.01\\
10	&	ST	&	.34{$^{***}$}	&	.15	{$^{*}$}	&	.23{$^{**}$}	&	.30{$^{***}$}	&	.31{$^{***}$}	&	.22{$^{**}$}	&	.37{$^{***}$}	&	-.03			&	.22{$^{**}$}	&		&.04&.02&.00&-.05&.03&.02&.04&.03\\
11	&	CH	&	.33{$^{***}$}	&	.49{$^{***}$}	&	.51{$^{***}$}	&	.29{$^{***}$}	&	.50{$^{***}$}	&	.14				&	.30				&	.12				&	.31{$^{***}$}	&	.13	&	&-.01&.01&.01&.00&.02&.07&.02	\\
12	&	TG	&	.33{$^{***}$}	&	.36{$^{***}$}	&	.30{$^{***}$}	&	.48{$^{***}$}	&	.45{$^{***}$}	&	.19{$^{*}$}		&	.47{$^{***}$}	&	.18{$^{*}$}		&	.26{$^{***}$}&	.36{$^{***}$}	&	.23{$^{**}$}	&	&-.08&.00&-.05&.02&.00&.00	\\
13	&	RZ	&	.32{$^{***}$}	&	.52{$^{***}$}	&	.42{$^{***}$}	&	.55{$^{***}$}	&	.44{$^{***}$}	&	.29{$^{***}$}	&	.45{$^{***}$}	&	.13				&	.44{$^{***}$}&	.34{$^{***}$}	&	.37{$^{***}$}	&	.41{$^{***}$}&&-.03&.01&.01&.02&.01	\\
14	&	WM	&	.41{$^{***}$}	&	.10				&	.24{$^{**}$}	&	.18{$^{*}$}		&	.20{$^{**}$}	&	.26{$^{***}$}	&	.34{$^{***}$}	&	.13				&	.13			&	.45{$^{***}$}	&	.16{$^{*}$}		&	.18{$^{*}$}		&	.15{$^{*}$}	&&-.01&.01&-.02&-.03	\\
15	&	KW	&	.26{$^{***}$}	&	.24{$^{***}$}	&	.28{$^{***}$}	&	.39{$^{***}$}	&	.39{$^{***}$}	&	.24{$^{**}$}	&	.54{$^{***}$}	&	.19{$^{*}$}		&	.26{$^{***}$}&	.49{$^{***}$}	&	.21{$^{**}$}	&	.60{$^{***}$}	&	.35{$^{***}$}	&	.34{$^{***}$}	&&.03&.04&-.01	\\
16	&	ZZ	&	.31{$^{***}$}	&	.02				&	.03				&	.20{$^{**}$}	&	.00				&	.34{$^{***}$}	&	.09				&	.11				&	.04			&	.28{$^{***}$}	&	.05				&	.05				&	.08				&	.38{$^{***}$}	&	.10	&&.02&.01	\\
17	&	OE	&	.06				&	-.01			&	-.01			&	.07				&	-.05			&	.04				&	.12				&	.46{$^{***}$}	&	.15{$^{*}$}&	-.01			&	-.13			&	.15{$^{*}$}		&	.13					&	.03	&	.13		& -.05&&.02	\\
18	&	WE	&	.42{$^{***}$}	&	.27{$^{***}$}	&	.26{$^{***}$}	&	.19{$^{*}$}		&	.29{$^{***}$}	&	.25{$^{***}$}	&	.06				&	.04				&	.14			&	.21{$^{**}$}	&	.25{$^{**}$}	&	.20{$^{**}$}	&	.33{$^{***}$}	&.19{$^{*}$}&	.23{$^{**}$}&.18{$^{*}$}&-.12&\\
\hline			
			\end{tabular}
			
			\begin{tablenotes}[flushleft]
				\footnotesize				% font size
				\setlength\labelsep{0pt}	% no indent on second line
				\item \textit{Anmerkung.} Siehe \autoref{tab:bis_subtest_description} für eine Beschreibung der Subtests.
				\item {$^a$}Der exakte Zusammenhang betrug $r_{s}=.147$, $p=.051$. Alle restlichen in der Tabelle mit $r_{s}=.15$ bezeichneten Korrelationskoeffizienten wiesen \textit{p}-Werte $<.05$ auf.
				\item {$^{*}$}$p<.05$. {$^{**}$}$p<.01$. {$^{***}$}$p<.001$ (zweiseitig).
			\end{tablenotes}
		\end{threeparttable}
	\end{adjustbox}
\end{sidewaystable}

\begin{sidewaystable}[htbp]
	\centering
	\captionsetup{labelsep = none}
	\caption[Spearmans Rangkorrelationen zwischen der \gls{ssauf}, dem \gls{si}, der \gls{ha}, dem \textit{z}-Wert und dem \gls{gfaktor} des \gls{bist}s]{\newline  \textit{Spearmans Rangkorrelationen (unterhalb der Diagonale) zwischen den Bedingungen der \gls{ssauf}, dem \gls{si}, den Bedingungen der \gls{ha}, dem \textit{z}-Wert und dem \gls{gfaktor} des \gls{bist}. Oberhalb der Diagonale sind die Differenzen zwischen der Produkt-Moment-Korrelation und Spearmans Rangkorrelation abgetragen} \vspace{.2cm}}
	\label{tab:correlations_between_manifest_parametric_and_nonparametric}
	\sisetup{table-space-text-post = $^{***}$}
%	\begin{adjustbox}{width=.85\textwidth, keepaspectratio}
		\begin{threeparttable}
			\begin{tabular}{
					l
					l
					S[table-format = 0.2, add-integer-zero=false]
					S[table-format = 0.2, add-integer-zero=false]
					S[table-format = 0.2, add-integer-zero=false]
					S[table-format = 0.2, add-integer-zero=false]
					S[table-format = 0.2, add-integer-zero=false]
					p{.001cm}
					S[table-format = 0.2, add-integer-zero=false]
					S[table-format = 0.2, add-integer-zero=false]
					S[table-format = 0.2, add-integer-zero=false]
					S[table-format = 0.2, add-integer-zero=false]
					p{.001cm}
					S[table-format = 0.2, add-integer-zero=false]
					S[table-format = 0.2, add-integer-zero=false]
					>{\centering\arraybackslash}p{1.2cm}
				}
				\hline
				
				
				&	& 	\multicolumn{5}{c}{\gls{ssauf}}	&	&	\multicolumn{4}{c}{\gls{ha}}	&	&	\multicolumn{2}{c}{\gls{bist}}	\\
				
				\cline{3-7}
				\cline{9-12}
				\cline{14-15}
				
				&	\multicolumn{1}{c}{Mass}			&	{1}				&	{2}				&	{3}				&	 {4}	& {5}& 	& {6}	& {7}	& {8}	&{9}&&{10}&{11} \\
				\hline
1	&	$1.8^{\circ}$	&					&	.01				&	.00	&	-.01	&	.05	&&	.05	&	.06	&	.01	&	.07	&&	-.02	&-.01\\
2	&	$3.6^{\circ}$	&	.84{$^{***}$}	&					&	.01	&	-.04	&	.02	&&	.05	&	.05	&	-.03&	.04	&&	-.05	&-.02\\
3	&	$5.4^{\circ}$	&	.73{$^{***}$}	&	.86{$^{***}$}	&		&	-.01	&	.08	&&	.09	&	.10	&	.01	&	.10	&&	-.07	&-.06\\
4	&	$7.2^{\circ}$	&	.55{$^{***}$}	&	.76{$^{***}$}	&	.88{$^{***}$}	&&	.13	&&	.02	&	.07	&	.00	&	.08	&&	-.06	&-.06\\
5	&	SI 				&	-.33{$^{***}$}	&	.03				&	.26{$^{***}$}	&	.53{$^{***}$}&&&.00&.05&.02 &	.05 &&	-.04 	&-.03 \\
\rule{0pt}{4ex}%  EXTRA vertical height
6	&	0-bit			&	.12				&	.19{$^{**}$}	&	.16{$^{*}$}	&	.12		&.01&	&		&.04&.06&.12&&-.04&-.04\\
7	&	1-bit			&	.03				&	.06				&	.03			&	.00		&-.05&	&.72{$^{***}$}	&	&.01	&.03	&	&.05&.04		\\
8	&	2-bit			&	.11				&	.11				&	.07			&	.04		&-.08&	&.52{$^{***}$}	&	.71{$^{***}$}	&	&.02	&	&.03&.04\\
9	&	2.58-bit		&	.07				&	.05				&	.02			&	-.01	&-.09&	&.40{$^{***}$}	&	.63{$^{***}$}	&	.81{$^{***}$}	&	&	&.03&.03		\\
\rule{0pt}{4ex}%  EXTRA vertical height
10	&	\textit{z}-Wert	&	-.14			&	-.12			&	-.09		&	-.06	&.04&	&	-.15{$^{*}$}	&	-.32{$^{***}$}	&	-.31{$^{***}$}	&	-.31{$^{***}$}	&				&	&.00	\\
11	&	\gls{gfaktor}	&	-.18{$^{*}$}	&	-.18{$^{*}$}	&	-.14		&	-.11	&.03&	&	-.15{$^{*}$}	&	-.29{$^{***}$}	&	-.29{$^{***}$}	&	-.28{$^{***}$}	&				&	.97{$^{***}$} & 	\\
				\hline
				
			\end{tabular}

			\begin{tablenotes}[flushleft]
				\footnotesize				% font size
				\setlength\labelsep{0pt}	% no indent on second line
				\item \textit{Anmerkungen}. SI = \gls{si}. \gls{zwert} = Mittelwert aus allen 18 \textit{z}-standardisierten Subtests.
				\item {$^{*}$}$p<.05$. {$^{**}$}$p<.01$. {$^{***}$}$p<.001$ (zweiseitig).
			\end{tablenotes}
		\end{threeparttable}
%	\end{adjustbox}
\end{sidewaystable}







\end{document}




